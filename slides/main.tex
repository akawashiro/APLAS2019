\documentclass[dvipdfmx,aspectratio=169, 20pt]{beamer}
\usepackage[absolute,overlay]{textpos}
\usepackage{amsmath}
\usepackage{amsthm}
\usepackage{ascmac}
\usepackage[backend=biber, style=numeric]{biblatex}
\usepackage{color}
\usepackage{mathtools}
\usepackage{url}
\usepackage{bcprules, proof}
\usepackage{xparse}
\usepackage{xspace}
\usepackage{fancybox}
\usepackage{float}
\usepackage{bcprules}
\usepackage{ebproof}
\usepackage{lscape}
\usepackage{pgfpages}
\usepackage{bm}

% beamer cheat sheet
% http://www.cpt.univ-mrs.fr/~masson/latex/Beamer-appearance-cheat-sheet.pdf

% Make note pages suer simple.
% \setbeamertemplate{note page}[plain]

% This option generate pdf files showing notes on the right hand side of each
% pages. But this optionn has some bugs. I recommend you tou do not use this
% option.
% \setbeameroption{show notes on second screen=right}

% Note options
% \setbeameroption{show notes}
% \setbeameroption{hide notes}
% \setbeameroption{show only notes}

% Show grid for debug
\setbeamertemplate{background}[grid][step=1cm]

% Use default design theme. I quit to use this theme because it has no boxed
% environments.
% \usetheme{default}
\usetheme{Boadilla}

% Add page numbers to footnotes. This line does not work for Boadilla theme.
% \setbeamertemplate{footline}[frame number]

% So, I use some hack in
% https://tex.stackexchange.com/questions/66995/modify-footer-of-slides.
% \makeatother and \makeatletter enable us to change internal package things.
\makeatother
\setbeamertemplate{footline}
{
    \leavevmode
    \hbox{
        \setbeamercolor{coloredboxstuff}{fg=black,bg=white}
        \hskip0.85\paperwidth
        \begin{beamercolorbox}[wd=0.10\paperwidth,ht=2.25ex,dp=1ex,right]{coloredboxstuff}
            {\textbf{\insertframenumber{} / \inserttotalframenumber}}
        \end{beamercolorbox}}
    \vskip0pt
}
\makeatletter

% Remove all navigation symbols
\setbeamertemplate{navigation symbols}{}
% using non standard fonts for beamer
\usefonttheme{professionalfonts}
% Set the font size of titles
% See other font category at
% https://tex.stackexchange.com/questions/183052/what-are-all-the-possible-first-arguments-to-setbeamerfont
\setbeamerfont{title}{size=\normalsize}
\setbeamerfont{frametitle}{size=\normalsize}
\setbeamerfont{normal text}{size=\small}
\setbeamerfont{itemize/enumerate body}{size=\small}
\setbeamerfont{itemize/enumerate subbody}{size=\footnotesize}
\setbeamerfont{frametitle continuation}{size=\small}
\setbeamerfont{framesubtitle}{size=\small}
\setbeamerfont{abstract}{size=\small}
\setbeamerfont{projected text}{size=\small}
\setbeamerfont{block title}{size=\small}
\setbeamerfont{footline}{size=\tiny}

% You need this line in order to enforce normal text font setting.
% https://tex.stackexchange.com/questions/320223/how-to-enforce-a-font-series-in-beamer-for-normal-default-text
\AtBeginDocument{\usebeamerfont{normal text}}

% Settings of Table of Contents page.
% Append underline to currentsection.
% https://tex.stackexchange.com/questions/280813/beamer-change-color-underline-but-do-not-hide-shade-other-section-in-table-of-c/283585
% https://tex.stackexchange.com/questions/414250/numbers-shading-and-alert-in-beamer-table-of-contents
% Page 101 of http://tug.ctan.org/macros/latex/contrib/beamer/doc/beameruserguide.pdf
\setbeamertemplate{section in toc}{
    \inserttocsectionnumber.\ \bf{\underline{\inserttocsection}}
}
\setbeamertemplate{section in toc shaded}{
    \inserttocsectionnumber.\  \inserttocsection
}

\setbeamertemplate{bibliography item}{\insertbiblabel}
% \setbeamertemplate{blocks}[default]
\setbeamertemplate{blocks}[rounded]
\setbeamertemplate{theorems}[normal font]

% \bibliography{main}{}
\addbibresource{main.bib}
\DeclareSortingScheme{mysorting}{\sort{\citeorder}}
\ExecuteBibliographyOptions{sorting=mysorting}

\setlength\intextsep{0pt}
\setlength\textfloatsep{0pt}

\newcommand{\rulefbox}[1]{\fbox{\ensuremath{#1}} \hspace{1mm}}
\newcommand{\figheader}[2]{
  \begin{flushleft}
    #2 {\bf \normalsize #1}
\end{flushleft}}

\defbeamertemplate{description item}{align left}{\insertdescriptionitem\hfill}

% Load symbol macros
\newcommand{\LTP}{$\lambda^{\triangleright\%}$\xspace}
\newcommand{\LMD}{$\lambda^{\textrm{MD}}$\xspace}
\newcommand{\LLF}{$\lambda\textrm{LF}$\xspace}

\newcommand{\G}{\Gamma}
\newcommand{\D}{\Delta}
\newcommand{\V}{\vdash_\Sigma}
\newcommand{\VT}{\vdash\hspace{-.50em}\raisebox{0.28em}{\tiny{$\TB$}}}
\newcommand{\iskind}{\text{\ kind}}
\newcommand{\TW}{{\mathop{\triangleright}}}
\newcommand{\TWL}{{\mathop{\triangleleft}}}
\newcommand{\F}{\forall}
\newcommand{\TB}{{\mathop{\blacktriangleright}}}
\newcommand{\TBL}{{\mathop{\blacktriangleleft}}}
\newcommand{\E}{\equiv}
\newcommand{\FV}{\text{FV}}
\newcommand{\FTV}{\text{FSV}}

\newcommand{\WStar}{\textsc{W-Star}\xspace}
\newcommand{\WAbs}{\textsc{W-Abs}\xspace}
\newcommand{\WCsp}{\textsc{W-Csp}\xspace}
\newcommand{\WApp}{\textsc{W-App}\xspace}
\newcommand{\WTW}{\textsc{W-$\TW$}\xspace}

\newcommand{\WAStar}{\textsc{WA-Star}\xspace}
\newcommand{\WAAbs}{\textsc{WA-Abs}\xspace}
\newcommand{\WACsp}{\textsc{WA-Csp}\xspace}
\newcommand{\WAApp}{\textsc{WA-App}\xspace}
\newcommand{\WATW}{\textsc{WA-$\TW$}\xspace}

\newcommand{\KVar}{\textsc{K-Var}\xspace}
\newcommand{\KTConst}{\textsc{K-TConst}\xspace}
\newcommand{\KAbs}{\textsc{K-Abs}\xspace}
\newcommand{\KApp}{\textsc{K-App}\xspace}
\newcommand{\KConv}{\textsc{K-Conv}\xspace}
\newcommand{\KTW}{\textsc{K-$\TW$}\xspace}
\newcommand{\KTWL}{\textsc{K-$\TWL$}\xspace}
\newcommand{\KGen}{\textsc{K-Gen}\xspace}
\newcommand{\KCsp}{\textsc{K-Csp}\xspace}

\newcommand{\KAVar}{\textsc{KA-Var}\xspace}
\newcommand{\KATConst}{\textsc{KA-TConst}\xspace}
\newcommand{\KAAbs}{\textsc{KA-Abs}\xspace}
\newcommand{\KAApp}{\textsc{KA-App}\xspace}
\newcommand{\KAConv}{\textsc{KA-Conv}\xspace}
\newcommand{\KATW}{\textsc{KA-$\TW$}\xspace}
\newcommand{\KATWL}{\textsc{KA-$\TWL$}\xspace}
\newcommand{\KAGen}{\textsc{KA-Gen}\xspace}
\newcommand{\KACsp}{\textsc{KA-Csp}\xspace}

\newcommand{\TConst}{\textsc{T-Const}\xspace}
\newcommand{\TVar}{\textsc{T-Var}\xspace}
\newcommand{\TAbs}{\textsc{T-Abs}\xspace}
\newcommand{\TApp}{\textsc{T-App}\xspace}
\newcommand{\TConv}{\textsc{T-Conv}\xspace}
\newcommand{\TTB}{\textsc{T-$\TB$}\xspace}
\newcommand{\TTBL}{\textsc{T-$\TBL$}\xspace}
\newcommand{\TGen}{\textsc{T-Gen}\xspace}
\newcommand{\TIns}{\textsc{T-Ins}\xspace}
\newcommand{\TCsp}{\textsc{T-Csp}\xspace}

\newcommand{\TAConst}{\textsc{TA-Const}\xspace}
\newcommand{\TAVar}{\textsc{TA-Var}\xspace}
\newcommand{\TAAbs}{\textsc{TA-Abs}\xspace}
\newcommand{\TAApp}{\textsc{TA-App}\xspace}
\newcommand{\TAConv}{\textsc{TA-Conv}\xspace}
\newcommand{\TATB}{\textsc{TA-$\TB$}\xspace}
\newcommand{\TATBL}{\textsc{TA-$\TBL$}\xspace}
\newcommand{\TAGen}{\textsc{TA-Gen}\xspace}
\newcommand{\TAIns}{\textsc{TA-Ins}\xspace}
\newcommand{\TACsp}{\textsc{TA-Csp}\xspace}

\newcommand{\QKAbs}{\textsc{QK-Abs}\xspace}
\newcommand{\QKCsp}{\textsc{QK-Csp}\xspace}
\newcommand{\QKRefl}{\textsc{QK-Refl}\xspace}
\newcommand{\QKSym}{\textsc{QK-Sym}\xspace}
\newcommand{\QKTrans}{\textsc{QK-Trans}\xspace}

\newcommand{\QKAAbs}{\textsc{QKA-Abs}\xspace}
\newcommand{\QKACsp}{\textsc{QKA-Csp}\xspace}
\newcommand{\QKARefl}{\textsc{QKA-Refl}\xspace}
\newcommand{\QKASym}{\textsc{QKA-Sym}\xspace}
\newcommand{\QKATrans}{\textsc{QKA-Trans}\xspace}

\newcommand{\QTAbs}{\textsc{QT-Abs}\xspace}
\newcommand{\QTApp}{\textsc{QT-App}\xspace}
\newcommand{\QTTW}{\textsc{QT-$\TW$}\xspace}
\newcommand{\QTGen}{\textsc{QT-Gen}\xspace}
\newcommand{\QTCsp}{\textsc{QT-Csp}\xspace}
\newcommand{\QTRefl}{\textsc{QT-Refl}\xspace}
\newcommand{\QTSym}{\textsc{QT-Sym}\xspace}
\newcommand{\QTTrans}{\textsc{QT-Trans}\xspace}

\newcommand{\QTAAbs}{\textsc{QTA-Abs}\xspace}
\newcommand{\QTAApp}{\textsc{QTA-App}\xspace}
\newcommand{\QTATW}{\textsc{QTA-$\TW$}\xspace}
\newcommand{\QTAGen}{\textsc{QTA-Gen}\xspace}
\newcommand{\QTACsp}{\textsc{QTA-Csp}\xspace}
\newcommand{\QTAConst}{\textsc{QTA-Const}\xspace}
\newcommand{\QTASym}{\textsc{QTA-Sym}\xspace}
\newcommand{\QTATrans}{\textsc{QTA-Trans}\xspace}

\newcommand{\QAbs}{\textsc{Q-Abs}\xspace}
\newcommand{\QApp}{\textsc{Q-App}\xspace}
\newcommand{\QTB}{\textsc{Q-$\TB$}\xspace}
\newcommand{\QTBL}{\textsc{Q-$\TBL$}\xspace}
\newcommand{\QGen}{\textsc{Q-Gen}\xspace}
\newcommand{\QIns}{\textsc{Q-Ins}\xspace}
\newcommand{\QCsp}{\textsc{Q-Csp}\xspace}
\newcommand{\QRefl}{\textsc{Q-Refl}\xspace}
\newcommand{\QSym}{\textsc{Q-Sym}\xspace}
\newcommand{\QTrans}{\textsc{Q-Trans}\xspace}
\newcommand{\QBeta}{\textsc{Q-$\beta$}\xspace}
\newcommand{\QEta}{\textsc{Q-$\eta$}\xspace}
\newcommand{\QTBLTB}{\textsc{Q-$\TBL\TB$}\xspace}
\newcommand{\QLambda}{\textsc{Q-$\Lambda$}\xspace}
\newcommand{\QPercent}{\textsc{Q-\%}\xspace}

\newcommand{\QAAbs}{\textsc{QA-Abs}\xspace}
\newcommand{\QAApp}{\textsc{QA-App}\xspace}
\newcommand{\QAVar}{\textsc{QA-Var}\xspace}
\newcommand{\QAConst}{\textsc{QA-Const}\xspace}
\newcommand{\QATB}{\textsc{QA-$\TB$}\xspace}
\newcommand{\QATBL}{\textsc{QA-$\TBL$}\xspace}
\newcommand{\QAGen}{\textsc{QA-Gen}\xspace}
\newcommand{\QAIns}{\textsc{QA-Ins}\xspace}
\newcommand{\QACsp}{\textsc{QA-Csp}\xspace}
\newcommand{\QARefl}{\textsc{QA-Refl}\xspace}
\newcommand{\QASym}{\textsc{QA-Sym}\xspace}
\newcommand{\QATrans}{\textsc{QA-Trans}\xspace}
\newcommand{\QABeta}{\textsc{QA-$\beta$}\xspace}
\newcommand{\QAEta}{\textsc{QA-$\eta$}\xspace}
\newcommand{\QATBLTB}{\textsc{QA-$\TBL\TB$}\xspace}
\newcommand{\QALambda}{\textsc{QA-$\Lambda$}\xspace}
\newcommand{\QAPercent}{\textsc{QA-\%}\xspace}

\newcommand{\ID}[1]{\infer[]{#1}{\vdots}}
\newcommand{\MD}[1]{\mathcal{D}_#1}

\newcommand{\I}{\textrm{Int}}
\newcommand{\B}{\textrm{Bool}}
\newcommand{\M}{\textrm{Mat}}

\newcommand{\RWH}{\longrightarrow_{\text{wh}}}
\newcommand{\AER}{\longrightarrow_{\text{AE}}}
\newcommand{\AV}{\vdash\!\!\!\raisebox{0.4ex}{\scalebox{0.7}{$\TB$}}}
\newcommand{\AVS}{\AV_\Sigma\xspace}
\newcommand{\EWH}{\E_{\text{wh}}}
\newcommand{\EAE}{\E_{\text{AE}}}
\newcommand{\AERWH}{\text{AERWH}}


% I don't use sigma in this slides.
\renewcommand{\V}{\vdash}

\title{Dependently Typed Multi-stage Calculus}
\author{\underline{Akira Kawata} and Atsushi Igarashi}
\institute{Graduate School of Informatics, Kyoto University}
\date{Dec 2, 2019}

\begin{document}
\maketitle

% Output the outline slide
% \begin{frame}{Contents}
%     \tableofcontents[currentsection]
%   \note{
%   }
% \end{frame}

\section{Introduction to Multistage programming}

\begin{frame}{Contents}

    \setbeamertemplate{section in toc}{
        \inserttocsectionnumber.\ \inserttocsection
    }

    \tableofcontents
    \note{
    }
\end{frame}

\begin{frame}[fragile]{What is Multi-stage Programming}
    A programming paradigm that enables
    \begin{itemize}
        \item Generation of code at runtime
        \item Evaluation of code at runtime
    \end{itemize}
    Applications
    \begin{itemize}
        \item Speeding up programs using runtime information
    \end{itemize}
    \note{
    }
\end{frame}

\begin{frame}[fragile]{Example of Multi-stage Programming in MetaOCaml}
    \begin{center}
        \begin{verbatim}
 # let a = < (1 + 2) >;;
 val a : int code = < (1 + 2) >
 # run a;;
 - : int = 3

 # let b = <(~a) + (~a)>
 val b : int code = < ((1 + 2) + (1 + 2)) >
 # run b
 - : int = 6
        \end{verbatim}
    \end{center}
    \note{
    }
\end{frame}

\begin{frame}[fragile]{Application of Multi-stage Programming: vadd-gen}
    Generate efficient vector addition functions for given length.
    
    \begin{verbatim}
 val vadd-gen : Int -> (Vec -> Vec -> Vec) code
 # let vadd3 = vadd-gen 3
 val vadd3 : (Vec -> Vec -> Vec) code
 (* vadd3 v w =                            *)
 (*   < let (v, w)=(~v, ~w) in             *)
 (*      [v[0]+w[0];v[1]+w[1];v[2]+w[2]] > *)
    \end{verbatim}
    \note{
    }
\end{frame}

\begin{frame}[fragile]{The problem of vadd3 generated by vadd-gen}
    Runtime error occurs for wrong length vectors.
    \begin{verbatim}
 val vadd3 : (Vec -> Vec -> Vec) code
 # (run vadd3) [1;2;3] [4;5;6]
 - : Vec = [5;7;9]
 # (run vadd3) [1;2] [4;5]
 Exception: Failure
    \end{verbatim}
    \note{
    }
\end{frame}

\begin{frame}[fragile]{Dependent Types}
    \renewcommand{\V}{\text{Vec}\ }
    Types dependent on values
    \begin{itemize}
        \item Can prevents illegal use of {\verb|vadd3|}.
    \end{itemize}
 \begin{verbatim}
 - : [1;2]: Vec 2
 - : vadd3 : (Vec 3 -> Vec 3 -> Vec 3) code

 # (run vadd3) [1;2] [4;5]
 [1;2] has type Vec 2
 but was expected of type Vec 3
 \end{verbatim}
    \note{
        By the way, we can write types dependent on values using dependent types.
    }
\end{frame}

\begin{frame}[fragile]{Our Approach: \LMD}
    Introduce dependent types into \LTP [Hanada\&Igarashi'14].
    \begin{itemize}
        \item Features of multi-stage programming
        \item Features of dependent types
    \end{itemize}
    Our contribution
    \begin{itemize}
        \item Define typing, kinding, well-formed kind rules for multi-stage calculus
        \item Define equivalence rules including CSP
    \end{itemize}
    \note{
        Our approach to the problem of multi-stage programming is introducing dependent types into a multi-stage calculus by Hanada and Igarashi.
        We call our multi-stage calculus as \LMD.
    }
\end{frame}

\section{Existing Multi-stage Calculus \LTP}

\begin{frame}{Contents}
    \tableofcontents[currentsection]
    \note{
    }
\end{frame}

\begin{frame}[fragile]{\LTP [Hanada\&Igarashi'14]}
    Multi-stage calculus with Cross Stage Persistence (CSP)
    \begin{block}{Terms}
        \begin{tabbing}
        \hspace{5mm} \( M \) \= \( ::= c \) \hspace{20mm} \= (constants) \\
        \> \( \mid \TB M \) \> (quote) \\
        \> \( \mid \TBL M \) \> (unquote) \\
        \> \( \mid \textbf{run}\ M \) \> (run) \\
        \> \( \mid \% M \) \> (CSP) \\
        \> \( \mid x \mid \lambda x:\tau.M \mid M\ M \) \\
    \end{tabbing}
    \end{block}
\end{frame}

\begin{frame}[fragile]{Types of \LTP}
    \begin{block}{Types}
    \begin{tabbing}
        \hspace{5mm} \( \tau \) \= ::= X \hspace{20mm} \= (constants) \\
        \> \( \mid \TW\ \tau \) \> (code type) \\
        \> \( \mid \tau \to \tau \) \> (function type)
    \end{tabbing}
    \end{block}
   \note{
    }
\end{frame}

\begin{frame}[fragile]{Example of \LTP}
    \begin{table}
        \begin{tabular}{ c | c | c }
            & \LTP & MetaOCaml \\[2mm]
            \hline
            Quote & \( \textbf{let } a = \TB (1 + 2) \) & \verb| let a = .<(1+2)>.| \\[2mm]
            Unquote & \( (\TB (\TBL a + \TBL a)) \) & \verb| .<(.~a)+(.~a)>.| \\[2mm]
            Run & \( \textbf{run } (\TB (1 + 2)) \) & \verb| run .<(1+2)>.| \\[2mm]
            Code type & \( \TW\ \text{Int} \) & \verb| int code |
        \end{tabular}
    \end{table}
\end{frame}

\begin{frame}[fragile]{What is a stage?}
    Level of nested quote
    \begin{center}
        \( \underbrace{(\textbf{run}\ (\TB\ \overbrace{(2 + 3)}^{\let\scriptstyle\textstyle\substack{\text{stage }1}})) + 1}_{\let\scriptstyle\textstyle\substack{\text{stage }0}} \)
    \end{center}

    Typing judgements of \LTP include a stage.
    \begin{center}
        \( \G \vdash 2 + 3 : \I {\mathbf{@ 1}} \)
    \end{center}
    \note{
    }
\end{frame}

\begin{frame}[fragile]{Typing Rules of \LTP}
    Cannot apply a term to a function in a different stage.
    \begin{center}
        {\small \infrule[\TApp]{
            \G\V M:\sigma \to \tau @ n \andalso
            \G\V N:\sigma @ n
        }{
            \G\V M\ N : \tau @ n
        }}
    \end{center}
    Increase and decrease stage using \( \TB \) and \( \TBL \).
    \begin{center}
        {\small \infrule[{\TTB}]{
            \G\V M:\tau @ n+1
        }{
            \G\V\TB M:\TW \tau @ n
        } \hfil
        \infrule[{\TTBL}]{
            \G\V M:\TW \tau @ n
        }{
            \G\V\TBL M:\tau @ n + 1
        }}
    \end{center}
    In general, we cannot use terms of one stage at other stages.
    \note{
        Let's have a look on some typing rules.
    }
\end{frame}

\begin{frame}[fragile]{Cross Stage Persistence (CSP)}
    Enable using terms at lower stages at higher stages. \\[3mm]

    Want to use \( \text{mul5} \) in a code value.
    \begin{center}
        \( \textbf{let}\ \underbrace{\text{mul5}}_{\text{stage }0}\ x = 5 * x\ \textbf{in} \TB\ \underbrace{(\% \text{mul5}\ 4)}_{\text{stage }1} \)
    \end{center}
    CSP symbol \( \% \) {\bf{lift up}} an normal value into code value.
\end{frame}

\section{Extend \LTP with Dependent Types}

\begin{frame}{Contents}
    \tableofcontents[currentsection]
  \note{
  }
\end{frame}

\begin{frame}[fragile]{Types and Kinds of \LMD}
    Append function dependent types to types of \LTP.
    \begin{block}{Types}
        \begin{tabbing}
            \hspace{5mm} \( \tau \) \= ::= X \hspace{20mm} \= (constants) \\
            \> \( \mid \TW\ \tau \) \> (code type) \\
            \> \( \bm{\mid \Pi x:\tau.\tau} \) \> (function dependent type) \\
            \> \( \bm{\mid \tau\ M} \) \> (application, e.g. \(\text{Vec}\ 3\))
        \end{tabbing}
    \end{block}
    \begin{block}{Kinds}
        \( K ::= * \mid \Pi x:\tau.K \)
    \end{block}
    % Note: We need CSP to use normal variables in code types.
    % \begin{itemize}
    %     \item OK  \( \Pi n: \TW (\text{Vec}\ \%n) \)
    %     \item NG  \( \Pi n: \TW (\text{Vec}\ n) \)
    % \end{itemize}
    \note{
    }
\end{frame}

\begin{frame}[fragile]{Kinding rules of \LMD}
    Types have also stages because types may contains terms.
    \begin{center}
    \infrule[{\KApp}]{
        \G\V \sigma:: (\Pi x:\tau.K) @ n \andalso \G\V M:\tau @ n
    }{
        \G\V \sigma\ M::K[x\mapsto M] @ n
    }
    \infrule[{\KAbs}]{
        \G\V \tau :: * @ n \andalso \G,x:\tau@ n \V \sigma::J@A
    }{
        \G\V(\Pi x:\tau.\sigma) :: (\Pi x:\tau.J) @ n
    }
    \end{center}
    So, we need CSP in types also.
    \begin{center}
    \( \underbrace{\Pi n:\I. \TW (\overbrace{\text{Vec}\ \%n}^{\text{stage }1})}_{\text{stage }0} \)
    \end{center}
\end{frame}

\begin{frame}[fragile]{Constants and type-level constants of \LMD}
    Constants and type-level constants can be used at any stage.
    \begin{itemize}
        \item Integers: \( 0,1,2,3,\dots \)
        \item Int, Vec, \dots
    \end{itemize}

    \begin{tabbing}
        \hspace{5mm} \= \text{val doub3} : \( \text{Vec }3 \to \text{Vec }3 \) \\[2mm]
        \> \textbf{let } \text{double} = \( \TB (\% \text{doub3}) \) \\
        \> \text{val double} : \( \TW ( \text{Vec }3 \to \text{Vec }3) \)
    \end{tabbing}
\end{frame}

\begin{frame}[fragile]{vadd-gen in \LMD}
    Type tells us the length of vector.
    \newcommand{\Vn}{\text{Vec}\ \%n}
    \newcommand{\Vt}{\text{Vec}\ \%3}
    \begin{tabbing}
        \hspace{5mm} \= \( \text{vadd-gen} : \Pi n:\I. \TW (\Vn \to \Vn \to \Vn) \) \\[2mm]
        \> \( \textbf{let}\ \text{vadd3} = \text{vadd-gen}\ 3;; \) \\
        \> \( \text{val}\ \text{vadd3} : \TW (\Vt \to \Vt \to \Vt) \) \\[2mm]
    \end{tabbing}
    Now, vadd3 takes only vectors of length 3.
    \note{
        vadd-gen is changed as follows in \LMD.
        Thanks to dependent types ...
    }
\end{frame}

\begin{frame}[fragile]{Problem of type of vadd3}
    Hard to compose with other functions.
    \newcommand{\Vn}{\text{Vec}\ \%n}
    \newcommand{\Vpt}{\text{Vec}\ \%3}
    \newcommand{\Vt}{\text{Vec}\ 3}
    \begin{tabbing}
        \hspace{5mm} \= \( \text{val}\ \text{vadd3} : \TW (\Vpt \to \Vpt \to \Vpt) \) \\
        \> \( \text{val}\ \text{double} : \TW (\Vt \to \Vt) \) \\[2mm]
        \> \( \textbf{let}\ \text{add-double}\ v\ w = \) \\
        \> \hspace{5mm} \( \TB (\underbrace{\TBL\text{double}}_{\Vt \to \Vt}\ \underbrace{(\TBL \text{vadd3}\ \TBL v\ \TBL w)}_{\Vpt});; \) \\
    \end{tabbing}
    We cannot compose because of type mismatch.
\end{frame}

\begin{frame}[fragile]{Solution}
    \newcommand{\Vt}{\text{Vec}\ 3}
    \newcommand{\Vpt}{\text{Vec}\ \%3}
    Introduce \QPercent.
    \begin{center}
        \infrule[{\QPercent}]{
            M \text{ is a closed term.}
        }{
            \G\vdash \% M \E M
        }
    \end{center}
    Erase the symbol of \( \% \) under the condition.
    \begin{itemize}
        \item \( \text{Vec}\ \%3 \E \text{Vec}\ 3 \)
        \item \( \text{vadd3} : \TW (\Vt \to \Vt \to \Vt) \) \\
            \hspace{15mm} \( (\E  \TW (\Vpt \to \Vpt \to \Vpt)) \)
    \end{itemize}
\end{frame}

\begin{frame}[fragile]{Properties}
    We proved
    \begin{itemize}
        \item Subject Reduction
        \item Strong Normalization
        \item Confluence
        \item Progress
    \end{itemize}
    \note{
    }
\end{frame}

\begin{frame}[fragile]{Related Work}
    \begin{itemize}
        \item \LTP [Hanada\&Igarashi'14]
            \begin{itemize}
                \item Multi-stage programming including explicit CSP
            \end{itemize}
        \item Concoqtion [Forgarty et al.]
            \begin{itemize}
                \item Introduce indexed type into MetaOCaml.
                \item Types are indexed by Coq terms.
            \end{itemize}
    \end{itemize}
    \note{
    }
\end{frame}

\begin{frame}[fragile]{Conclusion}
    We introduced dependent types into multi-stage calculus.
    \begin{itemize}
        \item Define typing, kinding, well-formed kinding rules for multi-stage calculus.
        \item Define equality rules.
            \begin{itemize}
                \item Especially, dealing with CSP.
            \end{itemize}
        \item Prove type soundness.
        \item Future work: algorithmic typing
    \end{itemize}
    \note{
    }
\end{frame}

% \begin{frame}[fragile]{Styles of Multi-stage Programming}
%     \begin{itemize}
%         \item Strings
%             \begin{itemize}
%                     \item Ruby, Perl, etc.
%             \end{itemize}
%         \item ASTs
%             \begin{itemize}
%                     \item Lisp
%             \end{itemize}
%         \item ASTs and code types
%             \begin{itemize}
%                     \item MetaOCaml
%             \end{itemize}
%     \end{itemize}
%     \note{
%         There are three kinds of multi-stage Programming.
%     }
% \end{frame}

% \begin{frame}[fragile]{Generated functions by vadd-gen in \LMD}
%     Type system prevents illegal use of generated function.
%     \renewcommand{\V}{\text{Vec}}
%     \begin{tabbing}
%         \( \text{vadd3}\ [1;2;3]\ [4;5;6];; \) \\
%         \( - : \V = [5;7;9] \) \\[2mm]
%         \( \text{vadd3}\ [1;2;3;4]\ [4;5;6;7];; \) \\
%         \( \longrightarrow \textbf{TYPE ERROR!} \text{ (not RUNTIME ERROR)} \)
%     \end{tabbing}
%     \note{
%         Type of generated functions by vadd-gen in \LMD have the length information in their types.
%     }
% \end{frame}

% \begin{frame}[fragile]{Terms of \LMD}
%     \begin{block}{Terms}
%         \( M ::= c \mid x \mid \lambda x:\tau.M\ \mid M\ M \mid \TB_\alpha M \mid \textbf{run}\ M \mid \TBL_\alpha M \)
%     \end{block}
%     \begin{itemize}
%         \item \( c \): constants
%             \begin{itemize}
%                     \item true, false, etc.
%             \end{itemize}
%         \item \( \TB_\alpha M \): code value
%         \item \( \textbf{run}\ M \): run a code value \( M \)
%             \begin{itemize}
%                 \item \( \textbf{run }(\text{vadd-gen }3)\)
%             \end{itemize}
%     \end{itemize}
%     \note{
%         Then, let me explain details of \LMD. \\
%         \( \TBL_\alpha M \) is not needed?
%     }
% \end{frame}

% \begin{frame}[fragile]{Types of \LMD}
%     \begin{block}{Types}
%     \( \tau,\sigma ::= X \mid \Pi x:\tau.\sigma \mid \tau\ M \mid \TW_{\alpha} \tau \mid \F\alpha.\tau \)
%     \end{block}
%     \begin{itemize}
%         \item \( X \): type-level constants 
%             \begin{itemize}
%                 \item \( \text{Vec} \), \( \text{Int} \).
%             \end{itemize}
%         \item \( \Pi x:\tau.\sigma \): dependent type.
%         \item \( \tau\ M \): Application of term to dependent type.
%             \begin{itemize}
%                 \item \( \text{Vec } 3 \)
%             \end{itemize}
%         \item \( \TW_{\alpha} \tau \): a type of code of type \( \tau \).
%             \begin{itemize}
%                 \item \( \text{vadd-gen 3} : \TW_\alpha (\text{Vec}\ 3 \to \text{Vec}\ 3 \to \text{Vec}\ 3) \)
%             \end{itemize}
%     \end{itemize}
%     \note{
%     }
% \end{frame}

% \begin{frame}[fragile]{Kinds of \LMD}
%     \begin{block}{Kinds}
%         \( K ::= * \mid \Pi x:\tau.K \)
%     \end{block}
%     \begin{itemize}
%         \item \( * \): kind of proper type
%             \begin{itemize}
%                 \item \( \text{Int} :: * \)
%             \end{itemize}
%         \item \( \Pi x:\tau.K \): kind dependent on term.
%             \begin{itemize}
%                 \item \( \text{Vec} :: (\Pi x:\text{Int}.*) \)
%             \end{itemize}
%     \end{itemize}
%     \note{
%     }
% \end{frame}

% \begin{frame}[fragile]{Judgements of \LMD}
%     \begin{itemize}
%         \item \( \G \V K @ A \)
%             \begin{itemize}
%                 \item \( K \) is a proper kind under signature \( \Sigma \) at stage \( A \).
%             \end{itemize}
%         \item \( \G \V \tau :: K @ A \)
%             \begin{itemize}
%                 \item Type \( \tau \) has kind \( K \) under signature \( \Sigma \) at stage \( A \).
%             \end{itemize}
%         \item \( \G \V M : \tau @ A \)
%         \item \( \G \V K \E K' @ A \)
%             \begin{itemize}
%                 \item Kind \( K \) is equivalent to \( K' \) under signature \( \Sigma \) at stage \( A \).
%             \end{itemize}
%         \item \( \G \V \tau \E \tau' @ A \)
%         \item \( \G \V M \E M' @ A \)
%     \end{itemize}
%     \note{
%     }
% \end{frame}

% \begin{frame}[fragile]{What is stage\ \( A \)\ ?}
%     % I copied this underwrite macro from
%     % https://tex.stackexchange.com/questions/141932/how-to-write-under-an-underline.
%     % \underwrite[<thickness>]{<numerator>}{<denominator>}
%     \newcommand{\underwrite}[3][]{
%         \genfrac{}{}{#1}{}{\textstyle #2}{\textstyle #3}
%     }
%
%     Level of nested brankets
%     \begin{block}{Stages}
%         \( A := \varepsilon (\text{empty}) \mid \alpha \mid A\alpha \)
%     \end{block}
%     \begin{center}
%         \( \underbrace{(\textbf{run}\ (\TB_\alpha\ \overbrace{(2 + 3)}^{\let\scriptstyle\textstyle\substack{@\alpha}})) + 1}_{\let\scriptstyle\textstyle\substack{@\varepsilon}} \)
%     \end{center}
%     \note{
%         Stage is a sequence of stage variables which appended to each brack triangles.
%         \( \varepsilon \) means the empty sequence
%
%         The wholes expression is a term on stage \( \varepsilon \),
%         But subexpression in the brack triangle is a term on stage \( \alpha \).
%     }
% \end{frame}

% \begin{frame}[fragile]{What is signature \( \Sigma \)\ ?}
%     Signature \( \Sigma \) is a sequence of type-level constants and constant values.
%     \begin{itemize}
%         \item For example,
%             \begin{itemize}
%                 \item \( \Sigma = \text{Int}::*, \text{Vec}::\Pi x:\text{Int}.*, 0:\text{Int} \)
%             \end{itemize}
%         \item Elements of \( \Sigma \) can be used at any stages.
%     \end{itemize}
%     \note{
%         自然数が無限個あるというツッコミを誘発する気がする。
%         後者関数が存在することにするか?
%     }
% \end{frame}

% \begin{frame}[fragile]{Typing Rules of \LMD}
%     \begin{center}
%         \infrule[{\TTB}]{
%             \G\V M:\tau@{A\alpha}
%         }{
%             \G\V\TB_{\alpha}M:\TW_{\alpha}\tau@A
%         } \\[2mm]
%         \infrule[{\textsc{T-App}}]{
%             \G\V M:(\Pi (x:\sigma).\tau)@A \andalso
%             \G\V N:\sigma@A
%         }{
%             \G\V M\ N : \tau[x\mapsto N]@A
%         } \\[2mm]
%         \infrule[{\TConst}]{
%             c:\tau \in \Sigma \andalso
%         }{
%             \G \V c:\tau@A
%         }
%     \end{center}
%     \begin{itemize}
%         \item Terms are stage sensitive so cannot ignore stages.
%         \item Constants in \( \Sigma \) can be used at any stages.
%     \end{itemize}
%     \note{
%         Let's have a look on some typing rules.
%     }
% \end{frame}

% \begin{frame}[fragile]{Kinding Rules of \LMD}
%     \begin{center}
%         \infrule[{\KTW}]{
%             \G\V \tau::*@A\alpha
%         }{
%             \G\V\TW_\alpha \tau::*@A
%         }\\[2mm]
%         \infrule[{\KApp}]{
%             \G\V \sigma:: (\Pi x:\tau.K)@A \andalso \G\V M:\tau@A
%         }{
%             \G\V \sigma\ M::K[x\mapsto M]@A
%         }
%         \infrule[{\KTConst}]{
%             X::K \in \Sigma
%         }{
%             \G \V X::K@A
%         }
%     \end{center}
%     \begin{itemize}
%         \item Types are also stage sensitive.
%         \item Type-level constants in \( \Sigma \) can be used at any stages.
%     \end{itemize}
%     \note{
%         Kinding rules have something in common with typing rules.
%     }
% \end{frame}

% \begin{frame}[fragile]{Example of Kinding Derivation}
%     \renewcommand{\Vec}{\text{Vec}}
%     \( \V \TB_\alpha [1,2,3] : \underbrace{\TW_\alpha (\Vec\ 3)}_{\TW_\alpha (\Vec\ 3) :: * @ \varepsilon} @ \varepsilon \) \\[2mm]
%     \pause
%     \( \Sigma = \Vec::\Pi x:\I.*, 0:\I, 1:\I, \cdots \)
%     \begin{center}
%         \footnotesize
%         \begin{minipage}{0.7\hsize}
%             \infer[\KTW]
%             {\V \TW_\alpha (\Vec\ 3) :: * @ \varepsilon}
%             {
%                 \infer[\KApp]
%                 {\V \Vec\ 3 :: * @ \alpha}
%                 {
%                     \infer[\KTConst]
%                     {\V \Vec :: \Pi x:\I.* @ \alpha}
%                     {\Vec :: \Pi x:\I.* \in \Sigma}
%                     \andalso
%                     \infer[\TConst]
%                     {\V 3 : \I @ \alpha}
%                     {3 : \I \in \Sigma}
%                 }
%             }
%         \end{minipage}
%     \end{center}
%     We can use \Vec\ and integers at any stages.
%     \note{
%         The constructed type Vec 3 is stage sensitive so we cannot use at arbitrary stages
%         but the elements, Vec and 3 are in the signature \( \Sigma \) and can be used at any stages.
%     }
% \end{frame}

% \begin{frame}[fragile]{Type and Term Equality Rules of \LMD}
%     \begin{center}
%         \footnotesize{
%             \infrule[{\QTApp}]{
%                 \G\V \tau \E \sigma :: (\Pi x:\rho.K)@A \andalso
%                 \G\V M \E N : \rho @A
%             }{
%                 \G\V \tau\ M \E \sigma\ N :: K[x \mapsto M]@A
%             } \\[2mm]
%             \infrule[{\QBeta}]{
%                 \G,x:\sigma@A\V M:\tau@A \andalso
%                 \G\V N:\sigma@A
%             }{
%                 \G\V(\lambda x:\sigma.M)\ N\E M[x\mapsto N] : \tau[x \mapsto N]@A
%             } \\[2mm]
%             \infrule[{\QPercent}]{
%                 \G\V M:\tau@{A\alpha} \andalso
%                 \G\V M:\tau@A
%             }{
%                 \G\V\%_\alpha M \E M : \tau@{A\alpha}
%             } \\[2mm]
%             \infrule[{\textsc{T-Conv}}]{
%                 \G\V M:\tau@A \andalso
%                 \G\V \tau\equiv \sigma :: K@A
%             }{
%                 \G\V M:\sigma@A
%             }
%         }
%     \end{center}
%     \note{
%     }
% \end{frame}

% \begin{frame}[fragile]{Examples of Type Equality}
%     \renewcommand{\V}{\text{Vec}}
%     \begin{itemize}
%         \item \( (\lambda x:\text{Int}.x)\ 3\E 3 \)
%             \begin{itemize}
%                 \item From \QBeta.
%             \end{itemize}
%         \item \( \V\ ((\lambda x:\text{Int}.x)\ 3) \E \V\ 3 \)
%             \begin{itemize}
%                 \item From \QTApp and \QBeta.
%             \end{itemize}
%         \item \( \V\ (\%_\alpha 3) \E \V\ 3 \)
%             \begin{itemize}
%                 \item Because \( 3 \) has type \text{Int} at any stage, we can use \QPercent.
%             \end{itemize}
%     \end{itemize}
%     \note{
%         Shold I show complete derivation tree?
%     }
% \end{frame}

\end{document}

\documentclass[dvipdfmx,aspectratio=169, 20pt]{beamer}
\usepackage[absolute,overlay]{textpos}
\usepackage{amsmath}
\usepackage{amsthm}
\usepackage{ascmac}
\usepackage[backend=biber, style=numeric]{biblatex}
\usepackage{color}
\usepackage{mathtools}
\usepackage{url}
\usepackage{bcprules, proof}
\usepackage{xparse}
\usepackage{xspace}
\usepackage{fancybox}
\usepackage{float}
\usepackage{bcprules}
\usepackage{ebproof}
\usepackage{lscape}
\usepackage{pgfpages}

% beamer cheat sheet
% http://www.cpt.univ-mrs.fr/~masson/latex/Beamer-appearance-cheat-sheet.pdf

% Make note pages suer simple.
% \setbeamertemplate{note page}[plain]

% This option generate pdf files showing notes on the right hand side of each
% pages. But this optionn has some bugs. I recommend you tou do not use this
% option.
% \setbeameroption{show notes on second screen=right}

% Note options
% \setbeameroption{show notes}
% \setbeameroption{hide notes}
% \setbeameroption{show only notes}

% Show grid for debug
% \setbeamertemplate{background}[grid][step=1cm]

% Use default design theme. I quit to use this theme because it has no boxed
% environments.
% \usetheme{default}
\usetheme{Boadilla}

% Add page numbers to footnotes. This line does not work for Boadilla theme.
% \setbeamertemplate{footline}[frame number]

% So, I use some hack in
% https://tex.stackexchange.com/questions/66995/modify-footer-of-slides.
% \makeatother and \makeatletter enable us to change internal package things.
\makeatother
\setbeamertemplate{footline}
{
    \leavevmode
    \hbox{
        \setbeamercolor{coloredboxstuff}{fg=black,bg=white}
        \hskip0.85\paperwidth
        \begin{beamercolorbox}[wd=0.10\paperwidth,ht=2.25ex,dp=1ex,right]{coloredboxstuff}
            {\textbf{\insertframenumber{} / \inserttotalframenumber}}
        \end{beamercolorbox}}
    \vskip0pt
}
\makeatletter

% Remove all navigation symbols
\setbeamertemplate{navigation symbols}{}
% using non standard fonts for beamer
\usefonttheme{professionalfonts}
% Set the font size of titles
% See other font category at
% https://tex.stackexchange.com/questions/183052/what-are-all-the-possible-first-arguments-to-setbeamerfont
\setbeamerfont{title}{size=\normalsize}
\setbeamerfont{frametitle}{size=\normalsize}
\setbeamerfont{normal text}{size=\small}
\setbeamerfont{itemize/enumerate body}{size=\footnotesize}
\setbeamerfont{itemize/enumerate subbody}{size=\footnotesize}
\setbeamerfont{frametitle continuation}{size=\small}
\setbeamerfont{framesubtitle}{size=\small}
\setbeamerfont{abstract}{size=\small}
\setbeamerfont{projected text}{size=\small}
\setbeamerfont{block title}{size=\small}
\setbeamerfont{footline}{size=\tiny}

% You need this line in order to enforce normal text font setting.
% https://tex.stackexchange.com/questions/320223/how-to-enforce-a-font-series-in-beamer-for-normal-default-text
\AtBeginDocument{\usebeamerfont{normal text}}

\setbeamertemplate{bibliography item}{\insertbiblabel}
% \setbeamertemplate{blocks}[default]
\setbeamertemplate{blocks}[rounded]
\setbeamertemplate{theorems}[normal font]

% \bibliography{main}{}
\addbibresource{main.bib}
\DeclareSortingScheme{mysorting}{\sort{\citeorder}}
\ExecuteBibliographyOptions{sorting=mysorting}

\setlength\intextsep{0pt}
\setlength\textfloatsep{0pt}

\newcommand{\rulefbox}[1]{\fbox{\ensuremath{#1}} \hspace{1mm}}
\newcommand{\figheader}[2]{
  \begin{flushleft}
    #2 {\bf \normalsize #1}
\end{flushleft}}

\newcommand{\toc}[0]{
  \begin{frame}{Contents}
  \tableofcontents[currentsection]
  \end{frame}
}

\defbeamertemplate{description item}{align left}{\insertdescriptionitem\hfill}

% Load symbol macros
\newcommand{\LTP}{$\lambda^{\triangleright\%}$\xspace}
\newcommand{\LMD}{$\lambda^{\textrm{MD}}$\xspace}
\newcommand{\LLF}{$\lambda\textrm{LF}$\xspace}

\newcommand{\G}{\Gamma}
\newcommand{\D}{\Delta}
\newcommand{\V}{\vdash_\Sigma}
\newcommand{\VT}{\vdash\hspace{-.50em}\raisebox{0.28em}{\tiny{$\TB$}}}
\newcommand{\iskind}{\text{\ kind}}
\newcommand{\TW}{{\mathop{\triangleright}}}
\newcommand{\TWL}{{\mathop{\triangleleft}}}
\newcommand{\F}{\forall}
\newcommand{\TB}{{\mathop{\blacktriangleright}}}
\newcommand{\TBL}{{\mathop{\blacktriangleleft}}}
\newcommand{\E}{\equiv}
\newcommand{\FV}{\text{FV}}
\newcommand{\FTV}{\text{FSV}}

\newcommand{\WStar}{\textsc{W-Star}\xspace}
\newcommand{\WAbs}{\textsc{W-Abs}\xspace}
\newcommand{\WCsp}{\textsc{W-Csp}\xspace}
\newcommand{\WApp}{\textsc{W-App}\xspace}
\newcommand{\WTW}{\textsc{W-$\TW$}\xspace}

\newcommand{\WAStar}{\textsc{WA-Star}\xspace}
\newcommand{\WAAbs}{\textsc{WA-Abs}\xspace}
\newcommand{\WACsp}{\textsc{WA-Csp}\xspace}
\newcommand{\WAApp}{\textsc{WA-App}\xspace}
\newcommand{\WATW}{\textsc{WA-$\TW$}\xspace}

\newcommand{\KVar}{\textsc{K-Var}\xspace}
\newcommand{\KTConst}{\textsc{K-TConst}\xspace}
\newcommand{\KAbs}{\textsc{K-Abs}\xspace}
\newcommand{\KApp}{\textsc{K-App}\xspace}
\newcommand{\KConv}{\textsc{K-Conv}\xspace}
\newcommand{\KTW}{\textsc{K-$\TW$}\xspace}
\newcommand{\KTWL}{\textsc{K-$\TWL$}\xspace}
\newcommand{\KGen}{\textsc{K-Gen}\xspace}
\newcommand{\KCsp}{\textsc{K-Csp}\xspace}

\newcommand{\KAVar}{\textsc{KA-Var}\xspace}
\newcommand{\KATConst}{\textsc{KA-TConst}\xspace}
\newcommand{\KAAbs}{\textsc{KA-Abs}\xspace}
\newcommand{\KAApp}{\textsc{KA-App}\xspace}
\newcommand{\KAConv}{\textsc{KA-Conv}\xspace}
\newcommand{\KATW}{\textsc{KA-$\TW$}\xspace}
\newcommand{\KATWL}{\textsc{KA-$\TWL$}\xspace}
\newcommand{\KAGen}{\textsc{KA-Gen}\xspace}
\newcommand{\KACsp}{\textsc{KA-Csp}\xspace}

\newcommand{\TConst}{\textsc{T-Const}\xspace}
\newcommand{\TVar}{\textsc{T-Var}\xspace}
\newcommand{\TAbs}{\textsc{T-Abs}\xspace}
\newcommand{\TApp}{\textsc{T-App}\xspace}
\newcommand{\TConv}{\textsc{T-Conv}\xspace}
\newcommand{\TTB}{\textsc{T-$\TB$}\xspace}
\newcommand{\TTBL}{\textsc{T-$\TBL$}\xspace}
\newcommand{\TGen}{\textsc{T-Gen}\xspace}
\newcommand{\TIns}{\textsc{T-Ins}\xspace}
\newcommand{\TCsp}{\textsc{T-Csp}\xspace}

\newcommand{\TAConst}{\textsc{TA-Const}\xspace}
\newcommand{\TAVar}{\textsc{TA-Var}\xspace}
\newcommand{\TAAbs}{\textsc{TA-Abs}\xspace}
\newcommand{\TAApp}{\textsc{TA-App}\xspace}
\newcommand{\TAConv}{\textsc{TA-Conv}\xspace}
\newcommand{\TATB}{\textsc{TA-$\TB$}\xspace}
\newcommand{\TATBL}{\textsc{TA-$\TBL$}\xspace}
\newcommand{\TAGen}{\textsc{TA-Gen}\xspace}
\newcommand{\TAIns}{\textsc{TA-Ins}\xspace}
\newcommand{\TACsp}{\textsc{TA-Csp}\xspace}

\newcommand{\QKAbs}{\textsc{QK-Abs}\xspace}
\newcommand{\QKCsp}{\textsc{QK-Csp}\xspace}
\newcommand{\QKRefl}{\textsc{QK-Refl}\xspace}
\newcommand{\QKSym}{\textsc{QK-Sym}\xspace}
\newcommand{\QKTrans}{\textsc{QK-Trans}\xspace}

\newcommand{\QKAAbs}{\textsc{QKA-Abs}\xspace}
\newcommand{\QKACsp}{\textsc{QKA-Csp}\xspace}
\newcommand{\QKARefl}{\textsc{QKA-Refl}\xspace}
\newcommand{\QKASym}{\textsc{QKA-Sym}\xspace}
\newcommand{\QKATrans}{\textsc{QKA-Trans}\xspace}

\newcommand{\QTAbs}{\textsc{QT-Abs}\xspace}
\newcommand{\QTApp}{\textsc{QT-App}\xspace}
\newcommand{\QTTW}{\textsc{QT-$\TW$}\xspace}
\newcommand{\QTGen}{\textsc{QT-Gen}\xspace}
\newcommand{\QTCsp}{\textsc{QT-Csp}\xspace}
\newcommand{\QTRefl}{\textsc{QT-Refl}\xspace}
\newcommand{\QTSym}{\textsc{QT-Sym}\xspace}
\newcommand{\QTTrans}{\textsc{QT-Trans}\xspace}

\newcommand{\QTAAbs}{\textsc{QTA-Abs}\xspace}
\newcommand{\QTAApp}{\textsc{QTA-App}\xspace}
\newcommand{\QTATW}{\textsc{QTA-$\TW$}\xspace}
\newcommand{\QTAGen}{\textsc{QTA-Gen}\xspace}
\newcommand{\QTACsp}{\textsc{QTA-Csp}\xspace}
\newcommand{\QTAConst}{\textsc{QTA-Const}\xspace}
\newcommand{\QTASym}{\textsc{QTA-Sym}\xspace}
\newcommand{\QTATrans}{\textsc{QTA-Trans}\xspace}

\newcommand{\QAbs}{\textsc{Q-Abs}\xspace}
\newcommand{\QApp}{\textsc{Q-App}\xspace}
\newcommand{\QTB}{\textsc{Q-$\TB$}\xspace}
\newcommand{\QTBL}{\textsc{Q-$\TBL$}\xspace}
\newcommand{\QGen}{\textsc{Q-Gen}\xspace}
\newcommand{\QIns}{\textsc{Q-Ins}\xspace}
\newcommand{\QCsp}{\textsc{Q-Csp}\xspace}
\newcommand{\QRefl}{\textsc{Q-Refl}\xspace}
\newcommand{\QSym}{\textsc{Q-Sym}\xspace}
\newcommand{\QTrans}{\textsc{Q-Trans}\xspace}
\newcommand{\QBeta}{\textsc{Q-$\beta$}\xspace}
\newcommand{\QEta}{\textsc{Q-$\eta$}\xspace}
\newcommand{\QTBLTB}{\textsc{Q-$\TBL\TB$}\xspace}
\newcommand{\QLambda}{\textsc{Q-$\Lambda$}\xspace}
\newcommand{\QPercent}{\textsc{Q-\%}\xspace}

\newcommand{\QAAbs}{\textsc{QA-Abs}\xspace}
\newcommand{\QAApp}{\textsc{QA-App}\xspace}
\newcommand{\QAVar}{\textsc{QA-Var}\xspace}
\newcommand{\QAConst}{\textsc{QA-Const}\xspace}
\newcommand{\QATB}{\textsc{QA-$\TB$}\xspace}
\newcommand{\QATBL}{\textsc{QA-$\TBL$}\xspace}
\newcommand{\QAGen}{\textsc{QA-Gen}\xspace}
\newcommand{\QAIns}{\textsc{QA-Ins}\xspace}
\newcommand{\QACsp}{\textsc{QA-Csp}\xspace}
\newcommand{\QARefl}{\textsc{QA-Refl}\xspace}
\newcommand{\QASym}{\textsc{QA-Sym}\xspace}
\newcommand{\QATrans}{\textsc{QA-Trans}\xspace}
\newcommand{\QABeta}{\textsc{QA-$\beta$}\xspace}
\newcommand{\QAEta}{\textsc{QA-$\eta$}\xspace}
\newcommand{\QATBLTB}{\textsc{QA-$\TBL\TB$}\xspace}
\newcommand{\QALambda}{\textsc{QA-$\Lambda$}\xspace}
\newcommand{\QAPercent}{\textsc{QA-\%}\xspace}

\newcommand{\ID}[1]{\infer[]{#1}{\vdots}}
\newcommand{\MD}[1]{\mathcal{D}_#1}

\newcommand{\I}{\textrm{Int}}
\newcommand{\B}{\textrm{Bool}}
\newcommand{\M}{\textrm{Mat}}

\newcommand{\RWH}{\longrightarrow_{\text{wh}}}
\newcommand{\AER}{\longrightarrow_{\text{AE}}}
\newcommand{\AV}{\vdash\!\!\!\raisebox{0.4ex}{\scalebox{0.7}{$\TB$}}}
\newcommand{\AVS}{\AV_\Sigma\xspace}
\newcommand{\EWH}{\E_{\text{wh}}}
\newcommand{\EAE}{\E_{\text{AE}}}
\newcommand{\AERWH}{\text{AERWH}}


\title{Dependently Typed Multi-stage Calculus}
\author{\underline{Akira Kawata} and Atsushi Igarashi}
\institute{Gradudate School of Informatics, Kyoto University}
\date{Dec 2, 2019}

\begin{document}
\maketitle

% Output the outline slide
% \begin{frame}{Contents}
%   \tableofcontents
%   \note{
%   }
% \end{frame}

\section{Introduction to Multistage programming}

\begin{frame}[fragile]{What is Multi-stage Programming}
    A programming paradigm enables
    \begin{itemize}
        \item Generation of code at runtime
        \item Evaluation of code at runtime
    \end{itemize}
    Applications
    \begin{itemize}
        \item Speeding up programs using runtime information
    \end{itemize}
    \note{
    }
\end{frame}

\begin{frame}[fragile]{Styles of Multi-stage Programming}
    \begin{itemize}
        \item Strings
            \begin{itemize}
                    \item Ruby, Perl, etc.
            \end{itemize}
        \item ASTs
            \begin{itemize}
                    \item Lisp
            \end{itemize}
        \item ASTs and code types
            \begin{itemize}
                    \item MetaOCaml
            \end{itemize}
    \end{itemize}
    \note{
        There are three kinds of multi-stage Programming.
    }
\end{frame}

\begin{frame}[fragile]{Example of Multi-stage Programming in MetaOCaml}
    \begin{center}
        \begin{verbatim}
     # let a = .< 1 + 2 >.;;
     val a : int code = .< 1 + 2 >.
     # Runcode.run a;;
     - : int = 3

     # let b = .<.~a + .~a>.;;
     val b : int code = .<(1 + 2) + (1 + 2)>. 
     # Runcode.run b;;
     - : int = 6
        \end{verbatim}
    \end{center}
    \note{
    }
\end{frame}

\begin{frame}[fragile]{Application of Multi-stage Programming: vadd-gen}
    Generate efficient vector addition functions for given length vectors.
    \renewcommand{\V}{\text{Vec}}
    \begin{tabbing}
        \( \text{vadd-gen} : \I \to \TW_\alpha (\V \to \V \to \V) \) \\
        (* \( \TW_\alpha \) is corresponding to code in MetaOCaml.*) \\[2mm]
        \( \textbf{let}\ \text{vadd3} = \textbf{run}\ (\text{vadd-gen}\ 3);; \) \\
        \( \text{val}\ \text{vadd3} : \V \to \V \to \V \) \\[2mm]
        \( \text{vadd3}\ [1;2;3]\ [4;5;6];; \) \\
        \( - : \V = [5;7;9] \)
    \end{tabbing}
    \note{
    }
\end{frame}

\begin{frame}[fragile]{The problem of vadd3 generated by vadd-gen}
    Runtime error occurs for wrong length vectors.
    \renewcommand{\V}{\text{Vec}}
    \begin{tabbing}
        \( \text{vadd3}\ [1;2;3;4]\ [4;5;6;7];; \) \\
        \( \longrightarrow \textbf{RUNTIME ERROR!} \)
    \end{tabbing}
    \note{
    }
\end{frame}

\begin{frame}[fragile]{Dependent Types}
    Types dependent on values
    \begin{itemize}
            \item Can give more fine restriction on terms.
    \end{itemize}
    \renewcommand{\V}{\text{Vec}\ }
    \begin{tabbing}
        \( [1;2;3]:\ (\V 3) \) \\
        \( [1;2;3;4]:\ (\V 4) \) \\
        \( \text{vadd3}:\ (\V 3) \to (\V 3) \to (\V 3) \)
    \end{tabbing}

    \note{
        By the way, we can write types dependent on values using dependent types.
    }
\end{frame}

\begin{frame}[fragile]{Our Approach: \LMD}
    Introduce dependent types into \LTP [Hanada\&Igarashi'14].
    \begin{itemize}
        \item Features of multi-stage programming
            \begin{itemize}
                \item Code type
                \item Type safe {\bf{run}}
                \item Cross Stage Persistence(CSP)
            \end{itemize}
        \item Features of dependent types
            \begin{itemize}
                \item Typing, Kinding, Well-formed kind rules with stages
                \item Equivalence rules including CSP
            \end{itemize}
    \end{itemize}
    \note{
        Our approach to the problem of multi-stage programming is introducing dependent types into a multi-stage calculus by Hanada and Igarashi.
        We call our multi-stage calculus as \LMD.
    }
\end{frame}

\begin{frame}[fragile]{vadd-gen in \LMD}
    Type tells us the length of vector.
    \newcommand{\Vn}{\text{Vec}\ n}
    \newcommand{\Vt}{\text{Vec}\ 3}
    \begin{tabbing}
        \( \text{vadd-gen} : \Pi n:\I \to \TW_\alpha (\Vn \to \Vn \to \Vn) \) \\[2mm]
        \( \textbf{let}\ \text{vadd3} = \textbf{run}\ (\text{vadd-gen}\ 3);; \) \\
        \( \text{val}\ \text{vadd3} : \Vt \to \Vt \to \Vt \) \\[2mm]
    \end{tabbing}
    Now, vadd3 takes only vectors of 3 length.
    \note{
        vadd-gen is changed as follows in \LMD.
        Thanks to dependent types ...
    }
\end{frame}

\begin{frame}[fragile]{Generated functions by vadd-gen in \LMD}
    Type system prevents illegal use of generated function.
    \renewcommand{\V}{\text{Vec}}
    \begin{tabbing}
        \( \text{vadd3}\ [1;2;3]\ [4;5;6];; \) \\
        \( - : \V = [5;7;9] \) \\[2mm]
        \( \text{vadd3}\ [1;2;3;4]\ [4;5;6;7];; \) \\
        \( \longrightarrow \textbf{TYPE ERROR!} \text{ (not RUNTIME ERROR)} \)
    \end{tabbing}
    \note{
        Type of generated functions by vadd-gen in \LMD have the length information in their types.
    }
\end{frame}

\begin{frame}[fragile]{Terms of \LMD}
    \begin{block}{Terms}
        \( M ::= c \mid x \mid \lambda x:\tau.M\ \mid M\ M \mid \TB_\alpha M \mid \textbf{run}\ M \mid \TBL_\alpha M \)
    \end{block}
    \begin{itemize}
        \item \( c \): constants
            \begin{itemize}
                    \item true, false, etc.
            \end{itemize}
        \item \( \TB_\alpha M \): code value
        \item \( \textbf{run}\ M \): run a code value \( M \)
            \begin{itemize}
                \item \( \textbf{run }(\text{vadd-gen }3)\)
            \end{itemize}
    \end{itemize}
    \note{
        Then, let me explain details of \LMD. \\
        \( \TBL_\alpha M \) is not needed?
    }
\end{frame}

\begin{frame}[fragile]{Types of \LMD}
    \begin{block}{Types}
    \( \tau,\sigma ::= X \mid \Pi x:\tau.\sigma \mid \tau\ M \mid \TW_{\alpha} \tau \mid \F\alpha.\tau \)
    \end{block}
    \begin{itemize}
        \item \( X \): type-level constants 
            \begin{itemize}
                \item \( \text{Vec} \), \( \text{Int} \).
            \end{itemize}
        \item \( \Pi x:\tau.\sigma \): dependent type.
        \item \( \tau\ M \): Application of term to dependent type.
            \begin{itemize}
                \item \( \text{Vec } 3 \)
            \end{itemize}
        \item \( \TW_{\alpha} \tau \): a type of code of type \( \tau \).
            \begin{itemize}
                \item \( \text{vadd-gen 3} : \TW_\alpha (\text{Vec}\ 3 \to \text{Vec}\ 3 \to \text{Vec}\ 3) \)
            \end{itemize}
    \end{itemize}
    \note{
    }
\end{frame}

\begin{frame}[fragile]{Kinds of \LMD}
    \begin{block}{Kinds}
        \( K ::= * \mid \Pi x:\tau.K \)
    \end{block}
    \begin{itemize}
        \item \( * \): kind of proper type
            \begin{itemize}
                \item \( \text{Int} :: * \)
            \end{itemize}
        \item \( \Pi x:\tau.K \): kind dependent on term.
            \begin{itemize}
                \item \( \text{Vec} :: (\Pi x:\text{Int}.*) \)
            \end{itemize}
    \end{itemize}
    \note{
    }
\end{frame}

\begin{frame}[fragile]{Judgements of \LMD}
    \begin{itemize}
        \item \( \G \V K @ A \)
            \begin{itemize}
                \item \( K \) is a proper kind under signature \( \Sigma \) at stage \( A \).
            \end{itemize}
        \item \( \G \V \tau :: K @ A \)
            \begin{itemize}
                \item Type \( \tau \) has kind \( K \) under signature \( \Sigma \) at stage \( A \).
            \end{itemize}
        \item \( \G \V M : \tau @ A \)
        \item \( \G \V K \E K' @ A \)
            \begin{itemize}
                \item Kind \( K \) is equivalent to \( K' \) under signature \( \Sigma \) at stage \( A \).
            \end{itemize}
        \item \( \G \V \tau \E \tau' @ A \)
        \item \( \G \V M \E M' @ A \)
    \end{itemize}
    \note{
    }
\end{frame}

\begin{frame}[fragile]{What is stage\ \( A \)\ ?}
    % I copied this underwrite macro from
    % https://tex.stackexchange.com/questions/141932/how-to-write-under-an-underline.
    % \underwrite[<thickness>]{<numerator>}{<denominator>}
    \newcommand{\underwrite}[3][]{
        \genfrac{}{}{#1}{}{\textstyle #2}{\textstyle #3}
    }

    Level of nested brankets
    \begin{block}{Stages}
        \( A := \varepsilon (\text{empty}) \mid \alpha \mid A\alpha \)
    \end{block}
    \begin{center}
        \( \underbrace{(\textbf{run}\ (\TB_\alpha\ \overbrace{(2 + 3)}^{\let\scriptstyle\textstyle\substack{@\alpha}})) + 1}_{\let\scriptstyle\textstyle\substack{@\varepsilon}} \)
    \end{center}
    \note{
        Stage is a sequence of stage variables which appended to each brack triangles.
        \( \varepsilon \) means the empty sequence

        The wholes expression is a term on stage \( \varepsilon \),
        But subexpression in the brack triangle is a term on stage \( \alpha \).
    }
\end{frame}

\begin{frame}[fragile]{What is signature \( \Sigma \)\ ?}
    Signature \( \Sigma \) is a sequence of type-level constants and constant values.
    \begin{itemize}
        \item For example,
            \begin{itemize}
                \item \( \Sigma = \text{Int}::*, \text{Vec}::\Pi x:\text{Int}.*, 0:\text{Int} \)
            \end{itemize}
        \item Elements of \( \Sigma \) can be used at any stages.
    \end{itemize}
    \note{
        自然数が無限個あるというツッコミを誘発する気がする。
        後者関数が存在することにするか?
    }
\end{frame}

\begin{frame}[fragile]{Typing Rules of \LMD}
    \begin{center}
        \infrule[{\TTB}]{
            \G\V M:\tau@{A\alpha}
        }{
            \G\V\TB_{\alpha}M:\TW_{\alpha}\tau@A
        } \\[2mm]
        \infrule[{\textsc{T-App}}]{
            \G\V M:(\Pi (x:\sigma).\tau)@A \andalso
            \G\V N:\sigma@A
        }{
            \G\V M\ N : \tau[x\mapsto N]@A
        } \\[2mm]
        \infrule[{\TConst}]{
            c:\tau \in \Sigma \andalso
        }{
            \G \V c:\tau@A
        }
    \end{center}
    \begin{itemize}
        \item Terms are stage sensitive so cannot ignore stages.
        \item Constants in \( \Sigma \) can be used at any stages.
    \end{itemize}
    \note{
        Let's have a look on some typing rules.
    }
\end{frame}

\begin{frame}[fragile]{Kinding Rules of \LMD}
    \begin{center}
        \infrule[{\KTW}]{
            \G\V \tau::*@A\alpha
        }{
            \G\V\TW_\alpha \tau::*@A
        }\\[2mm]
        \infrule[{\KApp}]{
            \G\V \sigma:: (\Pi x:\tau.K)@A \andalso \G\V M:\tau@A
        }{
            \G\V \sigma\ M::K[x\mapsto M]@A
        }
        \infrule[{\KTConst}]{
            X::K \in \Sigma
        }{
            \G \V X::K@A
        }
    \end{center}
    \begin{itemize}
        \item Types are also stage sensitive.
        \item Type-level constants in \( \Sigma \) can be used at any stages.
    \end{itemize}
    \note{
        Kinding rules have something in common with typing rules.
    }
\end{frame}

\begin{frame}[fragile]{Example of Kinding Derivation}
    \renewcommand{\Vec}{\text{Vec}}
    \( \V \TB_\alpha [1,2,3] : \underbrace{\TW_\alpha (\Vec\ 3)}_{\TW_\alpha (\Vec\ 3) :: * @ \varepsilon} @ \varepsilon \) \\[2mm]
    \pause
    \( \Sigma = \Vec::\Pi x:\I.*, 0:\I, 1:\I, \cdots \)
    \begin{center}
        \footnotesize
        \begin{minipage}{0.7\hsize}
            \infer[\KTW]
            {\V \TW_\alpha (\Vec\ 3) :: * @ \varepsilon}
            {
                \infer[\KApp]
                {\V \Vec\ 3 :: * @ \alpha}
                {
                    \infer[\KTConst]
                    {\V \Vec :: \Pi x:\I.* @ \alpha}
                    {\Vec :: \Pi x:\I.* \in \Sigma}
                    \andalso
                    \infer[\TConst]
                    {\V 3 : \I @ \alpha}
                    {3 : \I \in \Sigma}
                }
            }
        \end{minipage}
    \end{center}
    We can use \Vec\ and integers at any stages.
    \note{
        The constructed type Vec 3 is stage sensitive so we cannot use at arbitrary stages
        but the elements, Vec and 3 are in the signature \( \Sigma \) and can be used at any stages.
    }
\end{frame}

\begin{frame}[fragile]{Type and Term Equality Rules of \LMD}
    \begin{center}
        \footnotesize{
            \infrule[{\QTApp}]{
                \G\V \tau \E \sigma :: (\Pi x:\rho.K)@A \andalso
                \G\V M \E N : \rho @A
            }{
                \G\V \tau\ M \E \sigma\ N :: K[x \mapsto M]@A
            } \\[2mm]
            \infrule[{\QBeta}]{
                \G,x:\sigma@A\V M:\tau@A \andalso
                \G\V N:\sigma@A
            }{
                \G\V(\lambda x:\sigma.M)\ N\E M[x\mapsto N] : \tau[x \mapsto N]@A
            } \\[2mm]
            \infrule[{\QPercent}]{
                \G\V M:\tau@{A\alpha} \andalso
                \G\V M:\tau@A
            }{
                \G\V\%_\alpha M \E M : \tau@{A\alpha}
            } \\[2mm]
            \infrule[{\textsc{T-Conv}}]{
                \G\V M:\tau@A \andalso
                \G\V \tau\equiv \sigma :: K@A
            }{
                \G\V M:\sigma@A
            }
        }
    \end{center}
    \note{
    }
\end{frame}

\begin{frame}[fragile]{Examples of Type Equality}
    \renewcommand{\V}{\text{Vec}}
    \begin{itemize}
        \item \( (\lambda x:\text{Int}.x)\ 3\E 3 \)
            \begin{itemize}
                \item From \QBeta.
            \end{itemize}
        \item \( \V\ ((\lambda x:\text{Int}.x)\ 3) \E \V\ 3 \)
            \begin{itemize}
                \item From \QTApp and \QBeta.
            \end{itemize}
        \item \( \V\ (\%_\alpha 3) \E \V\ 3 \)
            \begin{itemize}
                \item Because \( 3 \) has type \text{Int} at any stage, we can use \QPercent.
            \end{itemize}
    \end{itemize}
    \note{
        Shold I show complete derivation tree?
    }
\end{frame}

\begin{frame}[fragile]{Propaties}
    \begin{itemize}
        \item Subject Reduction
        \item Strong Normalization
        \item Confluence
        \item Progress
    \end{itemize}
    \note{
    }
\end{frame}

\begin{frame}[fragile]{Conclusion \& Future Works}
    \begin{block}{Conclusion}
        We introduced dependent types into muti-stage programming.
        \begin{itemize}
            \item Define typing, kinding, well-formed kinding rules including stages.
            \item Define equality rules.
            \item Prove type soundness.
        \end{itemize}
    \end{block}
    \begin{block}{Future Works}
        \begin{itemize}
            \item Algorithmic Typing
        \end{itemize}
    \end{block}
    \note{
    }
\end{frame}

\begin{frame}[fragile]{Related Works}
    \begin{itemize}
        \item \LTP [Hanada\&Igarashi'14]
            \begin{itemize}
                \item Multi-stage programming including explicit CSP
            \end{itemize}
    \end{itemize}
    \note{
    }
\end{frame}

\end{document}

% !TEX root = ../main.tex

\section{Properties of \LMD \label{sec:properties}}

In this section, we show basic properties of \LMD: preservation, strong normalization, confluence for full reduction, and progress for staged reduction.

% Substitution Lemma

Substitution Lemma in \LMD{} is a little more complicated than usual
because there are eight judgment forms and two kinds of substitution.
Term Substitution Lemma states that term substitution $[z \mapsto M]$
preserves derivability of judgments and Stage Substitution Lemma
states similarly for stage substitution $[\alpha\mapsto A]$.

We let $\mathcal{J}$ stand for the judgments $K \iskind @A$, $\tau::K@A$,
$M:\tau@A$, $K \E J @ A$, $\tau \E \sigma @ A$, and
$M \E N : \tau @ A$.  Substitutions $\mathcal{J}[z \mapsto M]$ and
$\mathcal{J}[\alpha \mapsto A]$ are defined in a straightforward
manner.  Using these notations, the two substitution lemmas are stated as follows:

\begin{lemma}[Term Substitution]
  If $\G, z:\xi@B, \D \V \mathcal{J}$ and $\G\V P:\xi @B$, then $\G, (\D[z \mapsto P]) \V \mathcal{J}[z \mapsto P]$.  Similarly, if $\V \G, z:\xi@B, \D$ and
  $\G\V P:\xi @B$, then $\V \G, (\D[z \mapsto P])$.
\end{lemma}

\begin{proof}
  By simultaneous induction on derivations.
 \qed
\end{proof}

\begin{lemma}[Stage Substitution]
	If $\G \V \mathcal{J}$, then $\G[\beta\mapsto B] \V \mathcal{J}[\beta\mapsto B]$.  Similarly, if $\G \V \mathcal{J}$, then $\V \G[\beta\mapsto B]$.
\end{lemma}

\begin{proof}
  By simultaneous induction on derivations.
  \qed
\end{proof}

% Inversion Lemma

The following Inversion Lemma is needed to prove main theorems.
As usual~\cite{TAPL}, Inversion Lemma enables us to infer the types of subterms of a term from the type of the term.

\begin{lemma}[Inversion]\ 
	\begin{enumerate}
		\item If $\G \V (\lambda x:\sigma.M) : \rho$ then there are $\sigma'$ and $\tau'$ such that
		      $\rho = \Pi x:\sigma'.\tau'$, $\G \V \sigma \E \sigma'@A$ and $\G ,x:\sigma'@A\V M:\tau'@A$.
		\item If $\G \V \TB_\alpha M : \tau@A$ then 
		      there is $\sigma$ such that $\tau = \TW_\alpha \sigma$ and $\G \V M : \sigma@A$.
		  \item If $\G \V \Lambda\alpha.M : \tau$ then 
		  there is $\sigma$ such that $\sigma = \forall\alpha.\sigma$ and $\G \V M : \sigma@A$.% and $\alpha \notin \FTV(\G) \cup \FV(A)$.
	\end{enumerate}
\end{lemma}

\begin{proof}
  Each item is strengthen by statements about type equivalence.
  For example, the first statement is augmented by
  \begin{quotation}
    If $\G \V \rho \E (\Pi x:\sigma.\tau) : K @A$, then there exist
    $\sigma'$ and $\tau'$ such that $\rho = \Pi x:\sigma'.\tau'$ and
    $\G \V \sigma \E \sigma' : K @A$ and
    $\G, x:\sigma@A\V \tau \E \tau' : J @A$.
  \end{quotation}
  and its symmetric version.  Then, they are proved simultaneously by induction on derivations.
  Similarly for the others.
 \qed
\end{proof}


% Preservation

Thanks to Term/Stage Substitution and Inversion, we can prove Preservation easily.

\begin{theorem}[Preservation]
	If $\G\V M:\tau @A$ and $M \longrightarrow M'$, then $\G\V M':\tau @A$.
\end{theorem}

\begin{proof}
	First, there are three cases for $M \longrightarrow M'$.
	They are $M \longrightarrow_\beta M'$, $M \longrightarrow_\Lambda M'$, and $M \longrightarrow_\blacklozenge M'$.
	For each case, we can use straightforward induction on typing derivations.
	% Difficult cases are \TApp, \TTBL, and \TIns.
	% We need Inversion Lemmas for them.
\end{proof}

% Strong Normalization

Strong Normalization is also an important property, which guarantees that
no typed term has an infinite reduction sequence.
Following standard proofs (see, e.g., \cite{harper1993framework}), we prove this theorem by translating \LMD to the simply typed lambda calculus.

\begin{theorem}[Strong Normalization]
	If $\G\V M_1:\tau@A$ then there is no infinite sequence $(M_i)_{i\ge1}$ of terms such that
	$M_i \longrightarrow M_{i+1}$ for $i\ge 1$.
\end{theorem}

\begin{proof}
	In order to prove this theorem, we define a translation $(\cdot)^\natural$ from \LMD\ to the simply typed lambda calculus.
	Second, we prove the $\natural$-translation preserves typing and reduction.
	Then, we can prove Strong Normalization of \LMD from Strong Normalization of the simply typed lambda calculus.
\end{proof}

Confluence is a property that any reduction sequences from one typed term converge.
Since we have proved Strong Normalization, we can use Newman's Lemma~\cite{DBLP:books/daglib/0092409} to prove Confluence.

\begin{theorem}[Confluence]
	For any term $M$, if $M \longrightarrow^* M'$ and $M \longrightarrow^* M''$ then
	there exists $M'''$ that satisfies $M' \longrightarrow^* M'''$ and $M'' \longrightarrow^* M'''$.
\end{theorem}

\begin{proof}
  We can easily show Weak Church-Rosser.  Use Newman's Lemma.
	% Because we proved Strong Normalization of \LMD, 
	% we can use Newman's lemma to prove Confluence of \LMD.
	% Then, what we must show is Weak Church-Rosser Property now.
	% When we consider two different redexes in a \LMD term, they can only be disjoint, or one is a part of the other.
	% In short, they are never overlapped each other.
	% So, we can reduce one of them after we reduce another.
\end{proof}

Now, we turn our attention to staged semantics.  First, the staged
reduction relation is a subrelation of full reduction, so Subject
Reduction holds also for the staged reduction.

\begin{theorem}
  If $M \longrightarrow_s M'$, then $M \longrightarrow M'$.
\end{theorem}
\begin{proof}
  Easy.
 \qed
\end{proof}

The following theorem Unique Decomposition ensures that, every typed
term is either a value or can be uniquely decomposed to an evaluation
context and a redex, ensuring that a well typed term is not
immediately stuck and the staged semantics is deterministic.

\begin{theorem}[Unique Decomposition]
  If $\G$ does not have any variable declared at stage $\varepsilon$ 
  and $\G \V M : \tau @ A$, then either
  \begin{enumerate}
  \item $ M \in V^A$, or
  \item $M$ can be uniquely decomposed into an evaluation context and a redex, that is, there uniquely exist $B, E^A_B$, and $R^B$ such that $M = E^A_B[R^B]$.
  \end{enumerate}
\end{theorem}

\begin{proof}
  We can prove by straightforward induction on typing derivations.
%  Main cases are \TApp, \TTBL, and \TIns, where we use Inversion.
  \qed
\end{proof}

The type environment $\G$ in the statement usually has to be empty;
in other words, the term has to be closed.  The condition is relaxed here
because variables at stages higher than \(\varepsilon\) are considered
symbols.  In fact, this relaxation is required for proof by induction
to work.

% This theorem is important because it guarantees
% that the evaluation context decides a redex to reduce deterministically.
% Specifically speaking, this theorem guarantee that 
% when you write a interpreter using the evaluation context of \LMD,
% your interpreter works just as intended.
      
Progress is a corollary of Unique Decomposition.

\begin{corollary}[Progress]
	If $\G$ does not have any variable declared at stage $\varepsilon$ and $\G \V M : \tau  @ A$, then
	$ M \in V^A $ or $M'$ exists such that $M \longrightarrow_s M'$.
\end{corollary}


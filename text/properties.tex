% !TEX root = ../main.tex

\section{Properties of \LMD \label{sec:properties}}

In this section, we show basic properties of \LMD: preservation, strong normalization, confluence, and progress.

% Substitution Lemma

Substitution Lemma in \LMD\ is little more complicated than an ordinary one 
because there are six types of judgment and two types of substitution in \LMD.
Substitution Lemma on Terms states that term substitution $[z \mapsto P]$ preserves
typing, kinding, well-formed kinding, term equivalence, type equivalence, and king equivalence.
Substitution Lemma on Terms states the same for stage substitution $[\beta\mapsto A]$.
\red{2つの代入補題を指してSubstitution Lemmasということが出来るのか?}

In order to write Substitution Lemma briefly, we write $\G\V J@A$ for an arbitary judgment
amongst $\G\V M:\tau@A$, $\G\V \tau::K@A$, $\G\V K \iskind @A$, $\G\V M \E N : \tau @ A$, 
$\G\V \tau \E \sigma @ A$, and $\G \V K \E J @ A$.

\begin{lemma}[Substitution Lemma on Terms]
	If $\G, z:\xi@B, \D \V J @ A$ and $\G\V P:\xi @B$ then $\G, \D[z \mapsto P] \V J[z \mapsto P] @ A$.
\end{lemma}

\red{TODO: Fix the proof.}
\red{$\G\V J@A$における代入を定義する必要があるか?}

\begin{proof}
	Straightforward induction on derivations.
\end{proof}

\begin{lemma}[Substitution Lemma on Stages]
	If $\G \V J @ A$ then $\G[\beta\mapsto B] \V J[\beta\mapsto B] @ A[\beta\mapsto B]$.
\end{lemma}

\begin{proof}
	Straightforward induction on derivations.
\end{proof}

% Inversion Lemma

The following three Inversion Lemmas are needed to prove main theorems.
As usual~\cite{TAPL}, Inversion Lemmas enable us to infer the types of subterms of a term from the type of the term.

\begin{lemma}[Inversion Lemma for $\Pi$-types]
	If $\G \V (\lambda x:\sigma.M) : (\Pi x:\sigma'.\tau)@A$
	then $\G \V \sigma \E \sigma'@A$ and $\G ,x:\sigma@A\V M:\tau @A$.
\end{lemma}

\begin{proof}
	First, we generalize this theorem by adding statements about $\Pi$-types in equivalence relation in order to use induction.
	After that, we can prove with straightforward induction on derivations.
\end{proof}

\begin{lemma}[Inversion Lemma for $\TW$-types]
	If $\G \V \TB_\alpha M : \TW_\alpha \tau @A$ then $\G \V M : \tau @A$.
\end{lemma}

\begin{proof}
	Prove as with Inversion Lemma for $\Pi$-types.
\end{proof}

\begin{lemma}[Inversion Lemma for $\forall$-types]
	If $\G \V \Lambda\alpha.M : \forall\alpha.\tau @A$ then $\G \V M : \tau @A$ and $\alpha \notin \FTV(\G) \cup \FV(A)$.
\end{lemma}

\begin{proof}
	Prove as with Inversion Lemma for $\Pi$-types.
\end{proof}

% Preservation

Thanks to Substitution Lemmas and Inversion Lemmas, we can prove Preservation easily.
Preservations ensure that any one step reduction preserves type.

\begin{theorem}[Preservation]
	If $\G\V M:\tau @A$ and $M \longrightarrow M'$, then $\G\V M':\tau @A$.
\end{theorem}

\begin{proof}
	First, there are three cases for $M \longrightarrow M'$.
	They are $M \longrightarrow_\beta M'$, $M \longrightarrow_\Lambda M'$, and $M \longrightarrow_\blacklozenge M'$.
	For each case, we can use straightforward induction on derivations.
	Difficult cases are \TApp, \TTBL, and \TIns.
	We need Inversion Lemmas for them.
\end{proof}

% Strong Normalization

Strong Normalization is also important property which guarantee that
no typed term has an infinite reduction sequence.
We prove this thorem by translating \LMD to the symply typed lambda calculus.

\begin{theorem}[Strong Normalization]
	If $\G\V^A M:\tau$ then there is no infinite sequence of terms $(M_i)_{i\ge1}$ and 
	$M_i \longrightarrow M_{i+1}$ for $i\ge 1$.
\end{theorem}

\begin{proof}
	In order prove this theorem, we define a translation $\natural$ from \LMD\ to the symply typed lambda calculus.
	Second, we prove $\natural$ preserves typing and $\beta$-reductions.
	Then, we can prove Strong Normalization of \LMD\ from Strong Normalization of the symply typed lambda calculus.
\end{proof}

Confluence is a property that any reduction sequences from one typed term converge.
Because we have proved Strong Normalization, we can use Newman's Lemma to prove Confluence.

\begin{theorem}[Confluence]
	For any term $M$, if $M \longrightarrow^* M'$ and $M \longrightarrow^* M''$ then
	there exists $M'''$ that satisfies $M' \longrightarrow^* M'''$ and $M'' \longrightarrow^* M'''$.
\end{theorem}

\begin{proof}
	Because we proved Strong Normalization of \LMD, 
	we can use Newman's lemma to prove Confluence of \LMD.
	Then, what we must show is Weak Church-Rosser Property now.
	When we consider two different redexes in a \LMD term, they can only be disjoint, or one is a part of the other.
	In short, they are never overlapped each other.
	So, we can reduce one of them after we reduce another.
\end{proof}

Unique Decomposition ensure that,
for every typed term, we can find just one redex to reduce by the evaluation context or it is a value.
This theorem is important because it guarantee
that the evaluation context decides a redex to reduce deterministically.
Specifically speaking, this theorem guarantee that 
when you write a interpreter using the evaluation context of \LMD,
your interpreter works just as intended.
Although, in ordinary calculi, there is a condition that $\G$ is $\emptyset$,
the condition is relaxed because variables at non-$\varepsilon$ stages are values in \LMD.

\begin{theorem}[Unique Decomposition]
	If $\G$ does not have any variable declared at stage $\varepsilon$ 
	and $\G \V M : \tau @ A$ then either
	\begin{enumerate}
		\item $ M \in V^A$, or
		\item there exist $B, E^A_B$, and $R^B$ such that $M = E^A_B[R^B]$ with $B = \varepsilon$ or $B = \beta$ for some $\beta$.
	\end{enumerate}
\end{theorem}

\begin{proof}
	We can prove by straightforward induction on derivations.
	Difficult cases are \TApp, \TTBL, and \TIns.
	We need Inversion Lemmas for them.
\end{proof}

Progress states $\longrightarrow_s$ defines appropriate reduction for typed terms of \LMD.
Thanks to Unique Decomposition, we can prove this theorem easily.

\begin{theorem}[Progress]
	If $\G$ does not have any variable declared at stage $\varepsilon$ and $\G \V M : \tau  @ A$ then
	$ M \in V^A $ or $M'$ exists such that $M \longrightarrow_s M'$.
\end{theorem}

\begin{proof}
	We can prove by straightforward induction on derivations.
	Difficult cases are \TApp, \TTBL, and \TIns.
	We need Inversion Lemmas for them.
\end{proof}
% !TEX root = ../main.tex

\section{Introduction}

\subsection{Multi-stage Programming}

% \subsubsection{多段階計算とは何か?}

Multi-stage programming enables us to generate and run fragments of code at runtime.
MetaOCaml\cite{oleg2014} provide features for multi-stage programming, which are brackets, escape, and run.
Brackets, written \verb|.< e >.| in MetaOCaml, makes a code value from \verb|e|.
Run, written \verb| run e | in MetaOCaml, runs a code value of \verb|e| and restore \verb|e|.
Escape, written \verb| .~ e | in MetaOCaml, expands a code value of \verb|e|.
Unlike run, escape is supposed to use in a fragment of code, that is, escape cannot be used as an alternative of run.
For example, the following MetaOCaml expression

\begin{verbatim}
        let plusone = .< fun x -> x + 1 >. in .< .~plusone 2 >.
\end{verbatim}
evaluates to \verb|.< (fun x -> x + 1) 2 >.|. We can get the result with run.
\begin{verbatim}
        run (let plusone = .< fun x -> x + 1 >. in .< .~plusone 2 >.)
\end{verbatim}
is reduced to 3.

% \subsubsection{CSPに関する補足説明}

Cross-stage Persistence (CSP) is another primitive of multi-stage programming, which enable us to embed computed values into a code value.
This is an example of CSP.
\begin{verbatim}
        let plusone = fun x -> x + 1 in
        let y = plusone 2 in  .< y >.
\end{verbatim}
This program evaluates to \verb|.< 3 >.| not  \verb|.< plusone 2 >.|.
This is because the variable \verb|y| is introduced into a code fragment using CSP.
Therefore, the variable \verb|y| is embedded after it was calculated.
CSP is very important when we write practical programs 
because CSP enables us to use library functions in code fragments as in \verb|.<List.combine [1;2] ['a';'b']>.|.

% \subsubsection{多段階計算のメリットをべき乗の例を用いて説明する}

The main application of multi-stage calculi is program optimization.
A famous example is power functions.
Usual power functions take two arguments, which are the base and the exponent.
We can make a specialized power function for given exponents with multi-stage programming and 
optimize them by unrolling a loop in functions for given exponents.
The optimized power function for the exponent of \verb|3| looks like \verb|power3 = fun x -> x * x * x|.
\verb|power3| is faster than ordinary \verb|power| function because it contains no loop.
Multi-stage programming can optimize functions which take more than two arguments 
by generating a code fragment optimized to a given argument.

% \subsubsection{ベクトル計算も高速化できる}

Multi-stage programming can also optimize vector calculation.
For example, \verb|vadd| function, which takes two vectors and return the sum of them,
is implemented with a loop in many cases.
We can unroll \verb|vadd| function for a given vector length and optimize it with multi-stage programming.
\red{MetaOCamlで書いた長さ付きベクトルの例?}

% \subsubsection{生成した行列計算のコードは特定のサイズに特化しており、他のサイズでは使えない。
% このため、間違って組み合わせると容易にエラーを引き起こす。既存の型システムではこのエラーを防止できない}

However, there is a sever problem in functions which are optimized by multi-stage programming.
Although unrolled \verb|vadd| function is optimized, we cannot use it for different vector length.
For example, when you optimize for the length of 5, you shouldn't use it for vectors of 3 lengths.
Otherwise, we will get a Segmentation Fault error.
This problem is serious but existing type systems for multi-stage calculi cannot prevent it.

% We call this problem \red{Specialized Code Problem}.

\subsection{Dependent Types}

% \subsubsection{依存型とは何か?}

Dependent types are types which is dependent on values.
We can use dependent types for securer programming.
For example, we can realize vectors with their sizes with them.
\begin{verbatim}
        Vector :: Int -> *
\end{verbatim}
\verb|Vector| is a type constor which takes the length of vectors, that is, 
\verb|Vector 3| is a type for vectors whose lengths are 3.
If \verb|vadd| function has the type of \verb|Vector n -> Vector n -> Vector n|,
we can confirm the two arguments of \verb|vadd| function has the same length.
\red{secure programming? finer grained typing? とか}

\subsection{Multi-stage Programming with Dependent Types}
\red{このsubsectionは短いので上とくっつけてもいいかもしれない}

% \subsubsection{「生成した行列計算のコードは特定のサイズに特化しており、他のサイズでは使えない。
% このため、間違って組み合わせると容易にエラーを引き起こす。」問題を依存型で解決する。}

As we pointed out in the above section,
functions optimized with multi-stage programming can take only restricted values as arguments.
When you optimize \verb|vadd| function for the length of 5, you should it only for 5 length vectors.
We introduce dependent types into a multi-stage calculus
so that the type system can guarantee optimized functions are used properly.

In this paper, we design new multi-stage calculus \LMD by 
merging dependent types into existing multi-stage calculus \LTP\cite{Hanada2014}.
\LMD is multi-stage calculus which contains dependent types.
\LMD makes multi-stage programming more safer with the power of dependent types.

\subsection{Organization of the Paper}

The organization of this paper is the following.
Section 2 gives an informal overview of \LMD.
Section 3 defines the syntax, type system, full reduction, evaluation contexts of \LMD.
Section 4 shows the properties of \LMD including Unique Decomposition.
Section 5 discusses future works.
Finally, Section 6 discusses related works.
\AI{Use $\backslash$label and $\backslash$ref!}
\AI{``Related work'' and ``future work'' (they are uncountable).}


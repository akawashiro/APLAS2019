% !TEX root = ../main.tex

\section{Introduction}
\label{sec:intro}

\subsection{Multi-stage Programming and MetaOCaml}

Multi-stage programming makes it easier for programmers to implement
generation and execution of code at run time by providing language
constructs for composing and running pieces of code as first-class
values.  A promising application of multi-stage programming is
(run-time) code specialization, which generates program code
specialized to partial inputs to the program and such applications are
studied in the literature~\cite{8384206,mainland2012metahaskell,taha2007gentle}.
% \AI{We should list applications of multi-stage programming, not necessarily of MetaOCaml.}

MetaOCaml~\cite{calcagno2003implementing,oleg2014} is an extension of
OCaml\footnote{\url{http://ocaml.org}} with special constructs for
multi-stage programming, including brackets and escape, which are
(hygienic) quasi-quotation, and \texttt{run}, which is similar to
\texttt{eval} in Lisp, and cross-stage persistence
(CSP)~\cite{MetaML}.  Programmers can easily write code generators by
using these features.  Moreover, MetaOCaml is equipped with a powerful
type system for safe code generation and execution.  The notion of
code types is introduced to prevent code values that represent
ill-typed expressions from being generated.  For example, a quotation
of expression \texttt{1 + 1} is given type \texttt{int code}
and a code-generating function, which takes a code value \(c\) as an
argument and returns \(c \texttt{ + } c\), is given type \texttt{int
  code -> int code} so that it cannot be applied to, say, a quotation
of \texttt{"Hello"}, which is given type \texttt{string
  code}.  Ensuring safety for \verb|run| is more challenging because
code types by themselves do not guarantee that the execution of code values never
results in unbound variable errors.  Taha and
Nielsen~\cite{taha2003environment} introduced the notion of
environment classifiers to address the problem, developed a type
system to ensure not only type-safe composition but also type-safe
execution of code values, and proved a type soundness theorem (for a formal calculus \(\lambda^\alpha\) modeling a pure subset of MetaOCaml).


% rackets, written \verb|.< e >.| in
% MetaOCaml, makes a value that represents the code (or abstract syntax
% tree) of expression \verb|e|; and escape, written \verb|.~ e| and is
% supposed to appear inside brackets, evaluates \verb|e| to a code value
% and expands the code into the surrounding code.  For example,
% \begin{verbatim}
%    let plusone = .< fun x -> x + 1 >. in .< .~plusone 2 >.
% \end{verbatim}
% evaluates to \verb|.< (fun x -> x + 1) 2 >.|.  
% Run, written \verb|run e|, evaluates \verb|e| to a code value
% and executes it.  For example,
% \begin{verbatim}
%    run (let plusone = .< fun x -> x + 1 >. in .< .~plusone 2 >.)
% \end{verbatim}
% yields \verb|3| (here, escape connects tighter than application).
% Finally, CSP is a primitive to embed values (not necessarily code
% values) into a code value.  In MetaOCaml, CSP is implicitly applied to
% occurrences of variables defined outside brackets.  For example,
% \begin{verbatim}
%    let plusone = fun x -> x + 1 in
%    let y = plusone 2 in  .< y * 2 >.
% \end{verbatim}
% evaluates to \verb|.< 3 * 2 >.|, not \verb|.< plusone 2 * 2 >.|.
% This is because the variable \verb|y| is bound to an integer \verb|3|
% before \verb|.<y>.| is evaluated and the value of \verb|y| is embedded
% into the code value.  In MetaOCaml, CSP can be applied to (variables of) any type.  CSP is practically important because it enables us to use library functions in brackets, e.g., \verb|.<List.combine [1;2] ['a';'b']>.|.

% \subsubsection{多段階計算のメリットをべき乗の例を用いて説明する}

% As we mentioned earlier, a main application of multi-stage
% calculi is program optimization. 
% A famous example is power functions.
% Usual power functions take two arguments, which are the base and the exponent.
% We can make a specialized power function for given exponents with multi-stage programming and 
% optimize them by unrolling a loop in functions for given exponents.
% The optimized power function for the exponent of \verb|3| looks like \verb|power3 = fun x -> x * x * x|.
% \verb|power3| is faster than ordinary \verb|power| function because it contains no loop.
% Multi-stage programming can optimize functions which take more than two arguments 
% by generating a code fragment optimized to a given argument.

% 多段階計算の型理論の既存研究の紹介

However, the type system, which is based on the Hindley--Milner
polymorphism~\cite{Milner78JCSS}, is not strong enough to guarantee
invariants beyond simple types.  For example, Kiselyov~\cite{8384206}
demonstrates specialization of vector/matrix computation with respect
to the sizes of vectors and matrices in MetaOCaml but the type system
of MetaOCaml cannot prevent such specialized functions from being applied to
vectors and matrices of different sizes.

% Multi-stage programming can also optimize vector calculation.
% For example, \verb|vadd| function, which takes two vectors and return the sum of them,
% is implemented with a loop in many cases.
% We can unroll \verb|vadd| function for a given vector length and optimize it with multi-stage programming.

% % 多段階計算で生成したコードは特殊化されているが故に問題も多い

% However, there is a sever problem in functions which are optimized by multi-stage programming.
% Although unrolled \verb|vadd| function is optimized, we cannot use it for different vector length.
% For example, when you optimize for the length of 5, you shouldn't use it for vectors of 3 lengths.
% Otherwise, we will get a exception.
% This problem is serious but existing type systems for multi-stage calculi cannot prevent it.

\subsection{Multi-stage Programming with Dependent Types}

% 依存型とは何か?

One natural idea to address this problem is the introduction of dependent types
to express the size of data structures in static types~\cite{Xi98}.
% Dependent types are types which are dependent on values.
% We can use dependent types for securer programming like Xi and Pfenning
For example, we could declare vector types indexed by the size of
vectors as follows.
\begin{verbatim}
    Vector :: Int -> *
\end{verbatim}
\verb|Vector| is a type constructor that takes an integer (which
represents the length of vectors): for example, \verb|Vector 3| is the
type for vectors whose lengths are 3.  Then, our hope is to specialize
vector/matrix functions with respect to their size and get a piece of
function code, whose type respects the given size, \emph{provided at
  specialization time}.  For example, we would like to specialize a
function to add two vectors with respect to the size of vectors, that
is, to implement a code generator that takes a (nonnegative) integer $n$ as an
input and generates a piece of function code of type
\verb|(Vector |$n$\verb| -> Vector |$n$\verb| -> Vector |$n$\verb|) code|.

% As we pointed out in the above section,
% functions optimized with multi-stage programming can take only restricted values as arguments.
% When you optimize \verb|vadd| function for the length of 5, you should it only for 5 length vectors.
% We introduce dependent types into a multi-stage calculus
% so that the type system can guarantee optimized functions are used properly.
% Although there is a trial to combine multi-stage programming and dependent type by Brady and Hammond~\cite{brady2006dependently},
% they didn't give a formal definition and properties for the calculus.

\subsection{Our Work}
In this paper, we develop a new multi-stage calculus \LMD by extending
the existing multi-stage calculus \LTP\cite{Hanada2014} with dependent
types and study its properties.  We base our work on \LTP, in which
the four multi-stage constructs are handled slightly differently from
MetaOCaml, because its type system and semantics are arguably simpler
than \(\lambda^\alpha\)~\cite{taha2003environment}, which formalizes
the design of MetaOCaml more faithfully.  Dependent types are based on
\LLF~\cite{attapl}, which has one of the simplest forms of dependent
types.  Our technical contributions are summarized as follows:
\begin{itemize}
\item We give a formal definition of \LMD with its syntax, type system and
two kinds of reduction: full reduction, allowing reduction of any redex,
including one under $\lambda$-abstraction and quotation, and staged reduction, a
small-step call-by-value operational semantics that is closer to the intended
multi-stage implementation.
\item We show preservation, strong normalization, and confluence for
  full reduction; and show unique decomposition (and progress as its
  corollary) for staged reduction, which formalizes program execution.
\end{itemize}
The combination of multi-stage programming and dependent types has
been discussed by Pasalic, Taha, and Sheard~\cite{pasalic2002tagless}
and Brady and Hammond~\cite{brady2006dependently} but, to our
knowledge, our work is a first formal calculus of dependently typed
multi-stage programming with all the key constructs~\cite{MetaML}.

\subsubsection{Organization of the Paper.}

The organization of this paper is as follows.
Section~\ref{sec:informal-overview} gives an informal overview of
\LMD.  Section~\ref{sec:formal-definition} defines \LMD and
Section~\ref{sec:properties} shows properties of \LMD.
Section~\ref{sec:related-work} discusses related work and Section
\ref{sec:conclusion} concludes the paper with discussion of future
work.  For reference, we give the full definition of \LMD and
 proofs of properties in the Appendix.
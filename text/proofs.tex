\AI{Put a comma before ``then''.}
\AI{A period is needed for each statement (because it's a sentence).}
\AI{TAPL gives plenty of examples of how to write proofs.}
$\mathcal{J}$ is a metavariable for judgments as in Section \ref{sec:properties}.
We say type environment \(\G\) is a subsequence of type environment \(\D\)
if and only we can get \(\G\) from \(\D\) by deleting some or no variables without changing the order of the remaining elements.
\begin{lemma}[Weakening]
	If \(\G \V \mathcal{J}@A\) and \(\G\) is a subsequence of \(\D\), then \(\D \V J@A\). 
	% \AI{Apply similar changes and remove \%.}
	% \red{このコメントはどういう意味ですか?}
\end{lemma}

\begin{proof}
	By straightforward induction on the derivation of typing, kinding, well-formed kinding,
	term equivalence, type equivalence or kind equivalence.
	We show only representative cases.
	\begin{itemize}
		\item[] Case \WAbs{} where $\mathcal{J} = (\Pi x:\tau.K)\iskind @A$ and 
		      \begin{align*} 
		      	  & \G \V \tau :: * @ A &   & \G,x:\tau@A \V K \iskind @A. 
		      \end{align*}
		      By the induction hypothesis, we have
		      \begin{align*}
		      	  & \D \V \tau::*@A &   & \D, x:\tau@A \V K \iskind @ A 
		      \end{align*}
		      from which $\D\V (\Pi x:\tau.K) \iskind @ A$ follows by $\WAbs$.
		      
		\item[] Case \KAbs{} where $\mathcal{J} = (\Pi x:\tau.) :: (\Pi x:\tau.J)@A$ and 
		      \begin{align*} 
		      	  & \G \V \tau::*@A &   & \G,x:\tau@A\V \sigma :: J @ A. 
		      \end{align*}
		      By the induction hypothesis, we have
		      \begin{align*} 
		      	  & \D \V \tau::*@A &   & \D,x:\tau@A\V \sigma :: J @ A 
		      \end{align*}
		      from which $\D \V (\Pi x:\tau.) :: (\Pi x:\tau.J)@A$ follows.
		      % \item \TAbs
		      % \item \QKAbs
		      
		\item[] Case \QTAbs{} where \( \mathcal{J} = \rho \E \pi :: * @ A \) and
		      \begin{align*} 
		      	  & \G\V \tau \E \sigma :: *@A &   & \G,x:\tau@A \V \rho \E \pi :: *@A. 
		      \end{align*}
		      By the induction hypothesis, we have
		      \begin{align*} 
		      	  & \D\V \tau \E \sigma :: *@A &   & \D,x:\tau@A \V \rho \E \pi :: *@A 
		      \end{align*}
		      from which \( \D\V \rho \E \pi :: * @ A \) follows.
		      
		      % \item \QAbs
		      % \item \QBeta
	\end{itemize}
\end{proof}

\begin{theorem}[Term Substitution]
	If $\G, z:\xi@B, \D \V \mathcal{J}$ and $\G\V P:\xi @B$, then $\G, \D[z \mapsto P] \V \mathcal{J}[z \mapsto P]$.  Similarly, if $\V \G, z:\xi@B, \D$ and
	$\G\V P:\xi @B$, then $\V \G, \D[z \mapsto P]$.
\end{theorem}
\begin{proof}
	The six items are proved simultaneously by induction on derivations with case analysis on the last rule used.
	We show only representative cases.
	\begin{itemize}
										
		\newcommand{\SB}{[z \mapsto P]}
		\newcommand{\GG}{\G}
		\newcommand{\GGV}{\G \V}
										
		\item[] \textit{Case} \TVar{} where \(\mathcal{J} = y:\tau@A\) and 
		      \begin{align*} 
		      	  & y:\tau@A \in \G, z:\xi@B, \D &   & \G, z:\xi@B,\D \V \tau::*@A. 
		      \end{align*}
		      By the induction hypothesis, we have \(\G, \D\SB \V \tau\SB::*@A.\)
		      		      		      		      
		      \begin{itemize}
		      	\item If $y:\tau@A \in \G$ or $y:\tau@A \in \D$ then\\
		      	      it is obvious that $y:\tau\SB@A \in \GG$.\\
		      	      \(\G,\D\SB \V y\SB:\tau\SB@A\) from \TVar.
		      	      		      	      	      	      		      	      	      		      	      
		      	\item If $y:\tau@A = z:\xi@B$ then\\
		      	      $y = z$, $\tau = \xi$, and $A = B$.
		      	      From the well-formedness of \( \G, z:\xi@B,\D \), there is no $z$ in $\xi$.
		      	      Therefore, $\tau\SB = \xi\SB = \xi$.
		      	      It is obvious that $y\SB = z\SB = P$.
		      	      Thus, $\G \V y\SB : \tau\SB@A$.
		      	      By Weakening, $\G,\D\SB \V y\SB : \tau\SB@A$
		      	      		      	      
		      \end{itemize}
		      \vspace{3mm}
		      
		\item[] \textit{Case} \WAbs{} where \( \mathcal{J} =  (\Pi x:\tau.K) \iskind @A\) and
		      \begin{align*}	      	      	      
		      	  & \G, z:\xi@B, \D \V \tau::*@A &   & \G, z:\xi@B, \D, x:\tau \V K \iskind@A. 
		      \end{align*}
		      We can assume $x \neq z$ because we can select fresh $x$ when we construct $\Pi$ type.
		      By the induction hypothesis,
		      \begin{align*}
		      	  & \G,\D\SB \V \tau\SB::*@A &   & \G,\D\SB, x:\tau\SB \V K\SB \iskind@. 
		      \end{align*}
		      From \WAbs, $\G,\D\SB \V (\Pi x:\tau\SB.K\SB) \iskind @A$ and 
		      it is equivalent to $\G,\D\SB \V (\Pi x:\tau.K)\SB \iskind @A$.
		      \vspace{3mm}
		      
		\item[] \textit{Case} \TApp{} where \( \G, z:\xi@B, \D \V M\ N:\tau[x\mapsto N]@A \)
		      \begin{align*}
		      	  & \G, z:\xi@B, \D \V M:\Pi(x:\sigma).\tau@A &   & \G, z:\xi@B, \D \V N:\sigma@A 
		      \end{align*}
		      We can assume $x \neq z$ because we can select fresh $x$ when we construct $\Pi$ type.
		      By the induction hypothesis,
		      \begin{align*}
		      	  & \G, \D\SB \V M\SB: (\Pi(x:\sigma).\tau)\SB@A \\
		      	  & \G,\D\SB\V N\SB: \sigma\SB@A.                
		      \end{align*}
		      and \(\G, \D\SB \V M\SB: (\Pi(x:\sigma).\tau)\SB@A\) is equivalent to \( \G, \D\SB \V M\SB: (\Pi(x:\sigma\SB).\tau\SB)@A \).
		      From \TApp, \(\G,\D\SB (M\SB\ N\SB): \tau\SB[x \mapsto N\SB]@A\) and this is equivalent to
		      \(\G,\D\SB \V (M\ N)\SB: \tau[x \mapsto N]\SB@A\).
	\end{itemize}
\end{proof}

% \iffullversion
% \begin{itemize}
				
% 	\newcommand{\SB}{[z \mapsto P]}
% 	\newcommand{\GG}{\G}
% 	\newcommand{\GGV}{\G \V}
				
				
% 	\item \WStar
	      	      	      	      
% 	      From the definition of \WStar, we can get $\mathcal{D}_1$.
	      	      	      	      
% 	      $\mathcal{D}_1$ = \infer[\WStar]
% 	      {\GGV * \iskind @A}
% 	      {}
	      	      	      	      
% 	\item \WAbs
	      	      	      	      
% 	      We can assume $x \neq z$ because we can select fresh $x$ when we construct $\Pi$ type.
	      	      	      	      
% 	      We have two derivation trees from the premise.
	      	      	      	      
% 	      $\mathcal{D}_1$ = \infer[\WAbs]
% 	      {\G, z:\xi@B, \D \V (\Pi x:\tau.K) \iskind @A}
% 	      {\G, z:\xi@B, \D \V T::*@A \andalso \G, z:\xi@B, \D, x:\tau \V K \iskind@A}
	      	      	      	      
% 	      $\mathcal{D}_2$ = \infer[]
% 	      {\G \V P:\xi@B}
% 	      {\vdots}\
	      	      	      	      
% 	      We can get $\mathcal{D}_3$ by use the induction hypothesis to $\mathcal{D}_1$
	      	      	      	      
% 	      $\mathcal{D}_3$ = \infer[\WAbs]
% 	      {\GGV (\Pi x:\tau\SB.K\SB) \iskind @A}
% 	      {
% 	      	\infer[]{\GGV T\SB::*@A}{\vdots} \andalso
% 	      	\infer[]{\GG, x:\tau\SB \V K\SB \iskind@}{\vdots}
% 	      	}\\
	      	      	      	      
% 	      Following relationship is obvious.\\
% 	      $\GGV (\Pi x:\tau\SB.K\SB) \iskind @A$\\
% 	      is equivalent with\\
% 	      $\GGV (\Pi x:\tau.K)\SB \iskind @A$.\\
	      	      	      	      
% 	      Then, we can get $\mathcal{D}'_3$ from $\mathcal{D}_3$.
	      	      	      	      
% 	      $\mathcal{D}'_3$ = \infer[\WAbs]
% 	      {\GGV (\Pi x:\tau.K)\SB \iskind @A}
% 	      {
% 	      	\infer[]{\GGV T\SB::*@A}{\vdots} \andalso
% 	      	\infer[]{\GG, x:\tau\SB \V K\SB \iskind@}{\vdots}
% 	      	}\\
% 	      \AI{$\mathcal{D}'_3$ is the same as $\mathcal{D}_3$ because substitution is a meta-level operation.}
	      	      	      	              
% 	      \AI{Your proofs are good but people prefer to treat judgments
% 	      	as if they are English sentences and treat derivations
% 	      implicitly (even when the proof is by induction on derivations.  So, I would write these cases as follows.}
% 	\item[] Case \WStar{} where $K = *$.  The conclusion is immediate since $K[z \mapsto P] = *$.
	      	      	      	      
% 	\item[] Case \WAbs{} where $K = \Pi x:\tau.K_0$ and 
% 	      \begin{align*}
% 	      	  & \G, z:\xi@B, \D \V T::*@A &   & \G, z:\xi@B, \D, x:\tau \V K \iskind@A. 
% 	      \end{align*}
	      	      	      	      
% 	      %	We can assume $x \neq z$ because we can select fresh $x$ when we construct $\Pi$ type.
	      	      	      	      
% 	      By the induction hypothesis, we have
% 	      \begin{align*}
% 	      	  & \GGV T\SB::*@A &   & \GG, x:\tau\SB \V K\SB \iskind@ 
% 	      \end{align*}
% 	      from which $\GGV (\Pi x:\tau\SB.K\SB) \iskind @A$ follows by \WAbs.
% 	      We have $\Pi x:\tau\SB.K\SB = (\Pi x:\tau.K)\SB$ by definition.
	      	      	      	              
% 	\item \WCsp
	      	      	      	      
% 	      From the induction hypothesis, we can get $\mathcal{D}_1$.
	      	      	      	      
% 	      $\mathcal{D}_1$ = \infer[]
% 	      {\GGV K\SB \iskind @A}
% 	      {\vdots}
	      	      	      	      
% 	      Use \WCsp,
	      	      	      	      
% 	      $\mathcal{D}_2$ = \infer[\WCsp]
% 	      {\GGV K\SB \iskind @A\alpha}
% 	      {\mathcal{D}_1}
	      	      	      	      
% 	\item \WTW
	      	      	      	      
% 	      From the induction hypothesis, we can get $\mathcal{D}_1$.
	      	      	      	      
% 	      $\mathcal{D}_1$ = \infer[]
% 	      {\GGV K\SB \iskind @A\alpha}
% 	      {\vdots}
	      	      	      	      
% 	      Use \WTW,
	      	      	      	      
% 	      $\mathcal{D}_2$ = \infer[\WTW]
% 	      {\GGV K\SB \iskind @A\alpha}
% 	      {\mathcal{D}_1}
	      	      	      	      
% 	\item \KVar
	      	      	      	      
% 	      From the induction hypothesis, we can get $\mathcal{D}_1$.
	      	      	      	      
% 	      $\mathcal{D}_1$ = \infer[]
% 	      {\GGV K\SB \iskind @A\alpha}
% 	      {\vdots}
	      	      	      	      
% 	      And we can easily show that $X::K\SB@A \in \GG$.
	      	      	      	      
% 	      Then we can use \KVar\ and get $\mathcal{D}_2$.
	      	      	      	      
% 	      $\mathcal{D}_2$ = \infer[\KVar]
% 	      {\GGV X :: K\SB @A}
% 	      {X::K\SB@A \in \GG \andalso \mathcal{D}_1}
	      	      	      	      
% 	\item \KAbs
	      	      	      	      
% 	      From the induction hypothesis, we can get $\mathcal{D}_1$ and $\mathcal{D}_2$.
	      	      	      	      
% 	      $\mathcal{D}_1$ = \infer[]
% 	      {\GGV \tau\SB :: * @ A}
% 	      {\vdots}
	      	      	      	      
% 	      $\mathcal{D}_2$ = \infer[]
% 	      {\GG, x:\tau\SB@A\V \sigma\SB::J\SB@A}
% 	      {\vdots}
	      	      	      	      
% 	      Use \KAbs,
	      	      	      	      
% 	      $\mathcal{D}_3$ = \infer[\KAbs]
% 	      {\GGV (\Pi x:\tau\SB.\sigma\SB)::(\Pi x:\tau\SB.J\SB)@A}
% 	      {\mathcal{D}_1 \andalso \mathcal{D}_2}
	      	      	      	      
% 	      We can arrange the substitution.
	      	      	      	      
% 	      $\mathcal{D}'_3$ = \infer[\KAbs]
% 	      {\GGV (\Pi x:\tau.\sigma)\SB::(\Pi x:\tau.J)\SB@A}
% 	      {\mathcal{D}_1 \andalso \mathcal{D}_2}
	      	      	      	      
% 	\item \KApp
	      	      	      	      
% 	      Because the last rule is \KApp, we have a derivation tree $\mathcal{D}_1$.
	      	      	      	      
% 	      $\mathcal{D}_1$ = \infer[\KApp]
% 	      {\G, z:\xi@B, \D\V \sigma\ M :: K[x \mapsto M]}
% 	      {\infer[]
% 	      	{\G, z:\xi@B, \D\V \sigma::(\Pi x:\tau.K)@A}{\vdots} \andalso
% 	      	\infer[]
% 	      	{\G, z:\xi@B, \D\V M:\tau@A}
% 	      	{\vdots}}
	      	      	      	      
% 	      From the induction hypothesis, we can get $\mathcal{D}_2$ and $\mathcal{D}_3$.
	      	      	      	      
% 	      $\mathcal{D}_2$ = \infer[]
% 	      {\GGV \sigma\SB::(\Pi x:\tau.K)\SB@A}
% 	      {\vdots}
	      	      	      	      
% 	      Because $x \neq z$, we can write
	      	      	      	      
% 	      $\mathcal{D}'_2$ = \infer[]
% 	      {\GGV \sigma\SB::(\Pi x:\tau\SB.K\SB)@A}
% 	      {\vdots}
	      	      	      	      
% 	      $\mathcal{D}_3$ = \infer[]
% 	      {\GGV M\SB:\tau\SB@A}
% 	      {\vdots}
	      	      	      	      
% 	      Use \KApp,
	      	      	      	      
% 	      $\mathcal{D}_4$ = \infer[\KApp]
% 	      {\GGV (\sigma\SB\ M\SB)::K\SB[x \mapsto M]@A}
% 	      {\mathcal{D}'_2 \andalso \mathcal{D}_3}
	      	      	      	      
% 	      There is no $x$ in $P$ and no $z$ in $M$ because of freshness. So we can rewrite the $\mathcal{D}_4$.
	      	      	      	      
% 	      $\mathcal{D}_4$ = \infer[\KApp]
% 	      {\GGV (\sigma\ M)\SB::K[x \mapsto M]\SB@A}
% 	      {\mathcal{D}'_2 \andalso \mathcal{D}_3}
	      	      	      	      
% 	\item \KConv
	      	      	      	      
% 	      From the induction hypothesis, we have $\mathcal{D}_1$ and $\mathcal{D}_2$.
	      	      	      	      
% 	      $\mathcal{D}_1$ = \infer[]
% 	      {\GGV \tau\SB : K[z\mapsto P]@A}
% 	      {\vdots}
	      	      	      	      
% 	      $\mathcal{D}_2$ = \infer[]
% 	      {\GGV K\SB \E J\SB @A}
% 	      {\vdots}
	      	      	      	      
% 	      Use \KConv,
	      	      	      	      
% 	      $\mathcal{D}_2$ = \infer[\KConv]
% 	      {\GGV \tau\SB : J[z\mapsto P]@A}
% 	      {\mathcal{D}_1 \andalso \mathcal{D}_2}
	      	      	      	      
% 	\item \KTW
	      	      	      	      
% 	      From the induction hypothesis, we have $\mathcal{D}_1$.
	      	      	      	      
% 	      $\mathcal{D}_1$ = \infer[]
% 	      {\GGV \tau\SB :: K\SB @ A\alpha}
% 	      {\vdots}
	      	      	      	      
% 	      Use \KTW,
	      	      	      	      
% 	      $\mathcal{D}_2$ = \infer[]
% 	      {\GGV \TW_\alpha \tau\SB :: K\SB @ A}
% 	      {\mathcal{D}_1}
	      	      	      	      
% 	\item \KTWL
	      	      	      	      
% 	      From the induction hypothesis, we have $\mathcal{D}_1$.
	      	      	      	      
% 	      $\mathcal{D}_1$ = \infer[]
% 	      {\GGV \TW_\alpha \tau\SB :: K\SB @ A\alpha}
% 	      {\vdots}
	      	      	      	      
% 	      Use \KTWL,
	      	      	      	      
% 	      $\mathcal{D}_2$ = \infer[]
% 	      {\GGV \tau\SB :: K\SB @ A\alpha}
% 	      {\mathcal{D}_1}
	      	      	      	      
% 	\item \KGen
	      	      	      	      
% 	      From the induction hypothesis, we have $\mathcal{D}_1$.
	      	      	      	      
% 	      $\mathcal{D}_1$ = \infer[]
% 	      {\GGV \tau\SB :: K\SB @ A}
% 	      {\vdots}
	      	      	      	      
% 	      And we can prove easily $\alpha \notin \FTV(\GG) \cup \FTV(A)$.
	      	      	      	      
% 	      Use \KGen,
	      	      	      	      
% 	      $\mathcal{D}_2$ = \infer[\KGen]
% 	      {\GGV \forall\alpha.\tau\SB :: K\SB @ A}
% 	      {\mathcal{D}_1 \andalso \alpha \notin \FTV(\GG) \cup \FTV(A)}
	      	      	      	      
% 	\item \KCsp
	      	      	      	      
% 	      From the induction hypothesis, we have $\mathcal{D}_1$.
	      	      	      	      
% 	      $\mathcal{D}_1$ = \infer[]
% 	      {\GGV \tau\SB :: K\SB @ A}
% 	      {\vdots}
	      	      	      	      
% 	      Use \KCsp,
	      	      	      	      
% 	      $\mathcal{D}_2$ = \infer[\KCsp]
% 	      {\GGV \tau\SB :: K\SB @ A\alpha}
% 	      {\GGV \tau\SB :: K\SB @ A}
	      	      	      	      
% 	      \fi
	      	      	      	      
% 	\item \TVar
	      	      	      	      
% 	      We have two derivation trees from the premise.
	      	      	      	      
% 	      $\mathcal{D}_1$ = \infer[\TVar]
% 	      {\G, z:\xi@B \V y:\tau@A}
% 	      {y:\tau@A \in \G, z:\xi@B  \andalso \infer[]{\G, z:\xi@B \V \tau::*@A}{\vdots}}
	      	      	      	      
% 	      $\mathcal{D}_2$ = \infer[]
% 	      {\G \V P:\xi@B}
% 	      {\vdots}\\
	      	      	      	      
% 	      We can get $\mathcal{D}_3$ by use the induction hypothesis to $\mathcal{D}_1$.
	      	      	      	      
% 	      $\mathcal{D}_3$ = \infer[]
% 	      {\GGV \tau\SB::*\SB@A}
% 	      {\vdots}\\
	      	      	      	      
% 	      \begin{itemize}
% 	      	\item $y:\tau@A \in \G$ or $y:\tau@A \in \D$
	      	      	      	      	      	      	      	      
% 	      	      $\mathcal{D}_4$ is obvious.
	      	      	      	      	      	      	      	      
% 	      	      $\mathcal{D}_4$ = $y:\tau\SB@A \in \GG$
	      	      	      	      	      	      	      	      
% 	      	      Get $\mathcal{D}_5$ by using \TVar\ for $\mathcal{D}_3$, $\mathcal{D}_4$.
	      	      	      	      	      	      	      	      
% 	      	      $\mathcal{D}_5$ = \infer[]
% 	      	      {\GGV y\SB:\tau\SB@A}
% 	      	      {\mathcal{D}_4 \andalso \mathcal{D}_5}
	      	      	      	      	      	      	      	      
% 	      	\item $y:\tau@A = z:\xi@B$
	      	      	      	      	      	      	      	      
% 	      	      In this case,
% 	      	      \begin{itemize}
% 	      	      	\item $y = z$ \AI{Needs a comma.}
% 	      	      	\item $\tau = \xi$ \AI{Needs ``, and''.}
% 	      	      	\item $A = B$ \AI{Needs a period.  In general, you have to be able read as if it's a usual sentence.  Usual grammar rules apply.}
% 	      	      \end{itemize}
	      	      	      	      	      	      	      	                            
	      	      	      	      	      	      	      	      
% 	      	      Because there is no $z$ in $\xi$, $\tau\SB = \xi\SB = \xi$.
% 	      	      And it is obvious that $y\SB = z\SB = P$.
	      	      	      	      	      	      	      	      
% 	      	      From $\mathcal{D}_2$, $\G \V y\SB : \tau\SB@A$. Use Weakening lemma, \AI{By Weakening,} $\GGV y\SB : \tau\SB$.
% 	      \end{itemize}
	      	      	      	      
% 	      From and \TVar\\
% 	      $\GGV y\SB:\tau\SB$
	      	      	      	      
% 	      \iffullversion
	      	      	      	      
% 	\item \TAbs
	      	      	      	      
% 	      From the induction hypothesis and \TAbs, we get
	      	      	      	      
% 	      $\mathcal{D}_1$ = \infer[\TAbs]
% 	      {\GGV (\lambda y:\sigma\SB.M\SB):(\Pi y:\sigma\SB.\tau\SB)@A}
% 	      {\infer[]{\GGV \sigma\SB::*@A}{\vdots} \andalso \infer[]{\GG, y:\sigma\SB@A \V M\SB:\tau\SB@A}{\vdots}}
	      	      	      	      
% 	      Arrange substitutions,
	      	      	      	      
% 	      $\mathcal{D}'_1$ = \infer[\TAbs]
% 	      {\GGV (\lambda y:\sigma.M)\SB:(\Pi y:\sigma.\tau)\SB@A}
% 	      {\infer[]{\GGV \sigma\SB::*@A}{\vdots} \andalso \infer[]{\GG, y:\sigma\SB@A \V M\SB:\tau\SB@A}{\vdots}}
	      	      	      	      
% 	\item \TApp
	      	      	      	      
% 	      We have a derivation trees from the premise.
	      	      	      	      
% 	      $\mathcal{D}_1$ = \infer[\TApp]
% 	      {\G, z:\xi@B, \D \V M\ L:\tau[y\mapsto L]@A}
% 	      {\infer[]{\G, z:\xi@B, \D \V M:\Pi(y:\rho).\tau@A \andalso \G, z:\xi@B, \D \V L:\rho@A}{\vdots}}
	      	      	      	      
% 	      We get 2 trees from the induction hypothesis and $\mathcal{D}_1$.
	      	      	      	      
% 	      $\mathcal{D}_2$ = \infer[]
% 	      {\GGV M\SB: (\Pi(y:\rho).\tau)\SB@A}
% 	      {\vdots}
	      	      	      	      
% 	      and
	      	      	      	      
% 	      $\mathcal{D}_3$ = \infer[]
% 	      {\GGV L\SB: \rho\SB@A}
% 	      {\vdots}
	      	      	      	      
% 	      \red{distribute the substitution} in $\mathcal{D}_2$
	      	      	      	      
% 	      $\mathcal{D'}_2$ = \infer[]
% 	      {\GGV M\SB: (\Pi(y:\rho\SB).\tau\SB)@A}
% 	      {\vdots}
	      	      	      	      
% 	      From $\mathcal{D'}_2$ and $\mathcal{D}_3$
	      	      	      	      
% 	      $\mathcal{D}_4$ = \infer[\TApp]
% 	      {\GGV (M\SB\ L\SB): \tau\SB[y \mapsto L]@A}
% 	      {\mathcal{D'}_2 \andalso \mathcal{D}_3}
	      	      	      	      
% 	      \red{We can transform the conclusion of $\mathcal{D}_4$ into} \\
% 	      $\GGV (M\ L)\SB: \tau[y \mapsto L]\SB@A$
	      	      	      	      
% 	\item \TConv
	      	      	      	      
% 	      We have 2 derivation trees from the premise and the induction hypothesis.
	      	      	      	      
% 	      $\mathcal{D}_1$ = \infer[]
% 	      {\GGV t\SB:T\SB@A}
% 	      {\vdots}
	      	      	      	      
% 	      $\mathcal{D}_2$ = \infer[]
% 	      {\GGV T\SB \E T'\SB@A}
% 	      {\vdots}
	      	      	      	      
% 	      And then \\
% 	      \infer[\TConv]
% 	      {\GGV t\SB:T'\SB@A}
% 	      {\mathcal{D}_1 \andalso \mathcal{D}_2}
	      	      	      	      
% 	\item \TTB
	      	      	      	      
% 	      From the induction hypothesis and \TTB, we get
	      	      	      	      
% 	      $\mathcal{D}_3$ = \infer[\TTB]
% 	      {\GGV \TB_\alpha M\SB:\tau\SB@A}
% 	      {\infer[]{\GGV M\SB:\tau\SB@A\alpha}{\vdots}}
	      	      	      	      
% 	\item \TTBL
	      	      	      	      
% 	      From the induction hypothesis and \TTBL, we get
	      	      	      	      
% 	      $\mathcal{D}_3$ = \infer[\TTBL]
% 	      {\GGV \TBL_\alpha M\SB:\tau\SB@A}
% 	      {
% 	      	\infer[]{\GGV M\SB: \TW_\alpha \tau\SB@A\alpha}{\vdots}
% 	      }
	      	      	      	      
% 	\item \TGen
	      	      	      	      
% 	      From the induction hypothesis, we get $\mathcal{D}_1$.
	      	      	      	      
% 	      $\mathcal{D}_1$ = \infer[]
% 	      {\GGV M\SB:\tau\SB@A}
% 	      {\vdots}
	      	      	      	      
% 	      And we can prove easily $\alpha \notin \FTV(\GG) \cup \FTV(A)$.
	      	      	      	      
% 	      Use \TGen,
	      	      	      	      
% 	      $\mathcal{D}_2$ = \infer[\TGen]
% 	      {\GGV (\Lambda\alpha.M)\SB:(\forall\alpha.\tau)\SB@A}
% 	      {\mathcal{D}_1 \andalso \alpha \notin \FTV(\GG) \cup \FTV(A)}
	      	      	      	      
% 	\item \TIns
	      	      	      	      
% 	      From the induction hypothesis, we get $\mathcal{D}_1$.
	      	      	      	      
% 	      $\mathcal{D}_1$ = \infer[]
% 	      {\GGV M\SB:(\forall\alpha.\tau)\SB@A}
% 	      {\vdots}
	      	      	      	      
% 	      Use \TIns,
	      	      	      	      
% 	      $\mathcal{D}_2$ = \infer[\TIns]
% 	      {\GGV (M\ \varepsilon)\SB:\tau\SB@A}
% 	      {\mathcal{D}_1}
	      	      	      	      
% 	\item \TCsp
	      	      	      	      
% 	      From the induction hypothesis, we get $\mathcal{D}_1$.
	      	      	      	      
% 	      $\mathcal{D}_1$ = \infer[]
% 	      {\GGV M\SB:\tau\SB@A}
% 	      {\vdots}
	      	      	      	      
% 	      Use \TCsp,
	      	      	      	      
% 	      $\mathcal{D}_2$ = \infer[\TCsp]
% 	      {\GGV (\%_\alpha M)\SB:\tau\SB@A\alpha}
% 	      {\mathcal{D}_1}
	      	      	      	      
% 	\item \QKAbs
	      	      	      	      
% 	      From the induction hypothesis and \QKAbs, we get $\mathcal{D}_1$.
	      	      	      	      
% 	      $\mathcal{D}_1$ = \infer[\QKAbs]
% 	      {\GGV (\Pi x:\tau.K)\SB \E (\Pi x:\sigma.J)\SB@A}
% 	      {\infer[]{\GGV \tau\SB \E \sigma\SB :: *@A}{\vdots} \andalso
% 	      	\infer[]{\GG,x:\tau@A \V K\SB\E J\SB@A}{\vdots} }
	      	      	      	      
% 	\item \QKCsp
	      	      	      	      
% 	      From the induction hypothesis and \QKCsp, we get $\mathcal{D}_1$.
	      	      	      	      
% 	      $\mathcal{D}_1$ = \infer[\QKCsp]
% 	      {\GGV K\SB \E J\SB @A\alpha}
% 	      {\infer[]{\GGV K\SB \E J\SB @A}{\vdots}}
	      	      	      	      
% 	\item \QKRefl
	      	      	      	      
% 	      From the induction hypothesis and \QKRefl, we get $\mathcal{D}_1$.
	      	      	      	      
% 	      $\mathcal{D}_1$ = \infer[\QKRefl]
% 	      {\GGV K\SB \E K\SB @A}
% 	      {\infer[]{\GGV K\SB \iskind @A}{\vdots}}
	      	      	      	      
% 	\item \QKSym
	      	      	      	      
% 	      From the induction hypothesis and \QKSym, we get $\mathcal{D}_1$.
	      	      	      	      
% 	      $\mathcal{D}_1$ = \infer[\QKSym]
% 	      {\GGV J\SB\E K\SB@A}
% 	      {\GGV K\SB\E J\SB@A}
	      	      	      	      
% 	\item \QKTrans
	      	      	      	      
% 	      From the induction hypothesis and \QKTrans, we get $\mathcal{D}_1$.
	      	      	      	      
% 	      $\mathcal{D}_1$ = \infer[\QKTrans]
% 	      {\GGV K\SB\E I\SB@A}
% 	      {\GGV K\SB\E J\SB@A \andalso \GGV J\SB\E I\SB@A}
	      	      	      	      
% 	\item \QTAbs
	      	      	      	      
% 	      From the induction hypothesis and, we get $\mathcal{D}_1$.
	      	      	      	      
% 	      $\mathcal{D}_1$ = \infer[\QTAbs]
% 	      {\GG \Pi x:\tau\SB.\rho\SB \E \Pi x:\sigma\SB.\pi\SB@A}
% 	      {\ID{\GG \tau\SB \E \sigma\SB@A} \andalso \ID{\G, \D, x:\tau \V \rho\SB \E \pi\SB}}
	      	      	      	      
% 	      Arrange substitutions,
	      	      	      	      
% 	      $\mathcal{D}'_1$ = \infer[\QTAbs]
% 	      {\GG (\Pi x:\tau.\rho)\SB \E (\Pi x:\sigma.\pi)\SB@A}
% 	      {\ID{\GG \tau\SB \E \sigma\SB@A} \andalso \ID{\G, \D, x:\tau \V \rho\SB \E \pi\SB}}
	      	      	      	      
% 	\item \QTApp
	      	      	      	      
% 	      From the induction hypothesis and \QTApp, we get $\mathcal{D}_1$.
	      	      	      	      
% 	      $\mathcal{D}_1$ = \infer[\QTApp]
% 	      {\GGV\tau\SB\ M\SB \E \sigma\SB\ N\SB@A}
% 	      {\ID{\GGV\tau\SB\E\sigma\SB :: (\Pi x:\rho.K)@A} \andalso \ID{\GGV M\SB\E N\SB:\rho@A}}
	      	      	      	      
% 	      Arrange substitutions,
	      	      	      	      
% 	      $\mathcal{D}'_1$ = \infer[\QTApp]
% 	      {\GGV(\pi\ M)\SB \E (\sigma\ N)\SB@A}
% 	      {\ID{\GGV\tau\SB\E\sigma\SB :: (\Pi x:\rho.K)@A} \andalso \ID{\GGV M\SB\E N\SB:\rho@A}}
	      	      	      	      
% 	\item \QTTW
	      	      	      	      
% 	      From the induction hypothesis and \QTTW, we get $\mathcal{D}_1$.
	      	      	      	      
% 	      $\mathcal{D}_1$ = \infer[\QTTW]
% 	      {\GGV(\TW_\alpha \tau)\SB\E(\TW_\alpha\sigma)\SB@A}
% 	      {\ID{\GGV\tau\SB\E\sigma\SB@A\alpha}}
	      	      	      	      
% 	\item \QTGen
	      	      	      	      
% 	      We can prove easily $\alpha \notin \FTV(\GG) \cup \FTV(A)$.
% 	      From the induction hypothesis and \QTGen, we get $\mathcal{D}_1$.
	      	      	      	      
% 	      $\mathcal{D}_1$ = \infer[\QTGen]
% 	      {\GGV (\forall\alpha.\tau)\SB \E (\forall\alpha.\sigma)\SB@A}
% 	      {\ID{\GGV \tau\SB \E \sigma\SB@A} \andalso \alpha \notin \FTV(\GG) \cup \FTV(A)}
	      	      	      	      
% 	\item \QTCsp
	      	      	      	      
% 	      From the induction hypothesis and \QTCsp, we get $\mathcal{D}_1$.
	      	      	      	      
% 	      $\mathcal{D}_1$ = \infer[\QTCsp]
% 	      {\GGV\tau\SB \E \sigma\SB@A\alpha}
% 	      {\ID{\GGV\tau\SB \E \sigma\SB@A}}
	      	      	      	      
% 	\item \QTRefl
	      	      	      	      
% 	      From the induction hypothesis and \QTRefl, we get $\mathcal{D}_1$.
	      	      	      	      
% 	      $\mathcal{D}_1$ = \infer[\QTRefl]
% 	      {\GGV\tau\SB\E\tau\SB@A}
% 	      {\ID{\GGV\tau\SB::K\SB@A}}
	      	      	      	      
% 	\item \QTSym
	      	      	      	      
% 	      From the induction hypothesis and \QTSym, we get $\mathcal{D}_1$.
	      	      	      	      
% 	      $\mathcal{D}_1$ = \infer[\QTSym]
% 	      {\GGV\sigma\SB\E\tau\SB@A}
% 	      {\ID{\GGV\tau\SB\E\sigma\SB@A}}
	      	      	      	      
% 	\item \QTTrans
	      	      	      	      
% 	      From the induction hypothesis and \QTTrans, we get $\mathcal{D}_1$.
	      	      	      	      
% 	      $\mathcal{D}_1$ = \infer[\QTTrans]
% 	      {\GGV \tau\SB\E\rho\SB@A}
% 	      {\ID{\GGV\tau\SB\E\sigma\SB@A} \andalso \ID{\GGV\sigma\SB\E\rho\SB@A}}
	      	      	      	      
% 	\item \QAbs
	      	      	      	      
% 	      From the induction hypothesis and \QAbs, we get $\mathcal{D}_1$.
	      	      	      	      
% 	      $\mathcal{D}_1$ = \infer[\QAbs]
% 	      {\GGV \Pi x:\tau\SB.\rho\SB \E \Pi x:\sigma\SB.\pi\SB@A}
% 	      {\ID{\GGV\tau\SB \E \sigma\SB :: * @A} \andalso \ID{\GG,x:\tau\SB@A\V\rho\SB \E \pi\SB@A}}
	      	      	      	      
% 	      Arrange substitutions,
	      	      	      	      
% 	      $\mathcal{D}'_1$ = \infer[\QAbs]
% 	      {\GGV (\Pi x:\tau.\rho)\SB \E (\Pi x:\sigma.\pi)\SB@A}
% 	      {\ID{\GGV\tau\SB \E \sigma\SB :: * @A} \andalso \ID{\GG,x:\tau\SB@A\V\rho\SB \E \pi\SB@A}}
	      	      	      	      
% 	\item \QApp
	      	      	      	      
% 	      From the induction hypothesis
	      	      	      	      
% 	      $\mathcal{D}_1$ = \ID{\GGV M\SB \E L\SB :: (\Pi x:\sigma.\tau)\SB@A}
	      	      	      	      
% 	      Arrange substitutions,
	      	      	      	      
% 	      $\mathcal{D}'_1$ = \ID{\GGV M\SB \E L\SB :: (\Pi x:\sigma\SB.\tau\SB)@A}
	      	      	      	      
% 	      Using \QApp to $\mathcal{D}'_1$ and the induction hypothesis, we get $\mathcal{D}_2$.
	      	      	      	      
% 	      $\mathcal{D}_2$ = \infer[\QApp]
% 	      {\GGV M\SB\ N\SB \E L\SB\ O\SB @A}
% 	      {\mathcal{D}'_1 \andalso \ID{\GGV N\SB \E O\SB : \sigma\SB @A}}
	      	      	      	      
% 	      Arrange substitutions,
	      	      	      	      
% 	      $\mathcal{D}'_2$ = \infer[\QApp]
% 	      {\GGV (M\ N)\SB \E (L\ O)\SB @A}
% 	      {\mathcal{D}'_1 \andalso \ID{\GGV N\SB \E O\SB : \sigma\SB @A}}
	      	      	      	      
% 	\item \QTB
	      	      	      	      
% 	      From the induction hypothesis and \QTB, we get $\mathcal{D}_1$.
	      	      	      	      
% 	      $\mathcal{D}_1$ = \infer[\QTB]
% 	      {\GGV\TB_\alpha M\SB \E \TB_\alpha N\SB @A}
% 	      {\GGV M\SB \E N\SB @A\alpha}
	      	      	      	      
% 	\item \QTBL
	      	      	      	      
% 	      From the induction hypothesis and \QTBL, we get $\mathcal{D}_1$.
	      	      	      	      
% 	      $\mathcal{D}_1$ = \infer[\QTBL]
% 	      {\GGV\TBL_\alpha M\SB \E \TBL_\alpha N\SB@A\alpha}
% 	      {\GGV M\SB \E N\SB : \TW_\alpha \tau@A}
	      	      	      	      
% 	\item \QGen
	      	      	      	      
% 	      We can prove easily $\alpha \notin \FTV(\GG) \cup \FTV(A)$.
% 	      From the induction hypothesis and \QGen, we get $\mathcal{D}_1$.
	      	      	      	      
% 	      $\mathcal{D}_1$ = \infer[\QGen]
% 	      {\GGV\Lambda\alpha.M\SB \E \Lambda\alpha.N\SB@A}
% 	      {\ID{\GGV M\SB \E N\SB @A} \andalso \alpha \notin \FTV(\GG) \cup \FTV(A)}
	      	      	      	      
% 	\item \QIns
	      	      	      	      
% 	      From the induction hypothesis and \QIns, we get $\mathcal{D}_1$.
	      	      	      	      
% 	      $\mathcal{D}_1$ = \infer[\QIns]
% 	      {\GGV M\SB \E N\SB : \TW_\alpha.\tau @A}
% 	      {\ID{\GGV M\SB\ \varepsilon \E N\SB\ \varepsilon @A }}
	      	      	      	      
% 	\item \QCsp
	      	      	      	      
% 	      From the induction hypothesis and \QCsp, we get $\mathcal{D}_1$.
	      	      	      	      
% 	      $\mathcal{D}_1$ = \infer[\QCsp]
% 	      {\GGV \%_\alpha M\SB \E \%_\alpha N\SB @A}
% 	      {\ID{\GGV M\SB \E N\SB @A\alpha}}
	      	      	      	      
% 	\item \QRefl
	      	      	      	      
% 	      From the induction hypothesis and \QRefl, we get $\mathcal{D}_1$.
	      	      	      	      
% 	      $\mathcal{D}_1$ = \infer[\QRefl]
% 	      {\GGV M\SB \E M\SB @A}
% 	      {\ID{\GGV M\SB : \tau\SB @A}}
	      	      	      	      
% 	\item \QSym
	      	      	      	      
% 	      From the induction hypothesis and \QSym, we get $\mathcal{D}_1$.
	      	      	      	      
% 	      $\mathcal{D}_1$ = \infer[\QSym]
% 	      {\GGV N\SB \E M\SB @A}
% 	      {\ID{\GGV M\SB \E N\SB @A}}
	      	      	      	      
% 	\item \QTrans
	      	      	      	      
% 	      From the induction hypothesis and \QTrans, we get $\mathcal{D}_1$.
	      	      	      	      
% 	      $\mathcal{D}_1$ = \infer[\QTrans]
% 	      {\GGV M\SB \E L\SB @A}
% 	      {\ID{\GGV M\SB \E N\SB @A } \andalso \ID{\GGV N\SB \E L\SB @A}}
	      	      	      	      
% 	\item \QBeta
	      	      	      	      
% 	      From the induction hypothesis and \QBeta, we get $\mathcal{D}_1$.
	      	      	      	      
% 	      $\mathcal{D}_1$ = \infer[\QBeta]
% 	      {\GGV (\lambda x:\sigma\SB:M\SB)\ N\SB \E (M\SB)[x \mapsto N\SB]@A}
% 	      {\ID{\GG, x: \sigma\SB@A \V M\SB:\tau\SB@A} \andalso \ID{\GGV N\SB:\sigma\SB @A }}
	      	      	      	      
% 	      Arrange substitutions,
	      	      	      	      
% 	      $\mathcal{D}'_1$ = \infer[\QBeta]
% 	      {\GGV ((\lambda x:\sigma:M)\ N)\SB \E (M[x \mapsto N])\SB@A}
% 	      {\ID{\GG, x: \sigma\SB@A \V M\SB:\tau\SB@A} \andalso \ID{\GGV N\SB:\sigma\SB @A }}
	      	      	      	      
% 	\item \QEta
	      	      	      	      
% 	      From the induction hypothesis and \QEta, we get $\mathcal{D}_1$.
	      	      	      	      
% 	      $\mathcal{D}_1$ = \infer[\QEta]
% 	      {\GGV (\lambda x:\sigma\SB.M\SB\ x) \E M\SB@A}
% 	      {\ID{\GGV M\SB : (\Pi x:\sigma\SB.\tau\SB)@A} \andalso x \notin \FV(M\SB)}
	      	      	      	      
% 	      Arrange substitutions,
	      	      	      	      
% 	      $\mathcal{D}'_2$ = \infer[\QEta]
% 	      {\GGV (\lambda x:\sigma.M\ x)\SB \E M\SB@A}
% 	      {\ID{\GGV M\SB : (\Pi x:\sigma\SB.\tau\SB)\SB@A} \andalso x \notin \FV(M\SB)}
	      	      	      	      
% 	\item \QTBLTB
	      	      	      	      
% 	      From the induction hypothesis and \QTBLTB, we get $\mathcal{D}_1$.
	      	      	      	      
% 	      $\mathcal{D}_1$ = \infer[\QTBLTB]
% 	      {\GGV \TBL_\alpha \TB_\alpha M\SB \E N\SB@A}
% 	      {\ID{\GGV M\SB \E N\SB @A}}
	      	      	      	      
% 	\item \QLambda
	      	      	      	      
% 	      From the induction hypothesis and \QLambda, we get $\mathcal{D}_1$.
	      	      	      	      
% 	      $\mathcal{D}_1$ = \infer[\QLambda]
% 	      {\GGV (\Lambda\alpha.M\SB)\ \varepsilon \E M\SB[\alpha \mapsto \varepsilon]}
% 	      {\ID{\GGV (\Lambda\alpha.M\SB) : \forall\alpha.\tau\SB @A}}
	      	      	      	      
% 	      Arrange substitutions,
	      	      	      	      
% 	      $\mathcal{D}'_1$ = \infer[\QLambda]
% 	      {\GGV ((\Lambda\alpha.M)\ \varepsilon)\SB \E M[\alpha \mapsto \varepsilon]\SB}
% 	      {\ID{\GGV (\Lambda\alpha.M\SB) : \forall\alpha.\tau\SB @A}}
	      	      	      	      
	      	      	      	      
% 	\item \QPercent
	      	      	      	      
% 	      From the induction hypothesis and \QPercent, we get $\mathcal{D}_1$.
	      	      	      	      
% 	      $\mathcal{D}_1$ = \infer[\QPercent]
% 	      {\GGV \%_\alpha M\SB \E M\SB @ A\alpha}
% 	      {\ID{\GGV M\SB : \tau\SB @A\alpha} \andalso \ID{\GGV M\SB : \sigma\SB @A} }
	      	      	      	      
% \end{itemize}
% \fi

\begin{lemma}[Stage Substitution]
	If $\G \V \mathcal{J}$, then $\G[\beta\mapsto B] \V \mathcal{J}[\beta\mapsto B]$.  Similarly, if $\G \V \mathcal{J}$, then $\V \G[\beta\mapsto B]$.
\end{lemma}

\begin{proof}
	The six items are proved simultaneously by induction on derivations with case analysis on the last rule used.
	We show only representative cases.
	\begin{itemize}
								
		\newcommand{\SB}{[\beta \mapsto B]}
		\newcommand{\GG}{\G\SB}
		\newcommand{\GGV}{\G\SB \V}
				
		\item \textit{Case} \TGen{} where $\mathcal{J} = \Lambda\alpha.M:\forall\alpha.\tau@A$ and
		      \begin{align*}
		      	  & \G\V M:\tau@A &   & \alpha\notin\rm{FTV}(\G)\cup\rm{FTV}(A). 
		      \end{align*}
		      We can assume $\alpha \notin B$ because $\alpha$ is a bound variable.
		      By the induction hypothesis, we have \(\G\SB\V M\SB:\tau\SB@A\).
		      We can prove easily $\alpha \notin \FTV(\GG) \cup \FTV(A)$.
		      Then, \(\GGV (\Lambda\alpha.M)\SB:(\forall\alpha.\tau)\SB@A\SB\) by \TGen.
		      		      
		\item \textit{Case} \KTW{} where \(\mathcal{J} = \G\V \TW_\alpha \tau :: * @ A \) and \( \G\V \tau :: * @ A\alpha \).
		      \begin{itemize}
		      	\item If $\alpha \neq \beta$,\\
		      	      \( \GGV \tau\SB :: *\SB @ A\alpha\SB \) from the induction hypothesis.
		      	      \( \GGV \TW_\alpha \tau\SB :: *\SB @ A\SB \) from \KTW.
		      	      	      	      	      	      	      	      	      	      	     	      
		      	\item If $\alpha = \beta$, \\
		      	      \( \GGV \tau\SB :: *\SB @ A\alpha\SB \) from the induction hypothesis and
		      	      it is identical with \( \GGV \tau\SB :: * @ AB \).
		      	      We can get \( \GGV \TW_B \tau\SB :: * @ A \) after repeat using \KTW the length of $B$ times.
		      	      
		      \end{itemize}
		      		
		\item \textit{Case} \QGen{} where \(\mathcal{J} = \Lambda\alpha.M \E \Lambda\alpha.N : \forall\alpha.\tau@A\) and 
		      \begin{align*}
		      	  & \G\V M\E N : \tau@A &   & \alpha \notin \FTV(\G)\cup\FTV(A). 
		      \end{align*}
		      From the induction hypothesis, \( \GGV M\SB \E N\SB : \tau\SB@A\SB \).
		      We can assume \(\alpha \notin B \) because \(\alpha\) is a bound variable thus
		      \( \alpha \notin \FTV(\G\SB)\cup\FTV(A\SB) \).
		      \( \GGV \Lambda\alpha.M\SB \E \Lambda\alpha.N\SB : \tau\SB@A\SB \) from \QGen{} and
		      it is identical with \( \GGV (\Lambda\alpha.M)\SB \E (\Lambda\alpha.N)\SB : \tau\SB@A\SB \).
		      		      
	\end{itemize}
\end{proof}

% \iffullversion
% \begin{itemize}
					
% 	\newcommand{\SB}{[\beta \mapsto B]}
% 	\newcommand{\GG}{\G\SB}
% 	\newcommand{\GGV}{\G\SB \V}
					
					
% 	\item \WStar
	      	      	      	      	      
% 	      From the definition of \WStar, we can get $\mathcal{D}_1$.
	      	      	      	      	      
% 	      $\mathcal{D}_1$ = \infer[\WStar]
% 	      {\GGV * \iskind @A\SB}
% 	      {}
	      	      	      	      	      
% 	\item \WAbs
	      	      	      	      	      
% 	      From the induction hypothesis and \WAbs, we get $\mathcal{D}_1$.
	      	      	      	      	      
% 	      $\mathcal{D}_1$ = \infer[\WAbs]
% 	      {\GGV (\Pi x:\tau.K)\SB \iskind @A}
% 	      {
% 	      	\ID{\GGV \tau\SB::*@A\SB} \andalso
% 	      	\ID{\GG, x:\tau\SB \V K\SB \iskind@}
% 	      }
	      	      	      	      	      
% 	\item \WCsp
	      	      	      	      	      
% 	      \begin{itemize}
	      		      		      		      		      	
% 	      	\item $\alpha \neq \beta$
	      	      	      	      	      	      	      	      	      	      
% 	      	      From the induction hypothesis and \WCsp, we can get $\mathcal{D}_1$.
	      	      	      	      	      	      	      	      	      	      
% 	      	      $\mathcal{D}_1$ = \infer[\WCsp]
% 	      	      {\GGV K\SB \iskind @A\alpha\SB}
% 	      	      {\ID{\GGV K\SB \iskind @A\SB}}
	      	      	      	      	      	      	      	      	      	      
% 	      	\item $\alpha = \beta$
	      	      	      	      	      	      	      	      	      	      
% 	      	      The conclusion is identical with the induction hypothesis.
	      	      	      	      	      	      	      	      	      	      
% 	      \end{itemize}
	      	      	      	      	      
% 	\item \WTW
	      	      	      	      	      
% 	      \begin{itemize}
	      		      		      		      		      	
% 	      	\item $\alpha \neq \beta$
	      	      	      	      	      	      	      	      	      	      
% 	      	      From the induction hypothesis and \WTW, we can get $\mathcal{D}_1$.
	      	      	      	      	      	      	      	      	      	      
% 	      	      $\mathcal{D}_1$ = \infer[\WTW]
% 	      	      {\GGV K\SB \iskind @A\SB}
% 	      	      {\ID{\GGV K\SB \iskind @A\alpha\SB}}
	      	      	      	      	      	      	      	      	      	      
% 	      	\item $\alpha = \beta$
	      	      	      	      	      	      	      	      	      	      
% 	      	      The conclusion is identical with the induction hypothesis.
	      	      	      	      	      	      	      	      	      	      
% 	      \end{itemize}
	      	      	      	      	      
% 	\item \KVar
	      	      	      	      	      
% 	      We can easily show that $X::K\SB \in \GG$.
% 	      From the induction hypothesis and \KVar, we can get $\mathcal{D}_1$.
	      	      	      	      	      
% 	      $\mathcal{D}_1$ = \infer[\KVar]
% 	      {\GGV X :: K\SB @A\SB}
% 	      {X::K\SB \in \GG \andalso \ID{\GGV K\SB \iskind @A\SB}}
	      	      	      	      	      
% 	\item \KAbs
	      	      	      	      	      
% 	      From the induction hypothesis, we can get $\mathcal{D}_1$ and $\mathcal{D}_2$.
	      	      	      	      	      
% 	      $\mathcal{D}_1$ = \infer[]
% 	      {\GGV \tau\SB :: * @ A\SB}
% 	      {\vdots}
	      	      	      	      	      
% 	      $\mathcal{D}_2$ = \infer[]
% 	      {\GG, x:\tau\SB@A\V \sigma\SB::J\SB@A\SB}
% 	      {\vdots}
	      	      	      	      	      
% 	      Use \KAbs,
	      	      	      	      	      
% 	      $\mathcal{D}_3$ = \infer[\KAbs]
% 	      {\GGV (\Pi x:\tau\SB.\sigma\SB)::(\Pi x:\tau\SB.J\SB)@A\SB}
% 	      {\mathcal{D}_1 \andalso \mathcal{D}_2}
	      	      	      	      	      
% 	      We can arrange the substitution.
	      	      	      	      	      
% 	      $\mathcal{D}'_3$ = \infer[\KAbs]
% 	      {\GGV (\Pi x:\tau.\sigma)\SB::(\Pi x:\tau.J)\SB@A\SB}
% 	      {\mathcal{D}_1 \andalso \mathcal{D}_2}
	      	      	      	      	      
% 	\item \KApp
	      	      	      	      	      
% 	      From the induction hypothesis and \KApp, we can get $\mathcal{D}_1$.
	      	      	      	      	      
% 	      $\mathcal{D}_1$ = \infer[\KApp]
% 	      {\GGV (\sigma\SB\ M\SB)::K\SB[x \mapsto M\SB]@A\SB}
% 	      {\ID{\GGV \sigma\SB::(\Pi x:\tau\SB.K\SB)@A\SB} \andalso \ID{\GGV M\SB:\tau\SB@A}}
	      	      	      	      	      
% 	      Arrange substitutions,
	      	      	      	      	      
% 	      $\mathcal{D}'_1$ = \infer[\KApp]
% 	      {\GGV (\sigma\ M)\SB::K[x \mapsto M]\SB@A\SB}
% 	      {\ID{\GGV \sigma\SB::(\Pi x:\tau\SB.K\SB)@A\SB} \andalso \ID{\GGV M\SB:\tau\SB@A\SB}}
	      	      	      	      	      
% 	\item \KConv
	      	      	      	      	      
% 	      From the induction hypothesis, we have $\mathcal{D}_1$ and $\mathcal{D}_2$.
	      	      	      	      	      
% 	      $\mathcal{D}_1$ = \infer[]
% 	      {\GGV \tau\SB : K[z\mapsto P]@A\SB}
% 	      {\vdots}
	      	      	      	      	      
% 	      $\mathcal{D}_2$ = \infer[]
% 	      {\GGV K\SB \E J\SB @A\SB}
% 	      {\vdots}
	      	      	      	      	      
% 	      Use \KConv,
	      	      	      	      	      
% 	      $\mathcal{D}_2$ = \infer[\KConv]
% 	      {\GGV \tau\SB : J[z\mapsto P]@A\SB}
% 	      {\mathcal{D}_1 \andalso \mathcal{D}_2}
	      	      	      	      	      
% 	\item \KTW
	      	      	      	      	      
% 	      \begin{itemize}
	      		      		      		      		      	
% 	      	\item $\alpha \neq \beta$
	      	      	      	      	      	      	      	      	      	      
% 	      	      From the induction hypothesis, we have $\mathcal{D}_1$.
	      	      	      	      	      	      	      	      	      	      
% 	      	      $\mathcal{D}_1$ = \infer[]
% 	      	      {\GGV \tau\SB :: K\SB @ A\alpha\SB}
% 	      	      {\vdots}
	      	      	      	      	      	      	      	      	      	      
% 	      	      Use \KTW,
	      	      	      	      	      	      	      	      	      	      
% 	      	      $\mathcal{D}_2$ = \infer[]
% 	      	      {\GGV \TW_\alpha \tau\SB :: K\SB @ A\SB}
% 	      	      {\mathcal{D}_1}
	      	      	      	      	      	      	      	      	      	      
% 	      	\item $\alpha = \beta$
	      	      	      	      	      	      	      	      	      	      
% 	      	      The conclusion is identical with the induction hypothesis.
	      	      	      	      	      	      	      	      	      	      
% 	      \end{itemize}
	      	      	      	      	      
% 	\item \KTWL
	      	      	      	      	      
% 	      \begin{itemize}
	      		      		      		      		      	
% 	      	\item $\alpha \neq \beta$
	      	      	      	      	      	      	      	      	      	      
% 	      	      From the induction hypothesis, we have $\mathcal{D}_1$.
	      	      	      	      	      	      	      	      	      	      
% 	      	      $\mathcal{D}_1$ = \infer[]
% 	      	      {\GGV \TW_\alpha \tau\SB :: K\SB @ A\alpha\SB}
% 	      	      {\vdots}
	      	      	      	      	      	      	      	      	      	      
% 	      	      Use \KTWL,
	      	      	      	      	      	      	      	      	      	      
% 	      	      $\mathcal{D}_2$ = \infer[]
% 	      	      {\GGV \tau\SB :: K\SB @ A\alpha\SB}
% 	      	      {\mathcal{D}_1}
	      	      	      	      	      	      	      	      
% 	      	\item $\alpha = \beta$
	      	      	      	      	      	      	      	      
% 	      	      The conclusion is identical with the induction hypothesis.
	      	      	      	      	      	      	      	      
% 	      \end{itemize}
	      	      	      	      
	      	      	      	      
% 	\item \KGen
	      	      	      	      
% 	      From the induction hypothesis, we have $\mathcal{D}_1$.
	      	      	      	      
% 	      $\mathcal{D}_1$ = \infer[]
% 	      {\GGV \tau\SB :: K\SB @ A\SB}
% 	      {\vdots}
	      	      	      	      
% 	      And we can prove easily $\alpha \notin \FTV(\GG) \cup \FTV(A)$.
	      	      	      	      
% 	      Use \KGen,
	      	      	      	      
% 	      $\mathcal{D}_2$ = \infer[\KGen]
% 	      {\GGV \forall\alpha.\tau\SB :: K\SB @ A\SB}
% 	      {\mathcal{D}_1 \andalso \alpha \notin \FTV(\GG) \cup \FTV(A)}
	      	      	      	      
% 	\item \KCsp
	      	      	      	      
% 	      \begin{itemize}
	      		      		      		      	
% 	      	\item $\alpha \neq \beta$
	      	      	      	      	      	      
% 	      	      From the induction hypothesis and \KCsp, we have $\mathcal{D}_1$.
	      	      	      	      	      	      
% 	      	      $\mathcal{D}_1$ = \infer[\KCsp]
% 	      	      {\GGV \tau\SB :: K\SB @ A\alpha\SB}
% 	      	      {\ID{\GGV \tau\SB :: K\SB @ A\SB}}
	      	      	      	      	      	      
% 	      	\item $\alpha = \beta$
	      	      	      	      	      	      
% 	      	      The conclusion is identical with the induction hypothesis.
	      	      	      	      	      	      
% 	      \end{itemize}
	      	      	      
% 	\item \TVar
	      	      	      
% 	      We can easily prove $x:\tau\SB \in \GG$.
	      	      	      
% 	      From the induction hypothesis and \TVar, we have $\mathcal{D}_1$.
	      	      	      
% 	      $\mathcal{D}_1$ = \infer[]
% 	      {\GGV x:\tau\SB @A\SB}
% 	      {x:\tau\SB \in \GG \andalso \ID{\GGV \tau\SB::*@A\SB}}
	      	      	      
% 	\item \TAbs
	      	      	      
% 	      From the induction hypothesis and \TAbs, we get
	      	      	      
% 	      $\mathcal{D}_1$ = \infer[\TAbs]
% 	      {\GGV (\lambda x:\sigma\SB.M\SB):(\Pi x:\sigma\SB.\tau\SB)@A\SB}
% 	      {\ID{\GGV \sigma\SB::*@A\SB} \andalso \ID{\GG, x:\sigma\SB@A\SB \V M\SB:\tau\SB@A\SB}}
	      	      	      
% 	      Arrange substitutions,
	      	      	      
% 	      $\mathcal{D}'_1$ = \infer[\TAbs]
% 	      {\GGV (\lambda x:\sigma.M)\SB:(\Pi x:\sigma.\tau)\SB@A\SB}
% 	      {\ID{\GGV \sigma\SB::*@A\SB} \andalso \ID{\GG, x:\sigma\SB@A\SB \V M\SB:\tau\SB@A\SB}}
	      	      	      
% 	\item \TApp
	      	      	      
% 	      We have a derivation trees from the premise.
	      	      	      
% 	      $\mathcal{D}_1$ = \infer[\TApp]
% 	      {\G, z:\xi@B, \D \V M\ L:\tau[y\mapsto L]@A}
% 	      {\infer[]{\G, z:\xi@B, \D \V M:\Pi(y:\rho).\tau@A \andalso \G, z:\xi@B, \D \V L:\rho@A}{\vdots}}
	      	      	      
% 	      We get 2 trees from the induction hypothesis and $\mathcal{D}_1$.
	      	      	      
% 	      $\mathcal{D}_2$ = \infer[]
% 	      {\GGV M\SB: (\Pi(y:\rho).\tau)\SB@A}
% 	      {\vdots}
	      	      	      
% 	      and
	      	      	      
% 	      $\mathcal{D}_3$ = \infer[]
% 	      {\GGV L\SB: \rho\SB@A}
% 	      {\vdots}
	      	      	      
% 	      \red{distribute the substitution} in $\mathcal{D}_2$
	      	      	      
% 	      $\mathcal{D'}_2$ = \infer[]
% 	      {\GGV M\SB: (\Pi(y:\rho\SB).\tau\SB)@A}
% 	      {\vdots}
	      	      	      
% 	      From $\mathcal{D'}_2$ and $\mathcal{D}_3$
	      	      	      
% 	      $\mathcal{D}_4$ = \infer[\TApp]
% 	      {\GGV (M\SB\ L\SB): \tau\SB[y \mapsto L]@A}
% 	      {\mathcal{D'}_2 \andalso \mathcal{D}_3}
	      	      	      
% 	      \red{We can transform the conclusion of $\mathcal{D}_4$ into} \\
% 	      $\GGV (M\ L)\SB: \tau[y \mapsto L]\SB@A$
	      	      	      
% 	\item \TConv
	      	      	      
% 	      We have 2 derivation trees from the premise and the induction hypothesis.
	      	      	      
% 	      $\mathcal{D}_1$ = \infer[]
% 	      {\GGV t\SB:T\SB@A}
% 	      {\vdots}
	      	      	      
% 	      $\mathcal{D}_2$ = \infer[]
% 	      {\GGV T\SB \E T'\SB@A}
% 	      {\vdots}
	      	      	      
% 	      And then \\
% 	      \infer[\TConv]
% 	      {\GGV t\SB:T'\SB@A}
% 	      {\mathcal{D}_1 \andalso \mathcal{D}_2}
	      	      	      
% 	\item \TTB
	      	      	      
% 	      \begin{itemize}
	      		      		      	
% 	      	\item $\alpha \neq \beta$
% 	      	      From the induction hypothesis and \TTB, we get
	      	      	      	      	      	      
% 	      	      $\mathcal{D}_1$ = \infer[\TTB]
% 	      	      {\GGV \TB_\alpha M\SB:\tau\SB@A\SB}
% 	      	      {\infer[]{\GGV M\SB:\tau\SB@A\alpha\SB}{\vdots}}
	      	      	      	      	      	      
% 	      	\item $\alpha = \beta$
	      	      	      	      	      	      
% 	      	      The conclusion is identical with the induction hypothesis.
	      	      	      	      	      	      
% 	      \end{itemize}
	      	      	      
% 	\item \TTBL
	      	      	      
% 	      \begin{itemize}
	      		      		      	
% 	      	\item $\alpha \neq \beta$
% 	      	      From the induction hypothesis and \TTBL, we get
	      	      	      	      	      	      
% 	      	      $\mathcal{D}_3$ = \infer[\TTBL]
% 	      	      {\GGV \TBL_\alpha M\SB:\tau\SB@A\SB}
% 	      	      {
% 	      	      	\infer[]{\GGV M\SB: \TW_\alpha \tau\SB@A\alpha\SB}{\vdots}
% 	      	      }
	      	      	      	      	      	      
% 	      	\item $\alpha = \beta$
	      	      	      	      	      	      
% 	      	      The conclusion is identical with the induction hypothesis.
	      	      	      	      	      	      
% 	      \end{itemize}
	      	      	      
% 	\item \TGen
	      	      	      
% 	      From the induction hypothesis, we get $\mathcal{D}_1$.
	      	      	      
% 	      $\mathcal{D}_1$ = \infer[]
% 	      {\GGV M\SB:\tau\SB@A\SB}
% 	      {\vdots}
	      	      	      
% 	      And we can prove easily $\alpha \notin \FTV(\GG) \cup \FTV(A)$.
	      	      	      
% 	      Use \TGen,
	      	      	      
% 	      $\mathcal{D}_2$ = \infer[\TGen]
% 	      {\GGV (\Lambda\alpha.M)\SB:(\forall\alpha.\tau)\SB@A\SB}
% 	      {\mathcal{D}_1 \andalso \alpha \notin \FTV(\GG) \cup \FTV(A)}
	      	      	      
% 	\item \TIns
	      	      	      
% 	      We can assume $\alpha \neq \beta$ because $\alpha$ appears only in $M:\forall\alpha.\tau$ and we can rename $\alpha$ to an arbitary name.
	      	      	      
% 	      From the induction hypothesis and \TIns, we get $\mathcal{D}_1$.
	      	      	      
% 	      $\mathcal{D}_1$ = \infer[\TIns]
% 	      {\GGV (M\ \varepsilon)\SB:\tau\SB@A\SB}
% 	      {\ID{\GGV M\SB:(\forall\alpha.\tau)\SB@A\SB}}
	      	      	      
% 	\item \TCsp
	      	      	      
% 	      \begin{itemize}
	      		      		      	
% 	      	\item $\alpha \neq \beta$
	      	      	      	      	      	      
% 	      	      From the induction hypothesis and \TCsp, we get $\mathcal{D}_1$.
	      	      	      	      	      	      
% 	      	      $\mathcal{D}_2$ = \infer[\TCsp]
% 	      	      {\GGV (\%_\alpha M)\SB:\tau\SB@A\alpha}
% 	      	      {\ID{\GGV M\SB:\tau\SB@A}}
	      	      	      	      	      	      
% 	      	\item $\alpha = \beta$
	      	      	      	      	      	      
% 	      	      The conclusion is identical with the induction hypothesis.
	      	      	      	      	      	      
% 	      \end{itemize}
	      	      	      
% 	\item \QKAbs
	      	      	      
% 	      From the induction hypothesis and \QKAbs, we get $\mathcal{D}_1$.
	      	      	      
% 	      $\mathcal{D}_1$ = \infer[\QKAbs]
% 	      {\GGV (\Pi x:\tau.K)\SB \E (\Pi x:\sigma.J)\SB@A\SB}
% 	      {\infer[]{\GGV \tau\SB \E \sigma\SB :: *@A\SB}{\vdots} \andalso
% 	      	\infer[]{\GG,x:\tau@A \V K\SB\E J\SB@A\SB}{\vdots} }
	      	      	      
% 	\item \QKCsp
	      	      	      
% 	      \begin{itemize}
	      		      		      	
% 	      	\item $\alpha \neq \beta$
	      	      	      	      	      	      
% 	      	      From the induction hypothesis and \QKCsp, we get $\mathcal{D}_1$.
	      	      	      	      	      	      
% 	      	      $\mathcal{D}_1$ = \infer[\QKCsp]
% 	      	      {\GGV K\SB \E J\SB @A\alpha\SB}
% 	      	      {\infer[]{\GGV K\SB \E J\SB @A\SB}{\vdots}}
	      	      	      	      	      	      
	      	      	      	      	      	      
% 	      	\item $\alpha = \beta$
	      	      	      	      	      	      
% 	      	      The conclusion is identical with the induction hypothesis.
	      	      	      	      	      	      
% 	      \end{itemize}
	      	      	      
% 	\item \QKRefl
	      	      	      
% 	      From the induction hypothesis and \QKRefl, we get $\mathcal{D}_1$.
	      	      	      
% 	      $\mathcal{D}_1$ = \infer[\QKRefl]
% 	      {\GGV K\SB \E K\SB @A\SB}
% 	      {\infer[]{\GGV K\SB \iskind @A\SB}{\vdots}}
	      	      	      
% 	\item \QKSym
	      	      	      
% 	      From the induction hypothesis and \QKSym, we get $\mathcal{D}_1$.
	      	      	      
% 	      $\mathcal{D}_1$ = \infer[\QKSym]
% 	      {\GGV J\SB\E K\SB@A\SB}
% 	      {\GGV K\SB\E J\SB@A\SB}
	      	      	      
% 	\item \QKTrans
	      	      	      
% 	      From the induction hypothesis and \QKTrans, we get $\mathcal{D}_1$.
	      	      	      
% 	      $\mathcal{D}_1$ = \infer[\QKTrans]
% 	      {\GGV K\SB\E I\SB@A\SB}
% 	      {\GGV K\SB\E J\SB@A\SB \andalso \GGV J\SB\E I\SB@A\SB}
	      	      	      
% 	\item \QTAbs
	      	      	      
% 	      From the induction hypothesis and, we get $\mathcal{D}_1$.
	      	      	      
% 	      $\mathcal{D}_1$ = \infer[\QTAbs]
% 	      {\GGV \Pi x:\tau\SB.\rho\SB \E \Pi x:\sigma\SB.\pi\SB@A\SB}
% 	      {\ID{\GGV \tau\SB \E \sigma\SB :: *@A\SB} \andalso \ID{\GG, x:\tau\SB@A\SB \V \rho\SB \E \pi\SB @A\SB}}
	      	      	      
% 	      Arrange substitutions,
	      	      	      
% 	      $\mathcal{D}_1$ = \infer[\QTAbs]
% 	      {\GGV (\Pi x:\tau.\rho)\SB \E (\Pi x:\sigma.\pi)\SB@A\SB}
% 	      {\ID{\GGV \tau\SB \E \sigma\SB :: *@A\SB} \andalso \ID{\GG, x:\tau\SB@A\SB \V \rho\SB \E \pi\SB @A\SB}}
	      	      	      
% 	\item \QTApp
	      	      	      
% 	      From the induction hypothesis and \QTApp, we get $\mathcal{D}_1$.
	      	      	      
% 	      $\mathcal{D}_1$ = \infer[\QTApp]
% 	      {\GGV\pi\SB\ M\SB \E \sigma\SB\ N\SB@A\SB}
% 	      {\ID{\GGV\tau\SB\E\sigma\SB :: (\Pi x:\rho\SB.K\SB)@A\SB} \andalso \ID{\GGV M\SB\E N\SB:\rho\SB@A\SB}}
	      	      	      
% 	      Arrange substitutions,
	      	      	      
% 	      $\mathcal{D}_1$ = \infer[\QTApp]
% 	      {\GGV(\pi\ M)\SB \E (\sigma\ N)\SB@A\SB}
% 	      {\ID{\GGV\tau\SB\E\sigma\SB :: (\Pi x:\rho\SB.K\SB)@A\SB} \andalso \ID{\GGV M\SB\E N\SB:\rho\SB@A\SB}}
	      	      	      
% 	\item \QTTW
	      	      	      
% 	      \begin{itemize}
	      		      		      	
% 	      	\item $\alpha \neq \beta$
	      	      	      	      	      	      
% 	      	      From the induction hypothesis and \QTTW, we get $\mathcal{D}_1$.
	      	      	      	      	      	      
% 	      	      $\mathcal{D}_1$ = \infer[\QTTW]
% 	      	      {\GGV(\TW_\alpha \tau)\SB\E(\TW_\alpha\sigma)\SB@A\SB}
% 	      	      {\ID{\GGV\tau\SB\E\sigma\SB@A\alpha\SB}}
	      	      	      	      	      	      
% 	      	\item $\alpha = \beta$
	      	      	      	      	      	      
% 	      	      The conclusion is identical with the induction hypothesis.
	      	      	      	      	      	      
% 	      \end{itemize}
	      	      	      
% 	\item \QTGen
	      	      	      
% 	      We can assume $\alpha \neq \beta$ because $\alpha$ appears only in $M:\forall\alpha.\tau$ and we can rename $\alpha$ to an arbitary name.
	      	      	      
% 	      We can prove easily $\alpha \notin \FTV(\GG) \cup \FTV(A)$.
% 	      From the induction hypothesis and \QTGen, we get $\mathcal{D}_1$.
	      	      	      
% 	      $\mathcal{D}_1$ = \infer[\QTGen]
% 	      {\GGV (\forall\alpha.\tau)\SB \E (\forall\alpha.\sigma)\SB@A\SB}
% 	      {\ID{\GGV \tau\SB \E \sigma\SB@A\SB} \andalso \alpha \notin \FTV(\GG) \cup \FTV(A)}
	      	      	      
% 	\item \QTCsp
	      	      	      
% 	      \begin{itemize}
	      		      		      	
% 	      	\item $\alpha \neq \beta$
	      	      	      	      	      	      
% 	      	      From the induction hypothesis and \QTCsp, we get $\mathcal{D}_1$.
	      	      	      	      	      	      
% 	      	      $\mathcal{D}_1$ = \infer[\QTCsp]
% 	      	      {\GGV\tau\SB \E \sigma\SB@A\alpha\SB}
% 	      	      {\ID{\GGV\tau\SB \E \sigma\SB@A\SB}}
	      	      	      	      	      	      
% 	      	\item $\alpha = \beta$
	      	      	      	      	      	      
% 	      	      The conclusion is identical with the induction hypothesis.
	      	      	      	      	      	      
% 	      \end{itemize}
	      	      	      
% 	\item \QTRefl
	      	      	      
% 	      From the induction hypothesis and \QTRefl, we get $\mathcal{D}_1$.
	      	      	      
% 	      $\mathcal{D}_1$ = \infer[\QTRefl]
% 	      {\GGV\tau\SB\E\tau\SB@A\SB}
% 	      {\ID{\GGV\tau\SB::K\SB@A\SB}}
	      	      	      
% 	\item \QTSym
	      	      	      
% 	      From the induction hypothesis and \QTSym, we get $\mathcal{D}_1$.
	      	      	      
% 	      $\mathcal{D}_1$ = \infer[\QTSym]
% 	      {\GGV\sigma\SB\E\tau\SB@A\SB}
% 	      {\ID{\GGV\tau\SB\E\sigma\SB@A\SB}}
	      	      	      
% 	\item \QTTrans
	      	      	      
% 	      From the induction hypothesis and \QTTrans, we get $\mathcal{D}_1$.
	      	      	      
% 	      $\mathcal{D}_1$ = \infer[\QTTrans]
% 	      {\GGV \tau\SB\E\rho\SB@A\SB}
% 	      {\ID{\GGV\tau\SB\E\sigma\SB@A\SB} \andalso \ID{\GGV\sigma\SB\E\rho\SB@A\SB}}
	      	      	      
% 	\item \QAbs
	      	      	      
% 	      From the induction hypothesis and \QAbs, we get $\mathcal{D}_1$.
	      	      	      
% 	      $\mathcal{D}_1$ = \infer[\QAbs]
% 	      {\GGV \Pi x:\tau\SB.\rho\SB \E \Pi x:\sigma\SB.\pi\SB@A\SB}
% 	      {\ID{\GGV\tau\SB \E \sigma\SB :: * @A\SB} \andalso \ID{\GG,x:\tau\SB@A\SB\V\rho\SB \E \pi\SB@A\SB}}
	      	      	      
% 	      Arrange substitutions,
	      	      	      
% 	      $\mathcal{D}'_1$ = \infer[\QAbs]
% 	      {\GGV (\Pi x:\tau.\rho)\SB \E (\Pi x:\sigma.\pi)\SB@A\SB}
% 	      {\ID{\GGV\tau\SB \E \sigma\SB :: * @A\SB} \andalso \ID{\GG,x:\tau\SB@A\SB\V\rho\SB \E \pi\SB@A\SB}}
	      	      	      
% 	\item \QApp
	      	      	      
% 	      From the induction hypothesis
	      	      	      
% 	      $\mathcal{D}_1$ = \ID{\GGV M\SB \E L\SB :: (\Pi x:\sigma.\tau)\SB@A\SB}
	      	      	      
% 	      Arrange substitutions,
	      	      	      
% 	      $\mathcal{D}'_1$ = \ID{\GGV M\SB \E L\SB :: (\Pi x:\sigma\SB.\tau\SB)@A\SB}
	      	      	      
% 	      Using \QApp\ to $\mathcal{D}'_1$ and the induction hypothesis, we get $\mathcal{D}_2$.
	      	      	      
% 	      $\mathcal{D}_2$ = \infer[\QApp]
% 	      {\GGV M\SB\ N\SB \E L\SB\ O\SB @A\SB}
% 	      {\mathcal{D}'_1 \andalso \ID{\GGV N\SB \E O\SB : \sigma\SB @A\SB}}
	      	      	      
% 	      Arrange substitutions,
	      	      	      
% 	      $\mathcal{D}'_2$ = \infer[\QApp]
% 	      {\GGV (M\ N)\SB \E (L\ O)\SB @A\SB}
% 	      {\mathcal{D}'_1 \andalso \ID{\GGV N\SB \E O\SB : \sigma\SB @A\SB}}
	      	      	      
% 	\item \QTB
	      	      	      
% 	      \begin{itemize}
	      		      		      	
% 	      	\item $\alpha \neq \beta$
	      	      	      	      	      	      
% 	      	      From the induction hypothesis and \QTB, we get $\mathcal{D}_1$.
	      	      	      	      	      	      
% 	      	      $\mathcal{D}_1$ = \infer[\QTB]
% 	      	      {\GGV\TB_\alpha M\SB \E \TB_\alpha N\SB @A\SB}
% 	      	      {\GGV M\SB \E N\SB @A\alpha\SB}
	      	      	      	      	      	      
% 	      	\item $\alpha = \beta$
	      	      	      	      	      	      
% 	      	      The conclusion is identical with the induction hypothesis.
	      	      	      	      	      	      
% 	      \end{itemize}
	      	      	      
% 	\item \QTBL
	      	      	      
% 	      \begin{itemize}
	      		      		      	
% 	      	\item $\alpha \neq \beta$
	      	      	      	      	      	      
% 	      	      From the induction hypothesis and \QTBL, we get $\mathcal{D}_1$.
	      	      	      	      	      	      
% 	      	      $\mathcal{D}_1$ = \infer[\QTBL]
% 	      	      {\GGV\TBL_\alpha M\SB \E \TBL_\alpha N\SB@A\alpha\SB}
% 	      	      {\GGV M\SB \E N\SB : \TW_\alpha \tau@A\SB}
	      	      	      	      	      	      
% 	      	\item $\alpha = \beta$
	      	      	      	      	      	      
% 	      	      The conclusion is identical with the induction hypothesis.
	      	      	      	      	      	      
% 	      \end{itemize}
	      	      	      
% 	\item \QGen
	      	      	      
% 	      \begin{itemize}
	      		      		      	
% 	      	\item $\alpha \neq \beta$
	      	      	      	      	      	      
% 	      	      We can prove easily $\alpha \notin \FTV(\GG) \cup \FTV(A)$.
% 	      	      From the induction hypothesis and \QGen, we get $\mathcal{D}_1$.
	      	      	      	      	      	      
% 	      	      $\mathcal{D}_1$ = \infer[\QGen]
% 	      	      {\GGV\Lambda\alpha.M\SB \E \Lambda\alpha.N\SB@A\SB}
% 	      	      {\ID{\GGV M\SB \E N\SB @A\SB} \andalso \alpha \notin \FTV(\GG) \cup \FTV(A)}
	      	      	      	      	      	      
% 	      	\item $\alpha = \beta$
	      	      	      	      	      	      
% 	      	      The conclusion is identical with the induction hypothesis.
	      	      	      	      	      	      
% 	      \end{itemize}
	      	      	      
% 	\item \QIns
	      	      	      
% 	      We can assume $\alpha \neq \beta$ because $\alpha$ appears only in $M:\forall\alpha.\tau$ and we can rename $\alpha$ to an arbitary name.
	      	      	      
% 	      From the induction hypothesis and \QIns, we get $\mathcal{D}_1$.
	      	      	      
% 	      $\mathcal{D}_1$ = \infer[\QIns]
% 	      {\GGV M\SB\ \varepsilon \E N\SB \ \varepsilon @A\SB }
% 	      {\ID{\GGV M\SB \E N\SB : (\forall\alpha.\tau\SB) @A\SB}}
	      	      	      
% 	\item \QCsp
	      	      	      
% 	      \begin{itemize}
	      		      		      	
% 	      	\item $\alpha \neq \beta$
	      	      	      	      	      	      
% 	      	      From the induction hypothesis and \QCsp, we get $\mathcal{D}_1$.
	      	      	      	      	      	      
% 	      	      $\mathcal{D}_1$ = \infer[\QCsp]
% 	      	      {\GGV \%_\alpha M\SB \E \%_\alpha N\SB @A\SB}
% 	      	      {\ID{\GGV M\SB \E N\SB @A\alpha\SB}}
	      	      	      	      	      	      
% 	      	\item $\alpha = \beta$
	      	      	      	      	      	      
% 	      	      The conclusion is identical with the induction hypothesis.
	      	      	      	      	      	      
% 	      \end{itemize}
	      	      	      
% 	\item \QRefl
	      	      	      
% 	      From the induction hypothesis and \QRefl, we get $\mathcal{D}_1$.
	      	      	      
% 	      $\mathcal{D}_1$ = \infer[\QRefl]
% 	      {\GGV M\SB \E M\SB @A\SB}
% 	      {\ID{\GGV M\SB : \tau\SB @A\SB}}
	      	      	      
% 	\item \QSym
	      	      	      
% 	      From the induction hypothesis and \QSym, we get $\mathcal{D}_1$.
	      	      	      
% 	      $\mathcal{D}_1$ = \infer[\QSym]
% 	      {\GGV N\SB \E M\SB @A\SB}
% 	      {\ID{\GGV M\SB \E N\SB @A\SB}}
	      	      	      
% 	\item \QTrans
	      	      	      
% 	      From the induction hypothesis and \QTrans, we get $\mathcal{D}_1$.
	      	      	      
% 	      $\mathcal{D}_1$ = \infer[\QTrans]
% 	      {\GGV M\SB \E L\SB @A\SB}
% 	      {\ID{\GGV M\SB \E N\SB @A\SB } \andalso \ID{\GGV N\SB \E L\SB @A\SB}}
	      	      	      
% 	\item \QBeta
	      	      	      
% 	      From the induction hypothesis and \QBeta, we get $\mathcal{D}_1$.
	      	      	      
% 	      $\mathcal{D}_1$ = \infer[\QBeta]
% 	      {\GGV (\lambda x:\sigma\SB:M\SB)\ N\SB \E (M\SB)[x \mapsto N\SB]@A\SB}
% 	      {\ID{\GG, x: \sigma\SB@A\SB \V M\SB:\tau\SB@A\SB} \andalso \ID{\GGV N\SB:\sigma\SB @A\SB }}
	      	      	      
% 	      Arrange substitutions,
	      	      	      
% 	      $\mathcal{D}'_1$ = \infer[\QBeta]
% 	      {\GGV ((\lambda x:\sigma:M)\ N)\SB \E (M[x \mapsto N])\SB@A\SB}
% 	      {\ID{\GG, x: \sigma\SB@A\SB \V M\SB:\tau\SB@A\SB} \andalso \ID{\GGV N\SB:\sigma\SB @A\SB }}
	      	      	      
% 	\item \QEta
	      	      	      
% 	      From the induction hypothesis and \QEta, we get $\mathcal{D}_1$.
	      	      	      
% 	      $\mathcal{D}_1$ = \infer[\QEta]
% 	      {\GGV (\lambda x:\sigma\SB.M\SB\ x) \E M\SB@A\SB}
% 	      {\ID{\GGV M\SB : (\Pi x:\sigma\SB.\tau\SB)@A\SB} \andalso x \notin \FV(M\SB)}
	      	      	      
% 	      Arrange substitutions,
	      	      	      
% 	      $\mathcal{D}'_2$ = \infer[\QEta]
% 	      {\GGV (\lambda x:\sigma.M\ x)\SB \E M\SB@A\SB}
% 	      {\ID{\GGV M\SB : (\Pi x:\sigma\SB.\tau\SB)\SB@A\SB} \andalso x \notin \FV(M\SB)}
	      	      	      
% 	\item \QTBLTB
	      	      	      
% 	      From the induction hypothesis and \QTBLTB, we get $\mathcal{D}_1$.
	      	      	      
% 	      $\mathcal{D}_1$ = \infer[\QTBLTB]
% 	      {\GGV \TBL_\alpha \TB_\alpha M\SB \E N\SB@A\SB}
% 	      {\ID{\GGV M\SB \E N\SB @A\SB}}
	      	      	      
% 	\item \QLambda
	      	      	      
% 	      From the induction hypothesis and \QLambda, we get $\mathcal{D}_1$.
	      	      	      
% 	      $\mathcal{D}_1$ = \infer[\QLambda]
% 	      {\GGV (\Lambda\alpha.M\SB)\ \varepsilon \E M\SB[\alpha \mapsto \varepsilon]}
% 	      {\ID{\GGV (\Lambda\alpha.M\SB) : \forall\alpha.\tau\SB @A\SB}}
	      	      	      
% 	      Arrange substitutions,
	      	      	      
% 	      $\mathcal{D}'_1$ = \infer[\QLambda]
% 	      {\GGV ((\Lambda\alpha.M)\ \varepsilon)\SB \E M[\alpha \mapsto \varepsilon]\SB}
% 	      {\ID{\GGV (\Lambda\alpha.M\SB) : \forall\alpha.\tau\SB @A\SB}}
	      	      	      
% 	\item \QPercent
	      	      	      
% 	      From the induction hypothesis and \QPercent, we get $\mathcal{D}_1$.
	      	      	      
% 	      $\mathcal{D}_1$ = \infer[\QPercent]
% 	      {\GGV \%_\alpha M\SB \E M\SB @ A\alpha}
% 	      {\ID{\GGV M\SB : \tau\SB @A\alpha\SB} \andalso \ID{\GGV M\SB : \sigma\SB @A\SB} }
	      	      	      
% \end{itemize}
% \fi

\begin{lemma}[Agreement]
	\begin{flalign*}
		\text{If\ } \G\V \tau::K@A &\text{\ then\ } \G\V K\iskind@A. &\\
		\text{If\ } \G\V M:\tau@A &\text{\ then\ } \G\V \tau::*@A.&\\
		\text{If\ } \G\V K\E J@A &\text{\ then\ } \G\V K\iskind@A \text{\ and\ } \G\V J\iskind@A.&\\
		\text{If\ } \G\V \tau\E \sigma :: K@A &\text{\ then\ } \G\V \tau::K@A \text{\ and\ } \G\V \sigma::K@A.&\\
		\text{If\ } \G\V M\E N : \tau@A &\text{\ then\ } \G\V M:\tau@A \text{\ and\ } \G\V N:\tau@A.&\\
	\end{flalign*}
\end{lemma}

We can prove using induction on the derivation tree.
We show some cases as examples.

\begin{itemize}
	\item \textit{Case} \KCsp{} where \(\G\V \tau::* @A\) \\
	      From \WStar, \(\G\V *\iskind @A\alpha\).
	      	  
	\item \textit{Case} \TCsp{} where \( \G\V M:\tau@A \) \\
	      From the induction hypothesis, \( \G \V \tau :: * @ A \).
	      From \KCsp, \( \G \V \tau :: * @ A\alpha \).
	      
	\item \textit{Case} \QBeta{} where
	      \begin{align*} 
	      	  & \G,x:\sigma@A\V M:\tau@A &   & \G\V N:\sigma@A. 
	      \end{align*}
	      From \TAbs{} and \TApp, \( \G\V(\lambda x:\sigma.M)\ N : \tau[x \mapsto N]@A \).
	      From Term Substitution, \( \G\V M[x\mapsto N] : \tau[x \mapsto N]@A \).
\end{itemize}

As we said in Section \ref{sec:properties}, we generalize Inversion Lemma to use induction.
\begin{lemma}[Inversion Lemma for $\Pi$ type]
	\item If $\G \V (\lambda x:\sigma.M) : \rho$ then there are $\sigma'$ and $\tau'$ such that
	$\rho = \Pi x:\sigma'.\tau'$, $\G \V \sigma \E \sigma'@A$ and $\G ,x:\sigma'@A\V M:\tau'@A$.
	\item If $\G \V \rho \E (\Pi x:\sigma.\tau) : K @A$ then there are $\sigma', \tau', K$, and $J$ such that
	$\rho = \Pi x:\sigma'.\tau'$, $\G \V \sigma \E \sigma' : K @A$, and $\G, x:\sigma@A\V \tau \E \tau' : J @A$.
	\item If $\G \V (\Pi x:\sigma.\tau) \E \rho : K @A$ then there are $\sigma', \tau', K$, and $J$ such that
	$\rho = \Pi x:\sigma'.\tau'$, $\G \V \sigma \E \sigma' : K @A$, and $\G, x:\sigma@A\V \tau \E \tau' : J @A$.
\end{lemma}

\begin{proof}
	We can prove using induction on the derivation tree.
	We show some cases as examples.
				
	\begin{itemize}
		\item \textit{Case} \TAbs{} where $\G\V \sigma::*@A$ and $\G,x:\sigma@A\V M:\tau@A$. \\
		      We can take $\sigma$ and $\tau$ as $\sigma$ and $\tau$. 
		      		      		      		      	      	      	      
		\item \textit{Case} \TConv{} where \(\G \V (\lambda x:\sigma.M) : \rho@A\) and \(\G \V \rho \E (\Pi x:\sigma'.\tau)@A\).
		      There are $\sigma'$ and $\tau'$ such that
		      $\rho = \Pi x:\sigma'.\tau'$, $\G \V \sigma \E \sigma'@A$ and $\G ,x:\sigma'@A\V M:\tau'@A$.
		      by using the induction hypothesis to \(\G \V (\lambda x:\sigma.M) : \rho@A\).
		      %   There are $\sigma'', \tau'', K$, and $J$ such that
		      %   $\rho = \Pi x:\sigma''.\tau''$, $\G \V \sigma \E \sigma'' : K @A$, and $\G, x:\sigma@A\V \tau \E \tau'' : J @A$ 
		      %   by using the induction hypothesis to \(\G \V \rho \E (\Pi x:\sigma'.\tau)@A\).
		      	      	     		      	      	      	      
		\item \textit{Case} \QTRefl.
		      There two cases for the conclusion.
		      \begin{itemize}
		      	\item If $\G \V \rho \E (\Pi x:\sigma.\tau) : K @A$,\\
		      	      we can use the third statement of the lemma as the induction hypothesis.
		      	\item If $\G \V (\Pi x:\sigma.\tau) \E \rho : K @A$,\\
		      	      we can use the second statement of the lemma as the induction hypothesis.
		      \end{itemize}
	\end{itemize}
\end{proof}

\begin{lemma}[Inversion Lemma for $\TW$ type]
	\begin{item}
		  \item If $\G \V \TB_\alpha M : \tau@A$ then 
		  there is $\sigma$ such that $\tau = \TW_\alpha \sigma$ and $\G \V M : \sigma@A$.
	      \item If $\G \V \rho \E \TW_\alpha \tau : K @A$ then there are $\tau', K$, and $J$ such that
	      $\rho = \TW_\alpha \tau'$ and $\G \V \tau \E \tau' : K @A$.
	      \item If $\G \V \TW_\alpha \tau \E \rho : K @A$ then there are $\tau', K$, and $J$ such that
	      $\rho = \TW_\alpha \tau'$ and $\G \V \tau \E \tau' : K @A$.
	\end{item}
\end{lemma}

\begin{proof}
	We can prove using induction on the derivation tree.
	We show some cases as examples.
				
\begin{itemize}
	\item \textit{Case} \TTB{} where \(\G \V M : \sigma'@A\alpha\).
		  We can take $\sigma'$ as $\sigma$.	      	      		      	      	      	      	      
	\item \textit{Case} \TConv{} where \( \G \V \TB_\alpha M : \tau'@A \) and \( \G\V\tau' \E \tau :: K@A \).\\
		  There are $\sigma$ such that $\tau' = \TW_\alpha \sigma$ and $\G \V M : \sigma@A$
		  by using the induction hypothesis to \( \G \V \TB_\alpha M : \tau' @A\).
		  Then, we can rewrite $\tau'$ with $\TW_\alpha \sigma$.
		  There are $\sigma'$ and $K'$ such that $\tau = \TW_\alpha \sigma'$ and $\G \V \sigma \E \sigma' : K @ A$
		  by using the induction hypothesis to \( \G \V \TW_\alpha \sigma : \tau :: K' @A\).
		  This $\sigma'$ also satisfies $\G \V M : \sigma' @ A $ from \TConv.
\end{itemize}
\end{proof}	

\begin{lemma}[Inversion for $\Lambda$ type]
	\begin{item}
	      \item If $\G \V \Lambda\alpha.M : \tau$ then 
	      there is $\sigma$ such that $\sigma = \forall\alpha.\sigma$ and $\G \V M : \sigma@A$.% and $\alpha \notin \FTV(\G) \cup \FV(A)$.
	      \item If $\G \V \rho \E \forall\alpha.\tau : K @A$ then there are $\tau', K$ such that
	      $\rho = \forall\alpha.\tau'$ and $\G \V \tau \E \tau' : K @A$.
	      \item If $\G \V \forall\alpha.\tau \E \rho : K @A$ then there are $\tau', K$ such that
	      $\rho = \forall\alpha.\tau'$ and $\G \V \tau \E \tau' : K @A$.
	\end{item}
\end{lemma}

\begin{proof}
	We can prove using induction on the derivation tree.
	We show some cases as examples.
		
	\begin{itemize}
		\newcommand{\MC}[1]{\mathcal{#1}}
		\item \textit{Case} \TGen\ where $\G\V M:\sigma'@A$.\\
		      We can take $\sigma'$ as $\sigma$.	      	      		      	      	      	      
	\end{itemize}
\end{proof}
		
\begin{lemma}[Inversion Lemma for Application]
	\begin{item}
	      \item If $\G \V (\lambda x:\sigma.M)\ N: \tau@A$ then there are $x$ and $\rho$ such that
	      $\G, x:\sigma \V M : \rho @A$ and $\G \V N:\sigma @ A$.
	\end{item}
\end{lemma}

\begin{proof}
	We can prove using induction on the derivation tree.
	We show some cases as examples.
	\begin{itemize}
		\item \textit{Case} \TConv{} where $\G \V (\lambda x:\sigma.M)\ N: \rho@A$ and $\G \V \rho \E \tau : K @A$.
			  By using the induction hypothesis to $\G \V (\lambda x:\sigma.M)\ N: \rho@A$, 
			  we get $\G, x:\sigma \V M : \pi @A$ and $\G \V N:\sigma @ A$.
		      Then, we can fix $x, \pi$ as $x, \rho$ in the statement correspondingly.
	\end{itemize}
\end{proof}
	
\begin{theorem}[Preservation for $\beta$ reduction]
	If $\G\V M:\tau@A$ and $M \longrightarrow_{\beta} M'$, then $\G\V M':\tau@A$.
\end{theorem}
	
\begin{proof}
	We can prove using induction on the type derivation tree.
	We show some cases as examples.
	\begin{itemize}
		\newcommand{\LB}{\longrightarrow_{\beta}}
																
		\item \textit{Case} \TApp{} where the shape of the reduction is one of following three.
		      \begin{itemize}
		      	\item $(\lambda x:\sigma.N)\ L \LB N[x\mapsto L]$\\
		      	      Because the last rule of the type derivation tree is \TApp, 
		      	      we have $\G \V (\lambda x:\sigma.N) : (\Pi x:\sigma'.\tau')@A$ and
		      	      $\G \V L:\sigma' @A$.
		      	      By using Inversion Lemma for $\Pi$ type to $\G \V (\lambda x:\sigma.N) : (\Pi x:\sigma'.\tau')@A$,
		      	      we get $\G, x:\sigma \V N:\tau$ and $\G \V \sigma \E \sigma'$ and $\G ,x:\sigma \V \tau \E \tau'@A$.
		      	      From \TConv , $\G \V L:\sigma @A$.
		      	      By Term Substituition Lemma to $\G, x:\sigma \V N:\tau$ and $\G \V L:\sigma @A$, 
		      	      we get $\G \V N[x\mapsto L]:\tau[x\mapsto L]$.
		      	      		      	      	      	      	      	      	      	      	      	      	      	      	      	      		      	      	      	      	      	      	      	      	      	      
		      	\item $M\ N \LB M'\ N$\\
		      	      From the induction hypothesis and \TApp, the type is preserved for the reduction.
		      	\item $M\ N \LB M\ N'$\\
		      	      From the induction hypothesis and \TApp, the type is preserved for the reduction.
		      \end{itemize}
		      		      	      	      	      	      	      	      		      	      	      	      	      
		%       \iffullversion
		      		      	      	      	      	      	      	      		      	      	      	      	      
		% \item \TVar
		      		      	      	      	      	      	      	      		      	      	      	      	      
		%       In this case, there is no reduction from $x$.
		      		      	      	      	      	      	      	      		      	      	      	      	      
		% \item \TAbs
		      		      	      	      	      	      	      	      		      	      	      	      	      
		%       We can assume the reduction has following shape.
		      		      	      	      	      	      	      	      		      	      	      	      	      
		%       $\lambda x:\sigma.M \LB \lambda x:\sigma.M'$
		      		      	      	      	      	      	      	      		      	      	      	      	      
		%       From the induction hypothesis and \TAbs, the type is preserved for the reduction.
		      		      	      	      	      	      	      	      		      	      	      	      	      
		      		      	      	      	      	      	      	      		      	      	      	      	      
		% \item \TConv
		      		      	      	      	      	      	      	      		      	      	      	      	      
		%       We can assume the reduction has following shape.
		      		      	      	      	      	      	      	      		      	      	      	      	      
		%       $M \LB M'$
		      		      	      	      	      	      	      	      		      	      	      	      	      
		%       From the induction hypothesis and \TConv, the type is preserved for the reduction.
		      		      	      	      	      	      	      	      		      	      	      	      	      
		% \item \TTB
		      		      	      	      	      	      	      	      		      	      	      	      	      
		%       We can assume the reduction has following shape.
		      		      	      	      	      	      	      	      		      	      	      	      	      
		%       $\TB M \LB \TB M'$
		      		      	      	      	      	      	      	      		      	      	      	      	      
		%       From the induction hypothesis and \TTB, the type is preserved for the reduction.
		      		      	      	      	      	      	      	      		      	      	      	      	      
		% \item \TTBL
		      		      	      	      	      	      	      	      		      	      	      	      	      
		%       We can assume the reduction has following shape.
		      		      	      	      	      	      	      	      		      	      	      	      	      
		%       $\TBL M \LB \TBL M'$
		      		      	      	      	      	      	      	      		      	      	      	      	      
		%       From the induction hypothesis and \TTBL, the type is preserved for the reduction.
		      		      	      	      	      	      	      	      		      	      	      	      	      
		% \item \TGen
		      		      	      	      	      	      	      	      		      	      	      	      	      
		%       We can assume the reduction has following shape.
		      		      	      	      	      	      	      	      		      	      	      	      	      
		%       $\Lambda\alpha. M \LB \Lambda\alpha. M'$
		      		      	      	      	      	      	      	      		      	      	      	      	      
		%       From the induction hypothesis and \TGen, the type is preserved for the reduction.
		      		      	      	      	      	      	      	      		      	      	      	      	      
		% \item \TIns
		      		      	      	      	      	      	      	      		      	      	      	      	      
		%       We can assume the reduction has following shape.
		      		      	      	      	      	      	      	      		      	      	      	      	      
		%       $M\ \varepsilon \LB M'\ \varepsilon$
		      		      	      	      	      	      	      	      		      	      	      	      	      
		%       From the induction hypothesis and \TIns, the type is preserved for the reduction.
		      		      	      	      	      	      	      	      		      	      	      	      	      
		% \item \TCsp
		      		      	      	      	      	      	      	      		      	      	      	      	      
		%       We can assume the reduction has following shape.
		      		      	      	      	      	      	      	      		      	      	      	      	      
		%       $\%_\alpha M \LB \%_\alpha M'$
		      		      	      	      	      	      	      	      		      	      	      	      	      
		%       From the induction hypothesis and \TCsp, the type is preserved for the reduction.
		      		      	      	      	      	      	      	      		      	      	      	      	      
		%       \fi
		      		      	      	      	      	      	      	      		      	      	      	      	      
	\end{itemize}
\end{proof}

\begin{theorem}[Preservation for term on $\TBL\TB$ reduction]
	If $\G\V M:\tau@A$ and $M\longrightarrow_\blacklozenge N$, then $\G\V N:\tau@A$\\
\end{theorem}
	
\begin{proof}
	We can prove using induction on the type derivation tree.
	We show the case of \TTBL{} as examples.
	Other cases are easy.
	\begin{itemize}
		\newcommand{\R}{\longrightarrow_{\blacklozenge}}
																	
		\item \textit{Case} \TTBL{}\\
		      There are 2 cases for $\R$.
		      \begin{itemize}
		      	\item $\TBL\TB M \R M$\\
		      	      Because the last rule is \TTBL, we have $\G \V \TB M : \TW_\alpha \tau @A$.
		      	      By using Inversion Lemma for $\TW$ type to $\G \V \TB M : \TW_\alpha \tau @A$,
		      	      we get $\G \V M : \tau @A$
		      	\item $\TBL M \R \TBL M'$\\
		      	      We can use the induction hypothesis and \TTBL.
		      \end{itemize}
	\end{itemize}
\end{proof}
	

\begin{theorem}[Preservation for term on $\Lambda$ reduction]
	If $\G\V M:\tau@A$ and $M \longrightarrow_{\Lambda} N$, then $\G\V N:\tau@A$.
\end{theorem}
	
\begin{proof}
	We can prove using induction on the type derivation tree.
	We show some cases as examples.
	\begin{itemize}
		\newcommand{\R}{\longrightarrow_{\Lambda}}
		\item \TIns
		      		      	      	      	      	      	      	      	      	      		      	      	      	      
		      There are two cases for the reduction.
		      \begin{itemize}
		      	\item $\Lambda\alpha.M\ B \R M[\alpha \mapsto B]$
		      	      		      	      	      	      	      	      	      	      	      	      	      	      	      	      	      	      	      	      		      	      	      	      	      	      	      	      
		      	      The type derivation tree looks like $\mathcal{D}_1$.
		      	      		      	      	      	      	      	      	      	      	      	      	      	      	      	      	      	      	      	      		      	      	      	      	      	      	      	      
		      	      $\mathcal{D}_1$ = \infer[\TIns]
		      	      {\G \V \Lambda\alpha.M\ B : \tau[\alpha \mapsto B] @ A}
		      	      {\ID{\G \V \Lambda\alpha.M\ : \forall\alpha.\tau @ A}}
		      	      		      	      	      	      	      	      	      	      	      	      	      	      	      	      	      	      	      	      		      	      	      	      	      	      	      	      
		      	      Use "Inversion Lemma for $\Lambda$ type" to $\G \V \Lambda\alpha.M\ : \forall\alpha.\tau @ A$,
		      	      get $\G \V M : \tau @ A$ and $\alpha \notin \FTV(\G) \cup \FV(A)$.
		      	      		      	      	      	      	      	      	      	      	      	      	      	      	      	      	      	      	      	      		      	      	      	      	      	      	      	      
		      	      Use "Stage Substituition Lemma" to $\G \V M : \tau @ A$,
		      	      get $\G[\alpha \mapsto B] \V M[\alpha \mapsto B] : \tau[\alpha \mapsto B] @ A[\alpha \mapsto B]$.
		      	      		      	      	      	      	      	      	      	      	      	      	      	      	      	      	      	      	      	      		      	      	      	      	      	      	      	      
		      	      Because $\alpha \notin \FTV(\G) \cup \FV(A)$, $\G[\alpha \mapsto B] = \G$ and $A[\alpha \mapsto B] = A$.
		      	      		      	      	      	      	      	      	      	      	      	      	      	      	      	      	      	      	      	      		      	      	      	      	      	      	      	      
		      	      So, we can rewrite $\G[\alpha \mapsto B] \V M[\alpha \mapsto B] : \tau[\alpha \mapsto B] @ A[\alpha \mapsto B]$ to
		      	      $\G \V M[\alpha \mapsto B] : \tau[\alpha \mapsto B] @ A$.
		      	\item $M\ B \R M'\ B$
		      	      		      	      	      	      	      	      	      	      	      	      	      	      	      	      	      	      	      	      		      	      	      	      	      	      	      	      
		      	      We can use induction hypothesis directly.
		      \end{itemize}
		      		      	      	      	      	      	      	      	      	      		      	      	      	      
		\item Otherwise
		      		      	      	      	      	      	      	      	      	      		      	      	      	      
		      we can use induction hypothesis directly.
	\end{itemize}
\end{proof}
	
\begin{dfn}[$\natural$ translation]
	$\natural$ translation is a translation from $\lambda^\text{MD}$ to $\lambda^\to$.
	\begin{itemize}
		\item Term
		      \begin{flalign*}
		      	\natural(x) &= x & \\
		      	\natural(\lambda x:\tau.M) &= \lambda x:\natural(\tau).\natural(M) & \\
		      	\natural(M\ N) &= \natural(M)\ \natural(N)& \\
		      	\natural(\TB_\alpha M) &= \natural(M) & \\
		      	\natural(\TBL_\alpha M) &= \natural(M)& \\
		      	\natural(\Lambda\alpha.M) &= \natural(M)& \\
		      	\natural(M\ B) &= \natural(M) &
		      \end{flalign*}
		\item Type
		      \begin{flalign*}
		      	\natural(X) &= X & \\
		      	\natural(\Pi x:\tau.\sigma) &= \natural(\tau) \to \natural(\sigma) & \\
		      	\natural(\tau\ x) &= \natural(\tau) & \\
		      	\natural(\TW_\alpha \tau) &= \natural(\tau) & \\
		      	\natural(\forall \alpha.\tau) &= \natural(\tau) &
		      \end{flalign*}
		\item Kind
		      \begin{flalign*}
		      	\natural(K) &= * &
		      \end{flalign*}
		\item Context
		      \begin{flalign*}
		      	\natural(\phi) &= \phi & \\
		      	\natural(\G, x:T@A) &= \natural(\G), \natural(x):\natural(\tau) & \\
		      	\natural(\G, X:K@A) &= \natural(\G) &
		      \end{flalign*}
	\end{itemize}
\end{dfn}
	
\begin{lemma}[Preservation of equality in $\natural$]
	If $\G \V \tau \E \sigma @ A$ then $\natural(\tau) = \natural(\sigma)$.
\end{lemma}
	
Prove by induction on the derivation tree.
	
\begin{lemma}[Preservation of typing in $\natural$]
	If $\G \V M:\tau@A$ in $\lambda^{\text{MD}}$ then $\natural(\G) \V \natural(M): \natural(\tau)$ in $\lambda^\to$.
\end{lemma}
	
Prove by induction on the type derivation tree.
	
\begin{itemize}
	\item \TApp
	      	      	      	      	      	      	      	      	      	      		      	      	      	      
	      We have a derivation tree $\MD{1}$.
	      	      	      	      	      	      	      	      	      	      		      	      	      	      
	      $\MD{1}$ = \infer[\TApp]
	      {\G \V M N : \tau[x \mapsto N] @A}
	      {\ID{\G \V M : (\Pi(x:\sigma).\tau) @ A} \andalso \ID{\G \V N :\sigma @A}}
	      	      	      	      	      	      	      	      	      	      		      	      	      	      
	      From the induction hypothesis, we have $\natural(\G) \V \natural(M) : \natural(\sigma) \to \natural(\tau)$ and $\natural(\G) \V \natural(N) : \natural(\sigma)$.
	      Use the Application rule in $\lambda^\to$, we get $\natural(\G) \V \natural(M)\ \natural(N) : \natural(\tau)$.
	      Because $\natural(M)\ \natural(N) = \natural(M\ N)$ from the definition of $\natural$, $\natural(\G) \V \natural(M\ N) : \natural(\tau)$ in $\lambda^\to$.
	\item \TConv
	      	      	      	      	      	      	      	      	      	      		      	      	      	      
	      We have a derivation tree $\MD{1}$.
	      	      	      	      	      	      	      	      	      	      		      	      	      	      
	      $\MD{1}$ = \infer[\TConv]
	      {\G\V M:\sigma@A}
	      {\ID{\G\V M:\tau@A} \andalso \ID{\G\V \tau\E\sigma@A}}
	      	      	      	      	      	      	      	      	      	      		      	      	      	      
	      Use "Preservation of typing in $\natural$" to $\G\V \tau\E\sigma@A$, we get $\natural(\tau) = \natural(\sigma)$.
	      On the otherhand, $\natural(\G) \V \natural(M):\natural(\tau)$ from the induction hypothesis.
	      Then $\natural(\G) \V \natural(M):\natural(\sigma)$.
	      	      	      	      	      	      	      	      	      	      		      	      	      	      
\end{itemize}

\begin{lemma}[Preservation of substitution in $\natural$]
	If $\G, x:\sigma \V M:\tau@A$ and $\G \V N:\sigma@A$ in $\lambda^{\text{MD}}$
	then $\natural(M[x \mapsto N])$ = $\natural(M)[x\mapsto\natural(N)]$
\end{lemma}
	
Prove by induction on the type derivation tree of $\G, x:\sigma \V M:\tau@A$.
	
\begin{lemma}[Preservation of $\beta$ reduction in $\natural$]
	If $\G \V M:\tau@A$ and $M \longrightarrow_\beta N$ in $\lambda^{\text{MD}}$
	then $\natural(M) \longrightarrow_\beta^+ \natural(N)$.
\end{lemma}
	
Prove by induction on the derivation of $\beta$ reduction.
	
\begin{itemize}
	\newcommand{\R}{\longrightarrow_{\beta}}
	\item $(\lambda x:\tau.M)\ N \R M[x \mapsto N]$
	      	      	      	      	      	      	      	      	      	      		      	      	      	      
	      From the definition of $\natural$, $\natural((\lambda x:\tau.M)\ N)$ = $\lambda x:\natural(\tau).\natural(M)\ \natural(N)$.
	      	      	      	      	      	      	      	      	      	      		      	      	      	      
	      $\lambda x:\natural(\tau).\natural(M)\ \natural(N)$ is a typed term in $\lambda^\to$, we can do $\beta$ reduction from it.\\
	      As a result of the reduction, we get $\natural(M)[x\mapsto\natural(N)]$.
	      	      	      	      	      	      	      	      	      	      		      	      	      	      
	      On the otherside, use "Inversion Lemma for Application" to $(\lambda x:\tau.M)\ N$, get $\G, x:\sigma \V M:\tau@A$ and $\G \V N:\sigma@A$.
	      From "Preservation of substitution in $\natural$", $\natural(M[x \mapsto N])$ = $\natural(M)[x\mapsto\natural(N)]$.
	      	      	      	      	      	      	      	      	      	      		      	      	      	      
	\item Otherwise
	      	      	      	      	      	      	      	      	      	      		      	      	      	      
	      Use the induction hypothesis.
\end{itemize}
	
\begin{theorem}[Strong Normalization]
	If $\G\V^A t:T$ then there is no infinite sequence of terms $(t_i)_{i\ge1}$ and $t_i \longrightarrow_{\beta, \TBL \TB,\Lambda} t_{i+1}$ for $i\ge 1$
\end{theorem}
	
Prove if there are infinite reductions in $\lambda^{\text{MD}}$ then there are infinite beta reductions in $\lambda^{\text{MD}}$.\\
This is because other reductions reduce the size of term.\\
	
Now, we can conclude there are no infinite reductions in a typed $\lambda^{\text{MD}}$ term. \\
	
This is because if there are no infinite reduction in a typed $\lambda^{\text{MD}}$ term $M$,
we can construct a typed term of simply typed lambda calculus $\natural(M)$ from "Preservation of typing in $\natural$".
And $\natural(M)$ has infinite reductions from "Preservation of $\beta$ reduction in $\natural$".\\
	
But, indeed, $\lambda^\to$ has a property of Strong Normalization, so there is no infinite reductions.
	
\begin{theorem}[Confluence(Church-Rosser Property)]
	Define $M \longrightarrow N$ as $M \longrightarrow_{\beta} N$ or $M\longrightarrow_\blacklozenge N$ or  $M \longrightarrow_{\Lambda} N$.\\
	For any term $M$, if $M \longrightarrow^* N$ and $M \longrightarrow^* L$,
	there exists $O$ that satisfies $N \longrightarrow^* O$ and $L \longrightarrow^* O$.
\end{theorem}
	
\textsc{Proof.}
	
Because we show the Strong Normalization of $\lambda^{\text{MD}}$, we can use Newman's lemma to prove Church-Rosser property of $\lambda^{\text{MD}}$.
Then, what we must show is Weak Church-Rosser Property now.
	
When we consider two dirfferent redux in a $\lambda^{\text{MD}}$ term, they can only be disjoint, or one is a part of the other.
In short, they are never overlapped each other.
So, we can reduce one of them after we reduce another.\\
	
\figheader{Values and Redexes}{}
$A \neq \varepsilon$\\
\begin{align*}
	\textrm{Values}  &   & v^\varepsilon \in V^\varepsilon & ::= \lambda x:\tau.M \mid\ \TB_\alpha v^\alpha \mid \Lambda\alpha.v^\varepsilon              &   \\
	&   & v^A \in V^A                     & ::= x \mid \lambda x:\tau.v^A \mid v^A\ v^A \mid\ \TB_\alpha v^{A\alpha}
	\mid \Lambda\alpha.v^A \mid v^A\ \varepsilon &                                                                                                                                      \\
	                 &   &                                 & \quad\   \mid\ \TBL_\alpha v^{A'} (\text{if } A'\alpha = A \text{ and } A' \neq \varepsilon) &   \\
	                 &   &                                 & \quad\   \mid \%_\alpha v^{A'} (\text{if } A'\alpha = A)                                     &   \\
	\textrm{Redexes} &   & R^\varepsilon                   & ::= (\lambda x:\tau.M)\ v^\varepsilon \mid (\Lambda\alpha.v^\varepsilon)\ \varepsilon        &   \\
	                 &   & R^\alpha                        & ::=\ \TBL_\alpha \TB_\alpha v^\alpha                                                         &   \\
\end{align*}
	
\begin{dfn}[Reduction]
	$ M \longrightarrow M'$ iff \\
	$ M \longrightarrow_\Lambda M' $, $ M \longrightarrow_\blacklozenge M' $ or $ M \longrightarrow_\beta M' $.
\end{dfn}
	
\begin{theorem}[Progress]
	If $x:\tau@\varepsilon \notin \G$ and $\G \V M : \tau @ A$ then $ M \in V^A $ or $\exists M'$ such that $M \longrightarrow M'$.
\end{theorem}
	
Prove by induction on the type derivation tree of $\G \V M:\tau@A$.
	
\begin{itemize}
	\item \TVar
	      \begin{itemize}
	      	\item $ A = \varepsilon$
	      	      	      	      	      	      	      	      	      	      	      	      	      	      	      	      	      	      	      	      		      	      	      	      	      	      	      	      
	      	      This case is impossible because $x:\tau@\varepsilon \notin \G$.
	      	\item Otherwise
	      	      	      	      	      	      	      	      	      	      	      	      	      	      	      	      	      	      	      	      		      	      	      	      	      	      	      	      
	      	      This case is obvious because $x \in V^A$.
	      \end{itemize}
	      	      	      	      	      	      	      	      	      	      		      	      	      	      
	\item \TTBL
	      	      	      	      	      	      	      	      	      	      		      	      	      	      
	      The derivation is $\MD{1}$.
	      	      	      	      	      	      	      	      	      	      		      	      	      	      
	      $\MD{1}$ = \infer[\TTBL]
	      {\G \V \TBL_\alpha M :\tau @ A\alpha}
	      {\ID{\G \V M : \TW_\alpha \tau @ A}}
	      	      	      	      	      	      	      	      	      	      		      	      	      	      
	      There are two cases for the induction hypothesis.
	      	      	      	      	      	      	      	      	      	      		      	      	      	      
	      \begin{itemize}
	      		      		      		      		      		      		      		      		      		      			      		      		      		      	
	      	\item $ M \in V^A $
	      	      	      	      	      	      	      	      	      	      	      	      	      	      	      	      	      	      	      	      		      	      	      	      	      	      	      	      
	      	      \begin{itemize}
	      	      	\item $ A = \varepsilon $
	      	      	      	      	      	      	      	      	      	      	      	      	      	      	      	      	      	      	      	      	      	      	      	      	      	      	      	      	      	      		      	      	      	      	      	      	      	      	      	      	      	      
	      	      	      Use "Inversion Lemma" for all cases of $v^\varepsilon$, the case of $ M = \TB_\alpha v^\alpha $ is only reasonable.\\
	      	      	      From the definition of $ \longrightarrow $, $\TBL_\alpha \TB_\alpha v^\alpha \longrightarrow v^\alpha$
	      	      	      	      	      	      	      	      	      	      	      	      	      	      	      	      	      	      	      	      	      	      	      	      	      	      	      	      	      	      		      	      	      	      	      	      	      	      	      	      	      	      
	      	      	\item Otherwise
	      	      	      	      	      	      	      	      	      	      	      	      	      	      	      	      	      	      	      	      	      	      	      	      	      	      	      	      	      	      		      	      	      	      	      	      	      	      	      	      	      	      
	      	      	      $ \TBL_\alpha M \in V^{A\alpha}$.
	      	      \end{itemize}
	      	      	      	      	      	      	      	      	      	      	      	      	      	      	      	      	      	      	      	      		      	      	      	      	      	      	      	      
	      	\item $\exists M'$ such that $M \longrightarrow M'$
	      	      	      	      	      	      	      	      	      	      	      	      	      	      	      	      	      	      	      	      		      	      	      	      	      	      	      	      
	      	      From the definition of $ \longrightarrow $, $\TBL_\alpha M \longrightarrow \TBL_\alpha M'$
	      	      	      	      	      	      	      	      	      	      	      	      	      	      	      	      	      	      	      	      		      	      	      	      	      	      	      	      
	      \end{itemize}
	      	      	      	      	      	      	      	      	      	      		      	      	      	      
	\item \TApp
	      	      	      	      	      	      	      	      	      	      		      	      	      	      
	      The derivation looks like $\MD{1}$.
	      	      	      	      	      	      	      	      	      	      		      	      	      	      
	      $\MD{1}$ = \infer[\TApp]
	      {\G \V M\ N :\tau[x\mapsto N] @ A}
	      {\ID{\G \V M : (\Pi x:\sigma.\tau) @ A} \andalso \ID{\G \V N : \sigma @ A}}
	      	      	      	      	      	      	      	      	      	      		      	      	      	      
	      \begin{itemize}
	      	\item $M \in V^A$ and $N \in V^A$
	      	      	      	      	      	      	      	      	      	      	      	      	      	      	      	      	      	      	      	      		      	      	      	      	      	      	      	      
	      	      \begin{itemize}
	      	      	\item $A=\varepsilon$
	      	      	      	      	      	      	      	      	      	      	      	      	      	      	      	      	      	      	      	      	      	      	      	      	      	      	      	      	      	      		      	      	      	      	      	      	      	      	      	      	      	      
	      	      	      $M = \lambda x:\sigma.L$ from the definition of $V^\varepsilon$ and Inversion Lemma.\\
	      	      	      Then, $\lambda x:\sigma.L\ N \longrightarrow_\beta L[x\mapsto N]$.
	      	      	\item Otherwise
	      	      	      	      	      	      	      	      	      	      	      	      	      	      	      	      	      	      	      	      	      	      	      	      	      	      	      	      	      	      		      	      	      	      	      	      	      	      	      	      	      	      
	      	      	      $M\ N \in V^A$.
	      	      \end{itemize}
	      	\item Otherwise
	      	      	      	      	      	      	      	      	      	      	      	      	      	      	      	      	      	      	      	      		      	      	      	      	      	      	      	      
	      	      $M\ N \longrightarrow M'\ N$ or $M\ N \longrightarrow M\ N'$.
	      \end{itemize}
	      	      	      	      	      	      	      	      	      	      		      	      	      	      
	\item \TIns
	      	      	      	      	      	      	      	      	      	      		      	      	      	      
	      The derivation looks like $\MD{1}$.
	      	      	      	      	      	      	      	      	      	      		      	      	      	      
	      $\MD{1}$ = \infer[\TIns]
	      {\G \V M\ B : \tau[\alpha \mapsto B] @ A}
	      {\ID{\G \V M : \forall\alpha.\tau @ A}}
	      	      	      	      	      	      	      	      	      	      		      	      	      	      
	      \begin{itemize}
	      	\item $ M \in V^A $
	      	      	      	      	      	      	      	      	      	      	      	      	      	      	      	      	      	      	      	      		      	      	      	      	      	      	      	      
	      	      \begin{itemize}
	      	      	\item $A=\varepsilon$
	      	      	      	      	      	      	      	      	      	      	      	      	      	      	      	      	      	      	      	      	      	      	      	      	      	      	      	      	      	      		      	      	      	      	      	      	      	      	      	      	      	      
	      	      	      $M = \Lambda\alpha.v^\varepsilon$ from the definition of $V^\varepsilon$ and Inversion Lemma.\\
	      	      	      Then, $\Lambda\alpha.v^\varepsilon\ B \longrightarrow_\Lambda v^\varepsilon[x\mapsto B]$.
	      	      	      	      	      	      	      	      	      	      	      	      	      	      	      	      	      	      	      	      	      	      	      	      	      	      	      	      	      	      		      	      	      	      	      	      	      	      	      	      	      	      
	      	      	\item Otherwise
	      	      	      	      	      	      	      	      	      	      	      	      	      	      	      	      	      	      	      	      	      	      	      	      	      	      	      	      	      	      		      	      	      	      	      	      	      	      	      	      	      	      
	      	      	      $v^A B \in V^A$.
	      	      \end{itemize}
	      	      	      	      	      	      	      	      	      	      	      	      	      	      	      	      	      	      	      	      		      	      	      	      	      	      	      	      
	      	\item $\exists M'$ such that $M \longrightarrow M'$
	      	      	      	      	      	      	      	      	      	      	      	      	      	      	      	      	      	      	      	      		      	      	      	      	      	      	      	      
	      	      $M\ \varepsilon \longrightarrow M'\ B$
	      \end{itemize}
	      	      	      	      	      	      	      	      	      	      		      	      	      	      
	\item Others
	      	      	      	      	      	      	      	      	      	      		      	      	      	      
	      It is obvious from the induction hypothesis.
	      	      	      	      	      	      	      	      	      	      		      	      	      	      
\end{itemize}
	
\section{ Staged Semantics }
	
\AI{You don't have to repeat definitions.}
	
\figheader{Staged Reduction}{}
$A \neq \varepsilon$\\
\begin{align*}
	E^A_\varepsilon [(\lambda x:\tau.M)\ v^\varepsilon] & \longrightarrow_s E^A_\varepsilon[M[x\mapsto v^\varepsilon]]      \\
	E^A_\varepsilon [(\Lambda\alpha.v^\varepsilon)\ B]  & \longrightarrow_s E^A_\varepsilon[v^\varepsilon[\alpha\mapsto B]] \\
	E^A_\alpha [\TBL_\alpha \TB_\alpha v^\alpha]        & \longrightarrow_s E^A_\alpha[v^\alpha]                            \\
\end{align*}
	
\figheader{Evaluation Context}{}
$A \neq \varepsilon$\\
\begin{align*}
	E^\varepsilon_B \in ECtx^\varepsilon_B & ::= \square\ (\text{if\ } B = \varepsilon) \mid E^\varepsilon_B\ M \mid v^e\ E^\varepsilon_B 
	\mid \TB_\alpha E^\alpha_B \mid \Lambda\alpha.E^\varepsilon_B
	\mid E^\varepsilon_B\ B                                                                                                     \\
	E^A_B \in ECtx^A_B                     & ::= \square\ (\text{if } A = B) \mid \lambda x:\tau.E^A_B \mid E^A_B\ M \mid v^A\ E^A_B      
	\mid E^\varepsilon_B \mid \TB_\alpha E^{A\alpha}_B
	\mid \TBL_\alpha E^{A'}_B \ (\text{where } A'\alpha = A)                                                                              \\
	                                       & \quad \mid \Lambda\alpha.E^\varepsilon_B                                                     
	\mid E^A_B\ B \mid \%_\alpha\ E^{A'}_B \ (\text{where } A'\alpha = A)                                                       \\
\end{align*}
	
\begin{lemma}[Unique Decomposition]
	If $x:\tau@\varepsilon \notin \G$ and $\G \V M : \tau @ A$ then 1 or 2 is true.
	\begin{enumerate}
		\item $ M \in V^A$
		\item $\exists ! B, E^A_B, R^B$ such that ($B = \varepsilon$ or $B = \beta$) and $M = E^A_B[R^B]$.
	\end{enumerate}
\end{lemma}
	
Prove by induction on the type derivation tree of $\G \V M:\tau@A$.
	
\begin{itemize}
	\item \TVar
	      \begin{itemize}
	      	\item $ A = \varepsilon$
	      	      	      	      	      	      	      	      	      	      	      	      	      	      	      	      	      	      	      	      		      	      	      	      	      	      	      	      
	      	      This case is impossible because $x:\tau@\varepsilon \notin \G$.
	      	\item Otherwise
	      	      	      	      	      	      	      	      	      	      	      	      	      	      	      	      	      	      	      	      		      	      	      	      	      	      	      	      
	      	      This case is obvious because $x \in V^A$.
	      \end{itemize}
	      	      	      	      	      	      	      	      	      	      		      	      	      	      
	\item \TTBL
	      \begin{itemize}
	      	\item $ A = \varepsilon$
	      	      	      	      	      	      	      	      	      	      	      	      	      	      	      	      	      	      	      	      		      	      	      	      	      	      	      	      
	      	      This case is impossible because the stage of the conclusion of \TTBL\ cannot be $\varepsilon$.
	      	      	      	      	      	      	      	      	      	      	      	      	      	      	      	      	      	      	      	      		      	      	      	      	      	      	      	      
	      	\item Otherwise
	      	      	      	      	      	      	      	      	      	      	      	      	      	      	      	      	      	      	      	      		      	      	      	      	      	      	      	      
	      	      The derivation is $\MD{1}$.
	      	      	      	      	      	      	      	      	      	      	      	      	      	      	      	      	      	      	      	      		      	      	      	      	      	      	      	      
	      	      $\MD{1}$ = \infer[\TTBL]
	      	      {\G \V \TBL_\alpha M :\tau @ \alpha}
	      	      {\ID{\G \V M : \TW_\alpha \tau @ \varepsilon}}
	      	      	      	      	      	      	      	      	      	      	      	      	      	      	      	      	      	      	      	      		      	      	      	      	      	      	      	      
	      	      From the induction hypothesis, 1 or 2 is true.
	      	      \begin{enumerate}
	      	      	\item $ M \in V^\varepsilon$
	      	      	\item $\exists ! B, E^\varepsilon_B, R^B$ such that ($B = \varepsilon$ or $B = \beta$) and $M = E^\varepsilon_B[R^B]$.
	      	      \end{enumerate}
	      	      	      	      	      	      	      	      	      	      	      	      	      	      	      	      	      	      	      	      		      	      	      	      	      	      	      	      
	      	      \begin{itemize}
	      	      	\item $ M \in V^\varepsilon$ is true
	      	      	      	      	      	      	      	      	      	      	      	      	      	      	      	      	      	      	      	      	      	      	      	      	      	      	      	      	      	      		      	      	      	      	      	      	      	      	      	      	      	      
	      	      	      Use "Inversion Lemma" for all cases of $v^\varepsilon$, the case of $ M = \TB_\alpha v^\alpha $ is only reasonable.
	      	      	      	      	      	      	      	      	      	      	      	      	      	      	      	      	      	      	      	      	      	      	      	      	      	      	      	      	      	      		      	      	      	      	      	      	      	      	      	      	      	      
	      	      	      Then, $\TBL_\alpha \TB_\alpha v^\alpha = E^\alpha_\alpha [R^\alpha]$.
	      	      	      	      	      	      	      	      	      	      	      	      	      	      	      	      	      	      	      	      	      	      	      	      	      	      	      	      	      	      		      	      	      	      	      	      	      	      	      	      	      	      
	      	      	\item $\exists ! B, E^\varepsilon_B, R^B$ such that ($B = \varepsilon$ or $B = \beta$) and $M = E^\varepsilon_B[R^B]$ is true.
	      	      	      	      	      	      	      	      	      	      	      	      	      	      	      	      	      	      	      	      	      	      	      	      	      	      	      	      	      	      		      	      	      	      	      	      	      	      	      	      	      	      
	      	      	      \begin{itemize}
	      	      	      	\item $ M = \TB_\alpha E^\alpha_B[R^B] $
	      	      	      	      	      	      	      	      	      	      	      	      	      	      	      	      	      	      	      	      	      	      	      	      	      	      	      	      	      	      	      	      	      	      	      	      	      	      	      	      		      	      	      	      	      	      	      	      	      	      	      	      	      	      	      	      
	      	      	      	      $ \TBL_\alpha \TB_\alpha E^\alpha_B[R^B] \longrightarrow_s E^\alpha_B[R^B]$ doesn't hold because $ E^\alpha_B[R^B] \notin v^\alpha$.
	      	      	      	      So, given $B, E^\varepsilon_B, R^B$ are the unique tuples satisfies the condition.
	      	      	      	\item Otherwise
	      	      	      	      	      	      	      	      	      	      	      	      	      	      	      	      	      	      	      	      	      	      	      	      	      	      	      	      	      	      	      	      	      	      	      	      	      	      	      	      		      	      	      	      	      	      	      	      	      	      	      	      	      	      	      	      
	      	      	      	      It is obvious from the induction hypothesis and the definition of $E^A_B$.
	      	      	      \end{itemize}
	      	      \end{itemize}
	      	      	      	      	      	      	      	      	      	      	      	      	      	      	      	      	      	      	      	      		      	      	      	      	      	      	      	      
	      \end{itemize}
	      	      	      	      	      	      	      	      	      	      		      	      	      	      
	\item \TIns
	      	      	      	      	      	      	      	      	      	      		      	      	      	      
	      \begin{itemize}
	      	\item $ A = \varepsilon$
	      	      	      	      	      	      	      	      	      	      	      	      	      	      	      	      	      	      	      	      		      	      	      	      	      	      	      	      
	      	      The derivation is $\MD{1}$.
	      	      	      	      	      	      	      	      	      	      	      	      	      	      	      	      	      	      	      	      		      	      	      	      	      	      	      	      
	      	      $\MD{1}$ = \infer[\TIns]
	      	      {\G \V M\ C :\tau[\alpha \mapsto C] @ \varepsilon}
	      	      {\ID{\G \V M : \forall\alpha.\tau @ \varepsilon}}
	      	      	      	      	      	      	      	      	      	      	      	      	      	      	      	      	      	      	      	      		      	      	      	      	      	      	      	      
	      	      From the induction hypothesis, 1 or 2 is true.
	      	      	      	      	      	      	      	      	      	      	      	      	      	      	      	      	      	      	      	      		      	      	      	      	      	      	      	      
	      	      \begin{enumerate}
	      	      	\item $ M \in V^\varepsilon$
	      	      	\item $\exists ! B, E^\varepsilon_B, R^B$ such that ($B = \varepsilon$ or $B = \beta$) and $M = E^\varepsilon_B[R^B]$.
	      	      \end{enumerate}
	      	      	      	      	      	      	      	      	      	      	      	      	      	      	      	      	      	      	      	      		      	      	      	      	      	      	      	      
	      	      \begin{itemize}
	      	      	\item $ M \in V^\varepsilon$
	      	      	      	      	      	      	      	      	      	      	      	      	      	      	      	      	      	      	      	      	      	      	      	      	      	      	      	      	      	      		      	      	      	      	      	      	      	      	      	      	      	      
	      	      	      Use Inversion Lemma for all shape in $v^\varepsilon$, the case of $ M = \Lambda\alpha.v^\varepsilon$ is only reasonable.
	      	      	      	      	      	      	      	      	      	      	      	      	      	      	      	      	      	      	      	      	      	      	      	      	      	      	      	      	      	      		      	      	      	      	      	      	      	      	      	      	      	      
	      	      	      Then, $ \Lambda\alpha.v^\varepsilon\ C = E^\varepsilon_\varepsilon [R^\varepsilon]$
	      	      	\item $\exists ! B, E^\varepsilon_B, R^B$ such that ($B = \varepsilon$ or $B = \beta$) and $M = E^\varepsilon_B[R^B]$
	      	      	      	      	      	      	      	      	      	      	      	      	      	      	      	      	      	      	      	      	      	      	      	      	      	      	      	      	      	      		      	      	      	      	      	      	      	      	      	      	      	      
	      	      	      Because $ E^\varepsilon_B[R^B] \neq \Lambda\alpha.v^\varepsilon$, we can decompose $E^\varepsilon_B[R^B]\ B$ uniquely.
	      	      \end{itemize}
	      	      	      	      	      	      	      	      	      	      	      	      	      	      	      	      	      	      	      	      		      	      	      	      	      	      	      	      
	      	\item $ A \neq \varepsilon $
	      	      	      	      	      	      	      	      	      	      	      	      	      	      	      	      	      	      	      	      		      	      	      	      	      	      	      	      
	      	      The derivation is $\MD{1}$.
	      	      	      	      	      	      	      	      	      	      	      	      	      	      	      	      	      	      	      	      		      	      	      	      	      	      	      	      
	      	      $\MD{1}$ = \infer[\TIns]
	      	      {\G \V M\ C :\tau[\alpha \mapsto C] @ A}
	      	      {\ID{\G \V M : \forall\alpha.\tau @ A}}
	      	      	      	      	      	      	      	      	      	      	      	      	      	      	      	      	      	      	      	      		      	      	      	      	      	      	      	      
	      	      From the induction hypothesis, 1 or 2 is true.
	      	      	      	      	      	      	      	      	      	      	      	      	      	      	      	      	      	      	      	      		      	      	      	      	      	      	      	      
	      	      \begin{enumerate}
	      	      	\item $ M \in V^A$
	      	      	\item $\exists ! B, E^A_B, R^B$ such that ($B = \varepsilon$ or $B = \beta$) and $M = E^A_B[R^B]$.
	      	      \end{enumerate}
	      	      	      	      	      	      	      	      	      	      	      	      	      	      	      	      	      	      	      	      		      	      	      	      	      	      	      	      
	      	      \begin{itemize}
	      	      	\item $ M \in V^A$
	      	      	      	      	      	      	      	      	      	      	      	      	      	      	      	      	      	      	      	      	      	      	      	      	      	      	      	      	      	      		      	      	      	      	      	      	      	      	      	      	      	      
	      	      	      It is clear that $v^A\ C \in V^A$.
	      	      	      	      	      	      	      	      	      	      	      	      	      	      	      	      	      	      	      	      	      	      	      	      	      	      	      	      	      	      		      	      	      	      	      	      	      	      	      	      	      	      
	      	      	\item $\exists ! B, E^A_B, R^B$ such that ($B = \varepsilon$ or $B = \beta$) and $M = E^A_B[R^B]$
	      	      	      	      	      	      	      	      	      	      	      	      	      	      	      	      	      	      	      	      	      	      	      	      	      	      	      	      	      	      		      	      	      	      	      	      	      	      	      	      	      	      
	      	      	      Because we cannot $\Lambda$ reduction at stage $A$, we can decompose $E^A_B[R^B]\ C$ uniquely.
	      	      \end{itemize}
	      	      	      	      	      	      	      	      	      	      	      	      	      	      	      	      	      	      	      	      		      	      	      	      	      	      	      	      
	      	      	      	      	      	      	      	      	      	      	      	      	      	      	      	      	      	      	      	      		      	      	      	      	      	      	      	      
	      \end{itemize}
	      	      	      	      	      	      	      	      	      	      		      	      	      	      
	\item \TApp
	      	      	      	      	      	      	      	      	      	      		      	      	      	      
	      \begin{itemize}
	      	\item $ A = \varepsilon$
	      	      	      	      	      	      	      	      	      	      	      	      	      	      	      	      	      	      	      	      		      	      	      	      	      	      	      	      
	      	      The derivation is $\MD{1}$.
	      	      	      	      	      	      	      	      	      	      	      	      	      	      	      	      	      	      	      	      		      	      	      	      	      	      	      	      
	      	      $\MD{1}$ = \infer[\TApp]
	      	      {\G \V M\ N :\tau[\tau \mapsto N] @ \varepsilon}
	      	      {\ID{\G \V M : \Pi x:\sigma.\tau @ \varepsilon} \andalso \ID{\G \V N : \sigma @ \varepsilon}}
	      	      	      	      	      	      	      	      	      	      	      	      	      	      	      	      	      	      	      	      		      	      	      	      	      	      	      	      
	      	      From the induction hypothesis, 1 or 2 is true.
	      	      	      	      	      	      	      	      	      	      	      	      	      	      	      	      	      	      	      	      		      	      	      	      	      	      	      	      
	      	      \begin{enumerate}
	      	      	\item $ M \in V^\varepsilon$ and $ N \in V^\varepsilon$
	      	      	\item $ M \in V^\varepsilon$ and $\exists ! B, E^\varepsilon_B, R^B$ such that ($B = \varepsilon$ or $B = \beta$) and $N = E^\varepsilon_B[R^B]$
	      	      	\item $\exists ! B, E^\varepsilon_B, R^B$ such that ($B = \varepsilon$ or $B = \beta$) and $M = E^\varepsilon_B[R^B]$ and $ N \in V^\varepsilon$
	      	      	\item $\exists ! B, E^\varepsilon_B, R^B$ such that ($B = \varepsilon$ or $B = \beta$) and $M = E^\varepsilon_B[R^B]$ and $\exists ! B', E^\varepsilon_{B'}, R^{B'}$ such that ($B' = \varepsilon$ or $B' = \beta$) and $N = E^\varepsilon_{B'}[R^{B'}]$
	      	      \end{enumerate}
	      	      	      	      	      	      	      	      	      	      	      	      	      	      	      	      	      	      	      	      		      	      	      	      	      	      	      	      
	      	      \begin{itemize}
	      	      	\item Case of 1
	      	      	      	      	      	      	      	      	      	      	      	      	      	      	      	      	      	      	      	      	      	      	      	      	      	      	      	      	      	      		      	      	      	      	      	      	      	      	      	      	      	      
	      	      	      Use Inversion Lemma for all shape in $v^\varepsilon$, the case of $ M = \lambda x:\sigma.v^\varepsilon$ is only reasonable.
	      	      	      	      	      	      	      	      	      	      	      	      	      	      	      	      	      	      	      	      	      	      	      	      	      	      	      	      	      	      		      	      	      	      	      	      	      	      	      	      	      	      
	      	      	      Then $M N = R^\varepsilon$.
	      	      	      	      	      	      	      	      	      	      	      	      	      	      	      	      	      	      	      	      	      	      	      	      	      	      	      	      	      	      		      	      	      	      	      	      	      	      	      	      	      	      
	      	      	\item Otherwise
	      	      	      	      	      	      	      	      	      	      	      	      	      	      	      	      	      	      	      	      	      	      	      	      	      	      	      	      	      	      		      	      	      	      	      	      	      	      	      	      	      	      
	      	      	      It is clear.
	      	      \end{itemize}
	      	      	      	      	      	      	      	      	      	      	      	      	      	      	      	      	      	      	      	      		      	      	      	      	      	      	      	      
	      	\item $ A \neq \varepsilon $
	      	      	      	      	      	      	      	      	      	      	      	      	      	      	      	      	      	      	      	      		      	      	      	      	      	      	      	      
	      	      $M N \notin R^\varepsilon$ because $ A \neq \varepsilon$.
	      	      So, we can decompose uniquely.
	      	      	      	      	      	      	      	      	      	      	      	      	      	      	      	      	      	      	      	      		      	      	      	      	      	      	      	      
	      \end{itemize}
	      	      	      	      	      	      	      	      	      	      		      	      	      	      
	      	      	      	      	      	      	      	      	      	      		      	      	      	      
	\item \TConv
	      	      	      	      	      	      	      	      	      	      		      	      	      	      
	      We can use the induction hypothesis directly.
	      	      	      	      	      	      	      	      	      	      		      	      	      	      
\end{itemize}
	
\AI{Removed the work-in-progress part.}
\endinput
\blue{\huge{It is work in progress from here.}}
	
\begin{dfn}[\% Powerset of a Term]
	For a term $M$, $M^\%$ is a set of terms.\\
	$M' \in M^\%$ iff you can get $M'$ from $M$ by removing arbitary number $\%$.
\end{dfn}
	
\begin{lemma}[Equality and Reduction]
	$\G \V M \E N : \tau @A$ if and only if\\
	$\exists M' \in M^\%, M' \in M^\%, \exists N' \in N^\%, \exists L$
	such that $\G \V M' : \tau$, $\G \V N' : \tau$, $M' \longrightarrow^* L$ and $N' \longrightarrow^* L$.
\end{lemma}
	
\red{TODO}
	
	
% \section{ Deterministic Typechecking }
%
% \figheader{Well-formed kinds}{\rulefbox{\Gamma \vdash K\iskind}}
% \begin{center}
%     \infrule{}{\G\VT *\iskind @A}{\WStar} \andalso
%     \infrule{\G\VT \tau::*@A \andalso \G,x:\tau@A\VT K\iskind @A}{\G\VT(\Pi x:\tau.K)\iskind @A}{\WAbs}\andalso
%     \infrule{\G\VT K\iskind @A}{\G\VT K\iskind @A\alpha}{\WCsp}\\[2mm]
% \end{center}
% 
% \figheader{Kinding}{\rulefbox{\G \VT T::K}}
% \begin{center}
%     \infrule{X::K@A \in \G \andalso \G\VT K\iskind @A}{\G \VT X::K@A}{\KVar} \andalso
%     \infrule{\G\VT \tau :: *@A \andalso \G,x:\tau@A\VT \sigma::J@A}{\G\VT(\Pi x:\tau.\sigma) :: (\Pi x:\tau.J)@A}{\KAbs} \\[2mm]
%     \infrule{\G\VT \sigma:: (\Pi x:\tau.K)@A \andalso \G\VT M:\tau'@A \andalso \G\VT \tau\E\tau' @A}{\G\VT \sigma\ M::K[x\mapsto M]@A}{\KApp} \andalso
%     \infrule{\G\VT \tau::*@A\alpha}{\G\VT\TW_\alpha \tau::*@A}{\KTW}\andalso
%     \infrule{\G\VT \tau::K@A \andalso \alpha\notin\rm{FTV}(\G)\cup\rm{FTV}(A)}{\G\VT\forall\alpha.\tau::K@A}{\KGen} \andalso
%     \infrule{\G\VT \tau::K@A}{\G\VT \tau::K@A\alpha}{\KCsp} \end{center}
% 
% \figheader{Typing}{\rulefbox{\G\VT t:\tau}}
% \begin{center}
%     \infrule{x:\tau@A \in \G \andalso \G\VT \tau::*@A}{\G \VT x:\tau@A}{\TVar} \andalso
%     \infrule{\G\VT \sigma::*@A\andalso\G,x:\sigma@A\VT M:\tau@A}{\G\VT(\lambda (x:\sigma).M):(\Pi (x:\sigma).\tau)@A}{\textsc{T-Abs}} \\[2mm]
%     \infrule{\G\VT M:(\Pi (x:\sigma).\tau)@A \andalso \G\VT N:\sigma'@A \andalso \G\VT \sigma\E\sigma' @A}{\G\VT M\ N : \tau[x\mapsto N]@A}{\textsc{T-App}} \andalso
%     \infrule{\G\VT M:\tau@{A\alpha}}{\G\VT\TB_{\alpha}M:\TW_{\alpha}\tau@A}{\textsc{T-$\TB$}} \andalso
%     \infrule{\G\VT M:\TW_{\alpha}\tau@A}{\G\VT\TBL_{\alpha}M:\tau@{A\alpha}}{\TTBL} \\[2mm]
%     \infrule{\G\VT M:\tau@A \andalso \alpha\notin\rm{FTV}(\G)\cup\rm{FTV}(A)}{\G\VT\Lambda\alpha.M:\forall\alpha.\tau@A}{\textsc{T-Gen}} \andalso
%     \infrule{\G\VT M:\forall\alpha.\tau@A}{\G\VT M\ \varepsilon:\tau[\alpha \mapsto \varepsilon]@A}{\textsc{T-Ins}} \andalso
%     \infrule{\G\VT M:\tau@A}{\G\VT \%_\alpha M:\tau@{A\alpha}}{\textsc{T-Csp}} \andalso
% \end{center}
% 
% \figheader{Kind Equivalence}{\rulefbox{\G\VT K\E J@A}}
% \begin{center}
%     \infrule{\G\VT \tau \E \sigma :: *@A \andalso \G,x:\tau@A \VT K \E J@A}{\G\VT\Pi x:\tau.K \E \Pi x:\sigma.J@A}{\QKAbs} \andalso
%     \infrule{\G\VT K \E J@A}{\G\VT K \E J@{A\alpha}}{\textsc{QK-Csp}} \\[4mm]
% \end{center}
% 
% \figheader{Type Equivalence}{\rulefbox{\G\VT S\E T @A}}
% \begin{center}
%     \infrule{\G\VT \tau \E \sigma @A \andalso \G,x:\tau@A \VT \rho \E \pi @A}{\G\VT\Pi x:\tau.\rho \E \Pi x:\sigma.\pi @A}{\QTAbs} \andalso
%     \infrule{\G\VT \tau \E \sigma @A \andalso \G\VT M \E N @A}{\G\VT \tau\ M \E \sigma\ N @A}{\QTApp} \\[2mm]
%     \infrule{\G\VT \tau \E \sigma @{A\alpha}}{\G\VT \TW_{\alpha} \tau \E \TW_{\alpha} \sigma @A}{\textsc{QT-$\TW$}}\andalso
%     \infrule{\G\VT \tau \E \sigma @A \andalso \alpha\notin\rm{FTV}(\G)\cup\rm{FTV}(A)}{\G\VT \forall\alpha.\tau \E  \forall\alpha.\sigma @A}{\textsc{QT-Gen}} \andalso
%     \infrule{\G\VT \tau \E \sigma @A}{\G\VT \tau \E \sigma @{A\alpha}}{\textsc{QT-Csp}} \\[4mm]
% \end{center}
% 
% \figheader{Term Equivalence}{\rulefbox{\G\VT M\E N : \tau @A}}
% \begin{center}
%     Demands from syntax rules\\[2mm]
%     \infrule{\G\VT \tau \E \sigma :: *@A \andalso \G,x:\tau@A \VT M \E N : \rho @A}{\G\VT\lambda x:\tau.M \E \lambda x:\sigma.N : (\Pi x:\tau.\rho)@A}{\QAbs} \andalso
%     \infrule{\G\VT M \E L : (\Pi x:\sigma.\tau)@A \andalso \G\VT N \E O : \sigma@A}{\G\VT M\ N \E L\ O : \tau[x \mapsto N]@A}{\QApp} \\[2mm]
%     \infrule{\G\VT M \E N : \tau@{A\alpha}}{\G\VT \TB_\alpha M \E \TB_\alpha N : \TW_\alpha \tau@A}{\QTB} \andalso
%     \infrule{\G\VT M \E N : \TW_\alpha \tau@A}{\G\VT \TBL_\alpha M \E \TBL_\alpha N : \tau@{A\alpha}}{\QTBL} \\[2mm]
%     \infrule{\G\VT M\E N : \tau@A \andalso \alpha \notin \FTV(\G)\cup\FTV(A)}{\G\VT \Lambda\alpha.M \E \Lambda\alpha.N : \forall\alpha.\tau@A}{\QGen} \andalso
%     \infrule{\G\VT M \E N:\forall\alpha.\tau@A}{\G\VT M\ \varepsilon \E N\ \varepsilon : \tau[\alpha \mapsto \varepsilon]@A}{\QIns}\andalso
%     \infrule{\G\VT M \E N : \tau @A}{\G\VT\%_\alpha M \E \%_\alpha N : \tau@{A\alpha}}{\QCsp} \\[4mm]
%     Demands from equivalence relationship\\[2mm]
%     \infrule{\G\VT M:\tau@A}{\G\VT M\E M : \tau@A}{\QRefl} \andalso
%     \infrule{\G\VT M\E N : \tau@A}{\G\VT N\E M : \tau@A}{\QSym} \andalso
%     \infrule{\G\VT M\E N : \tau@A \andalso \G\VT N\E L : \tau@A}{\G\VT M\E L : \tau@A}{\QTrans} \\[4mm]
%     Demands from reduction rule\\[2mm]
%     \infrule{\G,x:\sigma@A\VT M:\tau@A \andalso \G\VT N:\sigma@A}{\G\VT(\lambda x:\sigma.M)\ N\E M[x\mapsto N] : \tau[x \mapsto N]@A}{\QBeta} \andalso
%     \infrule{\G\VT M:(\Pi x:\sigma.\tau)@A \andalso x\notin \text{FV}(M)}{\G\VT(\lambda x:\sigma.M\ x)\E M: (\Pi x:\sigma.\tau)@A}{\QEta} \\[2mm]
%     \infrule{\G\VT M \E N : \tau@A}{\G\VT \TBL_\alpha(\TB_\alpha M) \E N : \tau @A}{\QTBLTB} \andalso
%     \infrule{\G\VT (\Lambda\alpha.M) : \forall\alpha.\tau@A}{\G\VT (\Lambda\alpha.M)\ \varepsilon \E M[\alpha \mapsto \varepsilon] : \tau[\alpha \mapsto \varepsilon]@A}{\QLambda} \\[2mm]
%     \infrule{\G\VT M:\tau@{A\alpha} \andalso \G\VT M:\tau@A}{\G\VT\%_\alpha M \E M : \tau@{A\alpha}}{\QPercent} \andalso
% \end{center}
	
% \figheader{Well-formed kinds}{\rulefbox{\Gamma \vdash K\iskind}}
% \begin{center}
%     \infrule{}{\G\VT*\iskind @A}{\WStar} \andalso
%     \infrule{\G\VT \tau::*@A \andalso \G,x:\tau@A\VT K\iskind @A}{\G\VT(\Pi x:\tau.K)\iskind @A}{\WAbs}\andalso
%     \infrule{\G\VT K\iskind @A}{\G\VT K\iskind @A\alpha}{\WCsp}\\[2mm]
% \end{center}
% 
% \figheader{Kinding}{\rulefbox{\G \VT T::K}}
% \begin{center}
%     \infrule{X::K@A \in \G \andalso \G\VT K\iskind @A}{\G \VT X::K@A}{\KVar} \andalso
%     \infrule{\G\VT \tau :: *@A \andalso \G,x:\tau@A\VT \sigma::J@A}{\G\VT(\Pi x:\tau.\sigma) :: (\Pi x:\tau.J)@A}{\KAbs} \\[2mm]
%     \infrule{\G\VT \sigma:: (\Pi x:\tau.K)@A \andalso \G\VT M:\tau@A}{\G\VT \sigma\ M::K[x\mapsto M]@A}{\KApp} \andalso
%     \infrule{\G\VT \tau::K@A \andalso \G\VT K\equiv J@A}{\G\VT \tau::J@A}{\KConv} \\[2mm]
%     \infrule{\G\VT \tau::*@A\alpha}{\G\VT\TW_\alpha \tau::*@A}{\KTW}\andalso
%     \infrule{\G\VT \tau::K@A \andalso \alpha\notin\rm{FTV}(\G)\cup\rm{FTV}(A)}{\G\VT\forall\alpha.\tau::K@A}{\KGen} \andalso
%     \infrule{\G\VT \tau::K@A}{\G\VT \tau::K@A\alpha}{\KCsp}
% \end{center}
% 
% \figheader{Typing}{\rulefbox{\G\VT t:\tau}}
% \begin{center}
%     \infrule{x:\tau@A \in \G \andalso \G\VT \tau::*@A}{\G \VT x:\tau@A}{\TVar} \andalso
%     \infrule{\G\VT \sigma::*@A\andalso\G,x:\sigma@A\VT M:\tau@A}{\G\VT(\lambda (x:\sigma).M):(\Pi (x:\sigma).\tau)@A}{\textsc{T-Abs}} \\[2mm]
%     \infrule{\G\VT M:(\Pi (x:\sigma).\tau)@A \andalso \G\VT N:\sigma@A}{\G\VT M\ N : \tau[x\mapsto N]@A}{\textsc{T-App}} \andalso
%     \infrule{\G\VT M:\tau@A \andalso \G\VT \tau\equiv \sigma :: K@A}{\G\VT M:\sigma@A}{\textsc{T-Conv}} \\[2mm]
%     \infrule{\G\VT M:\tau@{A\alpha}}{\G\VT\TB_{\alpha}M:\TW_{\alpha}\tau@A}{\textsc{T-$\TB$}} \andalso
%     \infrule{\G\VT M:\TW_{\alpha}\tau@A}{\G\VT\TBL_{\alpha}M:\tau@{A\alpha}}{\TTBL} \\[2mm]
%     \infrule{\G\VT M:\tau@A \andalso \alpha\notin\rm{FTV}(\G)\cup\rm{FTV}(A)}{\G\VT\Lambda\alpha.M:\forall\alpha.\tau@A}{\textsc{T-Gen}} \andalso
%     \infrule{\G\VT M:\forall\alpha.\tau@A}{\G\VT M\ \varepsilon:\tau[\alpha \mapsto \varepsilon]@A}{\textsc{T-Ins}} \andalso
%     \infrule{\G\VT M:\tau@A}{\G\VT \%_\alpha M:\tau@{A\alpha}}{\textsc{T-Csp}} \andalso
% \end{center}
% 
% \figheader{Kind Equivalence}{\rulefbox{\G\VT K\E J@A}}
% \begin{center}
%     \infrule{\G\VT \tau \E \sigma :: *@A \andalso \G,x:\tau@A \VT K \E J@A}{\G\VT\Pi x:\tau.K \E \Pi x:\sigma.J@A}{\QKAbs} \andalso
%     \infrule{\G\VT K \E J@A}{\G\VT K \E J@{A\alpha}}{\textsc{QK-Csp}} \\[2mm]
%     \infrule{\G\VT K \iskind @A}{\G\VT K\E K@A}{\textsc{QK-Refl}} \andalso
%     % 簡約規則の要請\\[2mm]
% \end{center}
% 
% \figheader{Type Equivalence}{\rulefbox{\G\VT S\E T :: K @A}}
% \begin{center}
%     \infrule{\G\VT \tau \E \sigma :: *@A \andalso \G,x:\tau@A \VT \rho \E \pi :: *@A}{\G\VT\Pi x:\tau.\rho \E \Pi x:\sigma.\pi :: *@A}{\QTAbs} \andalso
%     \infrule{\G\VT \tau \E \sigma :: (\Pi x:\rho.K)@A \andalso \G\VT M \E N : \rho @A}{\G\VT \tau\ M \E \sigma\ N :: K[x \mapsto M]@A}{\QTApp} \\[2mm]
%     \infrule{\G\VT \tau \E \sigma :: *@{A\alpha}}{\G\VT \TW_{\alpha} \tau \E \TW_{\alpha} \sigma :: *@A}{\textsc{QT-$\TW$}}\andalso
%     \infrule{\G\VT \tau \E \sigma :: *@A \andalso \alpha\notin\rm{FTV}(\G)\cup\rm{FTV}(A)}{\G\VT \forall\alpha.\tau \E  \forall\alpha.\sigma :: *@A}{\textsc{QT-Gen}} \andalso
%     \infrule{\G\VT \tau \E \sigma :: K@A}{\G\VT \tau \E \sigma :: K@{A\alpha}}{\textsc{QT-Csp}} \\[2mm]
%     \infrule{\G\VT \tau::K@A}{\G\VT \tau\E\tau :: K@A}{\textsc{QT-Refl}} 
% \end{center}
% 
% \figheader{Term Equivalence}{\rulefbox{\G\VT M\E N : \tau @A}}
% \begin{center}
%     \infrule{\G\VT \tau \E \sigma :: *@A \andalso \G,x:\tau@A \VT M \E N : \rho @A}{\G\VT\lambda x:\tau.M \E \lambda x:\sigma.N : (\Pi x:\tau.\rho)@A}{\QAbs} \andalso
%     \infrule{\G\VT M \E L : (\Pi x:\sigma.\tau)@A \andalso \G\VT N \E O : \sigma@A}{\G\VT M\ N \E L\ O : \tau[x \mapsto N]@A}{\QApp} \\[2mm]
%     \infrule{\G\VT M \E N : \tau@{A\alpha}}{\G\VT \TB_\alpha M \E \TB_\alpha N : \TW_\alpha \tau@A}{\QTB} \andalso
%     \infrule{\G\VT M \E N : \TW_\alpha \tau@A}{\G\VT \TBL_\alpha M \E \TBL_\alpha N : \tau@{A\alpha}}{\QTBL} \\[2mm]
%     \infrule{\G\VT M\E N : \tau@A \andalso \alpha \notin \FTV(\G)\cup\FTV(A)}{\G\VT \Lambda\alpha.M \E \Lambda\alpha.N : \forall\alpha.\tau@A}{\QGen} \andalso
%     \infrule{\G\VT M \E N:\forall\alpha.\tau@A}{\G\VT M\ \varepsilon \E N\ \varepsilon : \tau[\alpha \mapsto \varepsilon]@A}{\QIns}\andalso
%     \infrule{\G\VT M \E N : \tau @A}{\G\VT\%_\alpha M \E \%_\alpha N : \tau@{A\alpha}}{\QCsp} \\[2mm]
%     \infrule{\G\VT M:\tau@A}{\G\VT M\E M : \tau@A}{\QRefl} \\[2mm]
%     \infrule{\G,x:\sigma@A\VT M:\tau@A \andalso \G\VT N:\sigma@A}{\G\VT(\lambda x:\sigma.M)\ N\E M[x\mapsto N] : \tau[x \mapsto N]@A}{\QBeta} \andalso
%     \infrule{\G,x:\sigma@A\VT M:\tau@A \andalso \G\VT N:\sigma@A}{\G\VT M[x\mapsto N] \E (\lambda x:\sigma.M)\ N : \tau[x \mapsto N]@A}{\QBeta T} \\[2mm]
%     \infrule{\G\VT M:(\Pi x:\sigma.\tau)@A \andalso x\notin \text{FV}(M)}{\G\VT(\lambda x:\sigma.M\ x)\E M: (\Pi x:\sigma.\tau)@A}{\QEta} \andalso
%     \infrule{\G\VT M:(\Pi x:\sigma.\tau)@A \andalso x\notin \text{FV}(M)}{\G\VT M \E (\lambda x:\sigma.M\ x) @A}{\QEta T} \\[2mm]
%     \infrule{\G\VT M \E N : \tau@A}{\G\VT \TBL_\alpha(\TB_\alpha M) \E N : \tau @A}{\QTBLTB} \andalso
%     \infrule{\G\VT M \E N : \tau@A}{\G\VT N \E \TBL_\alpha(\TB_\alpha M) : \tau @A}{\QTBLTB T} \\[2mm]
%     \infrule{\G\VT (\Lambda\alpha.M) : \forall\alpha.\tau@A}{\G\VT (\Lambda\alpha.M)\ \varepsilon \E M[\alpha \mapsto \varepsilon] : \tau[\alpha \mapsto \varepsilon]@A}{\QLambda} \andalso
%     \infrule{\G\VT (\Lambda\alpha.M) : \forall\alpha.\tau@A}{\G\VT M[\alpha \mapsto \varepsilon] \E (\Lambda\alpha.M)\ \varepsilon : \tau[\alpha \mapsto \varepsilon]@A}{\QLambda} \\[2mm]
%     \infrule{\G\VT M:\tau@{A\alpha} \andalso \G\VT M:\tau@A}{\G\VT\%_\alpha M \E M : \tau@{A\alpha}}{\QPercent} \\[2mm]
%     \infrule{\G\VT M:\tau@{A\alpha} \andalso \G\VT M:\tau@A}{\G\VT M \E \%_\alpha M : \tau@{A\alpha}}{\QPercent}
% \end{center}

% !TEX root = ../main.tex

\section{Informal Overview of \LMD \label{sec:informal-overview}}

We describe our calculus \LMD informally.  \LMD is based on
\LTP~\cite{Hanada2014} by Hanada and Igarashi and so we start with a review of 
\LTP.

% \red{\LTP\cite{Hanada2014}が頻発するが同一論文の複数回引用に何かルールはあるのか?}
% \AI{ないです.たしかにあんまりつけるとうるさいですね.2.1 の冒頭とかは文ごと不要なのでは?}
\subsection{\LTP}

% quote and unquote

In \LTP, brackets (quasi-quotation) and escape (unquote) are written
$\TB_\alpha M$ and $\TBL_\alpha M$, respectively.  For example,
$\TB_\alpha (1 + 1)$ represents code of expression $1 + 1$ and thus
evaluates to itself.  Escape $\TBL_\alpha M$ may appear under
$\TB_\alpha$; it evaluates $M$ to a code value and splices it into the
surrounding code fragment.  Such splicing is expressed by the
following reduction rule:
\begin{align*}
	\TBL_\alpha (\TB_\alpha M) \longrightarrow M .
\end{align*}
% Using the type constructor $\TW_\alpha$ for code values, the type of
% $\TB_\alpha M$ is given $\TW_\alpha \tau$ if $M$ has type $\tau$; the type
% of $\TBL_\alpha M$ is $\tau$ if $M$ has $\TW_\alpha \tau$.

The subscript $\alpha$ in $\TB_\alpha$ and $\TBL_\alpha$ is a \textit{stage
  variable}\footnote{%
  In Hanada and Igarashi~\cite{Hanada2014}, it was called a
  \textit{transition variable}, which is derived from correspondence
  to modal logic, studied by Tsukada and Igarashi~\cite{Tsukada}.} and
a sequence of stage variables is called a \textit{stage}.  Intuitively,
they represent the depth of nested brackets.  Stage variables can be
abstracted by $\Lambda\alpha.M$ and instantiated by an application
$M\ A$ to stages.  For example,
$\Lambda\alpha.\TB_\alpha ((\lambda x:\I.x+10)\ 5)$ is a code value,
where \(\alpha\) is abstracted.  If it is applied to \(\alpha_1 \cdots \alpha_n\), \(\TB_\alpha\) becomes \(\TB_{\alpha_1} \cdots \TB_{\alpha_n}\); in particular,
if \(n = 0\), \(\TB_\alpha\) disappears.  So, an
application of $\Lambda\alpha.\TB_\alpha ((\lambda x:\I.x+10)\ 5)$
to the empty sequence \(\varepsilon\) reduces to
(unquoted) \((\lambda x:\I.x+10)\ 5\) and to 15.  In other words, application of
\(\Lambda\)-abstraction to $\varepsilon$ corresponds to \texttt{run}.
% If \(\alpha\) is instantiated with, say, \(\beta\gamma\), then
% \(\TB_\alpha\) becomes nested brackets \(\TB_{\beta} \TB_\gamma\).
This is expressed by the following reduction rule:
\begin{align*}
	(\Lambda\alpha.M)\ A \longrightarrow M[\alpha\mapsto A]
\end{align*}
where stage substitution \([\alpha \mapsto A]\) manipulates the nesting of
\(\TB_\alpha\) and \(\TBL_\alpha\) (and also \(\%_\alpha\) as we see later).

Cross-stage persistence (CSP), which is an important feature of \LTP,
is a primitive to embed values (not necessarily code values) into a
code value.  For example, a \LTP-term
\[
  M_1 = \lambda x:\I.\Lambda\alpha.(\TB_\alpha ((\%_\alpha x) * 2))
\]
takes an integer \(x\) as an input and returns a code value, into
which \(x\) is embedded.  If $M_1$ is applied to $38 + 4$ as in
\[
  M_2 = (\lambda x:\I.\Lambda\alpha.(\TB_\alpha ((\%_\alpha x) * 2)))\ (38 + 4),
\]
then it evaluates to
\(M_3 = \Lambda\alpha.(\TB_\alpha ((\%_\alpha 42) * 2))\).  According
to the semantics of \LTP, the subterm $\%_\alpha 42$ means that it
waits for the surrounding code to be run (by an application to
$\varepsilon$) and so it does not reduce further.  If \(M_3\) is run
by application to \(\varepsilon\), substitution of \(\varepsilon\) for
\(\alpha\) eliminates \(\TB_\alpha\) and \(\%_\alpha\) and so
\(42 * 2\), which reduces to 84, is obtained.
CSP is practically important because
one can call library functions from inside quotations.  

The type system of \LTP uses code types---the type of code of type
\(\tau\) is written \(\TW_\alpha \tau\)---for typing \(\TB_\alpha\),
\(\TBL_\alpha\) and \(\%_\alpha\).  It takes stages into account: a
variable declaration (written $x:\tau@A$) in a type environment is associated with its
declared stage $A$ as well as its type $\tau$ and the type judgement of \LTP is of
the form $\G \vdash M : \tau@A$, in which $A$ stands for the stage
of term $M$.\footnote{%
  In Hanada and Igarashi~\cite{Hanada2014}, it is written
  $\G \vdash^A M : \tau$.
  }
For example,
$y:\I@\alpha \vdash (\lambda x:\I.y) : \I \to \I @ \alpha$ holds, but
$y:\I@\alpha \vdash (\lambda x:\I.y) : \I \to \I @ \varepsilon$ does
not because the latter uses $y$ at stage \(\varepsilon\) but $y$ is
declared at $\alpha$.  Quotation \(\TB_\alpha M\) is given type
\(\TW_\alpha \tau\) at stage \(A\) if \(M\) is given type \(\tau\) at
stage \(A\alpha\); unquote \(\TBL_\alpha M\) is given type \(\tau\)
at stage \(A\alpha\) if \(M\) is given type \(\TW_\alpha \tau\) at
stage \(A\alpha\); and CSP \(\%_\alpha M\) is give type \(\tau\)
at stage \(A\alpha\) if \(M\) is given type \(\tau\) at \(A\).
These are expressed by the following typing rules.
\begin{center}
	\infrule{\G\vdash M:\tau @{A\alpha}}{\G\vdash \TB_{\alpha}M:\TW_{\alpha}\tau @A} \hfil
	\infrule{\G\vdash M:\TW_{\alpha}\tau @A}{\G\vdash \TBL_{\alpha}M:\tau @{A\alpha}} \hfil
	\infrule{\G\vdash M: \tau @A}{\G\vdash \%_{\alpha}M:\tau @{A\alpha}}
\end{center}
% \TTB, corresponding to brackets, means 
% if $M$ is typed $\tau$ at stage $A\alpha$ then $\TB_{\alpha}M$, quoted $M$, is typed $\TW_{\alpha}\tau$ at $A$.
% \TTBL\ is converse of \TTB.

% CSP


% Omitting Residualization
% この段落は3章のM eのあとに、この制限の結果として...という形で入れる

% There is another important feature called program residualization in \LTP.
% It means that a generated code can be dumped into a file.
% We can load the dumped file and run it.
% The difficulty arises when program residualization is used with CSP.
% Transition variables are classified into two kinds in \LTP in order to deal with this difficulty.

% \subsection{\LLF}

% % \LLF
% \LLF is a simple system of dependent types introduced in \cite{attapl}.
% It is made from Edinburgh LF\cite{harper1993framework} by omitting signatures and include declarative equality judgements.
% Hence, all constants and base types are declared in the signature.
% The \LLF type theory generalizes simply typed lambda calculus
% by replacing the function type $\tau\to\sigma$ with the dependent function type $\Pi x:\tau.\sigma$.

% % Kind, Well-formed kind
% In addition to ordinary typing rules like simply typed lambda calculus,
% there are kinding rules, well-formed kinding rules, term equivalence rules, type equivalence rules, and kind equivalence rules in \LLF.
% Kinding rules and well-formed kinding rules are 
% introduced in order to prohibit making illegal types such as $\textrm{Vector}\ \textrm{true}$.
% For a well-formed type $\tau$, $\G \vdash \tau :: K$ means that $\tau$ has a kind $K$ under the environment $\G$ and 
% for a well-formed kind $K$, $\G \vdash K$ means that $K$ is a well-formed kind under an environment $\G$.

% % Type Equality
% Type equality rules are needed because the type equivalence is not obvious unlike simply typed lambda calculus.
% For example, $\textrm{Vector}\ 7$ should be equivalent to $\textrm{Vector}\ (3+4)$
% but they are not equivalent seemingly. Thus, we must define equivalence rules.
% In \LLF, equivalence is expressed with a symbol of $\E$.
% $\G \vdash M \E N$ means a term $M$ and a term $N$ are equivalent under the environment $\G$.
% $\G \vdash \tau \E \sigma$ means a type $\tau$ and a type $\sigma$ are equivalent under the environment $\G$.
% $\G \vdash K \E J$ means a kind $K$ and a kind $J$ are equivalent under the environment $\G$.

\subsection{Extending \LTP with Dependent Types}

In this paper, we add a simple form of dependent types---{\`a} la
Edinburgh LF~\cite{harper1993framework} and \LLF~\cite{attapl}---to \LTP.
Types can be indexed by terms as in \texttt{Vector} in
Section~\ref{sec:intro} and \(\lambda\)-abstractions can be given
dependent function types of the form \(\Pi x:\tau. \sigma\) but we do
not consider type operators (such as $\texttt{list } \tau$) or
abstraction over type variables.  We introduce kinds to classify
well-formed types and equivalences for kinds, types, and terms---as
in other dependent type systems---but we have to address a question
how the notion of stage (should) interact with kinds and types.

On the one hand, base types such as \(\I\) should be able to be used
at every stage as in \LTP so that
\(\lambda x:\I.\Lambda \alpha. \TB_\alpha \lambda y:\I.M\) is a valid
term (here, \(\I\) is used at \(\varepsilon\) and \(\alpha\)).
Similarly for indexed types such as Vector 4.  On the other hand, it
is not clear how a type indexed by a variable, which can be used only
at a declared stage, can be used.  For example,
consider
\[\TB_\alpha (\lambda x:\I. (\TBL_\alpha (\lambda y:\text{Vector
  }x.M)N) )
  \text{ and }
  \lambda x:\I. \TB_\alpha (\lambda y:\text{Vector }x.M).
\]
Is Vector\ \(x\) a legitimate type at \(\varepsilon\) (and \(\alpha\),
resp.)  even if \(x:\I\) is declared at stage \(\alpha\) (and
\(\varepsilon\), resp.)?  We will give our answer to this question in two
steps.

First, type-level constants such as \(\I\) and Vector can be used at
every stage in \LMD.  Technically, we introduce a signature that
declares kinds of type-level constants and types of constants.  For
example, a signature for the Boolean type and constants is given as
follows $\B::*, \text{true}:\B, \text{false}:\B$ (where $*$ is the
kind of proper types).  Declarations in a signature are not
associated to particular stages; so they can be used at every stage.

Second, an indexed type such as Vector\ 3 or Vector\ $x$ is well
formed only at the stage(s) where the index term is well-typed.  Since
constant \(3\) is well-typed at every stage (if it is declared in the
signature), Vector\ 3 is a well-formed type at every stage, too.
However, Vector\ $M$ is well-formed only at the stage where index term
$M$ is typed.  Thus, the kinding judgment
of \LMD takes the form \(\G\V \tau :: K @ A\), where stage $A$ stands for
where \(\tau\) is well-formed.  For example,
given \(\text{Vector}:: \I \rightarrow *\) in the signature \(\Sigma\),
\(x:\I@\varepsilon \V \text{Vector }x :: * @\varepsilon\) can be
derived but
neither
\(x:\I@\alpha \V \text{Vector }x :: * @\varepsilon\)
nor 
\(x:\I@\varepsilon \V \text{Vector }x :: * @\alpha\)
can be.

At first, the restriction above sounds too severe, because a term like
\(\lambda x:\I. \TB_\alpha (\lambda y:\text{Vector }x.M) \), which
models a typical code generator which takes the size $x$ and returns
code for vector manipulation specialized to the given size.  It seems
crucial for \(y\) to be given a type indexed by $x$.  We can address
this problem by CSP---In fact,
$\text{Vector }x$ is not well formed at $\alpha$ under $x:\I@\varepsilon$
but $\text{Vector }(\%_\alpha x)$ is!  So, we can still write
\(\lambda x:\I. \TB_\alpha (\lambda y:\text{Vector }(\%_\alpha x).M) \)
for the typical sort of code generators.

Our decision that well-formedness of types takes stages of index terms
into account will lead to the introduction of CSP at the type level
and special equivalence rules, as we will see later.

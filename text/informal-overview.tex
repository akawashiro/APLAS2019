% !TEX root = ../main.tex

\section{Informal Overview of \LMD}

We designed \LMD, a multi-stage calculus with dependent type.
\LMD is based on \LTP\cite{Hanada2014} by Hanada and Igarashi, which is a multi-stage calculus with CSP and
we introduced dependent types based on \LTP\cite{attapl}.
In this section, we check \LMD informally after checking \LTP and \LLF which are the basis of \LMD.

\red{\LTP\cite{Hanada2014}が頻発するが同一論文の複数回引用に何かルールはあるのか?}
\subsection{\LTP}

% quote and unquote

\LTP\cite{Hanada2014} is a multi-stage calculus with CSP by Hanada and Igarashi.
In \LTP, brackets and escape are written $\TB_\alpha M$ and $\TBL_\alpha M$, respectively.
The type of $\TB_\alpha M$ is $\TW_\alpha \tau$ if $M$ has type $\tau$.
The type of $\TBL_\alpha M$ is $\tau$ if $M$ has $\TW_\alpha \tau$.
% Please notice that if $\TBL_\alpha M$ is well-typed, $M$ has a code type from the typing rule of \LTP.
In addition to normal $\beta$-reduction, there is a reduction rule for brackets and escape.
\begin{align*}
	\TBL_\alpha (\TB_\alpha M) \longrightarrow M 
\end{align*}
It means escape cancels brackets.
% This reduction is called $\longrightarrow_\Lambda$ in \LMD.
% \red{この文はここに来るべきなのか? LTPの説明ではないが}

% transition variable and transition

The subscript $\alpha$ in $\TB_\alpha M$ is a \textit{transition variable} and
a sequence of transition variables is called \textit{transition}.
The empty transition is represented by $\epsilon$.
They are used to show the thickness of brackets.
For example, $\TB_\alpha (\lambda x:\I.x+10)$ is the fragment of code which becomes $(\lambda x:\I.x+10)$ after it was run once and
$\TB_\alpha \TB_\beta (\lambda x:\I.x+10)$, which is abbreviated as $\TB_{\alpha\beta} (\lambda x:\I.x+10)$,
becomes $(\lambda x:\I.x+10)$ after it was run it twice.
\red{run twiceは表現が微妙}

% transition abstraction and application

There are abstractions for transition variables and applications for transition abstraction in \LTP.
They look like $\Lambda\alpha.M$ and $M A$, respectively.
A transition abstraction binds a transition variable in a term.
For example, all $\alpha$ in $\Lambda\alpha.(\TB_\alpha (\lambda x:\I.x))$ are bound.
It is only natural there is a reduction rule for transition application in \LTP.
The rule is the following.
\begin{align*}
	(\Lambda\alpha.M)\ A \longrightarrow M[\alpha\mapsto A]
\end{align*}
For example, $\Lambda\alpha.(\TB_\alpha (\lambda x:\I.x))\ (\beta\gamma)$ reduces to $\TB_{\beta\gamma} (\lambda x:\I.x)$.

% transition-related symbols disappear when the empty transition is substituted.
% In short, about "run"

Another important rule about a transition variable is 
that symbols with transition variables disappear 
when the empty transition $\epsilon$ is substituted to the transition variable.
The purpose of this rule is express \verb|run| without any special symbol.
In multi-stage calculus, \verb|run| is a very important operator which changes quoted code to the original code.
For example, $(\TB_\epsilon (\lambda x:\I.x))$ is equivalent to $(\lambda x:\I.x)$.
Therefore, $\Lambda\alpha.(\TB_\alpha (\lambda x:\I.x))\ \epsilon$ becomes $\lambda x:\I.x$
In \LTP, \verb|run| is realized with application to the empty transition $\epsilon$.

% transition in typing rules

A type judgement of \LTP is of the form $\G \vdash M : \tau @ A$.
A transition in a judgement represents where the judgement is true.
For example, $\G \vdash (\lambda x:\I.x) : \I \to \I @ \alpha$ means 
term $\lambda x:\I.x$ has type $\I \to \I$ at transition $\alpha$.
Especially, terms without quoting exist at the empty transition $\epsilon$.
For example, $(\lambda x:\I.x)\ (1+2)$ is at $\epsilon$ transition and 
$\TB_\alpha (\lambda x:\I.x)$ is at $\alpha$ transition.
Terefore, transitions appear in typing rules, too.
\begin{center}
	\infrule{\G\vdash M:\tau @{A\alpha}}{\G\vdash \TB_{\alpha}M:\TW_{\alpha}\tau @A}{\TTB} \andalso
	\infrule{\G\vdash M:\TW_{\alpha}\tau @A}{\G\vdash \TBL_{\alpha}M:\tau @{A\alpha}}{\TTBL}
\end{center}
\TTB, corresponding to brackets, means 
if $M$ is typed $\tau$ at transition $A\alpha$ then $\TB_{\alpha}M$, quoted $M$, is typed $\TW_{\alpha}\tau$ at $A$.
\TTBL\ is converse of \TTB.

% CSP

CSP, cross-stage persistence, is an important feature of \LTP.
It enables us to embed value at an outer transition into an inner transition.
$\%$ is dedicated to CSP in \LTP.
For example, $\lambda a:\I.\Lambda\alpha.(\TB_\alpha (\lambda x:\I.x+\%_\alpha a))$
\red{inner / outer は適切か?}

% Omitting Residualization
% この段落は3章のM eのあとに、この制限の結果として...という形で入れる

% There is another important feature called program residualization in \LTP.
% It means that a generated code can be dumped into a file.
% We can load the dumped file and run it.
% The difficulty arises when program residualization is used with CSP.
% Transition variables are classified into two kinds in \LTP in order to deal with this difficulty.

\subsection{\LLF}

% \LLF
\LLF is a simple system of dependent types introduced in \cite{attapl}.
\red{system / type system / calculus?}
It is based on Edinburgh LF\cite{harper1993framework}.
Therefore, all constants and base types are declared in the signature.
The \LLF type theory generalizes simply typed lambda calculus
by replacing the function type $\tau\to\sigma$ with the dependent product type $\Pi x:\tau.\sigma$.

% Kind, Well-formed kind
In addition to ordinary typing rules like simply typed lambda calculus,
there are kinding rules, well-formed kinding rules, term equivalence rules, type equivalence rules, and kind equivalence rules in \LLF.
Kinding rules and well-formed kinding rules are 
introduced in order to prohibit making illegal types such as $\textrm{Vect}\ \textrm{Bool}$.
For a well-formed type $\tau$, $\G \vdash \tau :: K$ means that $\tau$ has a kind $K$ under the environment $\G$ and 
for a well-formed kind $K$, $\G \vdash K$ means that $K$ is a well-formed kind under an environment $\G$.

% Type Equality
Type equality rules are needed because the type equivalence is not obvious unlike simply typed lambda calculus.
For example, $\textrm{Vect}\ 7$ should be equivalent to $\textrm{Vect}\ (3+4)$
but they are not equivalent seemingly. Thus, we must define equivalence rules.
In \LLF, equivalence is expressed with a symbol of $\E$.
$\G \vdash M \E N$ means a term $M$ and a term $N$ are equivalent under the environment $\G$.
$\G \vdash \tau \E \sigma$ means a type $\tau$ and a type $\sigma$ are equivalent under the environment $\G$.
$\G \vdash K \E J$ means a kind $K$ and a kind $J$ are equivalent under the environment $\G$.

\subsection{Extending \LTP with Dependent Types}

Next, we develop \LMD by extending \LTP with \LLF-like dependent types.
From here, we use the word "stage" instead of "transition" 
because we develop a multi-stage calculus, not a multi-transition calculus.
\red{stageのほうが言葉としてふさわしいと言いたい}
There are three technical points in the extension from \LTP to \LMD.

% Constants and Base Types

First, the way of handling of constants and type-level constants is the difference between \LMD and \LTP.
We adopt a signature $\Sigma$ to handle constants and type-level constants.
This is because a signature simplifies kinding rules relating to type variables.
A signature $\Sigma$ is composed of pairs of a base type and its kind or a constant and its type.
For example, if you want to use boolean, $\Sigma = \B::*, \text{true}:\B, \text{false}:\B$
\red{具体的な導出例を出したほうがよいか? また、なぜsimpleになるのを書くべきか?}

% Kidinding and Well-formed Kinding Rules

Second, we need kinding rules and well-formed kinding rules in order to extend \LMD with dependent types.
It was lucky that almost all rules are determined easily.
This is because multi-stage calculus and dependent types are almost orthogonal.
\red{orthogonal は抽象的すぎるか?}
Therefore, we can get kinding rules and well-formed kinding rules of \LMD just by 
attaching stage anotations to ones of \LLF.
For example, \KAbs-LF is a kinding rule for a dependent type in \LLF and \KAbs\ is a corresponding one.
\begin{center}
	\infrule{\G\vdash \tau :: * \andalso \G,x:\tau @A\vdash \sigma::J}{\G\vdash(\Pi x:\tau.\sigma) :: (\Pi x:\tau.J)}{\KAbs-LF} \\[2mm]
	\infrule{\G\V \tau :: *@A \andalso \G,x:\tau @A\V \sigma::J@A}{\G\V(\Pi x:\tau.\sigma) :: (\Pi x:\tau.J)@A}{\KAbs} \\[2mm]
\end{center}

% Equivalence Rules

Third, we also need type equivalence rules in \LMD because its type system contains dependent types.
For example, vadd function, which takes a length of vector and returns addition function for vectors with the length, is following.
\begin{align*}
	  & \textbf{let}\ \text{vadd}_1 : \Pi n:\I.\TW_\alpha\Vpn\to\TW_\alpha\Vpn\to\TW_\alpha\Vpn                 \\
	  & \hspace{7mm} = \textbf{fix}\ f.\lambda n:\I.\ \lambda v_1:\TW_\alpha\Vpn.\ \lambda v_2:\TW_\alpha\Vpn. \\
	  & \hspace{14mm} \textbf{if}\ n = 0                                                                     \\
	  & \hspace{14mm} \textbf{then} \TB_\alpha \text{nil}                                                    \\
	  & \hspace{14mm} \textbf{else}\ \TB_\alpha (                                                            \\
	  & \hspace{21mm} \textbf{let}\ t_1 = \text{tail}\ (\TBL_\alpha v_1)\ \textbf{in}                        \\
	  & \hspace{21mm} \textbf{let}\ t_2 = \text{tail}\ (\TBL_\alpha v_2)\ \textbf{in}                        \\
	  & \hspace{21mm} \text{cons}\ (\text{head}\ t_1 + \text{head}\ t_2) (\TBL_\alpha f\ (n-1)\ t_1\ t_2))   \\
\end{align*}
\begin{align*}
	  & \textbf{let}\ \text{vadd}_\alpha : \Pi n:\I.\TW_\alpha(\Vpn\to\Vpn\to\Vpn)                                                                      \\
	  & \hspace{7mm} = \lambda n:\I.\TB_\alpha (\lambda v_1:\Vpn.\ \lambda v_2:\Vpn. \TBL_\alpha \text{vadd}_1\ n\ (\TB_\alpha v_1)\ (\TB_\alpha v_2)) \\
\end{align*}
Although there are new primitives relating to stages which aren't in \LLF,
we can design all rules except \QPercent\ easily.
\red{design?}
\red{この段落が短い。後ろとくっつけてもいいが、段落の趣旨がボケる。}

% Q-% rule

The biggest problem in extension is handling the CSP symbol $\%_\alpha$ in \LMD.
The equivalence simple rule to handle CSP is \QCsp.
\begin{center}
	\infrule{\G\V M \E N : \tau @A}{\G\V\%_\alpha M \E \%_\alpha N : \tau @{A\alpha}}{\QCsp}
\end{center}
However, this rule isn't enough when the parameters of dependent types are cross-staed.
We will discuss this problem in a later section and solve this problem with the new rule \QPercent.
\begin{center}
	\infrule{\G\V M:\tau @{A\alpha} \andalso \G\V M:\tau @A}{\G\V\%_\alpha M \E M : \tau @{A\alpha}}{\QPercent}
\end{center}
\red{この段落でQ-Cspの持つ問題点を指摘すべきか? ただ、ここでmulmatの例を使うと3章と被る。}


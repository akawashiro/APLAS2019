% !TEX root = ../main.tex

\section{Formal Definition of \LMD}

In this section, we give a formal definition of \LMD, including
the syntax, full reduction, and type system.  In addition to full reduction,
in which any redex at any stage can be reduced, we also give staged reduction,
which models execution as programs.

\subsection{Syntax}

% \AI{Rename ``type variables'' to ``type-level constants''. }
% \AI{I prefer $\varepsilon$ for the nonempty sequence.}  
We assume the denumerable set of type-level constants, ranged over by
metavariables \(X, Y, Z\), the denumerable set of variables, ranged
over by \(x,y,z\), the denumerable set of constants, ranged over by
\(c\), and the denumerable set of stage variables, ranged over by
\(\alpha, \beta, \gamma\).  The metavariables \(A, B, C\) range over
sequences of stage variables; we write \(\varepsilon\) for the empty
sequence. \LMD is defined by the following grammar:

% \AI{``Transition'' is still used.}

\begin{align*}
	% \textrm{Type variables}  &   &                          & X,Y,Z                                                                                                      \\
	% \textrm{Variables}       &   &                          & x,y,z                                                                                                      \\
	% \textrm{Stage variables} &   &                          & \alpha,\beta,\gamma                                                                                        \\
	% \textrm{Stage}           &   &                          & A,B,C                                                                                                      \\
	\textrm{Kinds}             &  & K,J,I,H,G                & ::= * \mid \Pi x:\tau.K                                                           \\
	\textrm{Types}             &  & \tau,\sigma,\rho,\pi,\xi & ::= X \mid \Pi x:\tau.\tau \mid \tau\ M \mid \TW_{\alpha} \tau \mid \F\alpha.\tau \\
	%     \textrm{Constants}       &   &                          & c                                                                                                          \\
	\textrm{Terms}             &  & M,N,L,O,P                & ::= c \mid x \mid \lambda x:\tau.M\ \mid M\ M \mid \TB_\alpha M                   \\
	                           &  &                          & \ \ \ \ \mid \TBL_\alpha M \mid \Lambda\alpha.M \mid M\ A \mid \%_\alpha M        \\
	\textrm{Signature}         &  & \Sigma                   & ::= \emptyset \mid \Sigma, X::K \mid \Sigma, c:\tau                               \\
	\textrm{Type environments} &  & \Gamma                   & ::= \emptyset \mid  \Gamma,x:\tau @A                                              \\
\end{align*}

% \AI{Add $M\,\alpha$ to terms.}

\AI{Is $O$ used as a metavariable?}
% \AI{Domain and FV are not defined (yet).  I think we should introduce the notion of well-formed type environments by prose below and assume every type environment is well formed.}

% Description on meta variables


% Kinds

A kind, which is used to classify types, is either $*$, the kind of
proper types (types that terms inhabit), or $\Pi x\colon\tau.K$, the kind
of type operators that takes $x$ as an argument of type $\tau$ and returns a type
of kind $K$.
% Types
% of terms have kind $*$ and dependent types have $\Pi$-kinds.  For
% example, $\lambda x:\I.x$ has type $\Pi x:\I.\I$, which has $*$ kind.
% \red{この段落は短いので型の段落と結合するか?}
% Types
A type is a type-level constant $X$, which is declared in the signature with its kind, a dependent function type $\Pi x:\tau_1.\tau_2$,
an application of a term to a type (operator) of $\Pi$-kind, a code type $\TW_\alpha$, or an $\alpha$-closed type $\F\alpha.\tau$.
An application of a term to a type (operator) of $\Pi$-kind is, for example, $\text{Vector}\ 10$
if type-level constant $\text{Vector}$ has the kind $\Pi x:\I.*$.
A code type $\TW_\alpha \tau$ denotes a code fragment of a term of type $\tau$.
An $\alpha$-closed type corresponds to a runnable code fragment.

% Type constructors tie strongly \AI{Is ``tie strongly'' standard?} in the order of 
An application of a term to a type (operator) of $\Pi$-kind: $\tau\ M$ connect tighter than
the type constructor $\TW_{\alpha}$ and $\TW_{\alpha}$ connect tighter than
$\Pi$ in dependent types such as $\Pi x:\tau.\tau$ and $\Pi$ connect tighter than
$\F$ in types for stage abstraction such as $\F\alpha.\tau$.
Therefore, $\F\alpha.\TW_{\alpha} \Pi x:\I.\text{Vector}\ 5$ is interpreted as
$\F\alpha.(\TW_{\alpha} (\Pi x:\I.(\text{Vector}\ 5)))$.
% \AI{I don't understand this rule...}

% Terms

Terms include ordinary (explicitly typed) \(\lambda\)-terms, constants,
whose type is declared in signature $\Sigma$, and the following five forms
related to multi-stage programming:
$\TB_\alpha M$ represents a code fragment; $\TB_\alpha M$ represents escape;
$\Lambda\alpha.M$ is a stage variable abstraction;
$M\ A$ is an application of stage $A$ to a stage abstraction; and
$\%_\alpha M$ is an operator for cross-stage persistence.
The prefix operators $\TB_\alpha, \TBL_\alpha$, and $\%_\alpha$ connect tighter than the two forms of applications
and applications are left-associative
and two abstractions extends as far to the right as possible.
For example, $\F\alpha.\lambda x:\I.\TB_\alpha x\ y$ means $\F\alpha.(\lambda x:\I.(\TB_\alpha x)\ y)$.
% \AI{I think the usage of ``application'' is wrong again} 

% Signature

We adopt a \LLF-like system, where constants and type-level constants are given in the signature $\Sigma$, which
are a sequence of declarations of the form $c:\tau$ and $X:K$.
For example, when we use integers in \LMD, $\Sigma = \B :: *, \textrm{true}:\B, \textrm{false}:\B$.
Please note that constants or type-level constants has no stage.
This means that we can use them at any stage.
% Type environments
Type environments are sequences of triples of a variable, its type, and its stage.
We define well-formed signature and well-formed type environments later.

% Free variables
As usual, the variable $x$ is bound in $\lambda x:\tau.M$ or $\Pi x:\tau.\tau$ and
and the stage variable $\alpha$ is bound in $\Lambda \alpha.M$ or $\F\alpha.\tau$.
We sometimes abbreviate $\Pi x:\tau_1.\tau_2$ to $\tau_1 \rightarrow \tau_2$ if
$x$ is not a free variable in $\tau_2$.
We identify $\alpha$-convertible terms and assume the names of bound variables are pairwise distinct.
We write $\FV(M)$ and $\FTV(M)$ for the set of free variables and the set of free stage variables in $M$, respectively.
% \AI{This should be mentioned after free variables are introduced.}
% \AI{What about other binders such as $\Pi$?}

% \AI{We should note that constants and type-level constants are not staged.}
% \paragraph{Remark:}  \AI{
	
% Comparison with Hanada--Igarashi

As the end of this subsection, we compare \LMD and \LTP.
Terms of \LMD is the same with \LTP except constants.
The difference between \LMD and \LTP is their types and kinds.
First, \LMD contains $\Pi$ type instead of function type in \LTP.
Second, there two kinds of $\F$ in \LTP in order to disscuss on program residualization
although \LMD contains only one $\F$ type.
Third, \LMD contains kinding and type equivalence rules to handle dependent types.

\subsection{Reduction}

Next, we define full reduction for \LMD.
Before giving the definition of reduction, we define six kinds of substitutions.
Substitution $M[x\mapsto N], \tau[x \mapsto N]$ and $K[x \mapsto N]$ are 
the ordinary capture-avoiding substitutions of
term $N$ for $x$ in term $M$, type $\tau$, and kind $K$, respectively
and we omit their definition here.
Substitution $M[\alpha \mapsto A], \tau [\alpha \mapsto A], K[\alpha \mapsto A]$ and $B[\alpha\mapsto A]$ are 
substitutions of stage $A$ for stage variable $\alpha$ in 
term $N$ for $x$ in term $M$, type $\tau$, kind $K$, and stage $B$ respectively.
We show replesentive cases below.
% \AI{What about $\tau[x \mapsto N], K[x \mapsto N], \tau [\alpha \mapsto \varepsilon]$ and $K[\alpha \mapsto \varepsilon]$?}

% \AI{Since we always substitute $\varepsilon$ for stage variables, we may want to introduce a special notation, something like $\textit{erase}_\alpha(M)$.}
% \AI{We will use $[\alpha \mapsto \varepsilon]$ and $[\alpha \mapsto \beta]$.  The latter is renaming, so we can omit the definition.}

\begin{align*}
	(\lambda x:\tau.M)[\alpha \mapsto A] & = \lambda x:(\tau[\alpha \mapsto A]).(M[\alpha \mapsto A])                                  \\
	(M\ B)[\alpha \mapsto A]             & = (M[\alpha \mapsto A])\ B[\alpha\mapsto A]                                                 \\
	(\TB_\beta M)[\alpha \mapsto A]      & = \TB_{\beta[\alpha \mapsto A]}M[\alpha \mapsto A]                                          \\
	(\TBL_\beta M)[\alpha \mapsto A]     & = \TBL_{\beta[\alpha \mapsto A]}M[\alpha \mapsto A]                                         \\
	(\%_\beta M)[\alpha \mapsto A]       & = \%_{\beta[\alpha \mapsto A]}M[\alpha \mapsto A]                                           \\
	(\beta B)[\alpha \mapsto A]          & = \beta (B[\alpha\mapsto A])                               & (\text{if } \alpha \neq \beta) \\
	(\beta B)[\alpha \mapsto A]          & = A (B[\alpha\mapsto A])                                   & (\text{if } \alpha = \beta)
\end{align*}

\begin{definition}[Reduction]
	The relations $M \longrightarrow_\beta N$, $M \longrightarrow_\blacklozenge N$, and $M \longrightarrow_\Lambda N$
	are the least compatible relations closed under the rules below.
	Congruence rules which are omitted from the definition.
	\begin{align*}
		 & (\lambda x:\tau.M) N \longrightarrow_\beta M[x \mapsto N]                             \\
		 & \TBL_\alpha \TB_\alpha M \longrightarrow_\blacklozenge M                              \\
		 & (\Lambda \alpha.M)\ A \longrightarrow_\Lambda M[\alpha \mapsto A]
	\end{align*}
	We write $ M \longrightarrow M'$ iff $ M \longrightarrow_\beta M'$, $ M \longrightarrow_\blacklozenge M'$, or $ M \longrightarrow_\Lambda M'$ and
	we call $\longrightarrow_\beta$, $\longrightarrow_\blacklozenge$, and $\longrightarrow_\Lambda$ $\beta$-reduction, $\blacklozenge$-reduction, and $\Lambda$-reduction, respectively.
\end{definition}

The relation $\longrightarrow_\beta$ represents ordinary $\beta-$reduction in the \(\lambda\)-calculus; the relation
$\longrightarrow_\blacklozenge$ represents that quotation $\TB_\alpha M$ is canceled by escape and $M$ is spliced into the code fragment surrounding the escape;
the relation $\longrightarrow_\Lambda$ means that a stage abstraction applied to the empty stage $\varepsilon$ reduces to the body of abstraction
where $\varepsilon$ is substituted for the stage variable.
There is no reduction rule for CSP as with Hanada and Igarashi \cite{Hanada2014}.
The CSP operator $\%_\alpha$ disappears when $\varepsilon$ is substituted to $\alpha$.
We show an example of reduction sequence below.
Underlines show the redexes.
\begin{align*}
	 & \hspace{10mm} \underline{(\lambda i:\I\to\I.(\Lambda\alpha.\TB_\alpha (\%_\alpha i\ 1 + (\TBL_\alpha \TB_\alpha 3))\ \varepsilon))\ (\lambda x:\I.x)} \\
	 & \longrightarrow_\blacklozenge (\Lambda\alpha.\TB_\alpha (\%_\alpha (\lambda x:\I.x)\ 1 + (\underline{\TBL_\alpha \TB_\alpha 3})))\ \varepsilon        \\
	 & \longrightarrow_\beta \underline{(\Lambda\alpha.\TB_\alpha (\%_\alpha (\lambda x:\I.x)\ 1 + 3))\ \varepsilon}                                         \\
	 & \longrightarrow_\Lambda \underline{(\lambda x:\I.x)\ 1} + 3                                                                                           \\
	 & \longrightarrow_\beta 1 + 3                                                                                                                           \\
	 & \longrightarrow^* 4
\end{align*}

% $\TB_\alpha$ and $\TBL_\alpha$ disappears in the same way as $\%_\alpha$.

\subsection{Type System}

% 総論

In this section, we define the type system of \LMD.
% \LMD is little complicated because it contains dependent types.
It consists of six judgment forms for typing, kinding, well-formed kinding, term equivalence, type equivalence, and kind equivalence.
We list the judgments forms in Figure~\ref{fig:LMD-six-judgments}.
% \AI{equality vs. equivalence?} \red{Select equivalence}
% \AI{A few judgment forms do not match rules.}

\begin{figure}
	\begin{center}
		\begin{align*}
			\G & \V M : \tau @ A       & \text{typing}              \\
			\G & \V \tau :: K @ A      & \text{kinding}             \\
			\G & \V K \iskind @ A      & \text{well-formed kinding} \\
			\G & \V M \E N @ A         & \text{term equivalence}    \\
			\G & \V \tau \E \sigma @ A & \text{type equivalence}    \\
			\G & \V K \E J @ A         & \text{kind equivalence}
		\end{align*}
		\caption{Six judgment forms of the type system of \LMD.}
		\label{fig:LMD-six-judgments}
	\end{center}
\end{figure}

% \AI{Insert a tilde (which is a space without a line break) between ``Figure'' and ref.}
\begin{definition}[Typing, Kinding, Well-formed Kinding, Term Equivalence, Type Equivalence, Kind Equivalence]
	The typing relation $ \G \V M : \tau @ A $,
	the kinding relation $\G \V \tau :: K @ A$,
	the well-formed kinding relation $\G \V K \iskind @ A$,
	the term equivalence relation $\G \V M \E N @ A$,
	the type equivalence relation $\G \V \tau \E \sigma @ A$,
	the kind equivalence relation $\G \V K \E J @ A$,
	are the least relation closed under the rules in Figure~\ref{fig:typing-rules}.
\end{definition}
This definition is little complicated to define six relations at the same time.
We canot define them one by one because they are dependent on each other.

We introduce a well-formed signature and well-formed type environments.
$\textit{dom}$ is a function from a content to a set of variables defined in the context.
\begin{definition}[Well-formed Signature]
	\begin{itemize}
		\item $\emptyset$ is a well-formed signature.
		\item if $\Sigma$ is a well-formed signature and $\V K :: * @ \varepsilon$ and $X\notin\textit{dom}(\Sigma)$ \\
		      then $\Sigma,X::K$ is a well-formed signature.
		\item if $\Sigma$ is a well-formed signature and $\V \tau :: K @ \varepsilon$ and $c\notin\textit{dom}(\Sigma)$ \\
		      then $\Sigma,c:\tau$ is a well-formed signature.
	\end{itemize}
\end{definition}
\begin{definition}[Well-formed Type Environments]
	\begin{itemize}
		\item $\emptyset$ are well-formed type environments.
		\item if $\G$ are well-formed type environments and $x\not\in\textit{dom}(\G)$ and $x\not\in\textrm{FV}(\tau)$ \\
		      then $\G,x:\tau@A$ are well-formed type environments.
	\end{itemize}
\end{definition}
From here, we assume that all signature and type environments are well-formed.

We start from defining typing of
\LMD.  Typing rules of \LMD are shown in Figure~\ref{fig:typing-rules}.
The rule \TConst{} means any constants in the
signature can appear at any stage.  \
AI{This should be discussed earlier.}
For example, if we have a signature $\Sigma$ which is
$\textrm{bool} :: *, \textrm{true}: \textrm{bool}, \textrm{false}:
	\textrm{bool}$, the derivation tree in
Figure~\ref{fig:tconst-derivation-tree} is admissible.
The rules \TVar,\TAbs, and \TApp{} are almost the same as those in the simply typed
lambda calculus or \LLF.  Additional conditions are that subterms must be
typed at the same stage (\TAbs{} and \TApp); the type
annotation/declaration on a variable has to be a proper type of kind
$*$ (\TAbs) at the stage where it is declared (\TVar{} and \TAbs).
% \AI{Don't we need signature well formedness?  If $c:\tau \in \Sigma$, then $\tau$ should be a well-formed type under ... what?}
% \AI{... or \(\lambda\)LF?  We are working in a dependently type system...}  

\begin{figure}
	\begin{center}
		\begin{minipage}{0.50\hsize}
			\infer[\TConst]
			{\G \V \textrm{true}:\textrm{bool}@\alpha\beta}
			{\textrm{true}:\textrm{bool} \in \Sigma \andalso
				\ID{\G\V\textrm{bool}::*@\alpha\beta} \andalso
			}
			\caption{A derivation tree using \TConst}
			\label{fig:tconst-derivation-tree}
		\end{minipage}
	\end{center}
\end{figure}


% \TConv
\AI{Looks like a paper is written for those who know multi-stage calculi fairly well but don't know dependent type systems.  The reality is opposite.}
As in standard dependent type systems, \TConv{} allows us to replace
the type of a term with an equivalent one.
In a type system which includes dependent types, this kind of rule is essential
because two types which have different shapes may be equivalent.
For example, when we use a vector type with its size ($\textrm{Vector}\ n$),
$\textrm{Vector}\ 5$ is equivalent to $\textrm{Vector}\ (4+1)$ obviously although they are not equivalent in apparently.

% Typing rules for a multi-stage calculus
% \AI{This paragraph is not very informaticve.  We shouldn't just refer redears to previous work.}
The rules \TTB, \TTBL, \TGen, \TIns, and \TCsp{} are rules for a multi-stage calculus.
The rule \TTB{}, which corresponds to brackets, means that if term $M$ is of type $\tau$ at stage $A\alpha$,
$\TB_\alpha M$ is of type $\TW_\alpha \tau$ at stage $A$.
The rule \TTBL{}, which corresponds to escape, is the converse of \TTB.
The rule \TGen{} is a rule for stage abstraction.
The condition of $\alpha\notin\rm{FTV}(\G)\cup\rm{FTV}(A)$ ensure that
substitutions to $\alpha$ don't affect type environments or stages.
The rule \TIns{} is for applications of stages to stage abstractions.
The rule \TCsp is the rule for CSP, which means that 
if term $M$ is of type $\tau$ at stage $A$, $\%_\alpha M$ is of type $\tau$ at stage $A\alpha$.

\begin{figure}
	\begin{center}
		\infrule[\TConst]{c:\tau \in \Sigma \andalso \G\V \tau::*@A}{\G \V c:\tau @A} \\[2mm]
		\infrule[\TVar]{x:\tau @A \in \G \andalso \G\V \tau::*@A}{\G \V x:\tau @A} \\[2mm]
		\infrule[\TAbs]{\G\V \sigma::*@A\andalso\G,x:\sigma@A\V M:\tau @A}{\G\V(\lambda (x:\sigma).M):(\Pi (x:\sigma).\tau)@A} \\[2mm]
		\infrule[\TApp]{\G\V M:(\Pi (x:\sigma).\tau)@A \andalso \G\V N:\sigma@A}{\G\V M\ N : \tau[x\mapsto N]@A} \\[2mm]
		\infrule[\TConv]{\G\V M:\tau @A \andalso \G\V \tau\equiv \sigma :: K@A}{\G\V M:\sigma@A} \\[2mm]
		\infrule[\TTB]{\G\V M:\tau @{A\alpha}}{\G\V\TB_{\alpha}M:\TW_{\alpha}\tau @A} \andalso
		\infrule[\TTBL]{\G\V M:\TW_{\alpha}\tau @A}{\G\V\TBL_{\alpha}M:\tau @{A\alpha}} \\[2mm]
		\infrule[\TGen]{\G\V M:\tau @A \andalso \alpha\notin\rm{FTV}(\G)\cup\rm{FTV}(A)}{\G\V\Lambda\alpha.M:\forall\alpha.\tau @A} \\[2mm]
		\infrule[\TIns]{\G\V M:\forall\alpha.\tau @A}{\G\V M\ B:\tau[\alpha \mapsto B]@A} \andalso
		\infrule[\TCsp]{\G\V M:\tau @A}{\G\V \%_\alpha M:\tau @{A\alpha}}
		\caption{Typing Rules.}
		\label{fig:typing-rules}
	\end{center}
\end{figure}

\subsubsection{Type and Term Equivalence}

As usual in dependent type systems, type equivalence, used in \TConv,
is important.  The type equivalence judgment of the form
$\G \V \tau \E \sigma : K @ A$ means that types $\tau$ and $\sigma$
are equivalent as types of kind $K$ at stage $A$ under $\G$.
Figure~\ref{fig:type-equivalence-rules} shows the rules for type
equivalence.  Type equivalence is basically the least congruence
closed under term equivalence.  The rules for compatibility (closure
under type formation) are derived from corresponding typing rules in a
straightforward manner.  The rules are a little simpler than some
dependent type systems \AI{such as?}  because there is no abstraction
at the type level.

% \QCsp以外の説明
% We show type equivalence rules in Figure \ref{fig:type-equivalence-rules}.
% All rules except \QTRefl, \QTSym, \QTTrans, and \QTApp\ are generated naturally from the typing rules.
% \QTRefl, \QTSym, \QTTrans\ exist in order to make the type equivalence relation an equivalence relationship.
% The rule \QTApp\ means that if there are two equivalent $\Pi$ type and two equivalent terms,
% the results of applications are also equivalent.

\begin{figure}
	\begin{center}
		\infrule[{\QTAbs }]{\G\V \tau \E \sigma :: *@A \andalso \G,x:\tau @A \V \rho \E \pi :: *@A}{\G\V\Pi x:\tau.\rho \E \Pi x:\sigma.\pi :: *@A} \\[2mm]
		\infrule[\QTApp]{\G\V \tau \E \sigma :: (\Pi x:\rho.K)@A \andalso \G\V M \E N : \rho @A}{\G\V \tau\ M \E \sigma\ N :: K[x \mapsto M]@A} \\[2mm]
		\infrule[\textsc{QT-$\TW$}]{\G\V \tau \E \sigma :: *@{A\alpha}}{\G\V \TW_{\alpha} \tau \E \TW_{\alpha} \sigma :: *@A}\andalso
		\infrule[\QTCsp]{\G\V \tau \E \sigma :: K@A}{\G\V \tau \E \sigma :: K@{A\alpha}} \\[2mm]
		\infrule[\QTGen]{\G\V \tau \E \sigma :: *@A \andalso \alpha\notin\rm{FTV}(\G)\cup\rm{FTV}(A)}{\G\V \forall\alpha.\tau \E  \forall\alpha.\sigma :: *@A} \\[2mm]
		\infrule[\QTRefl]{\G\V \tau::K@A}{\G\V \tau\E\tau :: K@A} \andalso
		\infrule[\QTSym]{\G\V \tau \E \sigma :: K@A}{\G\V \sigma \E \tau :: K@A} \\[2mm]
		\infrule[\QTTrans]{\G\V \tau \E \sigma :: K@A \andalso \G\V \sigma \E \rho  :: K@A}{\G\V \tau \E \rho  :: K@A}
		\caption{Type Equivalence Rules.}
		\label{fig:type-equivalence-rules}
	\end{center}
\end{figure}

The term equivalence judgment of the form $\G \V M \E N : \rho @ A$,
which means that terms $M$ and $N$ are equivalent as terms of type
$\rho$ at stage $A$ under $\G$, is defined by the rules in
Figure~\ref{fig:term-equivalence-rules}.  Most rules are
straightforward.  The rules \QAbs, \QApp, \QTB, \QTBL, \QGen, \QIns,
\QCsp, \QRefl, \QSym, and \QTrans{} make the relation congruence; the
rules \QBeta, \QTBLTB, and \QLambda{} correspond to
$\beta$-reduction, $\blacklozenge$-reduction, and $\Lambda$-reduction, respectively.

\begin{figure}
	\begin{center}
		\infrule[\QAbs]{\G\V \tau \E \sigma :: *@A \andalso \G,x:\tau @A \V M \E N : \rho @A}{\G\V\lambda x:\tau.M \E \lambda x:\sigma.N : (\Pi x:\tau.\rho)@A} \\[2mm]
		\infrule[\QAbs]{\G\V M \E L : (\Pi x:\sigma.\tau)@A \andalso \G\V N \E O : \sigma@A}{\G\V M\ N \E L\ O : \tau[x \mapsto N]@A} \\[2mm]
		\infrule[\QTB]{\G\V M \E N : \tau @{A\alpha}}{\G\V \TB_\alpha M \E \TB_\alpha N : \TW_\alpha \tau @A} \andalso
		\infrule[\QTBL]{\G\V M \E N : \TW_\alpha \tau @A}{\G\V \TBL_\alpha M \E \TBL_\alpha N : \tau @{A\alpha}} \\[2mm]
		\infrule[\QGen]{\G\V M\E N : \tau @A \andalso \alpha \notin \FTV(\G)\cup\FTV(A)}{\G\V \Lambda\alpha.M \E \Lambda\alpha.N : \forall\alpha.\tau @A} \\[2mm]
		\infrule[\QIns]{\G\V M \E N:\forall\alpha.\tau @A}{\G\V M\ \varepsilon \E N\ \varepsilon : \tau[\alpha \mapsto \varepsilon]@A} \andalso
		\infrule[\QCsp]{\G\V M \E N : \tau @A}{\G\V\%_\alpha M \E \%_\alpha N : \tau @{A\alpha}} \\[2mm]
		\infrule[\QRefl]{\G\V M:\tau @A}{\G\V M\E M : \tau @A} \andalso
		\infrule[\QSym]{\G\V M\E N : \tau @A}{\G\V N\E M : \tau @A} \\[2mm]
		\infrule[\QTrans]{\G\V M\E N : \tau @A \andalso \G\V N\E L : \tau @A}{\G\V M\E L : \tau @A} \\[2mm]
		\infrule[\QBeta]{\G,x:\sigma@A\V M:\tau @A \andalso \G\V N:\sigma@A}{\G\V(\lambda x:\sigma.M)\ N\E M[x\mapsto N] : \tau[x \mapsto N]@A} \\[2mm]
		% \infrule{\G\V M:(\Pi x:\sigma.\tau)@A \andalso x\notin \text{FV}(M)}{\G\V(\lambda x:\sigma.M\ x)\E M: (\Pi x:\sigma.\tau)@A}{\QEta} \\[2mm]
		\infrule[\QLambda]{\G\V (\Lambda\alpha.M) : \forall\alpha.\tau @A}{\G\V (\Lambda\alpha.M)\ \varepsilon \E M[\alpha \mapsto \varepsilon] : \tau[\alpha \mapsto \varepsilon]@A} \\[2mm]
		\infrule[\QTBLTB]{\G\V M \E N : \tau @A}{\G\V \TBL_\alpha(\TB_\alpha M) \E N : \tau @A} \hfil
		\infrule[\QPercent]{\G\V M:\tau @{A\alpha} \andalso \G\V M:\tau @A}{\G\V\%_\alpha M \E M : \tau @{A\alpha}}
		\caption{Term Equivalence Rules.}
		\label{fig:term-equivalence-rules}
	\end{center}
\end{figure}

% \QPercentの説明
The only rule that deserves elaboration is the last rule \QPercent.
Intuitively, it means that the CSP operator applied to term $M$ can be
removed if $M$ is also well typed at the next stage \(A\alpha\).
For example, constants do not depend on the stage (see \TConst) and
so \(\G\V \%_\alpha c \E c : \tau @ A\alpha\) holds but variables
do depend on stages and so this rule does not apply.

Interestingly, Hanada and Igarashi~\cite{Hanada2014} rejected the idea of
reduction to remove $\%_\alpha$ when they developed \LTP{}, as such
reduction does not match the operational behavior of the CSP operator
in implementation.  However, as an equational system for multi-stage
programs, the rule \QPercent makes sense and as we argue next
it is practically significant.


%%%%%%%%%%%%%%%%%%% SKIP FROM HERE %%%%%%%%%%%%%%%%%%%%
$\text{mulmat}$ function in the following code fragment generate code for matrix multiplication.
$\text{mulmat}$ takes two integers which are the size of the multiplicand matrix and generate a code.
The last integer to decide the size of the multiplier matrix is given at runtime.
You can generate code by applying two integers to $\text{mulmat}$.
We applied 3 and 5 in the second line and got $\TW_\alpha \Pi z:\I.(\M\ z\ \%_\alpha 5) \to (\M\ \%_\alpha 5\ \%_\alpha 3) \to (\M\ z\ \%_\alpha 3)$.
But this type is difficult to combine with other code because there are two $\%_\alpha$ for CSP.
\QPercent\ states that we can erase this $\%_\alpha$ under a condition which is explained in the next paragraph.

The condition is that a CSPed value equals to the original value when it has the same type in the original stage.
For example, $\V \%_\alpha 5 \E 5 @ \alpha$ because $ \V 5 : \I @ \alpha $ from \TConst\ and  $ \V \%_\alpha 5 : \I @ \alpha$.
In other words, we can remove a $\%_\alpha$ symbol of a value when it doesn't change the type.

	% Example of QCsp (mulmat)
	{
		\begin{align*}
			\text{mulmat}       & : \Pi x:\I.\Pi y:\I.(\TW_\alpha \Pi z:\I.(\M\ z\ \%_\alpha y) \to (\M\ \%_\alpha y\ \%_\alpha x) \to (\M\ z\ \%_\alpha x)) \\
			\text{mulmat}\ 3\ 5 & : \TW_\alpha \Pi z:\I.(\M\ z\ \%_\alpha 5) \to (\M\ \%_\alpha 5\ \%_\alpha 3) \to (\M\ z\ \%_\alpha 3)                     \\
			                    & (\E \TW_\alpha \Pi z:\I.(\M\ z\ 5) \to (\M\ 5\ 3) \to (\M\ z\ 3) )                                                         \\
		\end{align*}
	}

This kind of type equivalence is very useful when we combine $\text{mulmat}\ 3\ 5$ with another code.
We cannot remove $\%_\alpha$ in the type without \QPercent and it makes very difficult to combine
because types of codes don't contain $\%_\alpha$ generally.

%%%%%%%%%%%%%%%%%%% TO HERE %%%%%%%%%%%%%%%%%%%%

\subsubsection{Staged Semantics}

The reduction given above is full reduction and any redexes---even
under quotation---can be reduced in an arbitrary order.  
Following previous work~\cite{Hanada2014}, 
we introduce (small-step, call-by-value) staged semantics, 
where only $\beta$-reduction or $\Lambda$-reduction at stage or the outer-most $\blacklozenge$-reduction are allowed, 
modeling an implementation.

We start with the definition of values. Since terms under quotations are
not executed, the grammar is indexed by stages.

\begin{definition}[Values]
	The family $V^A$ of sets of values, ranged over by $v^A$,
	is defined by following grammar.  In the grammar, $A \neq \varepsilon$ is assumed.
	\begin{align*}
		v^\varepsilon \in V^\varepsilon & ::= \lambda x:\tau.M \mid\ \TB_\alpha v^\alpha \mid \Lambda\alpha.v^\varepsilon                                       & \\
		v^A \in V^A                     & ::= x \mid \lambda x:\tau.v^A \mid v^A\ v^A \mid\ \TB_\alpha v^{A\alpha} \mid \Lambda\alpha.v^A \mid v^A\ \varepsilon & \\
		                                & \quad\   \mid\ \TBL_\alpha v^{A'} (\text{if } A = A'\alpha \text{ for some } \alpha, A' \neq \varepsilon)             & \\
		                                & \quad\   \mid\ \%_\alpha v^{A'} (\text{if } A = A'\alpha  \text{ for some } \alpha, A')
	\end{align*}
\end{definition}

Values at $\varepsilon$ stage are a $\lambda$-abstraction, a quoted code,
or a $\Lambda$ abstraction.  The body of a $\lambda$-abstraction can
be any term but the body of $\Lambda$-abstraction has to a value.  It
means that the body of $\Lambda$-abstraction must be evaluated.  The
side condition for $\TBL_\alpha v^{A'}$ means that escapes in a value
can appear only under nested quotations (because an escape under a
single quotation will splice the code value into the surrounding
code).  See Hanada and Igarashi~\cite{Hanada2014} for details.

\begin{definition}[Evaluation Context]
	The family of sets $ECtx^A_B$ of evaluation contexts, ranged over by $E^A_B$.
	In the grammar, $A$ is assumed to be nonempty (but $B,A'$ can be empty).
	\begin{align*}
		 & E^\varepsilon_B \in ECtx^\varepsilon_B ::= \square\ (\text{if\ } B = \varepsilon)
		\mid E^\varepsilon_B\ M \mid v^e\ E^\varepsilon_B \mid \TB_\alpha E^\alpha_B
		\mid \Lambda\alpha.E^\varepsilon_B \mid E^\varepsilon_B\ A'                                                                       \\
		 & E^A_B \in ECtx^A_B                     ::= \square\ (\text{if } A = B) \mid \lambda x:\tau.E^A_B \mid E^A_B\ M \mid v^A\ E^A_B \\
		 & \hspace{25mm} \mid E^\varepsilon_B \mid \TB_\alpha E^{A\alpha}_B \mid \TBL_\alpha E^{A'}_B \ (\text{where } A'\alpha = A)      \\
		 & \hspace{25mm} \mid \Lambda\alpha.E^\varepsilon_B \mid E^A_B\ B' \mid \%_\alpha\ E^{A'}_B \ (\text{where } A'\alpha = A)        \\
	\end{align*}
\end{definition}

\begin{definition}[Redex]
	the sets of $\varepsilon$-redexes (ranged over by $R^\varepsilon$) and $\alpha$-redexes (ranged over by $R^\alpha$) are defined by the following grammar.
	\begin{align*}
		 & R^\varepsilon ::= (\lambda x:\tau.M)\ v^\varepsilon \mid (\Lambda\alpha.v^\varepsilon)\ \varepsilon \\
		 & R^\alpha      ::=\ \TBL_\alpha \TB_\alpha M                                                         \\
	\end{align*}
\end{definition}

\AI{some more explanation of the syntax of evaluation contexts.}

We write $E^A_B[M]$ for the term obtained by filling the hole in $E^A_B$ by $M$
and define the staged reduction relation.

\begin{definition}[Staged Reduction]
	The staged reduction relation, written $M \longrightarrow_s M'$, is defined by
	the least relation closed under the rules below.
	% \begin{figure}
	% 	\begin{center}
	\begin{align*}
		E^A_\varepsilon [(\lambda x:\tau.M)\ v^\varepsilon] & \longrightarrow_s E^A_\varepsilon[M[x\mapsto v^\varepsilon]]      \\
		E^A_\varepsilon [(\Lambda\alpha.v^\varepsilon)\ A]  & \longrightarrow_s E^A_\varepsilon[v^\varepsilon[\alpha\mapsto A]] \\
		E^A_\alpha [\TBL_\alpha \TB_\alpha v^\alpha]        & \longrightarrow_s E^A_\alpha[v^\alpha]                            \\
	\end{align*}
	%    	\caption{Staged Reduction}
	%    	\label{fig:staged-reduction}
	%    \end{center}
	% \end{figure}
\end{definition}


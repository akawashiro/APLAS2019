\section{Properties of \LMD}

In this section, we prove \AI{use ``show'' because we omit proofs from the paper} basic properties of \LMD: preservation, strong normalization, confluence, and progress.
These theorem shows that the type system of \LMD is strong enough to ensure a typed term of \LMD never go wrong.
\AI{In this setting, a simple type system is already sufficient to guarantee
  ``a well-typed term never goes wrong.''  If the language has
  case analysis that uses dependent types, then
  we could say more.
}

% Substitution Lemma

Substitution Lemma in \LMD\ is little more complicated than an ordinary one 
because there are six types of judgment and two types of substitution in \LMD.
Substitution Lemma on Terms states that term substitution $[z \mapsto P]$ preserves
typing, kinding, well-formed kinding, term equivalence, type equivalence, and king equivalence.
Substitution Lemma on Terms states the same for stage substitution $[\beta\mapsto\epsilon]$.

\begin{lemma}[Substitution Lemma on Terms]
	\begin{flalign*}
		\text{If\ } \G,z:\xi @B \V M:\tau @A \text{\ and\ } \G\V P:\xi @B
		&\text{\ then\ } \G\V M[z \mapsto P]:\tau[z \mapsto P] @A.&\\
		\text{If\ } \G,z:\xi @B \V \tau::K @A \text{\ and\ } \G\V P:\xi @B
		&\text{\ then\ } \G\V \tau[z \mapsto P]::K[z \mapsto P] @A.&\\
		\text{If\ } \G,z:\xi @B \V K\iskind @A \text{\ and\ } \G\V P:\xi @B
		&\text{\ then\ } \G\V K[z \mapsto P] \iskind  @A.&\\
		\text{If\ } \G,z:\xi @B \V M\E N : \tau @A \text{\ and\ } \G\V P:\xi @B
		&\text{\ then\ } \G\V M[z \mapsto P]\E N[z \mapsto P] : \tau[z \mapsto P] @A.&\\
		\text{If\ } \G,z:\xi @B \V \tau\E \sigma : K @A \text{\ and\ } \G\V P:\xi @B
		&\text{\ then\ } \G\V \tau[z \mapsto P]\E \sigma[z \mapsto P] : K[z \mapsto P] @A.&\\
		\text{If\ } \G,z:\xi @B \V K\E J @A \text{\ and\ } \G\V P:\xi @B
		&\text{\ then\ } \G\V K[z \mapsto P]\E J[z \mapsto P] @A.&
	\end{flalign*}
\end{lemma}

\AI{Since the order of variable declarations is significant,
  we cannot assume $z$ is always declared at the end.}

\begin{proof}
	Straightforward induction on derivations.
\end{proof}

\begin{theorem}[Substitution Lemma on Stages]
	\begin{flalign*}
		\text{If\ } \G \V M:\tau @A
		&\text{\ then\ } \G[\beta \mapsto \epsilon]\V M[\beta \mapsto \epsilon]:\tau[\beta \mapsto \epsilon] @A[\beta \mapsto \epsilon].&\\
		\text{If\ } \G \V \tau::K @A
		&\text{\ then\ } \G[\beta \mapsto \epsilon]\V \tau[\beta \mapsto \epsilon]::K[\beta \mapsto \epsilon] @A[\beta \mapsto \epsilon].&\\
		\text{If\ } \G \V K\iskind @A
		&\text{\ then\ } \G[\beta \mapsto \epsilon]\V K[\beta \mapsto \epsilon] \iskind @A[\beta \mapsto \epsilon].&\\
		\text{If\ } \G \V M\E N : \tau @A
		&\text{\ then\ } \G[\beta \mapsto \epsilon]\V M[\beta \mapsto \epsilon]\E N[\beta \mapsto \epsilon] : \tau[\beta \mapsto \epsilon]  @A[\beta \mapsto \epsilon].&\\
		\text{If\ } \G \V \tau\E \sigma : K @A
		&\text{\ then\ } \G[\beta \mapsto \epsilon]\V \tau[\beta \mapsto \epsilon]\E \sigma[\beta \mapsto \epsilon] : K[\beta \mapsto \epsilon] @A[\beta \mapsto \epsilon].&\\
		\text{If\ } \G \V K\E J @A
		&\text{\ then\ } \G[\beta \mapsto \epsilon]\V K[\beta \mapsto \epsilon]\E J[\beta \mapsto \epsilon] @A[\beta \mapsto \epsilon].&
	\end{flalign*}
\end{theorem}

\begin{proof}
	Straightforward induction on derivations.
\end{proof}

% Inversion Lemma

The following three Inversion Lemmas are needed to prove main theorems.
As usual~\cite{TAPL}, Inversion Lemmas enable us to infer the types of subterms of a term from the type of the term.
They are composed of three parts.
The first part states that we can infer types of subterms.
The second and third parts state 
that when we have a type equivalence, we can infer the shape of type from the left to the right, and vice versa.
We need both of the second and the third because of the reflexivity of the type equivalence. \AI{Symmetry?  We can omit the third one.}

\begin{lemma}[Inversion Lemma for $\Pi$-types]
	If $\G \V (\lambda x:\sigma.M) : (\Pi x:\sigma'.\tau)@A$ then
	\begin{enumerate}
		\item $\G \V \sigma \E \sigma'@A$ and
		\item $\G ,x:\sigma@A\V M:\tau @A$.
	\end{enumerate}
	\item If $\G \V \rho \E (\Pi x:\sigma.\tau) : K @A$ then $\exists \sigma', \tau', K, J$ such that
	\begin{enumerate}
		\item $\rho = \Pi x:\sigma'.\tau'$,
		\item $\G \V \sigma \E \sigma' : K @A$, and
		\item $\G, x:\sigma@A\V \tau \E \tau' : J @A$.
	\end{enumerate}
	\item If $\G \V (\Pi x:\sigma.\tau) \E \rho : K @A$ then $\exists \sigma', \tau', K, J$ such that
	\begin{enumerate}
		\item $\rho = \Pi x:\sigma'.\tau'$,
		\item $\G \V \sigma \E \sigma' : K @A$, and
		\item $\G, x:\sigma@A\V \tau \E \tau' : J @A$.
	\end{enumerate}
\end{lemma}

\begin{proof}
	Straightforward induction on derivations.
\end{proof}

\begin{theorem}[Inversion Lemma for $\TW$-types]
	\begin{item}
	      \item If $\G \V \TB_\alpha M : \TW_\alpha \tau @A$ then $\G \V M : \tau @A$.
	      \item If $\G \V \rho \E  \TW_\alpha \tau : K @A$ then $\exists \tau', K, J$ such that
	      \begin{enumerate}
		      \item $\rho = \TW_\alpha \tau'$ and
		      \item $\G @A\V \tau \E \tau' : K @A$.
	      \end{enumerate}
	      \item If $\G \V \TW_\alpha \tau \E \rho : K @A$ then $\exists \tau', K, J$ such that
	      \begin{enumerate}
		      \item $\rho = \TW_\alpha \tau'$ and
		      \item $\G @A\V \tau \E \tau' : K @A$.
	      \end{enumerate}
	\end{item}
\end{theorem}

\begin{proof}
	Straightforward induction on derivations.
\end{proof}

\begin{theorem}[Inversion Lemma for $\forall$-types]
	\begin{item}
	      \item If $\G \V \Lambda\alpha.M : \forall\alpha.\tau @A$ then $\G \V M : \tau @A$ and $\alpha \notin \FTV(\G) \cup \FV(A)$.
	      \item If $\G \V \rho \E \forall\alpha.\tau : K @A$ then $\exists \tau', K$ such that
	      \begin{enumerate}
		      \item $\rho = \forall\alpha.\tau'$ and
		      \item $\G \V \tau \E \tau' : K @A$.
	      \end{enumerate}
	      \item If $\G \V \forall\alpha.\tau \E \rho : K @A$ then $\exists \tau', K$ such that
	      \begin{enumerate}
		      \item $\rho = \forall\alpha.\tau'$ and
		      \item $\G \V \tau \E \tau' : K @A$.
	      \end{enumerate}
	\end{item}
\end{theorem}

\begin{proof}
	Straightforward induction on derivations.
\end{proof}

% Preservation

Thanks to Substitution Lemmas and Inversion Lemmas, we can prove Preservation easily.
Preservations ensure that any one step reduction preserves type.

\begin{theorem}[Preservation]
	If $\G\V M:\tau @A$ and $M \longrightarrow M'$, then $\G\V M':\tau @A$.
\end{theorem}

\begin{proof}
	First, there are three cases for $M \longrightarrow M'$.
	They are $M \longrightarrow_\beta M'$, $M \longrightarrow_\Lambda M'$, and $M \longrightarrow_\blacklozenge M'$.
	For each case, we can use straightforward induction on derivations.
	Difficult cases are \TApp, \TTBL, and \TIns.
	We need Inversion Lemmas for them.
\end{proof}

% Strong Normalization

Strong Normalization is also important property which guarantee that
no typed term has an infinite reduction sequence.
We prove this thorem by translating \LMD to the symply typed lambda calculus.

\begin{theorem}[Strong Normalization]
	If $\G\V^A M:\tau$ then there is no infinite sequence of terms $(M_i)_{i\ge1}$ and 
	$M_i \longrightarrow M_{i+1}$ for $i\ge 1$.
\end{theorem}

\begin{proof}
	In order prove this theorem, we define a translation $\natural$ from \LMD\ to the symply typed lambda calculus.
	Second, we prove $\natural$ preserves typing and $\beta$-reductions.
	Then, we can prove Strong Normalization of \LMD\ from Strong Normalization of the symply typed lambda calculus.
\end{proof}

Confluence is a property that any reduction sequences from one typed term converge.
Because we have proved Strong Normalization, we can use Newman's Lemma to prove Confluence.

\begin{theorem}[Confluence]
	For any term $M$, if $M \longrightarrow^* M'$ and $M \longrightarrow^* M''$ then
	there exists $M'''$ that satisfies $M' \longrightarrow^* M'''$ and $M'' \longrightarrow^* M'''$.
\end{theorem}

\begin{proof}
	Because we proved Strong Normalization of \LMD, 
	we can use Newman's lemma to prove Confluence of \LMD.
	Then, what we must show is Weak Church-Rosser Property now.
	When we consider two different redexes in a \LMD term, they can only be disjoint, or one is a part of the other.
	In short, they are never overlapped each other.
	So, we can reduce one of them after we reduce another.
\end{proof}

Unique Decomposition ensure that
,for every typed term, we can find just one redex to reduce by the evaluation context or it is a value.
This theorem is important because it guarantee
that the evaluation context decides a redex to reduce deterministically.
Specifically speaking, this theorem guarantee that 
when you write a interpreter using the evaluation context of \LMD,
your interpreter works just as intended.

\begin{theorem}[Unique Decomposition]
	If $x:\tau @\epsilon \notin \G$ and $\G \V M : \tau @ A$ then either
	\begin{enumerate}
		\item $ M \in V^A$, or
		\item there exist $B, E^A_B$, and $R^B$ such that $M = E^A_B[R^B]$ with $B = \epsilon$ or $B = \beta$ for some $\beta$.
	\end{enumerate}
\end{theorem}

\begin{proof}
	We can prove by straightforward induction on derivations.
	Difficult cases are \TApp, \TTBL, and \TIns.
	We need Inversion Lemmas for them.
\end{proof}

Progress states $\longrightarrow_s$ defines appropriate reduction for typed terms of \LMD.
Thanks to Unique Decomposition, we can prove this theorem easily.

\begin{theorem}[Progress]
	If $x:\tau @\epsilon \notin \G$ and $\G \V M : \tau  @ A$ then
	$ M \in V^A $ or $M'$ exists such that $M \longrightarrow_s M'$.
\end{theorem}

\begin{proof}
	We can prove by straightforward induction on derivations.
	Difficult cases are \TApp, \TTBL, and \TIns.
	We need Inversion Lemmas for them.
\end{proof}

\AI{Explain the condition on $\G$.}


% This is samplepaper.tex, a sample chapter demonstrating the
% LLNCS macro package for Springer Computer Science proceedings;
% Version 2.20 of 2017/10/04
%
\newif\ifnotanonymous \notanonymousfalse
\newif\iffullversion \fullversiontrue
\documentclass[runningheads]{llncs}
%
\usepackage{graphicx}
\usepackage{color}
\usepackage{amsmath}
\usepackage{amssymb}
\usepackage{bcprules, proof}
\usepackage{fancybox}
\usepackage{mathtools}
\usepackage{float}
\usepackage{xparse}
\usepackage{lscape}
\usepackage{xspace}
\usepackage{url}

% Used for displaying a sample figure. If possible, figure files should
% be included in EPS format.
%
% If you use the hyperref package, please uncomment the following line
% to display URLs in blue roman font according to Springer's eBook style:
% \renewcommand\UrlFont{\color{blue}\rmfamily}

\newcommand{\red}[1]{\textcolor{red}{#1 }}
\newcommand{\blue}[1]{\textcolor{blue}{#1 }}

\newcommand{\LTP}{$\lambda^{\triangleright\%}$\xspace}
\newcommand{\LMD}{$\lambda^{\textrm{MD}}$\xspace}
\newcommand{\LLF}{$\lambda\textrm{LF}$\xspace}

\newcommand{\G}{\Gamma}
\newcommand{\D}{\Delta}
\newcommand{\V}{\vdash_\Sigma}
\newcommand{\VT}{\vdash\hspace{-.50em}\raisebox{0.28em}{\tiny{$\TB$}}}
\newcommand{\iskind}{\text{\ kind}}
\newcommand{\TW}{{\mathop{\triangleright}}}
\newcommand{\TWL}{{\mathop{\triangleleft}}}
\newcommand{\F}{\forall}
\newcommand{\TB}{{\mathop{\blacktriangleright}}}
\newcommand{\TBL}{{\mathop{\blacktriangleleft}}}
\newcommand{\E}{\equiv}
\newcommand{\FV}{\text{FV}}
\newcommand{\FTV}{\text{FSV}}

\newcommand{\WStar}{\textsc{W-Star}}
\newcommand{\WAbs}{\textsc{W-Abs}}
\newcommand{\WCsp}{\textsc{W-Csp}}
\newcommand{\WApp}{\textsc{W-App}}
\newcommand{\WTW}{\textsc{W-$\TW$}}

\newcommand{\KVar}{\textsc{K-Var}}
\newcommand{\KTConst}{\textsc{K-TConst}}
\newcommand{\KAbs}{\textsc{K-Abs}}
\newcommand{\KApp}{\textsc{K-App}}
\newcommand{\KConv}{\textsc{K-Conv}}
\newcommand{\KTW}{\textsc{K-$\TW$}}
\newcommand{\KTWL}{\textsc{K-$\TWL$}}
\newcommand{\KGen}{\textsc{K-Gen}}
\newcommand{\KCsp}{\textsc{K-Csp}}

\newcommand{\TConst}{\textsc{T-Const}}
\newcommand{\TVar}{\textsc{T-Var}}
\newcommand{\TAbs}{\textsc{T-Abs}}
\newcommand{\TApp}{\textsc{T-App}}
\newcommand{\TConv}{\textsc{T-Conv}}
\newcommand{\TTB}{\textsc{T-$\TB$}}
\newcommand{\TTBL}{\textsc{T-$\TBL$}}
\newcommand{\TGen}{\textsc{T-Gen}}
\newcommand{\TIns}{\textsc{T-Ins}}
\newcommand{\TCsp}{\textsc{T-Csp}}

\newcommand{\QKAbs}{\textsc{QK-Abs}}
\newcommand{\QKCsp}{\textsc{QK-Csp}}
\newcommand{\QKRefl}{\textsc{QK-Refl}}
\newcommand{\QKSym}{\textsc{QK-Sym}}
\newcommand{\QKTrans}{\textsc{QK-Trans}}

\newcommand{\QTAbs}{\textsc{QT-Abs}}
\newcommand{\QTApp}{\textsc{QT-App}}
\newcommand{\QTTW}{\textsc{QT-$\TW$}}
\newcommand{\QTGen}{\textsc{QT-Gen}}
\newcommand{\QTCsp}{\textsc{QT-Csp}}
\newcommand{\QTRefl}{\textsc{QT-Refl}}
\newcommand{\QTSym}{\textsc{QT-Sym}}
\newcommand{\QTTrans}{\textsc{QT-Trans}}

\newcommand{\QAbs}{\textsc{Q-Abs}}
\newcommand{\QApp}{\textsc{Q-App}}
\newcommand{\QTB}{\textsc{Q-$\TB$}}
\newcommand{\QTBL}{\textsc{Q-$\TBL$}}
\newcommand{\QGen}{\textsc{Q-Gen}}
\newcommand{\QIns}{\textsc{Q-Ins}}
\newcommand{\QCsp}{\textsc{Q-Csp}}
\newcommand{\QRefl}{\textsc{Q-Refl}}
\newcommand{\QSym}{\textsc{Q-Sym}}
\newcommand{\QTrans}{\textsc{Q-Trans}}
\newcommand{\QBeta}{\textsc{Q-$\beta$}}
\newcommand{\QEta}{\textsc{Q-$\eta$}}
\newcommand{\QTBLTB}{\textsc{Q-$\TBL\TB$}}
\newcommand{\QLambda}{\textsc{Q-$\Lambda$}}
\newcommand{\QPercent}{\textsc{Q-\%}}

\newcommand{\ID}[1]{\infer[]{#1}{\vdots}}
\newcommand{\MD}[1]{\mathcal{D}_#1}

\newcommand{\I}{\textrm{Int}}
\newcommand{\B}{\textrm{Bool}}
\newcommand{\M}{\textrm{Mat}}

\newcommand{\rulefbox}[1]{\fbox{\ensuremath{#1}} \hspace{1mm}}

\newcommand{\AI}[1]{\textcolor{red}{[#1 -- AI]}}

\begin{document}
%
\title{A Dependently Typed Multi-Stage Calculus%
	\ifnotanonymous\thanks{Supported by organization x.}\fi
}
%
%\titlerunning{Abbreviated paper title}
% If the paper title is too long for the running head, you can set
% an abbreviated paper title here
%
\ifnotanonymous
	\author{Akira Kawata\inst{1} \and
		Atsushi Igarashi\inst{2}\orcidID{0000-0002-5143-9764}}
	% %
	\authorrunning{A. Kawata, A. Igarashi}
	% % First names are abbreviated in the running head.
	% % If there are more than two authors, 'et al.' is used.
	% %
	\institute{Graduate School of Informatics, Kyoto University, Kyoto, Japan\\
		\email{akira@fos.kuis.kyoto-u.ac.jp} \and
		\email{igarashi@kuis.kyoto-u.ac.jp}
	}
\fi
%
\maketitle              % typeset the header of the contribution
%
\begin{abstract}

	We study a dependently typed extension of a multi-stage programming
	language \`a la MetaOCaml, which supports quasi-quotation and
	cross-stage persistence for manipulation of code fragments as
	first-class values and eval for execution of programs dynamically
	generated by the code manipulation.  Dependent types are expected to
	bring to multi-stage programming enforcement of strong
	invariants---beyond simple type safety---on the behavior of
	dynamically generated code.  An extension is, however, not trivial
	because a type system would have to take stages---roughly speaking,
	the number of surrounding quotations---of types into account.
	
	To rigorously study properties of such an extension, we develop
	\LMD, which is an extension of Hanada and Igarashi's typed calculus
	\LTP with dependent types, and prove its properties including
	preservation, confluence, strong normalization for full
	reduction, and progress for staged reduction.  Motivated
	by code generators such that the type of generated code depends on a
	value from outside of quotations, we argue the significance of
	cross-stage persistence in dependently typed multi-stage programming
	and certain type equivalence that are not directly derived from
	reduction rules.
	
	% A multi-stage calculus enables us to generate and execute codes at runtime.
	% It can improve the performance of programs by generating optimized codes for given inputs.
	% Dependent types are types dependent on values. 
	% A vector with its length is a famous example of dependent types and it enables us to omit boundary checking.
	% In this paper, we design \LMD by introducing dependent type into $\lambda^{\TW\%}$.
	% You can make more efficient programs from existing dependent typed programs with \LMD.
	% \LMD has a simple, substitution-based full-reduction semantics and enjoys basic properties of subject reduction, confluence, and strong normalization, and progress.
	% It also includes an evaluation context which satisfies unique decomposition.
	% The main technical points of this paper are how to deal with Cross Stage Persistence of multi-stage calculuses which allows using a value in quoted code in a dependent type system.
	% Especially, the way of handling CSP in equivalence rules of dependent types wasn't clear.
	% In this paper, we give reasonable equivalence rules to handle them.
	
	\keywords{multi-stage programming, cross-stage persistence, dependent types}
\end{abstract}

\blue{181 words. The abstract should briefly summarize the contents of the paper in 150--250 words.}
\blue{コードとかはボールドを使ったほうが良いかもしれない。}
\red{Suspicious senetences are colored red.}
%
%
%
\red{Multi-stage Programming と Multi-stage Calculus をどう使い分けるか?}

% !TEX root = ../main.tex

\section{Introduction}

\subsection{Multi-stage Programming}

% \subsubsection{多段階計算とは何か?}

Multi-stage programming enables programmers to generate and run fragments of code at runtime.
We can treat a piece of code as a value with multi-stage programming.
It means code fragments can be substituted into variables, combined and run at runtime.
Multi-stage programming improves the performance of programs by generating optimized code for a given input at runtime. \cite{taha2007gentle}
% MetaOCamlを例に挙げて多段階計算について説明する

MetaOCaml~\cite{calcagno2003implementing}\cite{oleg2014} provides features for multi-stage programming, which are brackets, escape, run, and CSP.
Brackets, written \verb|.< e >.| in MetaOCaml, makes a code value from \verb|e|.
Run, written \verb| run e | in MetaOCaml, runs a code value of \verb|e| and restore \verb|e|.
Escape, written \verb| .~ e | in MetaOCaml, expands a code value of \verb|e|.
Unlike run, escape is supposed to use in a fragment of code, that is, escape cannot be used as an alternative of run.
For example, the following MetaOCaml expression

\begin{verbatim}
      let plusone = .< fun x -> x + 1 >. in .< .~plusone 2 >.
\end{verbatim}
evaluates to \verb|.< (fun x -> x + 1) 2 >.|. We can get the result with run.
\begin{verbatim}
    run (let plusone = .< fun x -> x + 1 >. in .< .~plusone 2 >.)
\end{verbatim}
is reduced to 3.

% \subsubsection{CSPに関する補足説明}

Cross-stage persistence (CSP) is another primitive of multi-stage programming, which enable us to embed computed values into a code value.
This is an example of CSP.
\begin{verbatim}
               let plusone = fun x -> x + 1 in
               let y = plusone 2 in  .< y >.
\end{verbatim}
This program evaluates to \verb|.< 3 >.| not  \verb|.< plusone 2 >.|.
This is because the variable \verb|y| is introduced into a code fragment using CSP.
Therefore, the variable \verb|y| is embedded after it was calculated.
CSP is very important when we write practical programs 
because CSP enables us to use library functions in code fragments as in \verb|.<List.combine [1;2] ['a';'b']>.|.

% \subsubsection{多段階計算のメリットをべき乗の例を用いて説明する}

The main application of multi-stage calculi is program optimization. 
A famous example is power functions.
Usual power functions take two arguments, which are the base and the exponent.
We can make a specialized power function for given exponents with multi-stage programming and 
optimize them by unrolling a loop in functions for given exponents.
The optimized power function for the exponent of \verb|3| looks like \verb|power3 = fun x -> x * x * x|.
\verb|power3| is faster than ordinary \verb|power| function because it contains no loop.
Multi-stage programming can optimize functions which take more than two arguments 
by generating a code fragment optimized to a given argument.

% 多段階計算の型理論の既存研究の紹介

Existing type systems for multi-stage calculi ensure that
the generated code is well-typed and code generation and evaluation are performed in the intended order.
For example, the type system of MetaOCaml~\cite{taha1999multi} classify expressions in a program
with the depth of brackets in order to handle them appropriately.
In $\lambda^\triangleright$ by Tsukada and Igarashi~\cite{Tsukada},
they use environment classifier to classify and 
succeeded in recognizing code values which are run safely at type-checking time.

% ベクトル計算も高速化できる

Multi-stage programming can also optimize vector calculation.
For example, \verb|vadd| function, which takes two vectors and return the sum of them,
is implemented with a loop in many cases.
We can unroll \verb|vadd| function for a given vector length and optimize it with multi-stage programming.

% 多段階計算で生成したコードは特殊化されているが故に問題も多い

However, there is a sever problem in functions which are optimized by multi-stage programming.
Although unrolled \verb|vadd| function is optimized, we cannot use it for different vector length.
For example, when you optimize for the length of 5, you shouldn't use it for vectors of 3 lengths.
Otherwise, we will get a exception.
This problem is serious but existing type systems for multi-stage calculi cannot prevent it.

\subsection{Multi-stage Programming with Dependent Types}

% 依存型とは何か?

One natural idea to address this problem is the introduction of dependent types.
Dependent types are types which are dependent on values.
We can use dependent types for securer programming like Xi and Pfenning \cite{Xi98}.
For example, we can realize vectors with their sizes with them.
\begin{verbatim}
        Vector :: Int -> *
\end{verbatim}
\verb|Vector| is a type constructor which takes the length of vectors, that is, 
\verb|Vector 3| is a type for vectors whose lengths are 3.
If \verb|vadd| function has the type of \verb|Vector n -> Vector n -> Vector n| for any positive number \verb|n|,
we can confirm the two arguments of \verb|vadd| function has the same length.
% \red{secure programming? finer grained typing? とか}

As we pointed out in the above section,
functions optimized with multi-stage programming can take only restricted values as arguments.
When you optimize \verb|vadd| function for the length of 5, you should it only for 5 length vectors.
We introduce dependent types into a multi-stage calculus
so that the type system can guarantee optimized functions are used properly.
Although there is a trial to combine multi-stage programming and dependent type by Brady and Hammond~\cite{brady2006dependently},
they didn't give a formal definition and properties for the calculus.

In this paper, we design a new multi-stage calculus \LMD by 
merging dependent types into existing multi-stage calculus \LTP\cite{Hanada2014}.
\LMD is multi-stage calculus which contains dependent types.
\LMD makes multi-stage programming safer with the power of dependent types.
Our technical contribution are summarized as follows:
\begin{itemize}
        \item we give the formal definition of \LMD with its syntax, 
        type system which contains type equivalence, full, reduction, and staged reduction;
        \item for the full reduction, we prove preservation, strong normalization, and confluence
        \item fot the staged reduction, we prove unique decomposition and progress.
\end{itemize}

\subsection{Organization of the Paper}

The organization of this paper is the following.
Section \ref{sec:informal-overview} gives an informal overview of \LMD.
Section \ref{sec:formal-definition} defines \LMD and
Section \ref{sec:properties} shows the properties of \LMD including preservation, strong normalization, confluence, and progress.
Section \ref{sec:related-work} discusses related work.
Finally, we conclude this paper in Section \ref{sec:conclusion} and discusses future work.


% !TEX root = ../main.tex

\section{Informal Overview of \LMD}

We designed \LMD, a multi-stage calculus with dependent type.
\LMD is based on \LTP\cite{Hanada2014} by Hanada and Igarashi, which is a multi-stage calculus with CSP and
we introduced dependent types based on \LTP\cite{attapl}.
In this section, we check \LMD informally after checking \LTP and \LLF which are the basis of \LMD.

\red{\LTP\cite{Hanada2014}が頻発するが同一論文の複数回引用に何かルールはあるのか?}
\subsection{\LTP}

% quote and unquote

\LTP\cite{Hanada2014} is a multi-stage calculus with CSP by Hanada and Igarashi.
In \LTP, brackets and escape are written $\TB_\alpha M$ and $\TBL_\alpha M$, respectively.
The type of $\TB_\alpha M$ is $\TW_\alpha \tau$ if $M$ has type $\tau$.
The type of $\TBL_\alpha M$ is $\tau$ if $M$ has $\TW_\alpha \tau$.
% Please notice that if $\TBL_\alpha M$ is well-typed, $M$ has a code type from the typing rule of \LTP.
In addition to normal $\beta$-reduction, there is a reduction rule for brackets and escape.
\begin{align*}
	\TBL_\alpha (\TB_\alpha M) \longrightarrow M 
\end{align*}
It means escape cancels brackets.
% This reduction is called $\longrightarrow_\Lambda$ in \LMD.
% \red{この文はここに来るべきなのか? LTPの説明ではないが}

% transition variable and transition

The subscript $\alpha$ in $\TB_\alpha M$ is a \textit{transition variable} and
a sequence of transition variables is called \textit{transition}.
The empty transition is represented by $\varepsilon$.
They are used to show the thickness of brackets.
For example, $\TB_\alpha (\lambda x:\I.x+10)$ is the fragment of code which becomes $(\lambda x:\I.x+10)$ after it was run once and
$\TB_\alpha \TB_\beta (\lambda x:\I.x+10)$, which is abbreviated as $\TB_{\alpha\beta} (\lambda x:\I.x+10)$,
becomes $(\lambda x:\I.x+10)$ after it was run it twice.
\red{run twiceは表現が微妙}

% transition abstraction and application

There are abstractions for transition variables and applications for transition abstraction in \LTP.
They look like $\Lambda\alpha.M$ and $M A$, respectively.
A transition abstraction binds a transition variable in a term.
For example, all $\alpha$ in $\Lambda\alpha.(\TB_\alpha (\lambda x:\I.x))$ are bound.
It is only natural there is a reduction rule for transition application in \LTP.
The rule is the following.
\begin{align*}
	(\Lambda\alpha.M)\ A \longrightarrow M[\alpha\mapsto A]
\end{align*}
For example, $\Lambda\alpha.(\TB_\alpha (\lambda x:\I.x))\ (\beta\gamma)$ reduces to $\TB_{\beta\gamma} (\lambda x:\I.x)$.

% transition-related symbols disappear when the empty transition is substituted.
% In short, about "run"

Another important rule about a transition variable is 
that symbols with transition variables disappear 
when the empty transition $\varepsilon$ is substituted to the transition variable.
The purpose of this rule is express \verb|run| without any special symbol.
In multi-stage calculus, \verb|run| is a very important operator which changes quoted code to the original code.
For example, $(\TB_\varepsilon (\lambda x:\I.x))$ is equivalent to $(\lambda x:\I.x)$.
Therefore, $\Lambda\alpha.(\TB_\alpha (\lambda x:\I.x))\ \varepsilon$ becomes $\lambda x:\I.x$
In \LTP, \verb|run| is realized with application to the empty transition $\varepsilon$.

% transition in typing rules

A type judgement of \LTP is of the form $\G \vdash M : \tau @ A$.
A transition in a judgement represents where the judgement is true.
For example, $\G \vdash (\lambda x:\I.x) : \I \to \I @ \alpha$ means 
term $\lambda x:\I.x$ has type $\I \to \I$ at transition $\alpha$.
Especially, terms without quoting exist at the empty transition $\varepsilon$.
For example, $(\lambda x:\I.x)\ (1+2)$ is at $\varepsilon$ transition and 
$\TB_\alpha (\lambda x:\I.x)$ is at $\alpha$ transition.
Terefore, transitions appear in typing rules, too.
\begin{center}
	\infrule{\G\vdash M:\tau @{A\alpha}}{\G\vdash \TB_{\alpha}M:\TW_{\alpha}\tau @A}{\TTB} \andalso
	\infrule{\G\vdash M:\TW_{\alpha}\tau @A}{\G\vdash \TBL_{\alpha}M:\tau @{A\alpha}}{\TTBL}
\end{center}
\TTB, corresponding to brackets, means 
if $M$ is typed $\tau$ at transition $A\alpha$ then $\TB_{\alpha}M$, quoted $M$, is typed $\TW_{\alpha}\tau$ at $A$.
\TTBL\ is converse of \TTB.

% CSP

CSP, cross-stage persistence, is an important feature of \LTP.
It enables us to embed value at an outer transition into an inner transition.
$\%$ is dedicated to CSP in \LTP.
For example, $\lambda a:\I.\Lambda\alpha.(\TB_\alpha (\lambda x:\I.x+\%_\alpha a))$
\red{inner / outer は適切か?}

% Omitting Residualization
% この段落は3章のM eのあとに、この制限の結果として...という形で入れる

% There is another important feature called program residualization in \LTP.
% It means that a generated code can be dumped into a file.
% We can load the dumped file and run it.
% The difficulty arises when program residualization is used with CSP.
% Transition variables are classified into two kinds in \LTP in order to deal with this difficulty.

\subsection{\LLF}

% \LLF
\LLF is a simple system of dependent types introduced in \cite{attapl}.
\red{system / type system / calculus?}
It is based on Edinburgh LF\cite{harper1993framework}.
Therefore, all constants and base types are declared in the signature.
The \LLF type theory generalizes simply typed lambda calculus
by replacing the function type $\tau\to\sigma$ with the dependent product type $\Pi x:\tau.\sigma$.

% Kind, Well-formed kind
In addition to ordinary typing rules like simply typed lambda calculus,
there are kinding rules, well-formed kinding rules, term equivalence rules, type equivalence rules, and kind equivalence rules in \LLF.
Kinding rules and well-formed kinding rules are 
introduced in order to prohibit making illegal types such as $\textrm{Vect}\ \textrm{Bool}$.
For a well-formed type $\tau$, $\G \vdash \tau :: K$ means that $\tau$ has a kind $K$ under the environment $\G$ and 
for a well-formed kind $K$, $\G \vdash K$ means that $K$ is a well-formed kind under an environment $\G$.

% Type Equality
Type equality rules are needed because the type equivalence is not obvious unlike simply typed lambda calculus.
For example, $\textrm{Vect}\ 7$ should be equivalent to $\textrm{Vect}\ (3+4)$
but they are not equivalent seemingly. Thus, we must define equivalence rules.
In \LLF, equivalence is expressed with a symbol of $\E$.
$\G \vdash M \E N$ means a term $M$ and a term $N$ are equivalent under the environment $\G$.
$\G \vdash \tau \E \sigma$ means a type $\tau$ and a type $\sigma$ are equivalent under the environment $\G$.
$\G \vdash K \E J$ means a kind $K$ and a kind $J$ are equivalent under the environment $\G$.

\subsection{Extending \LTP with Dependent Types}

Next, we develop \LMD by extending \LTP with \LLF-like dependent types.
From here, we use the word "stage" instead of "transition" 
because we develop a multi-stage calculus, not a multi-transition calculus.
\red{stageのほうが言葉としてふさわしいと言いたい}
There are three technical points in the extension from \LTP to \LMD.

% Constants and Base Types

First, the way of handling of constants and type-level constants is the difference between \LMD and \LTP.
We adopt a signature $\Sigma$ to handle constants and type-level constants.
This is because a signature simplifies kinding rules relating to type-level constants.
A signature $\Sigma$ is composed of pairs of a base type and its kind or a constant and its type.
For example, if you want to use boolean, $\Sigma = \B::*, \text{true}:\B, \text{false}:\B$
\red{具体的な導出例を出したほうがよいか? また、なぜsimpleになるのを書くべきか?}

% Kidinding and Well-formed Kinding Rules

Second, we need kinding rules and well-formed kinding rules in order to extend \LMD with dependent types.
It was lucky that almost all rules are determined easily.
This is because multi-stage calculus and dependent types are almost orthogonal.
\red{orthogonal は抽象的すぎるか?}
Therefore, we can get kinding rules and well-formed kinding rules of \LMD just by 
attaching stage anotations to ones of \LLF.
For example, \KAbs-LF is a kinding rule for a dependent type in \LLF and \KAbs\ is a corresponding one.
\begin{center}
	\infrule{\G\vdash \tau :: * \andalso \G,x:\tau @A\vdash \sigma::J}{\G\vdash(\Pi x:\tau.\sigma) :: (\Pi x:\tau.J)}{\KAbs-LF} \\[2mm]
	\infrule{\G\V \tau :: *@A \andalso \G,x:\tau @A\V \sigma::J@A}{\G\V(\Pi x:\tau.\sigma) :: (\Pi x:\tau.J)@A}{\KAbs} \\[2mm]
\end{center}

% Equivalence Rules

Third, we also need type equivalence rules in \LMD because its type system contains dependent types.
Although there are new primitives relating to stages which aren't in \LLF,
we can design all rules except \QPercent\ easily from the syntax or the reductions of \LMD.

\QPercent\ is different from other rules because it was designed from practical reason, not from the syntax or the reductions.
For example, we discuss $\text{vadd}_\beta$ function,
which takes a length of vector and returns addition function for vectors with the length.
$\text{vadd}_1$ is a helper function to make $\text{vadd}$, 
which takes a stage, a length of vectors, and two quoted vectors and returns a quoted vector.
We assumed that type checker can know $n=0$ when process the $\textbf{then}$ clause.
\newcommand{\Vpn}{\text{Vector}\ (\%_\alpha n)}
\begin{align*}
	  & \textbf{let}\ \text{vadd}_1 : \F\alpha.\Pi n:\I.\TW_\alpha\Vpn\to\TW_\alpha\Vpn\to\TW_\alpha\Vpn                                \\
	  & \hspace{7mm} = \textbf{fix}\ f.\Lambda\alpha.\lambda n:\I.\ \lambda v_1:\TW_\alpha\Vpn.\ \lambda v_2:\TW_\alpha\Vpn.            \\
	  & \hspace{14mm} \textbf{if}\ n = 0                                                                                                \\
	  & \hspace{14mm} \textbf{then} \TB_\alpha \text{nil}                                                                               \\
	  & \hspace{14mm} \textbf{else}\ \TB_\alpha (                                                                                       \\
	  & \hspace{21mm} \textbf{let}\ t_1 = \text{tail}\ (\TBL_\alpha v_1)\ \textbf{in}                                                   \\
	  & \hspace{21mm} \textbf{let}\ t_2 = \text{tail}\ (\TBL_\alpha v_2)\ \textbf{in}                                                   \\
	  & \hspace{21mm} \text{cons}\ (\text{head}\ (\TBL_\alpha v_1) + \text{head}\ (\TBL_\alpha v_2))\ (\TBL_\alpha f\ (n-1)\ t_1\ t_2)) \\
\end{align*}
Using $\text{vadd}_1$, we can make $\text{vadd}$ function easily.
This function takes the length of vectors and returns a quoted add function for vector of the given size.
\renewcommand{\Vpn}{\text{Vector}\ (\%_\beta n)}
\begin{align*}
	  & \textbf{let}\ \text{vadd}: \Pi n:\I.\F\beta.\TW_\beta(\Vpn\to\Vpn\to\Vpn)                \\ 
	  & \hspace{7mm} = \lambda n:\I.\Lambda\beta.\TB_\beta (\lambda v_1:\Vpn.\ \lambda v_2:\Vpn. \\
	  & \hspace{63mm} \TBL_\beta \text{vadd}_1\ \beta\ n\ (\TB_\beta\ v_1)\ (\TB_\beta\ v_2))    \\
\end{align*}
\renewcommand{\Vpn}{\text{Vector}\ (\%_\beta 10)}
When we use this function, we need to instantiate $\text{vadd}$ with a length.
If we instantiate it with a integer 10, we get function $\text{vadd10}$ whose type is \\
$\text{vadd10}: \F\beta.\TW_\beta(\Vpn\to\Vpn\to\Vpn)$.
This function is difficult to use because its type contains $\Vpn$.
In most cases, we have vectors of type $\text{Vector}\ 10$, not $\Vpn$.
Therefore, we want to identify $\text{Vector}\ 10$ and $\Vpn$.

In order to resolve this problem, we introduced \QPercent\ into \LMD.
\QPercent\ enables us to identify $\text{Vector}\ 10$ and $\Vpn$.
This rule says that if term $M$ has the same type at stage $A\alpha$ and $A$,
we can assume $\%_\alpha M \E M$ which roughly means $\%_\alpha M$ and $M$ are equivalent.
With this rule, we can identify $10$ and $\%_\beta 10$ and use $\text{vadd}$ to many vectors.
\begin{center}
	\infrule{\G\V M:\tau @{A\alpha} \andalso \G\V M:\tau @A}{\G\V\%_\alpha M \E M : \tau @{A\alpha}}{\QPercent}
\end{center}



% !TEX root = ../main.tex

\section{Formal Definition of \LMD}

In this section, we give a formal definition of \LMD, including
the syntax, full reduction, and type system.  In addition to full reduction,
in which any redex at any stage can be reduced, we also give staged reduction,
which models execution as programs.

\subsection{Syntax}

We assume the denumerable set of \emph{type-level constants}, ranged over by
metavariables \(X, Y, Z\), the denumerable set of \emph{variables}, ranged
over by \(x,y,z\), the denumerable set of \emph{constants}, ranged over by
\(c\), and the denumerable set of \emph{stage variables}, ranged over by
\(\alpha, \beta, \gamma\).  The metavariables \(A, B, C\) range over
sequences of stage variables; we write \(\varepsilon\) for the empty
sequence. \LMD is defined by the following grammar:

% \AI{``Transition'' is still used.}

\begin{align*}
    % \textrm{Type variables}  &   &                          & X,Y,Z                                                                                                      \\
    % \textrm{Variables}       &   &                          & x,y,z                                                                                                      \\
    % \textrm{Stage variables} &   &                          & \alpha,\beta,\gamma                                                                                        \\
    % \textrm{Stage}           &   &                          & A,B,C                                                                                                      \\
    \textrm{kinds}             &  & K,J,I,H,G                & ::= * \mid \Pi x:\tau.K                                                           \\
    \textrm{types}             &  & \tau,\sigma,\rho,\pi,\xi & ::= X \mid \Pi x:\tau.\tau \mid \tau\ M \mid \TW_{\alpha} \tau \mid \F\alpha.\tau \\
    %     \textrm{Constants}       &   &                          & c                                                                                                          \\
    \textrm{terms}             &  & M,N,L,O,P                & ::= c \mid x \mid \lambda x:\tau.M\ \mid M\ M \mid \TB_\alpha M                   \\
                               &  &                          & \ \ \ \ \mid \TBL_\alpha M \mid \Lambda\alpha.M \mid M\ A \mid \%_\alpha M        \\
    \textrm{signatures}         &  & \Sigma                   & ::= \emptyset \mid \Sigma, X::K \mid \Sigma, c:\tau                               \\
    \textrm{type environments} &  & \Gamma                   & ::= \emptyset \mid  \Gamma,x:\tau @A                                              \\
\end{align*}

% \AI{Add $M\,\alpha$ to terms.}

\AI{Is $O$ used as a metavariable?}
% \AI{Domain and FV are not defined (yet).  I think we should introduce the notion of well-formed type environments by prose below and assume every type environment is well formed.}

% Description of meta variables


% Kinds

A kind, which is used to classify types, is either $*$, the kind of
proper types (types that terms inhabit), or $\Pi x\colon\tau.K$, the kind
of type operators that takes $x$ as an argument of type $\tau$ and returns a type
of kind $K$.
% Types
% of terms have kind $*$ and dependent types have $\Pi$-kinds.  For
% example, $\lambda x:\I.x$ has type $\Pi x:\I.\I$, which has $*$ kind.
% \red{この段落は短いので型の段落と結合するか?}
% Types
A type is a type-level constant $X$, which is declared in the signature with its kind, a dependent function type $\Pi x:\tau_1.\tau_2$,
an application $\tau\ M$ of a type (operator) (of $\Pi$-kind) to a term, a code type $\TW_\alpha$, or an $\alpha$-closed type $\F\alpha.\tau$.
An application of a type (operator) of $\Pi$-kind to a term is, for example, $\text{Vector}\ 10$
if type-level constant $\text{Vector}$ has the kind $\Pi x:\I.*$.
A code type $\TW_\alpha \tau$ is for a code fragment of a term of type $\tau$.
An $\alpha$-closed type corresponds to a runnable code fragment.

% An application $\tau\ M$ of a type (operator) to a term connects tighter than
% $\TW_{\alpha}$ connects tighter than
% $\Pi$ in dependent types such as $\Pi x:\tau.\tau$ and $\Pi$ connect tighter than
% $\F$ in types for stage abstraction such as $\F\alpha.\tau$.
% Therefore, $\F\alpha.\TW_{\alpha} \Pi x:\I.\text{Vector}\ 5$ is interpreted as
% $\F\alpha.(\TW_{\alpha} (\Pi x:\I.(\text{Vector}\ 5)))$.
% \AI{I don't understand this rule...}

% Terms

Terms include ordinary (explicitly typed) \(\lambda\)-terms, constants,
whose type is declared in signature $\Sigma$, and the following five forms
related to multi-stage programming:
$\TB_\alpha M$ represents a code fragment; $\TB_\alpha M$ represents escape;
$\Lambda\alpha.M$ is a stage variable abstraction;
$M\ A$ is an application of stage $A$ to a stage abstraction; and
$\%_\alpha M$ is an operator for cross-stage persistence.

% Signature

We adopt the tradition of \LLF-like systems, where types of constants and kinds of type-level constants are globally declared in a signature $\Sigma$, which
is a sequence of declarations of the form $c:\tau$ and $X:K$.
For example, when we use Boolean in \LMD, $\Sigma$ includes $\B :: *, \textrm{true}:\B, \textrm{false}:\B$.
Type environments are sequences of triples of a variable, its type, and its stage.
We write \(\textit{dom}(\Sigma)\) and \(\textit{dom}(\Gamma)\)
for the set of (type-level) constants declared in \(\Sigma\) and \(\Gamma\), respectively.   As in other multi-stage calculi~\cite{Taha03,Tsukada,Hanada2014},
a variable declaration is associated with a stage
so that a variable can be referenced only at the declared stage.
Constants and type-level constants are not associated with stages;
so, we can use them at any stage.
We define well-formed signature and well-formed type environments later.

The variable $x$ is bound in $M$ of $\lambda x:\tau.M$ and in $\tau_2$
of $\Pi x:\tau_1.\tau_2$, as usual; the stage variable $\alpha$ is
bound in $M$ of $\Lambda \alpha.M$ and $\tau$ of $\F\alpha.\tau$.
We write $\FV(M)$ and $\FTV(M)$ for the set of free variables and the set of free stage variables in $M$, respectively.  Similarly, $\FV(\tau)$, $\FTV(\tau)$,
$\FV(K)$, and $\FTV(K)$ are defined.
We sometimes abbreviate $\Pi x:\tau_1.\tau_2$ to $\tau_1 \rightarrow \tau_2$ if
$x$ is not a free variable in $\tau_2$.
% Free variables
We identify $\alpha$-convertible terms and assume the names of bound variables are pairwise distinct.
% \AI{This should be mentioned after free variables are introduced.}
% \AI{What about other binders such as $\Pi$?}

The prefix operators $\TW_\alpha, \TB_\alpha, \TBL_\alpha$, and
$\%_\alpha$ connect tighter than the three forms $\tau\ M$, $M\ N$,
$M\ A$ of applications, which are left-associative; The binders $\Pi$,
$\forall$, and $\lambda$ extend as far to the right as possible.
Thus, $\F\alpha.\TW_{\alpha} (\Pi x:\I.\text{Vector}\ 5)$ is
interpreted as
$\F\alpha.(\TW_{\alpha} (\Pi x:\I.(\text{Vector}\ 5)))$; and
$\Lambda\alpha.\lambda x:\I.\TB_\alpha x\ y$ means
$\Lambda\alpha.(\lambda x:\I.(\TB_\alpha x)\ y)$.

\paragraph{Remark:} Basically, we define \LMD to be an extension of
\LTP with dependent types.  One notable difference is that \LMD has
only one kind of \(\alpha\)-closed types, whereas \LTP has two kinds
of \(\alpha\)-closed types \(\forall\alpha.\tau\) and
\(\forall^\varepsilon\alpha.\tau\).  We have omitted the first kind,
for simplicity, and dropped the superscript $\varepsilon$.  It would
not be difficult to recover the distinction to show properties related
to program residualization~\cite{Hanada2014}.


\subsection{Reduction}

Next, we define full reduction for \LMD.
Before giving the definition of reduction, we define six kinds of substitutions.
Substitution $M[x\mapsto N], \tau[x \mapsto N]$ and $K[x \mapsto N]$ are
ordinary capture-avoiding substitution of
term $N$ for $x$ in term $M$, type $\tau$, and kind $K$, respectively,
and we omit their definitions here.
Substitution $M[\alpha \mapsto A], \tau [\alpha \mapsto A], K[\alpha \mapsto A]$ and $B[\alpha\mapsto A]$ are
substitutions of stage $A$ for stage variable $\alpha$ in
term $N$ for $x$ in term $M$, type $\tau$, kind $K$, and stage $B$, respectively.
We show representative cases below.
%
\begin{align*}
    (\lambda x:\tau.M)[\alpha \mapsto A] & = \lambda x:(\tau[\alpha \mapsto A]).(M[\alpha \mapsto A])                                  \\
    (M\ B)[\alpha \mapsto A]             & = (M[\alpha \mapsto A])\ B[\alpha\mapsto A]                                                 \\
    (\TB_\beta M)[\alpha \mapsto A]      & = \TB_{\beta[\alpha \mapsto A]}M[\alpha \mapsto A]                                          \\
    (\TBL_\beta M)[\alpha \mapsto A]     & = \TBL_{\beta[\alpha \mapsto A]}M[\alpha \mapsto A]                                         \\
    (\%_\beta M)[\alpha \mapsto A]       & = \%_{\beta[\alpha \mapsto A]}M[\alpha \mapsto A]                                           \\
    (\beta B)[\alpha \mapsto A]          & = \beta (B[\alpha\mapsto A])                               & (\text{if } \alpha \neq \beta) \\
    (\beta B)[\alpha \mapsto A]          & = A (B[\alpha\mapsto A])                                   & (\text{if } \alpha = \beta)
\end{align*}
Here, $\TB_{\alpha_1\cdots\alpha_n} M$,
$\TBL_{\alpha_1\cdots\alpha_n} M$, and $\%_{\alpha_1\cdots\alpha_n} M$
$(n \geq 0)$ stand for $\TB_{\alpha_1} \cdots \TB_{\alpha_n} M$,
$\TBL_{\alpha_1}\cdots \TBL_{\alpha_n} M$, and
$\%_{\alpha_1}\cdots \%_{\alpha_n} M$, respectively.  In particular,
$\TB_{\varepsilon} M = \TBL_{\varepsilon} M = \%_{\varepsilon} M = M$.

\begin{definition}[Reduction]
    The relations $M \longrightarrow_\beta N$, $M \longrightarrow_\blacklozenge N$, and $M \longrightarrow_\Lambda N$
    are the least compatible relations closed under the rules below.
%    Congruence rules are omitted from the definition.
    \begin{align*}
         (\lambda x:\tau.M) N & \longrightarrow_\beta M[x \mapsto N]         \\
         \TBL_\alpha \TB_\alpha M & \longrightarrow_\blacklozenge M          \\
         (\Lambda \alpha.M)\ A & \longrightarrow_\Lambda M[\alpha \mapsto A]
    \end{align*}
\end{definition}
We write $ M \longrightarrow M'$ iff $ M \longrightarrow_\beta M'$,
$ M \longrightarrow_\blacklozenge M'$, or
$ M \longrightarrow_\Lambda M'$ and we call $\longrightarrow_\beta$,
$\longrightarrow_\blacklozenge$, and $\longrightarrow_\Lambda$
$\beta$-reduction, $\blacklozenge$-reduction, and $\Lambda$-reduction,
respectively.
\AI{Don't we need to define $\longrightarrow^*$?}

The relation $\longrightarrow_\beta$ represents ordinary $\beta$-reduction in the \(\lambda\)-calculus; the relation
$\longrightarrow_\blacklozenge$ represents that quotation $\TB_\alpha M$ is canceled by escape and $M$ is spliced into the code fragment surrounding the escape;
the relation $\longrightarrow_\Lambda$ means that a stage abstraction applied to  stage $A$ reduces to the body of the abstraction
where $A$ is substituted for the stage variable.
There is no reduction rule for CSP as with Hanada and Igarashi \cite{Hanada2014}.
The CSP operator $\%_\alpha$ disappears when $\varepsilon$ is substituted for $\alpha$.
We show an example of a reduction sequence below.
Underlines show the redexes.
\begin{align*}
     & \hspace{10mm} \underline{(\lambda i:\I\to\I.(\Lambda\alpha.\TB_\alpha (\%_\alpha i\ 1 + (\TBL_\alpha \TB_\alpha 3))\ \varepsilon))\ (\lambda x:\I.x)} \\
     & \longrightarrow_\blacklozenge (\Lambda\alpha.\TB_\alpha (\%_\alpha (\lambda x:\I.x)\ 1 + (\underline{\TBL_\alpha \TB_\alpha 3})))\ \varepsilon        \\
     & \longrightarrow_\beta \underline{(\Lambda\alpha.\TB_\alpha (\%_\alpha (\lambda x:\I.x)\ 1 + 3))\ \varepsilon}                                         \\
     & \longrightarrow_\Lambda \underline{(\lambda x:\I.x)\ 1} + 3                                                                                           \\
     & \longrightarrow_\beta 1 + 3                                                                                                                           \\
     & \longrightarrow^* 4
\end{align*}

% $\TB_\alpha$ and $\TBL_\alpha$ disappears in the same way as $\%_\alpha$.

\subsection{Type System}

In this section, we define the type system of \LMD.
It consists of eight judgment forms for signature well-formedness, type environment well-formedness, kind well-formedness, kinding, typing, term equivalence, type equivalence, and kind equivalence.
We list the judgments forms in Figure~\ref{fig:LMD-six-judgments}.
They are all defined in a mutual recursive manner.  We will discuss
each judgment below.

\begin{figure}
    \begin{center}
      \begin{align*}
            \vdash & \Sigma            & \text{signature well-formedness} \\
            \V & \G                & \text{type environment well-formedness} \\
            \G & \V K \iskind @ A      & \text{kind well-formedness} \\
            \G & \V \tau :: K @ A      & \text{kinding}             \\
            \G & \V M : \tau @ A       & \text{typing}              \\
            \G & \V K \E J @ A         & \text{kind equivalence}    \\
            \G & \V \tau \E \sigma @ A & \text{type equivalence}    \\
            \G & \V M \E N @ A         & \text{term equivalence}
        \end{align*}
        \caption{Eight judgment forms of the type system of \LMD.}
        \label{fig:LMD-six-judgments}
    \end{center}
\end{figure}


\subsubsection{Signature and Type Environment Well-formedness.}
The rules for Well-formed signatures and type environments are
shown below:
%
{\small
\begin{center}
  \infrule{
  }{
    \vdash \emptyset
  }
  \hfil
  \infrule{
    \vdash \Sigma \andalso
    \V K \iskind @ \varepsilon \\
    X\notin\textit{dom}(\Sigma)
  }{
    \vdash \Sigma, X::K
  }
  \hfil
  \infrule{
    \vdash \Sigma \andalso
    \V \tau :: * @ \varepsilon \\
    c\notin\textit{dom}(\Sigma)
  }{
   \vdash \Sigma, c:\tau
 }
 \\[2mm]
  \infrule{
  }{
    \V \emptyset
  }
  \hfil
  \infrule{
    \V \Gamma \andalso
    \Gamma \V \tau :: * @ A \andalso
    x\notin\textit{dom}(\Sigma)
  }{
   \V \Gamma, x:\tau@A
 }
\end{center}
}

To add declarations to a signature, the kind/type of a (type-level)
constant has to be well-formed at stage \(\varepsilon\) so that it is
used at any stage. \AI{We should discuss implicit CSP at the type
  level somewhere.}

In what follows, well-formedness is not explicitly mentioned but
we assume that all signatures and type environments are well-formed.

\subsubsection{Kind Wellformedness and Kinding.}

The rules for kind wellformedness and kiding is straightforward
adaptation from \LLF.  So, we omit them.

\subsubsection{Typing.}

\AI{The second premises of \TConst and \TVar are not needed?}
\begin{figure}
    \begin{center}
        \infrule[\TConst]{c:\tau \in \Sigma \andalso \G\V \tau::*@A}{\G \V c:\tau @A} \hfil
        \infrule[\TVar]{x:\tau @A \in \G \andalso \G\V \tau::*@A}{\G \V x:\tau @A} \\[2mm]
        \infrule[\TAbs]{\G\V \sigma::*@A\andalso\G,x:\sigma@A\V M:\tau @A}{\G\V(\lambda (x:\sigma).M):(\Pi (x:\sigma).\tau)@A} \\[2mm]
        \infrule[\TApp]{\G\V M:(\Pi (x:\sigma).\tau)@A \andalso \G\V N:\sigma@A}{\G\V M\ N : \tau[x\mapsto N]@A} \\[2mm]
        \infrule[\TConv]{\G\V M:\tau @A \andalso \G\V \tau\equiv \sigma :: K@A}{\G\V M:\sigma@A} \\[2mm]
        \infrule[\TTB]{\G\V M:\tau @{A\alpha}}{\G\V\TB_{\alpha}M:\TW_{\alpha}\tau @A} \andalso
        \infrule[\TTBL]{\G\V M:\TW_{\alpha}\tau @A}{\G\V\TBL_{\alpha}M:\tau @{A\alpha}} \\[2mm]
        \infrule[\TGen]{\G\V M:\tau @A \andalso \alpha\notin\rm{FTV}(\G)\cup\rm{FTV}(A)}{\G\V\Lambda\alpha.M:\forall\alpha.\tau @A} \\[2mm]
        \infrule[\TIns]{\G\V M:\forall\alpha.\tau @A}{\G\V M\ B:\tau[\alpha \mapsto B]@A} \andalso
        \infrule[\TCsp]{\G\V M:\tau @A}{\G\V \%_\alpha M:\tau @{A\alpha}}
        \caption{Typing Rules.}
        \label{fig:typing-rules}
    \end{center}
\end{figure}

Typing rules of \LMD are shown in Figure~\ref{fig:typing-rules}.
The rule \TConst{} means that a constant can appear at any stage.
% For example, if we have a signature $\Sigma$ which is
% $\textrm{bool} :: *, \textrm{true}: \textrm{bool}, \textrm{false}:
%     \textrm{bool}$, the derivation tree in
% Figure~\ref{fig:tconst-derivation-tree} is admissible.
The rules \TVar,\TAbs, and \TApp{} are almost the same as those in the simply typed
lambda calculus or \LLF.  Additional conditions are that subterms must be
typed at the same stage (\TAbs{} and \TApp); the type
annotation/declaration on a variable has to be a proper type of kind
$*$ (\TAbs) at the stage where it is declared (\TVar{} and \TAbs).
% \AI{Don't we need signature well formedness?  If $c:\tau \in \Sigma$, then $\tau$ should be a well-formed type under ... what?}
% \AI{... or \(\lambda\)LF?  We are working in a dependently type system...}  

% \begin{figure}
%     \begin{center}
%         \begin{minipage}{0.50\hsize}
%             \infer[\TConst]
%             {\G \V \textrm{true}:\textrm{bool}@\alpha\beta}
%             {\textrm{true}:\textrm{bool} \in \Sigma \andalso
%                 \ID{\G\V\textrm{bool}::*@\alpha\beta} \andalso
%             }
%             \caption{A derivation tree using \TConst}
%             \label{fig:tconst-derivation-tree}
%         \end{minipage}
%     \end{center}
% \end{figure}


% \TConv
\AI{Looks like a paper is written for those who know multi-stage calculi fairly well but don't know dependent type systems.  The reality is the opposite.}
As in standard dependent type systems, \TConv{} allows us to replace
the type of a term with an equivalent one.
In a type system with dependent types, this kind of rule is essential
because \AI{?} two types which have different shapes may be equivalent.
For example, when we use a vector type with its size ($\textrm{Vector}\ n$),
$\textrm{Vector}\ 5$ is equivalent to $\textrm{Vector}\ (4+1)$ obviously although they are not syntactically equivalent.

% Typing rules for a multi-stage calculus
% \AI{This paragraph is not very informative.  We shouldn't just refer readers to previous work.}
The rules \TTB, \TTBL, \TGen, \TIns, and \TCsp{} are rules for a multi-stage calculus.
The rule \TTB{}, which corresponds to brackets \AI{Sounds like the rule corresponds to brackets.  What corresponds to brackets is $\TB$}, means that if term $M$ is of type $\tau$ at stage $A\alpha$,
$\TB_\alpha M$ is of type $\TW_\alpha \tau$ at stage $A$.
The rule \TTBL{}, which corresponds to escape, is the converse of \TTB.
The rule \TGen{} for stage abstraction is straightforward.
The condition $\alpha\notin\rm{FTV}(\G)\cup\rm{FTV}(A)$ ensures that
the scope of $\alpha$ is in $M$.
The rule \TIns{} is for applications of stages to stage abstractions.
\AI{Any comments?}
The rule \TCsp is the rule for CSP, which means that
if term $M$ is of type $\tau$ at stage $A$, $\%_\alpha M$ is of type $\tau$ at stage $A\alpha$.  \AI{This does not explain what this rule means.
  It just gives how the rule is read.}

\subsubsection{Kind, Type and Term Equivalence.}

Since the syntax of kinds, types, and terms is mutually recursive,
corresponding notions of equivalence are also mutually recursive.
They are congruences closed under a few axioms for term equivalence.
So, the rules for kind and type equivalences are not very interesting
except implict CSP is allowed.
We show a few representative rules below.

{\small
\begin{center}
\infrule[\textsc{QK-Csp}]{%
  \G\V K \E J @ A
}{
  \G\V K \E J @ A\alpha
}
\hfil
\infrule[\QTCsp]{
  \G\V \tau \E \sigma :: K@A
}{
  \G\V \tau \E \sigma :: K@{A\alpha}
}
\\[2mm]
\infrule[\QTApp]{%
  \G\V \tau \E \sigma :: (\Pi x:\rho.K)@A \andalso
  \G\V M \E N : \rho @A
}{
  \G\V \tau\ M \E \sigma\ N :: K[x \mapsto M]@A
}
\end{center}
}

% The type equivalence judgment of the form
% $\G \V \tau \E \sigma : K @ A$ means that types $\tau$ and $\sigma$
% are equivalent as types of kind $K$ at stage $A$ under $\G$.
% Figure~\ref{fig:type-equivalence-rules} shows the rules for type
% equivalence.  Type equivalence is basically the least congruence
% closed under term equivalence.  The rules for compatibility (closure
% under type formation) are derived from corresponding typing rules in a
% straightforward manner.  The rules are a little simpler than some
% dependent type systems \AI{such as?}  because there is no abstraction
% at the type level.

% \QCsp以外の説明
% We show type equivalence rules in Figure \ref{fig:type-equivalence-rules}.
% All rules except \QTRefl, \QTSym, \QTTrans, and \QTApp\ are generated naturally from the typing rules.
% \QTRefl, \QTSym, \QTTrans\ exist in order to make the type equivalence relation an equivalence relationship.
% The rule \QTApp\ means that if there are two equivalent $\Pi$ type and two equivalent terms,
% the results of applications are also equivalent.

% \begin{figure}
%     \begin{center}
%         \infrule[{\QTAbs }]{\G\V \tau \E \sigma :: *@A \andalso \G,x:\tau @A \V \rho \E \pi :: *@A}{\G\V\Pi x:\tau.\rho \E \Pi x:\sigma.\pi :: *@A} \\[2mm]
%         \infrule[\QTApp]{\G\V \tau \E \sigma :: (\Pi x:\rho.K)@A \andalso \G\V M \E N : \rho @A}{\G\V \tau\ M \E \sigma\ N :: K[x \mapsto M]@A} \\[2mm]
%         \infrule[\textsc{QT-$\TW$}]{\G\V \tau \E \sigma :: *@{A\alpha}}{\G\V \TW_{\alpha} \tau \E \TW_{\alpha} \sigma :: *@A}\andalso
%         \infrule[\QTCsp]{\G\V \tau \E \sigma :: K@A}{\G\V \tau \E \sigma :: K@{A\alpha}} \\[2mm]
%         \infrule[\QTGen]{\G\V \tau \E \sigma :: *@A \andalso \alpha\notin\rm{FTV}(\G)\cup\rm{FTV}(A)}{\G\V \forall\alpha.\tau \E  \forall\alpha.\sigma :: *@A} \\[2mm]
%         \infrule[\QTRefl]{\G\V \tau::K@A}{\G\V \tau\E\tau :: K@A} \andalso
%         \infrule[\QTSym]{\G\V \tau \E \sigma :: K@A}{\G\V \sigma \E \tau :: K@A} \\[2mm]
%         \infrule[\QTTrans]{\G\V \tau \E \sigma :: K@A \andalso \G\V \sigma \E \rho  :: K@A}{\G\V \tau \E \rho  :: K@A}
%         \caption{Type Equivalence Rules.}
%         \label{fig:type-equivalence-rules}
%     \end{center}
% \end{figure}

% The term equivalence judgment of the form $\G \V M \E N : \rho @ A$,
% which means that terms $M$ and $N$ are equivalent as terms of type
% $\rho$ at stage $A$ under $\G$, is defined by the rules in
We show the rules for term equivalence are shown in 
Figure~\ref{fig:term-equivalence-rules}, omitting
straightforward rules for reflexivity, symmetry, transitivity,
and compatibility.
% Most rules are
% straightforward.  The rules \QAbs, \QApp, \QTB, \QTBL, \QGen, \QIns,
% \QCsp, \QRefl, \QSym, and \QTrans{} make the relation congruence;
The rules \QBeta, \QTBLTB, and \QLambda{} correspond to
$\beta$-reduction, $\blacklozenge$-reduction, and $\Lambda$-reduction, respectively.

\begin{figure}
    \begin{center}
        % \infrule[\QAbs]{\G\V \tau \E \sigma :: *@A \andalso \G,x:\tau @A \V M \E N : \rho @A}{\G\V\lambda x:\tau.M \E \lambda x:\sigma.N : (\Pi x:\tau.\rho)@A} \\[2mm]
        % \infrule[\QApp]{\G\V M \E L : (\Pi x:\sigma.\tau)@A \andalso \G\V N \E O : \sigma@A}{\G\V M\ N \E L\ O : \tau[x \mapsto N]@A} \\[2mm]
        % \infrule[\QTB]{\G\V M \E N : \tau @{A\alpha}}{\G\V \TB_\alpha M \E \TB_\alpha N : \TW_\alpha \tau @A} \andalso
        % \infrule[\QTBL]{\G\V M \E N : \TW_\alpha \tau @A}{\G\V \TBL_\alpha M \E \TBL_\alpha N : \tau @{A\alpha}} \\[2mm]
        % \infrule[\QGen]{\G\V M\E N : \tau @A \andalso \alpha \notin \FTV(\G)\cup\FTV(A)}{\G\V \Lambda\alpha.M \E \Lambda\alpha.N : \forall\alpha.\tau @A} \\[2mm]
        % \infrule[\QIns]{\G\V M \E N:\forall\alpha.\tau @A}{\G\V M\ \varepsilon \E N\ \varepsilon : \tau[\alpha \mapsto \varepsilon]@A} \andalso
        % \infrule[\QCsp]{\G\V M \E N : \tau @A}{\G\V\%_\alpha M \E \%_\alpha N : \tau @{A\alpha}} \\[2mm]
        % \infrule[\QRefl]{\G\V M:\tau @A}{\G\V M\E M : \tau @A} \andalso
        % \infrule[\QSym]{\G\V M\E N : \tau @A}{\G\V N\E M : \tau @A} \\[2mm]
        % \infrule[\QTrans]{\G\V M\E N : \tau @A \andalso \G\V N\E L : \tau @A}{\G\V M\E L : \tau @A} \\[2mm]
        \infrule[\QBeta]{\G,x:\sigma@A\V M:\tau @A \andalso \G\V N:\sigma@A}{\G\V(\lambda x:\sigma.M)\ N\E M[x\mapsto N] : \tau[x \mapsto N]@A} \\[2mm]
        % \infrule{\G\V M:(\Pi x:\sigma.\tau)@A \andalso x\notin \text{FV}(M)}{\G\V(\lambda x:\sigma.M\ x)\E M: (\Pi x:\sigma.\tau)@A}{\QEta} \\[2mm]
        \infrule[\QLambda]{\G\V (\Lambda\alpha.M) : \forall\alpha.\tau @A}{\G\V (\Lambda\alpha.M)\ \varepsilon \E M[\alpha \mapsto \varepsilon] : \tau[\alpha \mapsto \varepsilon]@A} \\[2mm]
        \infrule[\QTBLTB]{\G\V M \E N : \tau @A}{\G\V \TBL_\alpha(\TB_\alpha M) \E N : \tau @A} \hfil
        \infrule[\QPercent]{\G\V M:\tau @{A\alpha} \andalso \G\V M:\tau @A}{\G\V\%_\alpha M \E M : \tau @{A\alpha}}
        \caption{Term Equivalence Rules.}
        \label{fig:term-equivalence-rules}
    \end{center}
\end{figure}

% \QPercentの説明
The only rule that deserves elaboration is the last rule \QPercent.
Intuitively, it means that the CSP operator applied to term $M$ can be
removed if $M$ is also well typed at the next stage \(A\alpha\).
For example, constants do not depend on the stage (see \TConst) and
so \(\G\V \%_\alpha c \E c : \tau @ A\alpha\) holds but variables
do depend on stages and so this rule does not apply.

Interestingly, Hanada and Igarashi~\cite{Hanada2014} rejected the idea of
reduction that removes $\%_\alpha$ when they developed \LTP{}, as such
reduction does not match the operational behavior of the CSP operator
in implementation.  However, as an equational system for multi-stage
programs, the rule \QPercent makes sense and, as we argue next,
it is practically significant.

\AI{This paragraph should be polished later.}
\renewcommand{\Vpn}{\text{Vector}\ (\%_\alpha n)}
As we discussed in Section 2, $\text{vadd}$ function, 
which takes the length of vectors and returns a quoted add function for vector of the given size,
has the type of $\Pi n:\I.\F\beta.\TW_\beta(\Vpn\to\Vpn\to\Vpn)$.
When we use this function, we instansiate $\text{vadd}$ with some integer such as 5
and get code of a optimzed $\text{vadd}$ function such as $\text{vadd5}$.
\renewcommand{\Vpn}{\text{Vector}\ (\%_\beta 5)}
Although $\text{vadd5}$ is of type $\F\beta.\TW_\beta(\Vpn\to\Vpn\to\Vpn)$,
a code value of this type is difficult to combine with another code fragment.
For example, if we want to composite $\text{vadd5}$ with reverse fucntion, 
which is of type $\text{rev5}:\F\alpha.\TW_\alpha(\text{Vector}\ 5\to\text{Vector}\ 5)$,
we write a program like following.
\begin{align*}
\Lambda\gamma.\TB_\gamma.\lambda v_1:(\text{Vector}\ 5).\lambda v_2:(\text{Vector}\ 5).\TBL_\gamma 
(\text{rev5}\ \gamma)\ (\TBL_\gamma (\text{vadd5}\ \gamma)\ \TBL_\gamma v_1\ \TBL_\gamma v_2)
\end{align*}
However, this program cannot be typed without \QPercent{}
because the type of $v_1$: $\text{Vector}\ 5$ isn't equivalent with the type of arguments of $\text{vadd5}$: $\text{Vector}\ \%_\gamma 5$.

In order to resolve this problem, we introduce \QPercent{} into \LMD.
\begin{center}
\infrule[\QPercent]{\G\V M:\tau @{A\alpha} \andalso \G\V M:\tau @A}{\G\V\%_\alpha M \E M : \tau @{A\alpha}}
\end{center}
\QPercent\ states that we can erase this $\%_\alpha$ under a condition.
The condition is that a CSPed value equals to the original value when it has the same type in the original stage.
For example, $\V \%_\alpha 5 \E 5 @ \alpha$ because $ \V 5 : \I @ \alpha $ from \TConst\ and  $ \V \%_\alpha 5 : \I @ \alpha$.
In other words, we can remove a $\%_\alpha$ symbol of a value when it doesn't change the type.
Thanks to \QPercent, we can identify $\text{Vector}\ 5$ and $\text{Vector}\ \%_\gamma 5$
and compose code values of functions as we expected.

\subsection{Staged Semantics}

The reduction given above is full reduction and any redexes---even
under quotation---can be reduced in an arbitrary order.
Following previous work~\cite{Hanada2014},
we introduce (small-step, call-by-value) staged semantics,
where only $\beta$-reduction or $\Lambda$-reduction at stage $\varepsilon$ or the outer-most $\blacklozenge$-reduction are allowed,
modeling an implementation.

We start with the definition of values. Since terms under quotations are
not executed, the grammar is indexed by stages.

\begin{definition}[Values]
    The family $V^A$ of sets of values, ranged over by $v^A$,
    is defined by the following grammar.  In the grammar, $A \neq \varepsilon$ is assumed.
    \begin{align*}
        v^\varepsilon \in V^\varepsilon & ::= \lambda x:\tau.M \mid\ \TB_\alpha v^\alpha \mid \Lambda\alpha.v^\varepsilon                                       & \\
        v^A \in V^A                     & ::= x \mid \lambda x:\tau.v^A \mid v^A\ v^A \mid\ \TB_\alpha v^{A\alpha} \mid \Lambda\alpha.v^A \mid v^A\ \varepsilon & \\
                                        & \quad\   \mid\ \TBL_\alpha v^{A'} (\text{if } A = A'\alpha \text{ for some } \alpha, A' \neq \varepsilon)             & \\
                                        & \quad\   \mid\ \%_\alpha v^{A'} (\text{if } A = A'\alpha  \text{ for some } \alpha, A')
    \end{align*}
\end{definition}

Values at $\varepsilon$ stage are a $\lambda$-abstraction, a quoted code,
or a $\Lambda$ abstraction.  The body of a $\lambda$-abstraction can
be any term but the body of $\Lambda$-abstraction has to a value.  It
means that the body of $\Lambda$-abstraction must be evaluated.  The
side condition for $\TBL_\alpha v^{A'}$ means that escapes in a value
can appear only under nested quotations (because an escape under a
single quotation will splice the code value into the surrounding
code).  See Hanada and Igarashi~\cite{Hanada2014} for details.

In order to define staged reduction, we define redex and evaluation contexts.

\begin{definition}[Redex]
    The sets of $\varepsilon$-redexes (ranged over by $R^\varepsilon$) and $\alpha$-redexes (ranged over by $R^\alpha$) are defined by the following grammar.
    \begin{align*}
         & R^\varepsilon ::= (\lambda x:\tau.M)\ v^\varepsilon \mid (\Lambda\alpha.v^\varepsilon)\ \varepsilon \\
         & R^\alpha      ::=\ \TBL_\alpha \TB_\alpha M                                                         \\
    \end{align*}
\end{definition}

\begin{definition}[Evaluation Context]
  Let $B$ be either \(\varepsilon\) or a transition variable \(\beta\).
  The family of sets $ECtx^A_B$ of evaluation contexts, ranged over by $E^A_B$, is defined by the following grammar (in which $A'$ stands for a non-empty stage).
%  $A$ is assumed to be nonempty (but $B,A'$ can be empty).
    \begin{align*}
         E^\varepsilon_B \in ECtx^\varepsilon_B & ::= \square\ (\text{if\ } B = \varepsilon)
        \mid E^\varepsilon_B\ M \mid v^\varepsilon\ E^\varepsilon_B \mid \TB_\alpha E^\alpha_B
        \mid \Lambda\alpha.E^\varepsilon_B \mid E^\varepsilon_B\ A \\
         E^{A'}_B \in ECtx^{A'}_B  & ::= \square\ (\text{if } A' = B) \mid \lambda x:\tau.E^{A'}_B \mid E^{A'}_B\ M \mid v^{A'}\ E^{A'}_B \\
         & \mid \TB_\alpha E^{A'\alpha}_B \mid \TBL_\alpha E^{A}_B \ (\text{where } A\alpha = A')      \\
         & \mid \Lambda\alpha.E^{A'}_B \mid E^{A'}_B\ A \mid \%_\alpha\ E^{A'}_B \ (\text{where } A\alpha = A')
    \end{align*}
\end{definition}
\AI{some more explanation of the syntax of evaluation contexts.}

The subscripts $A$ and $B$ in $E^A_B$ stand for the stage of the evaluation context and of the hole, respectively.  The grammer represents that staged reduction is left-to-right and call-by-value and terms under \(\Lambda\) is reduced.
A few examples of evaluation context are shown below:
\begin{align*}
     \square\ (\lambda x:\I.x) & \in  ECtx^\varepsilon_\varepsilon            \\
     \Lambda\alpha.\square\ \epsilon & \in ECtx^\varepsilon_\varepsilon       \\
     \TBL_\alpha \TB_\alpha \TBL_\alpha \square & \in ECtx^\alpha_\varepsilon
\end{align*}
%
We write $E^A_B[M]$ for the term obtained by filling the hole $\square$ in $E^A_B$ by $M$.

Now we define staged reduction using the redex and evaluation contexts.

\begin{definition}[Staged Reduction]
    The staged reduction relation, written $M \longrightarrow_s M'$, is defined by
    the least relation closed under the rules below.
    \begin{align*}
        E^A_\varepsilon [(\lambda x:\tau.M)\ v^\varepsilon] & \longrightarrow_s E^A_\varepsilon[M[x\mapsto v^\varepsilon]]      \\
        E^A_\varepsilon [(\Lambda\alpha.v^\varepsilon)\ A]  & \longrightarrow_s E^A_\varepsilon[v^\varepsilon[\alpha\mapsto A]] \\
        E^A_\alpha [\TBL_\alpha \TB_\alpha v^\alpha]        & \longrightarrow_s E^A_\alpha[v^\alpha]                            \\
    \end{align*}
\end{definition}

This reduction relation represents a term reduces in a deterministic,
left-to-right, call-by-value manner.  An application of an abstraction
is executed only at stage \(\varepsilon\) and only a quotation at
stage \(\varepsilon\) is spiliced into the surrounding code---notice
that, if \(\TB_\alpha v^\alpha\) is at stage \(\varepsilon\), then the
redex \(\TBL_\alpha \TB_\alpha v^\alpha\) is at stage \(\alpha\).
In other words, terms in brackets are not evaluated until the terms are run
and arguments of a function are evaluated before the application.
We show an example of staged reduction.
Underlines show the redexes.
\begin{align*}
     & (\Lambda\alpha.(\TB_\alpha \underline{\TBL_\alpha \TB_\alpha ((\lambda x:\I.x)\ 10))})\ \varepsilon \\
      \longrightarrow_s & \underline{(\Lambda\alpha.(\TB_\alpha ((\lambda x:\I.x)\ 10)))\ \varepsilon}                   \\
      \longrightarrow_s & \underline{(\lambda x:\I.x)\ 10}                                                           \\
      \longrightarrow_s & 10
\end{align*}

% !TEX root = ../main.tex

\section{Properties of \LMD \label{sec:properties}}

In this section, we show basic properties of \LMD: preservation, strong normalization, confluence, and progress.

% Substitution Lemma

Substitution Lemma in \LMD\ is little more complicated than an ordinary one 
because there are six types of judgment and two types of substitution in \LMD.
Substitution Lemma on Terms states that term substitution $[z \mapsto P]$ preserves
typing, kinding, well-formed kinding, term equivalence, type equivalence, and king equivalence.
Substitution Lemma on Terms states the same for stage substitution $[\beta\mapsto A]$.
\red{2つの代入補題を指してSubstitution Lemmasということが出来るのか?}

In order to write Substitution Lemma briefly, we write $\G\V J@A$ for an arbitary judgment
amongst $\G\V M:\tau@A$, $\G\V \tau::K@A$, $\G\V K \iskind @A$, $\G\V M \E N : \tau @ A$, 
$\G\V \tau \E \sigma @ A$, and $\G \V K \E J @ A$.

\begin{lemma}[Substitution Lemma on Terms]
	If $\G, z:\xi@B, \D \V J @ A$ and $\G\V P:\xi @B$ then $\G, \D[z \mapsto P] \V J[z \mapsto P] @ A$.
\end{lemma}

\red{TODO: Fix the proof.}
\red{$\G\V J@A$における代入を定義する必要があるか?}

\begin{proof}
	Straightforward induction on derivations.
\end{proof}

\begin{lemma}[Substitution Lemma on Stages]
	If $\G \V J @ A$ then $\G[\beta\mapsto B] \V J[\beta\mapsto B] @ A[\beta\mapsto B]$.
\end{lemma}

\begin{proof}
	Straightforward induction on derivations.
\end{proof}

% Inversion Lemma

The following three Inversion Lemmas are needed to prove main theorems.
As usual~\cite{TAPL}, Inversion Lemmas enable us to infer the types of subterms of a term from the type of the term.

\begin{lemma}[Inversion Lemma for $\Pi$-types]
	If $\G \V (\lambda x:\sigma.M) : (\Pi x:\sigma'.\tau)@A$
	then $\G \V \sigma \E \sigma'@A$ and $\G ,x:\sigma@A\V M:\tau @A$.
\end{lemma}

\begin{proof}
	First, we generalize this theorem by adding statements about $\Pi$-types in equivalence relation in order to use induction.
	After that, we can prove with straightforward induction on derivations.
\end{proof}

\begin{lemma}[Inversion Lemma for $\TW$-types]
	If $\G \V \TB_\alpha M : \TW_\alpha \tau @A$ then $\G \V M : \tau @A$.
\end{lemma}

\begin{proof}
	Prove as with Inversion Lemma for $\Pi$-types.
\end{proof}

\begin{lemma}[Inversion Lemma for $\forall$-types]
	If $\G \V \Lambda\alpha.M : \forall\alpha.\tau @A$ then $\G \V M : \tau @A$ and $\alpha \notin \FTV(\G) \cup \FV(A)$.
\end{lemma}

\begin{proof}
	Prove as with Inversion Lemma for $\Pi$-types.
\end{proof}

% Preservation

Thanks to Substitution Lemmas and Inversion Lemmas, we can prove Preservation easily.
Preservations ensure that any one step reduction preserves type.

\begin{theorem}[Preservation]
	If $\G\V M:\tau @A$ and $M \longrightarrow M'$, then $\G\V M':\tau @A$.
\end{theorem}

\begin{proof}
	First, there are three cases for $M \longrightarrow M'$.
	They are $M \longrightarrow_\beta M'$, $M \longrightarrow_\Lambda M'$, and $M \longrightarrow_\blacklozenge M'$.
	For each case, we can use straightforward induction on derivations.
	Difficult cases are \TApp, \TTBL, and \TIns.
	We need Inversion Lemmas for them.
\end{proof}

% Strong Normalization

Strong Normalization is also important property which guarantee that
no typed term has an infinite reduction sequence.
We prove this thorem by translating \LMD to the symply typed lambda calculus.

\begin{theorem}[Strong Normalization]
	If $\G\V^A M:\tau$ then there is no infinite sequence of terms $(M_i)_{i\ge1}$ and 
	$M_i \longrightarrow M_{i+1}$ for $i\ge 1$.
\end{theorem}

\begin{proof}
	In order prove this theorem, we define a translation $\natural$ from \LMD\ to the symply typed lambda calculus.
	Second, we prove $\natural$ preserves typing and $\beta$-reductions.
	Then, we can prove Strong Normalization of \LMD\ from Strong Normalization of the symply typed lambda calculus.
\end{proof}

Confluence is a property that any reduction sequences from one typed term converge.
Because we have proved Strong Normalization, we can use Newman's Lemma to prove Confluence.

\begin{theorem}[Confluence]
	For any term $M$, if $M \longrightarrow^* M'$ and $M \longrightarrow^* M''$ then
	there exists $M'''$ that satisfies $M' \longrightarrow^* M'''$ and $M'' \longrightarrow^* M'''$.
\end{theorem}

\begin{proof}
	Because we proved Strong Normalization of \LMD, 
	we can use Newman's lemma to prove Confluence of \LMD.
	Then, what we must show is Weak Church-Rosser Property now.
	When we consider two different redexes in a \LMD term, they can only be disjoint, or one is a part of the other.
	In short, they are never overlapped each other.
	So, we can reduce one of them after we reduce another.
\end{proof}

Unique Decomposition ensure that,
for every typed term, we can find just one redex to reduce by the evaluation context or it is a value.
This theorem is important because it guarantee
that the evaluation context decides a redex to reduce deterministically.
Specifically speaking, this theorem guarantee that 
when you write a interpreter using the evaluation context of \LMD,
your interpreter works just as intended.
Although, in ordinary calculi, there is a condition that $\G$ is $\emptyset$,
the condition is relaxed because variables at non-$\varepsilon$ stages are values in \LMD.

\begin{theorem}[Unique Decomposition]
	If $\G$ does not have any variable declared at stage $\varepsilon$ 
	and $\G \V M : \tau @ A$ then either
	\begin{enumerate}
		\item $ M \in V^A$, or
		\item there exist $B, E^A_B$, and $R^B$ such that $M = E^A_B[R^B]$ with $B = \varepsilon$ or $B = \beta$ for some $\beta$.
	\end{enumerate}
\end{theorem}

\begin{proof}
	We can prove by straightforward induction on derivations.
	Difficult cases are \TApp, \TTBL, and \TIns.
	We need Inversion Lemmas for them.
\end{proof}

Progress states $\longrightarrow_s$ defines appropriate reduction for typed terms of \LMD.
Thanks to Unique Decomposition, we can prove this theorem easily.

\begin{theorem}[Progress]
	If $\G$ does not have any variable declared at stage $\varepsilon$ and $\G \V M : \tau  @ A$ then
	$ M \in V^A $ or $M'$ exists such that $M \longrightarrow_s M'$.
\end{theorem}

\begin{proof}
	We can prove by straightforward induction on derivations.
	Difficult cases are \TApp, \TTBL, and \TIns.
	We need Inversion Lemmas for them.
\end{proof}

% !TEX root = ../main.tex

\section{Related Work \label{sec:related-work}}

MetaOCaml is a programming language with quoting, unquoting, run, and
CSP.  Kiselyov give many applications of MetaOCaml in \cite{8384206},
which includes filtering in signal processing, matrix-vector product,
and DSL compiler.  

% 多段階計算の歴史

Theoretical studies on multi-stage programming owe a lot to seminal
work by Davies and Pfenning~\cite{DaviesPfenning01JACM} and
Davies~\cite{davies1996temporal}, which found Curry-Howard
correspondence between multi-stage calculi and modal logic.  In
particular, Davies' $\lambda^\circ$~\cite{davies1996temporal} has been
a basis for multi-stage calculi with quasi-quotation.  $\lambda^\circ$
did not have operators for run and CSP; a few
studies~\cite{benaissa1999logical,MoggiTahaBenaissaSheard99ESOP}
enhanced and improved $\lambda^\circ$ towards the development of a
type-safe multi-stage calculus with quasi-quotation, run, and CSP,
which were proposed by Taha and Sheard as constructs for multi-stage
programming~\cite{MetaML}.
% Benaissa et al.~\cite{benaissa1999logical} study the relationship between a multi-stage calculus and category theory or modal logic.
% Taha and Sheard introduced run and CSP to a multi-stage calculus in 
Finally, Taha and Nielsen invented the concept of environment
classifiers~\cite{taha2003environment} and developed a typed calculus
$\lambda^\alpha$, which was equipped with all the features above in a
type sound manner and formed a basis of earlier versions of MetaOCaml.
A different approach to type-safe multi-stage programming with
different semantics for quasi-quotations has been studied by Kim, Yi,
and Calcagno~\cite{DBLP:conf/popl/KimYC06}.

Later, Tsukada and Igarashi~\cite{Tsukada} found correspondence
between a variant of \(\lambda^\alpha\) called $\lambda^\TW$
and modal logic and showed that run could be represented as a special
case of application of a transition abstraction ($\Lambda\alpha$) to
the empty sequence $\varepsilon$.  Hanada and
Igarashi~\cite{Hanada2014} developed \LTP as an extension
$\lambda^\TW$ with CSP.
% and discuss code residualization which
% allows us to dump the quoted code into an external file.

% 他段階計算の機能拡張


% 多段階計算の応用



% 依存型の歴史

There is much work on dependent types and most of them are affected by
the pioneering work by Martin-L\"{o}f~\cite{martin1973intuitionstic}.
Among many dependent type systems such as
$\lambda^\Pi$~\cite{Meyer1986}, Calculus of
Constructions~\cite{coquand:inria-00076024}, and Edinburgh
LF~\cite{harper1993framework}, we base our work on \LLF~\cite{attapl}
(which is quite close to $\lambda^\Pi$ and Edinburgh LF) due to its
simplicity.  It has been well known that dependent types are useful to
express detailed properties of data structure at the type level such
as the size of data structures~\cite{Xi98} and typed abstract syntax
trees~\cite{DBLP:conf/dsl/LeijenM99,DBLP:conf/popl/XiCC03}.  The
vector addition discussed in Section~\ref{sec:formal} is also such an
example.

% 依存型の応用

% Practical applications of dependent types have been also studied.
% One can use dependent types in programming languages such as Idris~\cite{brady2013idris} or
% interactive theorem provers such as Coq~\cite{09thecoq} based on \cite{coquand:inria-00076024}.
% In Xi and Pfenning~\cite{Xi98}, they extended SML with restricted dependent types
% and succeeded in reducing the bounds checking of arrays.
% In Xi and Harper~\cite{xi2001dependently}, they design a type system for an assembly language and
% it is useful for speed up.
% Xi also gave dead code elimination and loop unrolling as applications
%  of dependent types~\cite{xi1999dependent}.

% 関連研究との比較

Although there are studies on combinations of multi-stage programming and other programming features such as mutable cells~\cite{kiselyov2016refined},
control operators~\cite{KameyamaKiselyovShan09PEPM,oishi2017staging},
a combination with dependent types has been little studied.
One exception is Brady and Hammond~\cite{brady2006dependently},
who have discussed a
combination of dependently typeed programming with staging in the
style of MetaOCaml to implement a staged interpreter, which is
statically guaranteed to generate well typed code.  However, they
focused on concrete programming examples and there is no theoretical investigation
of the programming language they used.

% 他の方法としてどのようなものが考えられたか

% Although we define type equivalence of \LMD with composition of equivalence rules,
% there is another candiate to define,
% which gives reduction on types and compare the results of reduction such as~\cite{sorensen2006lectures}.
% This method is better than one of \LMD because equivalence rules become simple.
% However, we reject it because it cannot handle CSP flexibly.


% !TEX root = ../main.tex

\section{Conclusion \label{sec:conclusion}}

We proposed new multi-stage calculus \LMD, which can use dependent types.
\LMD enable finer-grained typing of multi-stage programming comparing to existing type systems.
This extention prevent some bugs relating to generated functions with multi-stage programming,
such as \verb|vadd| function for vectors of a fixed length.
We proved properties of \LMD including preservation, confluence, 
strong normalization for full reduction, and progress for staged reduction.
The technical difficulty of extention is how to deal with the symbol of CSP $\%$ in the equation rules.
We resolve this problem by ignoring $\%$ under some condition.

Making algorithmic typing and equality rules for \LMD is a future work.
They would be complicated but possible 
because we can realize them by defining a new weak head normal form which contains $\%$ erasing rule.

% \begin{enumerate}
% 	\item Done: On Cross-Stage Persistence in Multi-Stage\cite{Hanada2014}
	      
% 	      CSPも入りました
% 	\item Done: Eliminating Array Bound Checking Through Dependent Types\cite{Xi98}
% 	\item Done: MetaML and Multi-stage Programming with Explicit Annotations\cite{MetaML}
% 	\item Done: Idris, a general-purpose dependently typed programming language: Design and implementation\cite{brady2013idris}
% 	\item Done: A Logical Foundation for Environment Classifiers\cite{Tsukada}
% 	\item Done: Environment classifiers\cite{taha2003environment}
% 	\item Done: A framework for defining logics\cite{harper1993framework}
	      
% 	      $\Sigma$ の使い方を確認した
% 	\item Done: The Design and Implementation of {BER} MetaOCaml - System Description\cite{oleg2014}
% 	\item Done: Dependent types in practical programming\cite{xi1999dependent}
	      
% 	      Section of Applicationを読んだ。Dead Code EliminationやLoop Unrollingにも使えるらしい。
	      
% 	\item Done: A dependently typed assembly language\cite{xi2001dependently}
	      
% 	      DTALの定義と制約solverを用いた型検査の定義
% 	\item Done: Refined Environment Classifiers\cite{kiselyov2016refined}
% 	\item Done: Staging with control: type-safe multi-stage programming with control operators\cite{oishi2017staging}
% 	\item Done: Logical Modalities and Multi-Stage Programming\cite{benaissa1999logical}
% 	\item \red{Partial evaluation and automatic program generation}\cite{jones1993partial}
% 	\item \red{Efficient multi-level generating extensions for program specialization}\cite{gluck1995efficient}
% 	\item \red{Run-time code generation and Modal-ML}\cite{wickline1998run}
% 	\item \red{C and tcc: a language and compiler for dynamic code generation}\cite{poletto1999c}
% 	\item \red{Run-time bytecode specialization}\cite{masuhara2001run}
% 	\item \red{A tour of Tempo: A program specializer for the C language}\cite{consel2004tour}
% 	\item \red{Optimizing ML with run-time code generation}\cite{lee1996optimizing}
% 	\item \red{Efficient incremental run-time specialization for free}\cite{marlet1999efficient}
% \end{enumerate}

% 2つの設計手法と今回採用した理由
% Conclusion送りか?
% When we design type equivalence rules of \LMD, there are two design choices.
% One is defining type equivalence by $\beta$-equality after we define $\beta$-reduction of types.
% Another is defining type equivalence directly by a combination of type equivalence rules.
% We adopt the latter one because it is convenient to handle CSP in the type equivalence.

% \section{First Section}
% \subsection{A Subsection Sample}
% Please note that the first paragraph of a section or subsection is
% not indented. The first paragraph that follows a table, figure,
% equation etc. does not need an indent, either.
% 
% Subsequent paragraphs, however, are indented.
% 
% \subsubsection{Sample Heading (Third Level)} Only two levels of
% headings should be numbered. Lower level headings remain unnumbered;
% they are formatted as run-in headings.
% 
% \paragraph{Sample Heading (Fourth Level)}
% The contribution should contain no more than four levels of
% headings. Table~\ref{tab1} gives a summary of all heading levels.
% 
% \begin{table}
% \caption{Table captions should be placed above the
% tables.}\label{tab1}
% \begin{tabular}{|l|l|l|}
% \hline
% Heading level &  Example & Font size and style\\
% \hline
% Title (centered) &  {\Large\bfseries Lecture Notes} & 14 point, bold\\
% 1st-level heading &  {\large\bfseries 1 Introduction} & 12 point, bold\\
% 2nd-level heading & {\bfseries 2.1 Printing Area} & 10 point, bold\\
% 3rd-level heading & {\bfseries Run-in Heading in Bold.} Text follows & 10 point, bold\\
% 4th-level heading & {\itshape Lowest Level Heading.} Text follows & 10 point, italic\\
% \hline
% \end{tabular}
% \end{table}
% 
% 
% \noindent Displayed equations are centered and set on a separate
% line.
% \begin{equation}
% x + y = z
% \end{equation}
% Please try to avoid rasterized images for line-art diagrams and
% schemas. Whenever possible, use vector graphics instead (see
% Fig.~\ref{fig1}).
% 
% \begin{figure}
% \includegraphics[width=\textwidth]{fig1.eps}
% \caption{A figure caption is always placed below the illustration.
% Please note that short captions are centered, while long ones are
% justified by the macro package automatically.} \label{fig1}
% \end{figure}
% 
% \begin{theorem}
% This is a sample theorem. The run-in heading is set in bold, while
% the following text appears in italics. Definitions, lemmas,
% propositions, and corollaries are styled the same way.
% \end{theorem}
% %
% % the environments 'definition', 'lemma', 'proposition', 'corollary',
% % 'remark', and 'example' are defined in the LLNCS documentclass as well.
% %
% \begin{proof}
% Proofs, examples, and remarks have the initial word in italics,
% while the following text appears in normal font.
% \end{proof}
% For citations of references, we prefer the use of square brackets
% and consecutive numbers. Citations using labels or the author/year
% convention are also acceptable. The following bibliography provides
% a sample reference list with entries for journal
% articles~\cite{ref_article1}, an LNCS chapter~\cite{ref_lncs1}, a
% book~\cite{ref_book1}, proceedings without editors~\cite{ref_proc1},
% and a homepage~\cite{ref_url1}. Multiple citations are grouped
% \cite{ref_article1,ref_lncs1,ref_book1},
% \cite{ref_article1,ref_book1,ref_proc1,ref_url1}.
%
% ---- Bibliography ----
%
% BibTeX users should specify bibliography style 'splncs04'.
% References will then be sorted and formatted in the correct style.
%
\bibliographystyle{splncs04}
\bibliography{main}
%
% \begin{thebibliography}{8}
% \bibitem{ref_article1}
% Author, F.: Article title. Journal \textbf{2}(5), 99--110 (2016)
% 
% \bibitem{ref_lncs1}
% Author, F., Author, S.: Title of a proceedings paper. In: Editor,
% F., Editor, S. (eds.) CONFERENCE 2016, LNCS, vol. 9999, pp. 1--13.
% Springer, Heidelberg (2016). \doi{10.10007/1234567890}
% 
% \bibitem{ref_book1}
% Author, F., Author, S., Author, T.: Book title. 2nd edn. Publisher,
% Location (1999)
% 
% \bibitem{ref_proc1}
% Author, A.-B.: Contribution title. In: 9th International Proceedings
% on Proceedings, pp. 1--2. Publisher, Location (2010)
% 
% \bibitem{ref_url1}
% LNCS Homepage, \url{http://www.springer.com/lncs}. Last accessed 4
% Oct 2017
% \end{thebibliography}

\iffullversion
\appendix

%\newtheorem{thm}{Theorem}
\newtheorem{dfn}{Defnition}
\newtheorem{ex}{Example}
\newtheorem{cm}{Comment}
\newcommand{\figheader}[2]{
  \begin{flushleft}
    #2 {\bf \normalsize #1}
\end{flushleft}}

\newpage
\section{Full Definition of \LMD}
\input{text/fulldefinitions}

\section{Proofs}
\iffullversion
\AI{Put a comma before ``then''.}
\AI{A period is needed for each statement (because it's a sentence).}
\AI{TAPL gives plenty of examples of how to write proofs.}
\begin{lemma}[Weakening]
	If \(\G \V J@A\) and \(\G\) is a subsequence of \(\D\), then \(\D \V J@A\). 
	\AI{Apply similar changes and remove \%.}
	\red{このコメントはどういう意味ですか?}
\end{lemma}

\begin{proof}
	\AI{Use the proof environment.}
	By straightforward induction.
	\begin{itemize}
		\item \WAbs
		\item \KAbs
		\item \TAbs
		\item \QKAbs
		\item \QTAbs
		\item \QAbs
		\item \QBeta
	\end{itemize}
\end{proof}
\fi

\begin{theorem}[Term Substitution]
	\AI{The statements should be fixed.}
	\begin{flalign*}
		\text{If\ } \G,z:\xi@B \V M:\tau@A \text{\ and\ } \G\V P:\xi@B
		&\text{\ then\ } \G\V M[z \mapsto P]:\tau[z \mapsto P]@A.&\\
		\text{If\ } \G,z:\xi@B \V \tau::K@A \text{\ and\ } \G\V P:\xi@B
		&\text{\ then\ } \G\V \tau[z \mapsto P]::K[z \mapsto P]@A.&\\
		\text{If\ } \G,z:\xi@B \V K\iskind@A \text{\ and\ } \G\V P:\xi@B
		&\text{\ then\ } \G\V K[z \mapsto P] \iskind @A.&\\
		\text{If\ } \G,z:\xi@B \V M\E N : \tau@A \text{\ and\ } \G\V P:\xi@B
		&\text{\ then\ } \G\V M[z \mapsto P]\E N[z \mapsto P] : \tau[z \mapsto P]@A.&\\
		\text{If\ } \G,z:\xi@B \V \tau\E \sigma : K@A \text{\ and\ } \G\V P:\xi@B
		&\text{\ then\ } \G\V \tau[z \mapsto P]\E \sigma[z \mapsto P] : K[z \mapsto P]@A.&\\
		\text{If\ } \G,z:\xi@B \V K\E J@A \text{\ and\ } \G\V P:\xi@B
		&\text{\ then\ } \G\V K[z \mapsto P]\E J[z \mapsto P]@A.&
	\end{flalign*}
\end{theorem}


Prove by induction on derivation tree.
\AI{The six items are proved simultaneously by induction on derivations with
case analysis on the last rule used.}

\begin{itemize}

\newcommand{\SB}{[z \mapsto P]}
\newcommand{\GG}{\G}
\newcommand{\GGV}{\G \V}

\iffullversion

	\item \WStar

	From the definition of \WStar, we can get $\mathcal{D}_1$.

	$\mathcal{D}_1$ = \infer[\WStar]
	{\GGV * \iskind @A}
	{}

	\item \WAbs

	We can assume $x \neq z$ because we can select fresh $x$ when we construct $\Pi$ type.

	We have two derivation trees from the premise.

	$\mathcal{D}_1$ = \infer[\WAbs]
	{\G, z:\xi@B, \D \V (\Pi x:\tau.K) \iskind @A}
	{\G, z:\xi@B, \D \V T::*@A \andalso \G, z:\xi@B, \D, x:\tau \V K \iskind@A}

	$\mathcal{D}_2$ = \infer[]
	{\G \V P:\xi@B}
	{\vdots}\

	We can get $\mathcal{D}_3$ by use the induction hypothesis to $\mathcal{D}_1$

	$\mathcal{D}_3$ = \infer[\WAbs]
	{\GGV (\Pi x:\tau\SB.K\SB) \iskind @A}
	{
		\infer[]{\GGV T\SB::*@A}{\vdots} \andalso
		\infer[]{\GG, x:\tau\SB \V K\SB \iskind@}{\vdots}
	}\\

	Following relationship is obvious.\\
	$\GGV (\Pi x:\tau\SB.K\SB) \iskind @A$\\
	is equivalent with\\
	$\GGV (\Pi x:\tau.K)\SB \iskind @A$.\\

	Then, we can get $\mathcal{D}'_3$ from $\mathcal{D}_3$.

	$\mathcal{D}'_3$ = \infer[\WAbs]
	{\GGV (\Pi x:\tau.K)\SB \iskind @A}
	{
		\infer[]{\GGV T\SB::*@A}{\vdots} \andalso
		\infer[]{\GG, x:\tau\SB \V K\SB \iskind@}{\vdots}
	}\\
        \AI{$\mathcal{D}'_3$ is the same as $\mathcal{D}_3$ because substitution is a meta-level operation.}
        
        \AI{Your proofs are good but people prefer to treat judgments
          as if they are English sentences and treat derivations
          implicitly (even when the proof is by induction on derivations.  So, I would write these cases as follows.}
      \item[] Case \WStar{} where $K = *$.  The conclusion is immediate since $K[z \mapsto P] = *$.

      \item[] Case \WAbs{} where $K = \Pi x:\tau.K_0$ and 
        \begin{align*}
          & \G, z:\xi@B, \D \V T::*@A && \G, z:\xi@B, \D, x:\tau \V K \iskind@A.
        \end{align*}

%	We can assume $x \neq z$ because we can select fresh $x$ when we construct $\Pi$ type.

        By the induction hypothesis, we have
        \begin{align*}
	& \GGV T\SB::*@A && \GG, x:\tau\SB \V K\SB \iskind@
        \end{align*}
        from which $\GGV (\Pi x:\tau\SB.K\SB) \iskind @A$ follows by \WAbs.
        We have $\Pi x:\tau\SB.K\SB = (\Pi x:\tau.K)\SB$ by definition.
        
	\item \WCsp

	From the induction hypothesis, we can get $\mathcal{D}_1$.

	$\mathcal{D}_1$ = \infer[]
	{\GGV K\SB \iskind @A}
	{\vdots}

	Use \WCsp,

	$\mathcal{D}_2$ = \infer[\WCsp]
	{\GGV K\SB \iskind @A\alpha}
	{\mathcal{D}_1}

	\item \WTW

	From the induction hypothesis, we can get $\mathcal{D}_1$.

	$\mathcal{D}_1$ = \infer[]
	{\GGV K\SB \iskind @A\alpha}
	{\vdots}

	Use \WTW,

	$\mathcal{D}_2$ = \infer[\WTW]
	{\GGV K\SB \iskind @A\alpha}
	{\mathcal{D}_1}

	\item \KVar

	From the induction hypothesis, we can get $\mathcal{D}_1$.

	$\mathcal{D}_1$ = \infer[]
	{\GGV K\SB \iskind @A\alpha}
	{\vdots}

	And we can easily show that $X::K\SB@A \in \GG$.

	Then we can use \KVar\ and get $\mathcal{D}_2$.

	$\mathcal{D}_2$ = \infer[\KVar]
	{\GGV X :: K\SB @A}
	{X::K\SB@A \in \GG \andalso \mathcal{D}_1}

	\item \KAbs

	From the induction hypothesis, we can get $\mathcal{D}_1$ and $\mathcal{D}_2$.

	$\mathcal{D}_1$ = \infer[]
	{\GGV \tau\SB :: * @ A}
	{\vdots}

	$\mathcal{D}_2$ = \infer[]
	{\GG, x:\tau\SB@A\V \sigma\SB::J\SB@A}
	{\vdots}

	Use \KAbs,

	$\mathcal{D}_3$ = \infer[\KAbs]
	{\GGV (\Pi x:\tau\SB.\sigma\SB)::(\Pi x:\tau\SB.J\SB)@A}
	{\mathcal{D}_1 \andalso \mathcal{D}_2}

	We can arrange the substitution.

	$\mathcal{D}'_3$ = \infer[\KAbs]
	{\GGV (\Pi x:\tau.\sigma)\SB::(\Pi x:\tau.J)\SB@A}
	{\mathcal{D}_1 \andalso \mathcal{D}_2}

	\item \KApp

	Because the last rule is \KApp, we have a derivation tree $\mathcal{D}_1$.

	$\mathcal{D}_1$ = \infer[\KApp]
	{\G, z:\xi@B, \D\V \sigma\ M :: K[x \mapsto M]}
	{\infer[]
		{\G, z:\xi@B, \D\V \sigma::(\Pi x:\tau.K)@A}{\vdots} \andalso
		\infer[]
		{\G, z:\xi@B, \D\V M:\tau@A}
		{\vdots}}

	From the induction hypothesis, we can get $\mathcal{D}_2$ and $\mathcal{D}_3$.

	$\mathcal{D}_2$ = \infer[]
	{\GGV \sigma\SB::(\Pi x:\tau.K)\SB@A}
	{\vdots}

	Because $x \neq z$, we can write

	$\mathcal{D}'_2$ = \infer[]
	{\GGV \sigma\SB::(\Pi x:\tau\SB.K\SB)@A}
	{\vdots}

	$\mathcal{D}_3$ = \infer[]
	{\GGV M\SB:\tau\SB@A}
	{\vdots}

	Use \KApp,

	$\mathcal{D}_4$ = \infer[\KApp]
	{\GGV (\sigma\SB\ M\SB)::K\SB[x \mapsto M]@A}
	{\mathcal{D}'_2 \andalso \mathcal{D}_3}

	There is no $x$ in $P$ and no $z$ in $M$ because of freshness. So we can rewrite the $\mathcal{D}_4$.

	$\mathcal{D}_4$ = \infer[\KApp]
	{\GGV (\sigma\ M)\SB::K[x \mapsto M]\SB@A}
	{\mathcal{D}'_2 \andalso \mathcal{D}_3}

	\item \KConv

	From the induction hypothesis, we have $\mathcal{D}_1$ and $\mathcal{D}_2$.

	$\mathcal{D}_1$ = \infer[]
	{\GGV \tau\SB : K[z\mapsto P]@A}
	{\vdots}

	$\mathcal{D}_2$ = \infer[]
	{\GGV K\SB \E J\SB @A}
	{\vdots}

	Use \KConv,

	$\mathcal{D}_2$ = \infer[\KConv]
	{\GGV \tau\SB : J[z\mapsto P]@A}
	{\mathcal{D}_1 \andalso \mathcal{D}_2}

	\item \KTW

	From the induction hypothesis, we have $\mathcal{D}_1$.

	$\mathcal{D}_1$ = \infer[]
	{\GGV \tau\SB :: K\SB @ A\alpha}
	{\vdots}

	Use \KTW,

	$\mathcal{D}_2$ = \infer[]
	{\GGV \TW_\alpha \tau\SB :: K\SB @ A}
	{\mathcal{D}_1}

	\item \KTWL

	From the induction hypothesis, we have $\mathcal{D}_1$.

	$\mathcal{D}_1$ = \infer[]
	{\GGV \TW_\alpha \tau\SB :: K\SB @ A\alpha}
	{\vdots}

	Use \KTWL,

	$\mathcal{D}_2$ = \infer[]
	{\GGV \tau\SB :: K\SB @ A\alpha}
	{\mathcal{D}_1}

	\item \KGen

	From the induction hypothesis, we have $\mathcal{D}_1$.

	$\mathcal{D}_1$ = \infer[]
	{\GGV \tau\SB :: K\SB @ A}
	{\vdots}

	And we can prove easily $\alpha \notin \FTV(\GG) \cup \FTV(A)$.

	Use \KGen,

	$\mathcal{D}_2$ = \infer[\KGen]
	{\GGV \forall\alpha.\tau\SB :: K\SB @ A}
	{\mathcal{D}_1 \andalso \alpha \notin \FTV(\GG) \cup \FTV(A)}

	\item \KCsp

	From the induction hypothesis, we have $\mathcal{D}_1$.

	$\mathcal{D}_1$ = \infer[]
	{\GGV \tau\SB :: K\SB @ A}
	{\vdots}

	Use \KCsp,

	$\mathcal{D}_2$ = \infer[\KCsp]
	{\GGV \tau\SB :: K\SB @ A\alpha}
	{\GGV \tau\SB :: K\SB @ A}

\fi

\item \TVar

We have two derivation trees from the premise.

$\mathcal{D}_1$ = \infer[\TVar]
{\G, z:\xi@B \V y:\tau@A}
{y:\tau@A \in \G, z:\xi@B  \andalso \infer[]{\G, z:\xi@B \V \tau::*@A}{\vdots}}

$\mathcal{D}_2$ = \infer[]
{\G \V P:\xi@B}
{\vdots}\\

We can get $\mathcal{D}_3$ by use the induction hypothesis to $\mathcal{D}_1$.

$\mathcal{D}_3$ = \infer[]
{\GGV \tau\SB::*\SB@A}
{\vdots}\\

\begin{itemize}
	\item $y:\tau@A \in \G$ or $y:\tau@A \in \D$

	      $\mathcal{D}_4$ is obvious.

	      $\mathcal{D}_4$ = $y:\tau\SB@A \in \GG$

	      Get $\mathcal{D}_5$ by using \TVar\ for $\mathcal{D}_3$, $\mathcal{D}_4$.

	      $\mathcal{D}_5$ = \infer[]
	      {\GGV y\SB:\tau\SB@A}
	      {\mathcal{D}_4 \andalso \mathcal{D}_5}

	\item $y:\tau@A = z:\xi@B$

	      In this case,
	      \begin{itemize}
		      \item $y = z$ \AI{Needs a comma.}
		      \item $\tau = \xi$ \AI{Needs ``, and''.}
		      \item $A = B$ \AI{Needs a period.  In general, you have to be able read as if it's a usual sentence.  Usual grammar rules apply.}
                      \end{itemize}
                      

	      Because there is no $z$ in $\xi$, $\tau\SB = \xi\SB = \xi$.
	      And it is obvious that $y\SB = z\SB = P$.

	      From $\mathcal{D}_2$, $\G \V y\SB : \tau\SB@A$. Use Weakening lemma, \AI{By Weakening,} $\GGV y\SB : \tau\SB$.
\end{itemize}

From and \TVar\\
$\GGV y\SB:\tau\SB$

\iffullversion

	\item \TAbs

	From the induction hypothesis and \TAbs, we get

	$\mathcal{D}_1$ = \infer[\TAbs]
	{\GGV (\lambda y:\sigma\SB.M\SB):(\Pi y:\sigma\SB.\tau\SB)@A}
	{\infer[]{\GGV \sigma\SB::*@A}{\vdots} \andalso \infer[]{\GG, y:\sigma\SB@A \V M\SB:\tau\SB@A}{\vdots}}

	Arrange substitutions,

	$\mathcal{D}'_1$ = \infer[\TAbs]
	{\GGV (\lambda y:\sigma.M)\SB:(\Pi y:\sigma.\tau)\SB@A}
	{\infer[]{\GGV \sigma\SB::*@A}{\vdots} \andalso \infer[]{\GG, y:\sigma\SB@A \V M\SB:\tau\SB@A}{\vdots}}

	\item \TApp

	We have a derivation trees from the premise.

	$\mathcal{D}_1$ = \infer[\TApp]
	{\G, z:\xi@B, \D \V M\ L:\tau[y\mapsto L]@A}
	{\infer[]{\G, z:\xi@B, \D \V M:\Pi(y:\rho).\tau@A \andalso \G, z:\xi@B, \D \V L:\rho@A}{\vdots}}

	We get 2 trees from the induction hypothesis and $\mathcal{D}_1$.

	$\mathcal{D}_2$ = \infer[]
	{\GGV M\SB: (\Pi(y:\rho).\tau)\SB@A}
	{\vdots}

	and

	$\mathcal{D}_3$ = \infer[]
	{\GGV L\SB: \rho\SB@A}
	{\vdots}

	\red{distribute the substitution} in $\mathcal{D}_2$

	$\mathcal{D'}_2$ = \infer[]
	{\GGV M\SB: (\Pi(y:\rho\SB).\tau\SB)@A}
	{\vdots}

	From $\mathcal{D'}_2$ and $\mathcal{D}_3$

	$\mathcal{D}_4$ = \infer[\TApp]
	{\GGV (M\SB\ L\SB): \tau\SB[y \mapsto L]@A}
	{\mathcal{D'}_2 \andalso \mathcal{D}_3}

	\red{We can transform the conclusion of $\mathcal{D}_4$ into} \\
	$\GGV (M\ L)\SB: \tau[y \mapsto L]\SB@A$

	\item \TConv

	We have 2 derivation trees from the premise and the induction hypothesis.

	$\mathcal{D}_1$ = \infer[]
	{\GGV t\SB:T\SB@A}
	{\vdots}

	$\mathcal{D}_2$ = \infer[]
	{\GGV T\SB \E T'\SB@A}
	{\vdots}

	And then \\
	\infer[\TConv]
	{\GGV t\SB:T'\SB@A}
	{\mathcal{D}_1 \andalso \mathcal{D}_2}

	\item \TTB

	From the induction hypothesis and \TTB, we get

	$\mathcal{D}_3$ = \infer[\TTB]
	{\GGV \TB_\alpha M\SB:\tau\SB@A}
	{\infer[]{\GGV M\SB:\tau\SB@A\alpha}{\vdots}}

	\item \TTBL

	From the induction hypothesis and \TTBL, we get

	$\mathcal{D}_3$ = \infer[\TTBL]
	{\GGV \TBL_\alpha M\SB:\tau\SB@A}
	{
		\infer[]{\GGV M\SB: \TW_\alpha \tau\SB@A\alpha}{\vdots}
	}

	\item \TGen

	From the induction hypothesis, we get $\mathcal{D}_1$.

	$\mathcal{D}_1$ = \infer[]
	{\GGV M\SB:\tau\SB@A}
	{\vdots}

	And we can prove easily $\alpha \notin \FTV(\GG) \cup \FTV(A)$.

	Use \TGen,

	$\mathcal{D}_2$ = \infer[\TGen]
	{\GGV (\Lambda\alpha.M)\SB:(\forall\alpha.\tau)\SB@A}
	{\mathcal{D}_1 \andalso \alpha \notin \FTV(\GG) \cup \FTV(A)}

	\item \TIns

	From the induction hypothesis, we get $\mathcal{D}_1$.

	$\mathcal{D}_1$ = \infer[]
	{\GGV M\SB:(\forall\alpha.\tau)\SB@A}
	{\vdots}

	Use \TIns,

	$\mathcal{D}_2$ = \infer[\TIns]
	{\GGV (M\ \varepsilon)\SB:\tau\SB@A}
	{\mathcal{D}_1}

	\item \TCsp

	From the induction hypothesis, we get $\mathcal{D}_1$.

	$\mathcal{D}_1$ = \infer[]
	{\GGV M\SB:\tau\SB@A}
	{\vdots}

	Use \TCsp,

	$\mathcal{D}_2$ = \infer[\TCsp]
	{\GGV (\%_\alpha M)\SB:\tau\SB@A\alpha}
	{\mathcal{D}_1}

	\item \QKAbs

	From the induction hypothesis and \QKAbs, we get $\mathcal{D}_1$.

	$\mathcal{D}_1$ = \infer[\QKAbs]
	{\GGV (\Pi x:\tau.K)\SB \E (\Pi x:\sigma.J)\SB@A}
	{\infer[]{\GGV \tau\SB \E \sigma\SB :: *@A}{\vdots} \andalso
		\infer[]{\GG,x:\tau@A \V K\SB\E J\SB@A}{\vdots} }

	\item \QKCsp

	From the induction hypothesis and \QKCsp, we get $\mathcal{D}_1$.

	$\mathcal{D}_1$ = \infer[\QKCsp]
	{\GGV K\SB \E J\SB @A\alpha}
	{\infer[]{\GGV K\SB \E J\SB @A}{\vdots}}

	\item \QKRefl

	From the induction hypothesis and \QKRefl, we get $\mathcal{D}_1$.

	$\mathcal{D}_1$ = \infer[\QKRefl]
	{\GGV K\SB \E K\SB @A}
	{\infer[]{\GGV K\SB \iskind @A}{\vdots}}

	\item \QKSym

	From the induction hypothesis and \QKSym, we get $\mathcal{D}_1$.

	$\mathcal{D}_1$ = \infer[\QKSym]
	{\GGV J\SB\E K\SB@A}
	{\GGV K\SB\E J\SB@A}

	\item \QKTrans

	From the induction hypothesis and \QKTrans, we get $\mathcal{D}_1$.

	$\mathcal{D}_1$ = \infer[\QKTrans]
	{\GGV K\SB\E I\SB@A}
	{\GGV K\SB\E J\SB@A \andalso \GGV J\SB\E I\SB@A}

	\item \QTAbs

	From the induction hypothesis and, we get $\mathcal{D}_1$.

	$\mathcal{D}_1$ = \infer[\QTAbs]
	{\GG \Pi x:\tau\SB.\rho\SB \E \Pi x:\sigma\SB.\pi\SB@A}
	{\ID{\GG \tau\SB \E \sigma\SB@A} \andalso \ID{\G, \D, x:\tau \V \rho\SB \E \pi\SB}}

	Arrange substitutions,

	$\mathcal{D}'_1$ = \infer[\QTAbs]
	{\GG (\Pi x:\tau.\rho)\SB \E (\Pi x:\sigma.\pi)\SB@A}
	{\ID{\GG \tau\SB \E \sigma\SB@A} \andalso \ID{\G, \D, x:\tau \V \rho\SB \E \pi\SB}}

	\item \QTApp

	From the induction hypothesis and \QTApp, we get $\mathcal{D}_1$.

	$\mathcal{D}_1$ = \infer[\QTApp]
	{\GGV\tau\SB\ M\SB \E \sigma\SB\ N\SB@A}
	{\ID{\GGV\tau\SB\E\sigma\SB :: (\Pi x:\rho.K)@A} \andalso \ID{\GGV M\SB\E N\SB:\rho@A}}

	Arrange substitutions,

	$\mathcal{D}'_1$ = \infer[\QTApp]
	{\GGV(\pi\ M)\SB \E (\sigma\ N)\SB@A}
	{\ID{\GGV\tau\SB\E\sigma\SB :: (\Pi x:\rho.K)@A} \andalso \ID{\GGV M\SB\E N\SB:\rho@A}}

	\item \QTTW

	From the induction hypothesis and \QTTW, we get $\mathcal{D}_1$.

	$\mathcal{D}_1$ = \infer[\QTTW]
	{\GGV(\TW_\alpha \tau)\SB\E(\TW_\alpha\sigma)\SB@A}
	{\ID{\GGV\tau\SB\E\sigma\SB@A\alpha}}

	\item \QTGen

	We can prove easily $\alpha \notin \FTV(\GG) \cup \FTV(A)$.
	From the induction hypothesis and \QTGen, we get $\mathcal{D}_1$.

	$\mathcal{D}_1$ = \infer[\QTGen]
	{\GGV (\forall\alpha.\tau)\SB \E (\forall\alpha.\sigma)\SB@A}
	{\ID{\GGV \tau\SB \E \sigma\SB@A} \andalso \alpha \notin \FTV(\GG) \cup \FTV(A)}

	\item \QTCsp

	From the induction hypothesis and \QTCsp, we get $\mathcal{D}_1$.

	$\mathcal{D}_1$ = \infer[\QTCsp]
	{\GGV\tau\SB \E \sigma\SB@A\alpha}
	{\ID{\GGV\tau\SB \E \sigma\SB@A}}

	\item \QTRefl

	From the induction hypothesis and \QTRefl, we get $\mathcal{D}_1$.

	$\mathcal{D}_1$ = \infer[\QTRefl]
	{\GGV\tau\SB\E\tau\SB@A}
	{\ID{\GGV\tau\SB::K\SB@A}}

	\item \QTSym

	From the induction hypothesis and \QTSym, we get $\mathcal{D}_1$.

	$\mathcal{D}_1$ = \infer[\QTSym]
	{\GGV\sigma\SB\E\tau\SB@A}
	{\ID{\GGV\tau\SB\E\sigma\SB@A}}

	\item \QTTrans

	From the induction hypothesis and \QTTrans, we get $\mathcal{D}_1$.

	$\mathcal{D}_1$ = \infer[\QTTrans]
	{\GGV \tau\SB\E\rho\SB@A}
	{\ID{\GGV\tau\SB\E\sigma\SB@A} \andalso \ID{\GGV\sigma\SB\E\rho\SB@A}}

	\item \QAbs

	From the induction hypothesis and \QAbs, we get $\mathcal{D}_1$.

	$\mathcal{D}_1$ = \infer[\QAbs]
	{\GGV \Pi x:\tau\SB.\rho\SB \E \Pi x:\sigma\SB.\pi\SB@A}
	{\ID{\GGV\tau\SB \E \sigma\SB :: * @A} \andalso \ID{\GG,x:\tau\SB@A\V\rho\SB \E \pi\SB@A}}

	Arrange substitutions,

	$\mathcal{D}'_1$ = \infer[\QAbs]
	{\GGV (\Pi x:\tau.\rho)\SB \E (\Pi x:\sigma.\pi)\SB@A}
	{\ID{\GGV\tau\SB \E \sigma\SB :: * @A} \andalso \ID{\GG,x:\tau\SB@A\V\rho\SB \E \pi\SB@A}}

	\item \QApp

	From the induction hypothesis

	$\mathcal{D}_1$ = \ID{\GGV M\SB \E L\SB :: (\Pi x:\sigma.\tau)\SB@A}

	Arrange substitutions,

	$\mathcal{D}'_1$ = \ID{\GGV M\SB \E L\SB :: (\Pi x:\sigma\SB.\tau\SB)@A}

	Using \QApp to $\mathcal{D}'_1$ and the induction hypothesis, we get $\mathcal{D}_2$.

	$\mathcal{D}_2$ = \infer[\QApp]
	{\GGV M\SB\ N\SB \E L\SB\ O\SB @A}
	{\mathcal{D}'_1 \andalso \ID{\GGV N\SB \E O\SB : \sigma\SB @A}}

	Arrange substitutions,

	$\mathcal{D}'_2$ = \infer[\QApp]
	{\GGV (M\ N)\SB \E (L\ O)\SB @A}
	{\mathcal{D}'_1 \andalso \ID{\GGV N\SB \E O\SB : \sigma\SB @A}}

	\item \QTB

	From the induction hypothesis and \QTB, we get $\mathcal{D}_1$.

	$\mathcal{D}_1$ = \infer[\QTB]
	{\GGV\TB_\alpha M\SB \E \TB_\alpha N\SB @A}
	{\GGV M\SB \E N\SB @A\alpha}

	\item \QTBL

	From the induction hypothesis and \QTBL, we get $\mathcal{D}_1$.

	$\mathcal{D}_1$ = \infer[\QTBL]
	{\GGV\TBL_\alpha M\SB \E \TBL_\alpha N\SB@A\alpha}
	{\GGV M\SB \E N\SB : \TW_\alpha \tau@A}

	\item \QGen

	We can prove easily $\alpha \notin \FTV(\GG) \cup \FTV(A)$.
	From the induction hypothesis and \QGen, we get $\mathcal{D}_1$.

	$\mathcal{D}_1$ = \infer[\QGen]
	{\GGV\Lambda\alpha.M\SB \E \Lambda\alpha.N\SB@A}
	{\ID{\GGV M\SB \E N\SB @A} \andalso \alpha \notin \FTV(\GG) \cup \FTV(A)}

	\item \QIns

	From the induction hypothesis and \QIns, we get $\mathcal{D}_1$.

	$\mathcal{D}_1$ = \infer[\QIns]
	{\GGV M\SB \E N\SB : \TW_\alpha.\tau @A}
	{\ID{\GGV M\SB\ \varepsilon \E N\SB\ \varepsilon @A }}

	\item \QCsp

	From the induction hypothesis and \QCsp, we get $\mathcal{D}_1$.

	$\mathcal{D}_1$ = \infer[\QCsp]
	{\GGV \%_\alpha M\SB \E \%_\alpha N\SB @A}
	{\ID{\GGV M\SB \E N\SB @A\alpha}}

	\item \QRefl

	From the induction hypothesis and \QRefl, we get $\mathcal{D}_1$.

	$\mathcal{D}_1$ = \infer[\QRefl]
	{\GGV M\SB \E M\SB @A}
	{\ID{\GGV M\SB : \tau\SB @A}}

	\item \QSym

	From the induction hypothesis and \QSym, we get $\mathcal{D}_1$.

	$\mathcal{D}_1$ = \infer[\QSym]
	{\GGV N\SB \E M\SB @A}
	{\ID{\GGV M\SB \E N\SB @A}}

	\item \QTrans

	From the induction hypothesis and \QTrans, we get $\mathcal{D}_1$.

	$\mathcal{D}_1$ = \infer[\QTrans]
	{\GGV M\SB \E L\SB @A}
	{\ID{\GGV M\SB \E N\SB @A } \andalso \ID{\GGV N\SB \E L\SB @A}}

	\item \QBeta

	From the induction hypothesis and \QBeta, we get $\mathcal{D}_1$.

	$\mathcal{D}_1$ = \infer[\QBeta]
	{\GGV (\lambda x:\sigma\SB:M\SB)\ N\SB \E (M\SB)[x \mapsto N\SB]@A}
	{\ID{\GG, x: \sigma\SB@A \V M\SB:\tau\SB@A} \andalso \ID{\GGV N\SB:\sigma\SB @A }}

	Arrange substitutions,

	$\mathcal{D}'_1$ = \infer[\QBeta]
	{\GGV ((\lambda x:\sigma:M)\ N)\SB \E (M[x \mapsto N])\SB@A}
	{\ID{\GG, x: \sigma\SB@A \V M\SB:\tau\SB@A} \andalso \ID{\GGV N\SB:\sigma\SB @A }}

	\item \QEta

	From the induction hypothesis and \QEta, we get $\mathcal{D}_1$.

	$\mathcal{D}_1$ = \infer[\QEta]
	{\GGV (\lambda x:\sigma\SB.M\SB\ x) \E M\SB@A}
	{\ID{\GGV M\SB : (\Pi x:\sigma\SB.\tau\SB)@A} \andalso x \notin \FV(M\SB)}

	Arrange substitutions,

	$\mathcal{D}'_2$ = \infer[\QEta]
	{\GGV (\lambda x:\sigma.M\ x)\SB \E M\SB@A}
	{\ID{\GGV M\SB : (\Pi x:\sigma\SB.\tau\SB)\SB@A} \andalso x \notin \FV(M\SB)}

	\item \QTBLTB

	From the induction hypothesis and \QTBLTB, we get $\mathcal{D}_1$.

	$\mathcal{D}_1$ = \infer[\QTBLTB]
	{\GGV \TBL_\alpha \TB_\alpha M\SB \E N\SB@A}
	{\ID{\GGV M\SB \E N\SB @A}}

	\item \QLambda

	From the induction hypothesis and \QLambda, we get $\mathcal{D}_1$.

	$\mathcal{D}_1$ = \infer[\QLambda]
	{\GGV (\Lambda\alpha.M\SB)\ \varepsilon \E M\SB[\alpha \mapsto \varepsilon]}
	{\ID{\GGV (\Lambda\alpha.M\SB) : \forall\alpha.\tau\SB @A}}

	Arrange substitutions,

	$\mathcal{D}'_1$ = \infer[\QLambda]
	{\GGV ((\Lambda\alpha.M)\ \varepsilon)\SB \E M[\alpha \mapsto \varepsilon]\SB}
	{\ID{\GGV (\Lambda\alpha.M\SB) : \forall\alpha.\tau\SB @A}}


	\item \QPercent

	From the induction hypothesis and \QPercent, we get $\mathcal{D}_1$.

	$\mathcal{D}_1$ = \infer[\QPercent]
	{\GGV \%_\alpha M\SB \E M\SB @ A\alpha}
	{\ID{\GGV M\SB : \tau\SB @A\alpha} \andalso \ID{\GGV M\SB : \sigma\SB @A} }

\fi

\end{itemize}

\begin{lemma}[Stage Substitution]
	\begin{flalign*}
		\text{If\ } \G \V M:\tau@A
		&\text{\ then\ } \G[\beta \mapsto B]\V M[\beta \mapsto B]:\tau[\beta \mapsto B]@A[\beta \mapsto B].&\\
		\text{If\ } \G \V \tau::K@A
		&\text{\ then\ } \G[\beta \mapsto B]\V \tau[\beta \mapsto B]::K[\beta \mapsto B]@A[\beta \mapsto B].&\\
		\text{If\ } \G \V K\iskind@A
		&\text{\ then\ } \G[\beta \mapsto B]\V K[\beta \mapsto B] \iskind@A[\beta \mapsto B].&\\
		\text{If\ } \G \V M\E N : \tau@A
		&\text{\ then\ } \G[\beta \mapsto B]\V M[\beta \mapsto B]\E N[\beta \mapsto B] : \tau[\beta \mapsto B] @A[\beta \mapsto B].&\\
		\text{If\ } \G \V \tau\E \sigma : K@A
		&\text{\ then\ } \G[\beta \mapsto B]\V \tau[\beta \mapsto B]\E \sigma[\beta \mapsto B] : K[\beta \mapsto B]@A[\beta \mapsto B].&\\
		\text{If\ } \G \V K\E J@A
		&\text{\ then\ } \G[\beta \mapsto B]\V K[\beta \mapsto B]\E J[\beta \mapsto B]@A[\beta \mapsto B].&
	\end{flalign*}
\end{lemma}

\begin{itemize}

	\newcommand{\SB}{[\beta \mapsto B]}
	\newcommand{\GG}{\G\SB}
	\newcommand{\GGV}{\G\SB \V}

	\iffullversion

	\item \WStar

	      From the definition of \WStar, we can get $\mathcal{D}_1$.

	      $\mathcal{D}_1$ = \infer[\WStar]
	      {\GGV * \iskind @A\SB}
	      {}

	\item \WAbs

	      From the induction hypothesis and \WAbs, we get $\mathcal{D}_1$.

	      $\mathcal{D}_1$ = \infer[\WAbs]
	      {\GGV (\Pi x:\tau.K)\SB \iskind @A}
	      {
		      \ID{\GGV \tau\SB::*@A\SB} \andalso
		      \ID{\GG, x:\tau\SB \V K\SB \iskind@}
	      }

	\item \WCsp

	      \begin{itemize}

		      \item $\alpha \neq \beta$

		            From the induction hypothesis and \WCsp, we can get $\mathcal{D}_1$.

		            $\mathcal{D}_1$ = \infer[\WCsp]
		            {\GGV K\SB \iskind @A\alpha\SB}
		            {\ID{\GGV K\SB \iskind @A\SB}}

		      \item $\alpha = \beta$

		            The conclusion is identical with the induction hypothesis.

	      \end{itemize}

	\item \WTW

	      \begin{itemize}

		      \item $\alpha \neq \beta$

		            From the induction hypothesis and \WTW, we can get $\mathcal{D}_1$.

		            $\mathcal{D}_1$ = \infer[\WTW]
		            {\GGV K\SB \iskind @A\SB}
		            {\ID{\GGV K\SB \iskind @A\alpha\SB}}

		      \item $\alpha = \beta$

		            The conclusion is identical with the induction hypothesis.

	      \end{itemize}

	\item \KVar

	      We can easily show that $X::K\SB \in \GG$.
	      From the induction hypothesis and \KVar, we can get $\mathcal{D}_1$.

	      $\mathcal{D}_1$ = \infer[\KVar]
	      {\GGV X :: K\SB @A\SB}
	      {X::K\SB \in \GG \andalso \ID{\GGV K\SB \iskind @A\SB}}

	\item \KAbs

	      From the induction hypothesis, we can get $\mathcal{D}_1$ and $\mathcal{D}_2$.

	      $\mathcal{D}_1$ = \infer[]
	      {\GGV \tau\SB :: * @ A\SB}
	      {\vdots}

	      $\mathcal{D}_2$ = \infer[]
	      {\GG, x:\tau\SB@A\V \sigma\SB::J\SB@A\SB}
	      {\vdots}

	      Use \KAbs,

	      $\mathcal{D}_3$ = \infer[\KAbs]
	      {\GGV (\Pi x:\tau\SB.\sigma\SB)::(\Pi x:\tau\SB.J\SB)@A\SB}
	      {\mathcal{D}_1 \andalso \mathcal{D}_2}

	      We can arrange the substitution.

	      $\mathcal{D}'_3$ = \infer[\KAbs]
	      {\GGV (\Pi x:\tau.\sigma)\SB::(\Pi x:\tau.J)\SB@A\SB}
	      {\mathcal{D}_1 \andalso \mathcal{D}_2}

	\item \KApp

	      From the induction hypothesis and \KApp, we can get $\mathcal{D}_1$.

	      $\mathcal{D}_1$ = \infer[\KApp]
	      {\GGV (\sigma\SB\ M\SB)::K\SB[x \mapsto M\SB]@A\SB}
	      {\ID{\GGV \sigma\SB::(\Pi x:\tau\SB.K\SB)@A\SB} \andalso \ID{\GGV M\SB:\tau\SB@A}}

	      Arrange substitutions,

	      $\mathcal{D}'_1$ = \infer[\KApp]
	      {\GGV (\sigma\ M)\SB::K[x \mapsto M]\SB@A\SB}
	      {\ID{\GGV \sigma\SB::(\Pi x:\tau\SB.K\SB)@A\SB} \andalso \ID{\GGV M\SB:\tau\SB@A\SB}}

	\item \KConv

	      From the induction hypothesis, we have $\mathcal{D}_1$ and $\mathcal{D}_2$.

	      $\mathcal{D}_1$ = \infer[]
	      {\GGV \tau\SB : K[z\mapsto P]@A\SB}
	      {\vdots}

	      $\mathcal{D}_2$ = \infer[]
	      {\GGV K\SB \E J\SB @A\SB}
	      {\vdots}

	      Use \KConv,

	      $\mathcal{D}_2$ = \infer[\KConv]
	      {\GGV \tau\SB : J[z\mapsto P]@A\SB}
	      {\mathcal{D}_1 \andalso \mathcal{D}_2}

	\item \KTW

	      \begin{itemize}

		      \item $\alpha \neq \beta$

		            From the induction hypothesis, we have $\mathcal{D}_1$.

		            $\mathcal{D}_1$ = \infer[]
		            {\GGV \tau\SB :: K\SB @ A\alpha\SB}
		            {\vdots}

		            Use \KTW,

		            $\mathcal{D}_2$ = \infer[]
		            {\GGV \TW_\alpha \tau\SB :: K\SB @ A\SB}
		            {\mathcal{D}_1}

		      \item $\alpha = \beta$

		            The conclusion is identical with the induction hypothesis.

	      \end{itemize}

	\item \KTWL

	      \begin{itemize}

		      \item $\alpha \neq \beta$

		            From the induction hypothesis, we have $\mathcal{D}_1$.

		            $\mathcal{D}_1$ = \infer[]
		            {\GGV \TW_\alpha \tau\SB :: K\SB @ A\alpha\SB}
		            {\vdots}

		            Use \KTWL,

		            $\mathcal{D}_2$ = \infer[]
		            {\GGV \tau\SB :: K\SB @ A\alpha\SB}
		            {\mathcal{D}_1}

		      \item $\alpha = \beta$

		            The conclusion is identical with the induction hypothesis.

	      \end{itemize}


	\item \KGen

	      From the induction hypothesis, we have $\mathcal{D}_1$.

	      $\mathcal{D}_1$ = \infer[]
	      {\GGV \tau\SB :: K\SB @ A\SB}
	      {\vdots}

	      And we can prove easily $\alpha \notin \FTV(\GG) \cup \FTV(A)$.

	      Use \KGen,

	      $\mathcal{D}_2$ = \infer[\KGen]
	      {\GGV \forall\alpha.\tau\SB :: K\SB @ A\SB}
	      {\mathcal{D}_1 \andalso \alpha \notin \FTV(\GG) \cup \FTV(A)}

	\item \KCsp

	      \begin{itemize}

		      \item $\alpha \neq \beta$

		            From the induction hypothesis and \KCsp, we have $\mathcal{D}_1$.

		            $\mathcal{D}_1$ = \infer[\KCsp]
		            {\GGV \tau\SB :: K\SB @ A\alpha\SB}
		            {\ID{\GGV \tau\SB :: K\SB @ A\SB}}

		      \item $\alpha = \beta$

		            The conclusion is identical with the induction hypothesis.

	      \end{itemize}

	\item \TVar

	      We can easily prove $x:\tau\SB \in \GG$.

	      From the induction hypothesis and \TVar, we have $\mathcal{D}_1$.

	      $\mathcal{D}_1$ = \infer[]
	      {\GGV x:\tau\SB @A\SB}
	      {x:\tau\SB \in \GG \andalso \ID{\GGV \tau\SB::*@A\SB}}

	\item \TAbs

	      From the induction hypothesis and \TAbs, we get

	      $\mathcal{D}_1$ = \infer[\TAbs]
	      {\GGV (\lambda x:\sigma\SB.M\SB):(\Pi x:\sigma\SB.\tau\SB)@A\SB}
	      {\ID{\GGV \sigma\SB::*@A\SB} \andalso \ID{\GG, x:\sigma\SB@A\SB \V M\SB:\tau\SB@A\SB}}

	      Arrange substitutions,

	      $\mathcal{D}'_1$ = \infer[\TAbs]
	      {\GGV (\lambda x:\sigma.M)\SB:(\Pi x:\sigma.\tau)\SB@A\SB}
	      {\ID{\GGV \sigma\SB::*@A\SB} \andalso \ID{\GG, x:\sigma\SB@A\SB \V M\SB:\tau\SB@A\SB}}

	\item \TApp

	      We have a derivation trees from the premise.

	      $\mathcal{D}_1$ = \infer[\TApp]
	      {\G, z:\xi@B, \D \V M\ L:\tau[y\mapsto L]@A}
	      {\infer[]{\G, z:\xi@B, \D \V M:\Pi(y:\rho).\tau@A \andalso \G, z:\xi@B, \D \V L:\rho@A}{\vdots}}

	      We get 2 trees from the induction hypothesis and $\mathcal{D}_1$.

	      $\mathcal{D}_2$ = \infer[]
	      {\GGV M\SB: (\Pi(y:\rho).\tau)\SB@A}
	      {\vdots}

	      and

	      $\mathcal{D}_3$ = \infer[]
	      {\GGV L\SB: \rho\SB@A}
	      {\vdots}

	      \red{distribute the substitution} in $\mathcal{D}_2$

	      $\mathcal{D'}_2$ = \infer[]
	      {\GGV M\SB: (\Pi(y:\rho\SB).\tau\SB)@A}
	      {\vdots}

	      From $\mathcal{D'}_2$ and $\mathcal{D}_3$

	      $\mathcal{D}_4$ = \infer[\TApp]
	      {\GGV (M\SB\ L\SB): \tau\SB[y \mapsto L]@A}
	      {\mathcal{D'}_2 \andalso \mathcal{D}_3}

	      \red{We can transform the conclusion of $\mathcal{D}_4$ into} \\
	      $\GGV (M\ L)\SB: \tau[y \mapsto L]\SB@A$

	\item \TConv

	      We have 2 derivation trees from the premise and the induction hypothesis.

	      $\mathcal{D}_1$ = \infer[]
	      {\GGV t\SB:T\SB@A}
	      {\vdots}

	      $\mathcal{D}_2$ = \infer[]
	      {\GGV T\SB \E T'\SB@A}
	      {\vdots}

	      And then \\
	      \infer[\TConv]
	      {\GGV t\SB:T'\SB@A}
	      {\mathcal{D}_1 \andalso \mathcal{D}_2}

	\item \TTB

	      \begin{itemize}

		      \item $\alpha \neq \beta$
		            From the induction hypothesis and \TTB, we get

		            $\mathcal{D}_1$ = \infer[\TTB]
		            {\GGV \TB_\alpha M\SB:\tau\SB@A\SB}
		            {\infer[]{\GGV M\SB:\tau\SB@A\alpha\SB}{\vdots}}

		      \item $\alpha = \beta$

		            The conclusion is identical with the induction hypothesis.

	      \end{itemize}

	\item \TTBL

	      \begin{itemize}

		      \item $\alpha \neq \beta$
		            From the induction hypothesis and \TTBL, we get

		            $\mathcal{D}_3$ = \infer[\TTBL]
		            {\GGV \TBL_\alpha M\SB:\tau\SB@A\SB}
		            {
			            \infer[]{\GGV M\SB: \TW_\alpha \tau\SB@A\alpha\SB}{\vdots}
		            }

		      \item $\alpha = \beta$

		            The conclusion is identical with the induction hypothesis.

	      \end{itemize}

	\item \TGen

	      From the induction hypothesis, we get $\mathcal{D}_1$.

	      $\mathcal{D}_1$ = \infer[]
	      {\GGV M\SB:\tau\SB@A\SB}
	      {\vdots}

	      And we can prove easily $\alpha \notin \FTV(\GG) \cup \FTV(A)$.

	      Use \TGen,

	      $\mathcal{D}_2$ = \infer[\TGen]
	      {\GGV (\Lambda\alpha.M)\SB:(\forall\alpha.\tau)\SB@A\SB}
	      {\mathcal{D}_1 \andalso \alpha \notin \FTV(\GG) \cup \FTV(A)}

	\item \TIns

	      We can assume $\alpha \neq \beta$ because $\alpha$ appears only in $M:\forall\alpha.\tau$ and we can rename $\alpha$ to an arbitary name.

	      From the induction hypothesis and \TIns, we get $\mathcal{D}_1$.

	      $\mathcal{D}_1$ = \infer[\TIns]
	      {\GGV (M\ \varepsilon)\SB:\tau\SB@A\SB}
	      {\ID{\GGV M\SB:(\forall\alpha.\tau)\SB@A\SB}}

	\item \TCsp

	      \begin{itemize}

		      \item $\alpha \neq \beta$

		            From the induction hypothesis and \TCsp, we get $\mathcal{D}_1$.

		            $\mathcal{D}_2$ = \infer[\TCsp]
		            {\GGV (\%_\alpha M)\SB:\tau\SB@A\alpha}
		            {\ID{\GGV M\SB:\tau\SB@A}}

		      \item $\alpha = \beta$

		            The conclusion is identical with the induction hypothesis.

	      \end{itemize}

	\item \QKAbs

	      From the induction hypothesis and \QKAbs, we get $\mathcal{D}_1$.

	      $\mathcal{D}_1$ = \infer[\QKAbs]
	      {\GGV (\Pi x:\tau.K)\SB \E (\Pi x:\sigma.J)\SB@A\SB}
	      {\infer[]{\GGV \tau\SB \E \sigma\SB :: *@A\SB}{\vdots} \andalso
		      \infer[]{\GG,x:\tau@A \V K\SB\E J\SB@A\SB}{\vdots} }

	\item \QKCsp

	      \begin{itemize}

		      \item $\alpha \neq \beta$

		            From the induction hypothesis and \QKCsp, we get $\mathcal{D}_1$.

		            $\mathcal{D}_1$ = \infer[\QKCsp]
		            {\GGV K\SB \E J\SB @A\alpha\SB}
		            {\infer[]{\GGV K\SB \E J\SB @A\SB}{\vdots}}


		      \item $\alpha = \beta$

		            The conclusion is identical with the induction hypothesis.

	      \end{itemize}

	\item \QKRefl

	      From the induction hypothesis and \QKRefl, we get $\mathcal{D}_1$.

	      $\mathcal{D}_1$ = \infer[\QKRefl]
	      {\GGV K\SB \E K\SB @A\SB}
	      {\infer[]{\GGV K\SB \iskind @A\SB}{\vdots}}

	\item \QKSym

	      From the induction hypothesis and \QKSym, we get $\mathcal{D}_1$.

	      $\mathcal{D}_1$ = \infer[\QKSym]
	      {\GGV J\SB\E K\SB@A\SB}
	      {\GGV K\SB\E J\SB@A\SB}

	\item \QKTrans

	      From the induction hypothesis and \QKTrans, we get $\mathcal{D}_1$.

	      $\mathcal{D}_1$ = \infer[\QKTrans]
	      {\GGV K\SB\E I\SB@A\SB}
	      {\GGV K\SB\E J\SB@A\SB \andalso \GGV J\SB\E I\SB@A\SB}

	\item \QTAbs

	      From the induction hypothesis and, we get $\mathcal{D}_1$.

	      $\mathcal{D}_1$ = \infer[\QTAbs]
	      {\GGV \Pi x:\tau\SB.\rho\SB \E \Pi x:\sigma\SB.\pi\SB@A\SB}
	      {\ID{\GGV \tau\SB \E \sigma\SB :: *@A\SB} \andalso \ID{\GG, x:\tau\SB@A\SB \V \rho\SB \E \pi\SB @A\SB}}

	      Arrange substitutions,

	      $\mathcal{D}_1$ = \infer[\QTAbs]
	      {\GGV (\Pi x:\tau.\rho)\SB \E (\Pi x:\sigma.\pi)\SB@A\SB}
	      {\ID{\GGV \tau\SB \E \sigma\SB :: *@A\SB} \andalso \ID{\GG, x:\tau\SB@A\SB \V \rho\SB \E \pi\SB @A\SB}}

	\item \QTApp

	      From the induction hypothesis and \QTApp, we get $\mathcal{D}_1$.

	      $\mathcal{D}_1$ = \infer[\QTApp]
	      {\GGV\pi\SB\ M\SB \E \sigma\SB\ N\SB@A\SB}
	      {\ID{\GGV\tau\SB\E\sigma\SB :: (\Pi x:\rho\SB.K\SB)@A\SB} \andalso \ID{\GGV M\SB\E N\SB:\rho\SB@A\SB}}

	      Arrange substitutions,

	      $\mathcal{D}_1$ = \infer[\QTApp]
	      {\GGV(\pi\ M)\SB \E (\sigma\ N)\SB@A\SB}
	      {\ID{\GGV\tau\SB\E\sigma\SB :: (\Pi x:\rho\SB.K\SB)@A\SB} \andalso \ID{\GGV M\SB\E N\SB:\rho\SB@A\SB}}

	\item \QTTW

	      \begin{itemize}

		      \item $\alpha \neq \beta$

		            From the induction hypothesis and \QTTW, we get $\mathcal{D}_1$.

		            $\mathcal{D}_1$ = \infer[\QTTW]
		            {\GGV(\TW_\alpha \tau)\SB\E(\TW_\alpha\sigma)\SB@A\SB}
		            {\ID{\GGV\tau\SB\E\sigma\SB@A\alpha\SB}}

		      \item $\alpha = \beta$

		            The conclusion is identical with the induction hypothesis.

	      \end{itemize}

	\item \QTGen

	      We can assume $\alpha \neq \beta$ because $\alpha$ appears only in $M:\forall\alpha.\tau$ and we can rename $\alpha$ to an arbitary name.

	      We can prove easily $\alpha \notin \FTV(\GG) \cup \FTV(A)$.
	      From the induction hypothesis and \QTGen, we get $\mathcal{D}_1$.

	      $\mathcal{D}_1$ = \infer[\QTGen]
	      {\GGV (\forall\alpha.\tau)\SB \E (\forall\alpha.\sigma)\SB@A\SB}
	      {\ID{\GGV \tau\SB \E \sigma\SB@A\SB} \andalso \alpha \notin \FTV(\GG) \cup \FTV(A)}

	\item \QTCsp

	      \begin{itemize}

		      \item $\alpha \neq \beta$

		            From the induction hypothesis and \QTCsp, we get $\mathcal{D}_1$.

		            $\mathcal{D}_1$ = \infer[\QTCsp]
		            {\GGV\tau\SB \E \sigma\SB@A\alpha\SB}
		            {\ID{\GGV\tau\SB \E \sigma\SB@A\SB}}

		      \item $\alpha = \beta$

		            The conclusion is identical with the induction hypothesis.

	      \end{itemize}

	\item \QTRefl

	      From the induction hypothesis and \QTRefl, we get $\mathcal{D}_1$.

	      $\mathcal{D}_1$ = \infer[\QTRefl]
	      {\GGV\tau\SB\E\tau\SB@A\SB}
	      {\ID{\GGV\tau\SB::K\SB@A\SB}}

	\item \QTSym

	      From the induction hypothesis and \QTSym, we get $\mathcal{D}_1$.

	      $\mathcal{D}_1$ = \infer[\QTSym]
	      {\GGV\sigma\SB\E\tau\SB@A\SB}
	      {\ID{\GGV\tau\SB\E\sigma\SB@A\SB}}

	\item \QTTrans

	      From the induction hypothesis and \QTTrans, we get $\mathcal{D}_1$.

	      $\mathcal{D}_1$ = \infer[\QTTrans]
	      {\GGV \tau\SB\E\rho\SB@A\SB}
	      {\ID{\GGV\tau\SB\E\sigma\SB@A\SB} \andalso \ID{\GGV\sigma\SB\E\rho\SB@A\SB}}

	\item \QAbs

	      From the induction hypothesis and \QAbs, we get $\mathcal{D}_1$.

	      $\mathcal{D}_1$ = \infer[\QAbs]
	      {\GGV \Pi x:\tau\SB.\rho\SB \E \Pi x:\sigma\SB.\pi\SB@A\SB}
	      {\ID{\GGV\tau\SB \E \sigma\SB :: * @A\SB} \andalso \ID{\GG,x:\tau\SB@A\SB\V\rho\SB \E \pi\SB@A\SB}}

	      Arrange substitutions,

	      $\mathcal{D}'_1$ = \infer[\QAbs]
	      {\GGV (\Pi x:\tau.\rho)\SB \E (\Pi x:\sigma.\pi)\SB@A\SB}
	      {\ID{\GGV\tau\SB \E \sigma\SB :: * @A\SB} \andalso \ID{\GG,x:\tau\SB@A\SB\V\rho\SB \E \pi\SB@A\SB}}

	\item \QApp

	      From the induction hypothesis

	      $\mathcal{D}_1$ = \ID{\GGV M\SB \E L\SB :: (\Pi x:\sigma.\tau)\SB@A\SB}

	      Arrange substitutions,

	      $\mathcal{D}'_1$ = \ID{\GGV M\SB \E L\SB :: (\Pi x:\sigma\SB.\tau\SB)@A\SB}

	      Using \QApp\ to $\mathcal{D}'_1$ and the induction hypothesis, we get $\mathcal{D}_2$.

	      $\mathcal{D}_2$ = \infer[\QApp]
	      {\GGV M\SB\ N\SB \E L\SB\ O\SB @A\SB}
	      {\mathcal{D}'_1 \andalso \ID{\GGV N\SB \E O\SB : \sigma\SB @A\SB}}

	      Arrange substitutions,

	      $\mathcal{D}'_2$ = \infer[\QApp]
	      {\GGV (M\ N)\SB \E (L\ O)\SB @A\SB}
	      {\mathcal{D}'_1 \andalso \ID{\GGV N\SB \E O\SB : \sigma\SB @A\SB}}

	\item \QTB

	      \begin{itemize}

		      \item $\alpha \neq \beta$

		            From the induction hypothesis and \QTB, we get $\mathcal{D}_1$.

		            $\mathcal{D}_1$ = \infer[\QTB]
		            {\GGV\TB_\alpha M\SB \E \TB_\alpha N\SB @A\SB}
		            {\GGV M\SB \E N\SB @A\alpha\SB}

		      \item $\alpha = \beta$

		            The conclusion is identical with the induction hypothesis.

	      \end{itemize}

	\item \QTBL

	      \begin{itemize}

		      \item $\alpha \neq \beta$

		            From the induction hypothesis and \QTBL, we get $\mathcal{D}_1$.

		            $\mathcal{D}_1$ = \infer[\QTBL]
		            {\GGV\TBL_\alpha M\SB \E \TBL_\alpha N\SB@A\alpha\SB}
		            {\GGV M\SB \E N\SB : \TW_\alpha \tau@A\SB}

		      \item $\alpha = \beta$

		            The conclusion is identical with the induction hypothesis.

	      \end{itemize}

	\item \QGen

	      \begin{itemize}

		      \item $\alpha \neq \beta$

		            We can prove easily $\alpha \notin \FTV(\GG) \cup \FTV(A)$.
		            From the induction hypothesis and \QGen, we get $\mathcal{D}_1$.

		            $\mathcal{D}_1$ = \infer[\QGen]
		            {\GGV\Lambda\alpha.M\SB \E \Lambda\alpha.N\SB@A\SB}
		            {\ID{\GGV M\SB \E N\SB @A\SB} \andalso \alpha \notin \FTV(\GG) \cup \FTV(A)}

		      \item $\alpha = \beta$

		            The conclusion is identical with the induction hypothesis.

	      \end{itemize}

	\item \QIns

	      We can assume $\alpha \neq \beta$ because $\alpha$ appears only in $M:\forall\alpha.\tau$ and we can rename $\alpha$ to an arbitary name.

	      From the induction hypothesis and \QIns, we get $\mathcal{D}_1$.

	      $\mathcal{D}_1$ = \infer[\QIns]
	      {\GGV M\SB\ \varepsilon \E N\SB \ \varepsilon @A\SB }
	      {\ID{\GGV M\SB \E N\SB : (\forall\alpha.\tau\SB) @A\SB}}

	\item \QCsp

	      \begin{itemize}

		      \item $\alpha \neq \beta$

		            From the induction hypothesis and \QCsp, we get $\mathcal{D}_1$.

		            $\mathcal{D}_1$ = \infer[\QCsp]
		            {\GGV \%_\alpha M\SB \E \%_\alpha N\SB @A\SB}
		            {\ID{\GGV M\SB \E N\SB @A\alpha\SB}}

		      \item $\alpha = \beta$

		            The conclusion is identical with the induction hypothesis.

	      \end{itemize}

	\item \QRefl

	      From the induction hypothesis and \QRefl, we get $\mathcal{D}_1$.

	      $\mathcal{D}_1$ = \infer[\QRefl]
	      {\GGV M\SB \E M\SB @A\SB}
	      {\ID{\GGV M\SB : \tau\SB @A\SB}}

	\item \QSym

	      From the induction hypothesis and \QSym, we get $\mathcal{D}_1$.

	      $\mathcal{D}_1$ = \infer[\QSym]
	      {\GGV N\SB \E M\SB @A\SB}
	      {\ID{\GGV M\SB \E N\SB @A\SB}}

	\item \QTrans

	      From the induction hypothesis and \QTrans, we get $\mathcal{D}_1$.

	      $\mathcal{D}_1$ = \infer[\QTrans]
	      {\GGV M\SB \E L\SB @A\SB}
	      {\ID{\GGV M\SB \E N\SB @A\SB } \andalso \ID{\GGV N\SB \E L\SB @A\SB}}

	\item \QBeta

	      From the induction hypothesis and \QBeta, we get $\mathcal{D}_1$.

	      $\mathcal{D}_1$ = \infer[\QBeta]
	      {\GGV (\lambda x:\sigma\SB:M\SB)\ N\SB \E (M\SB)[x \mapsto N\SB]@A\SB}
	      {\ID{\GG, x: \sigma\SB@A\SB \V M\SB:\tau\SB@A\SB} \andalso \ID{\GGV N\SB:\sigma\SB @A\SB }}

	      Arrange substitutions,

	      $\mathcal{D}'_1$ = \infer[\QBeta]
	      {\GGV ((\lambda x:\sigma:M)\ N)\SB \E (M[x \mapsto N])\SB@A\SB}
	      {\ID{\GG, x: \sigma\SB@A\SB \V M\SB:\tau\SB@A\SB} \andalso \ID{\GGV N\SB:\sigma\SB @A\SB }}

	\item \QEta

	      From the induction hypothesis and \QEta, we get $\mathcal{D}_1$.

	      $\mathcal{D}_1$ = \infer[\QEta]
	      {\GGV (\lambda x:\sigma\SB.M\SB\ x) \E M\SB@A\SB}
	      {\ID{\GGV M\SB : (\Pi x:\sigma\SB.\tau\SB)@A\SB} \andalso x \notin \FV(M\SB)}

	      Arrange substitutions,

	      $\mathcal{D}'_2$ = \infer[\QEta]
	      {\GGV (\lambda x:\sigma.M\ x)\SB \E M\SB@A\SB}
	      {\ID{\GGV M\SB : (\Pi x:\sigma\SB.\tau\SB)\SB@A\SB} \andalso x \notin \FV(M\SB)}

	\item \QTBLTB

	      From the induction hypothesis and \QTBLTB, we get $\mathcal{D}_1$.

	      $\mathcal{D}_1$ = \infer[\QTBLTB]
	      {\GGV \TBL_\alpha \TB_\alpha M\SB \E N\SB@A\SB}
	      {\ID{\GGV M\SB \E N\SB @A\SB}}

	\item \QLambda

	      From the induction hypothesis and \QLambda, we get $\mathcal{D}_1$.

	      $\mathcal{D}_1$ = \infer[\QLambda]
	      {\GGV (\Lambda\alpha.M\SB)\ \varepsilon \E M\SB[\alpha \mapsto \varepsilon]}
	      {\ID{\GGV (\Lambda\alpha.M\SB) : \forall\alpha.\tau\SB @A\SB}}

	      Arrange substitutions,

	      $\mathcal{D}'_1$ = \infer[\QLambda]
	      {\GGV ((\Lambda\alpha.M)\ \varepsilon)\SB \E M[\alpha \mapsto \varepsilon]\SB}
	      {\ID{\GGV (\Lambda\alpha.M\SB) : \forall\alpha.\tau\SB @A\SB}}


	      \fi

	\item \QPercent

	      From the induction hypothesis and \QPercent, we get $\mathcal{D}_1$.

	      $\mathcal{D}_1$ = \infer[\QPercent]
	      {\GGV \%_\alpha M\SB \E M\SB @ A\alpha}
	      {\ID{\GGV M\SB : \tau\SB @A\alpha\SB} \andalso \ID{\GGV M\SB : \sigma\SB @A\SB} }

\end{itemize}

\begin{lemma}[Agreement]
	\begin{flalign*}
		\text{If\ } \G\V \tau::K@A &\text{\ then\ } \G\V K\iskind@A. &\\
		\text{If\ } \G\V M:\tau@A &\text{\ then\ } \G\V \tau::*@A.&\\
		\text{If\ } \G\V K\E J@A &\text{\ then\ } \G\V K\iskind@A \text{\ and\ } \G\V J\iskind@A.&\\
		\text{If\ } \G\V \tau\E \sigma :: K@A &\text{\ then\ } \G\V \tau::K@A \text{\ and\ } \G\V \sigma::K@A.&\\
		\text{If\ } \G\V M\E N : \tau@A &\text{\ then\ } \G\V M:\tau@A \text{\ and\ } \G\V N:\tau@A.&\\
	\end{flalign*}
\end{lemma}

We can prove using induction on the derivation tree.

Basecases are \KVar, \TVar, \QKRefl, \QTRefl, \QRefl.
We can prove for these cases directly from the definition of rules.

Others are proved by using induction hypothesis and a rule. We show some cases as examples.
\begin{itemize}
	\item \KCsp

	      We have a derivation tree from the premise.

	      $\mathcal{D}_1$ = \infer[\KCsp]
	      {\G\V \tau::K@A\alpha}
	      {\infer[]{\G\V \tau::K @A}{\vdots}}

	      From the hypothesis of the reduction, we can get

	      $\mathcal{D}_2$ = \infer[]{\G\V K\iskind @A}{\vdots}

	      Use \WCsp\ to $\mathcal{D}_2$, we get the conclusion.

	      $\mathcal{D}_3$ = \infer[\WCsp]
	      {\G\V K \iskind @A\alpha}
	      {\mathcal{D}_2}

	\item \KTW

	      We have a derivation tree from the premise.

	      $\MD{1}$ = \infer[\KTW]
	      {\G\V\TW_\alpha \tau::*@A}
	      {\vdots}

	      From \WStar, $\G\V * \iskind @A$.

\end{itemize}

\begin{lemma}[Inversion Lemma for $\Pi$ type]
	If $\G \V (\lambda x:\sigma.M) : (\Pi x:\sigma'.\tau)@A$ then
	\begin{enumerate}
		\item $\G \V \sigma \E \sigma'@A$
		\item $\G ,x:\sigma@A\V M:\tau@A$
	\end{enumerate}
	\item If $\G \V \rho \E (\Pi x:\sigma.\tau) : K @A$ then $\exists \sigma', \tau', K, J$ such that
	\begin{enumerate}
		\item $\rho = \Pi x:\sigma'.\tau'$
		\item $\G \V \sigma \E \sigma' : K @A$
		\item $\G, x:\sigma@A\V \tau \E \tau' : J @A$
	\end{enumerate}
	\item If $\G \V (\Pi x:\sigma.\tau) \E \rho : K @A$ then $\exists \sigma', \tau', K, J$ such that
	\begin{enumerate}
		\item $\rho = \Pi x:\sigma'.\tau'$
		\item $\G \V \sigma \E \sigma' : K @A$
		\item $\G, x:\sigma@A\V \tau \E \tau' : J @A$
	\end{enumerate}
\end{lemma}

Proved by induction on the derivation tree.

\begin{itemize}
	\item \TAbs

	      We have a derivation tree $\mathcal{D}_1$.

	      $\mathcal{D}_1$ = \infer[\TAbs]
	      {\G \V ((\lambda x:\sigma).M) : (\Pi (x:\sigma).\tau) @ A}
	      {\ID{\G \V \sigma :: * @ A} \andalso \ID{\G, x:\sigma@A \V M:\tau@A}}

	      Following conclusion are obvious.
	      \begin{enumerate}
		      \item $\G \V \sigma \E \sigma'@A$
		      \item $\G ,x:\sigma@A\V M:\tau@A$
	      \end{enumerate}

	\item \TConv

	      We have a derivation tree $\mathcal{D}_1$.

	      $\mathcal{D}_1$ = \infer[\TConv]
	      {\G \V M : (\Pi x:\sigma.\tau)@A}
	      {\ID{\G \V M : \rho@A } \andalso \ID{\G \V \rho \E (\Pi x:\sigma.\tau)@A}}

	      Now, we can use the induction hypothesis to $\G \V M : \rho@A$ and $\G \V \rho \E (\Pi x:\sigma.\tau)@A$.

	\item \QTRefl

	      There two cases for the conclusion.
	      \begin{itemize}
		      \item $\G \V \rho \E (\Pi x:\sigma.\tau) : K @A$

		            In this case, we can use the induction hypothesis of statement 3.
		      \item $\G \V (\Pi x:\sigma.\tau) \E \rho : K @A$

		            In this case, we can use the induction hypothesis of statement 2.
	      \end{itemize}

	\item Otherwise

	      It is obvious.

\end{itemize}

\begin{theorem}[Preservation on $\beta$ reduction]
  \AI{I'm not sure ``on'' is the correct preposition.  Maybe ``for''?'}
  If $\G\V M:\tau@A$ and $M \longrightarrow_{\beta} M'$, then $\G\V M':\tau@A$\\
  \AI{Put a period.  You don't have to use a linebreak.}
\end{theorem}

Use induction on type derivation tree.

\begin{itemize}
	\newcommand{\LB}{\longrightarrow_{\beta}}

	\item \TApp

	      We can assume the reduction has following shape.

	      \begin{itemize}
		      \item $(\lambda x:\sigma.N)\ L \LB N[x\mapsto L]$

		            Because the last rule is \TApp, we have $\mathcal{D}_1$.

		            $\mathcal{D}_1$ = \infer[\TApp]
		            {\G \V (\lambda x:\sigma.N)\ L : \tau[x\mapsto N]@A}
		            {\ID{\G \V (\lambda x:\sigma.N) : (\Pi x:\sigma'.\tau')@A} \andalso \ID{\G \V L:\sigma' @A}} \\

		            Use "Inversion Lemma for $\Pi$ type" to $\G \V (\lambda x:\sigma.N) : (\Pi x:\sigma'.\tau')@A$,\\
		            get $\G, x:\sigma \V N:\tau$ and $\G \V \sigma \E \sigma'$ and $\G ,x:\sigma \V \tau \E \tau'@A$.

		            Use \TConv, $\G \V L:\sigma @A$.

		            Use "Substituition Lemma" to $\G, x:\sigma \V N:\tau$ and $\G \V L:\sigma @A$, get $\G \V N[x\mapsto L]:\tau[x\mapsto L]$.
		            %Get $\G \V \tau

		      \item $M\ N \LB M'\ N$

		            From the induction hypothesis and \TApp, the type is preserved for the reduction.
		      \item $M\ N \LB M\ N'$

		            From the induction hypothesis and \TApp, the type is preserved for the reduction.
	      \end{itemize}

	      \iffullversion

	\item \TVar

	      In this case, there is no reduction from $x$.

	\item \TAbs

	      We can assume the reduction has following shape.

	      $\lambda x:\sigma.M \LB \lambda x:\sigma.M'$

	      From the induction hypothesis and \TAbs, the type is preserved for the reduction.


	\item \TConv

	      We can assume the reduction has following shape.

	      $M \LB M'$

	      From the induction hypothesis and \TConv, the type is preserved for the reduction.

	\item \TTB

	      We can assume the reduction has following shape.

	      $\TB M \LB \TB M'$

	      From the induction hypothesis and \TTB, the type is preserved for the reduction.

	\item \TTBL

	      We can assume the reduction has following shape.

	      $\TBL M \LB \TBL M'$

	      From the induction hypothesis and \TTBL, the type is preserved for the reduction.

	\item \TGen

	      We can assume the reduction has following shape.

	      $\Lambda\alpha. M \LB \Lambda\alpha. M'$

	      From the induction hypothesis and \TGen, the type is preserved for the reduction.

	\item \TIns

	      We can assume the reduction has following shape.

	      $M\ \varepsilon \LB M'\ \varepsilon$

	      From the induction hypothesis and \TIns, the type is preserved for the reduction.

	\item \TCsp

	      We can assume the reduction has following shape.

	      $\%_\alpha M \LB \%_\alpha M'$

	      From the induction hypothesis and \TCsp, the type is preserved for the reduction.

	      \fi

\end{itemize}

\begin{lemma}[Inversion Lemma for $\TW$ type]
	\begin{item}
	      \item If $\G \V \TB_\alpha M : \TW_\alpha \tau@A$ then $\G \V M : \tau @A$.
	      \item If $\G \V \rho \E  \TW_\alpha \tau : K @A$ then $\exists \tau', K, J$ such that
	      \begin{enumerate}
		      \item $\rho = \TW_\alpha \tau'$
		      \item $\G @A\V \tau \E \tau' : K @A$
	      \end{enumerate}
	      \item If $\G \V \TW_\alpha \tau \E \rho : K @A$ then $\exists \tau', K, J$ such that
	      \begin{enumerate}
		      \item $\rho = \TW_\alpha \tau'$
		      \item $\G @A\V \tau \E \tau' : K @A$
	      \end{enumerate}
	\end{item}
\end{lemma}

\begin{itemize}
	\item \TTB

	      The derivation tree looks like $\mathcal{D}_1$.

	      $\mathcal{D}_1$ = \infer[\TTB]
	      {\G \V \TB_\alpha M : \TW_\alpha \tau@A}
	      {\ID{\G \V M : \tau@A\alpha}}

	\item Otherwise

	      It is obvious.

\end{itemize}

\begin{theorem}[Preservation for term on $\TBL\TB$ reduction]
	If $\G\V M:\tau@A$ and $M\longrightarrow_\blacklozenge N$, then $\G\V N:\tau@A$\\
\end{theorem}

Use induction on type derivation tree.

\begin{itemize}
\newcommand{\R}{\longrightarrow_{\blacklozenge}}

\item \TTBL

	      There are 2 cases for $\R$.

	      \begin{itemize}
		      \item $\TBL\TB M \R M$

		            Because the last rule is \TTBL, we have $\mathcal{D}_1$.

		            $\mathcal{D}_1$ = \infer[\TTBL]
		            {\G \V \TBL\TB M : \tau@A}
		            {\ID{\G \V \TB M : \TW_\alpha \tau @A}}

		            Use "Inversion Lemma for $\TW$ type" to $\G \V \TB M : \TW_\alpha \tau @A$,  get $\G \V M : \tau @A$

		      \item $\TBL M \R \TBL M'$

		            We can use the induction hypothesis directly.
	      \end{itemize}

	\item Otherwise

	      We can use the induction hypothesis directly.
\end{itemize}

\begin{lemma}[Inversion for $\Lambda$ type]
	\begin{item}
	      \item If $\G \V \Lambda\alpha.M : \forall\alpha.\tau@A$ then $\G \V M : \tau @A$ and $\alpha \notin \FTV(\G) \cup \FV(A)$.
	      \item If $\G \V \rho \E \forall\alpha.\tau : K @A$ then $\exists \tau', K$ such that
	      \begin{enumerate}
		      \item $\rho = \forall\alpha.\tau'$
		      \item $\G \V \tau \E \tau' : K @A$
	      \end{enumerate}
	      \item If $\G \V \forall\alpha.\tau \E \rho : K @A$ then $\exists \tau', K$ such that
	      \begin{enumerate}
		      \item $\rho = \forall\alpha.\tau'$
		      \item $\G \V \tau \E \tau' : K @A$
	      \end{enumerate}
	\end{item}
\end{lemma}

\begin{itemize}
	\newcommand{\MC}[1]{\mathcal{#1}}
	\item \TGen

	      The derivation tree is $\MC{D}_1$.

	      $\MC{D}_1$ = \infer[\TGen]
	      {\G \V \Lambda\alpha.M : \forall\alpha.\tau@A}
	      {\ID{\G\V M:\tau@A} \andalso \alpha \notin \FTV(\G)\cup\FTV(A)}

	\item Otherwise

	      It is obvious.
\end{itemize}

\begin{theorem}[Preservation for term on $\Lambda$ reduction]
	If $\G\V M:\tau@A$ and $M \longrightarrow_{\Lambda} N$, then $\G\V N:\tau@A$.
\end{theorem}

Use induction on type derivation tree.

\begin{itemize}
	\newcommand{\R}{\longrightarrow_{\Lambda}}
	\item \TIns

	      There are two cases for the reduction.
	      \begin{itemize}
		      \item $\Lambda\alpha.M\ B \R M[\alpha \mapsto B]$

		            The type derivation tree looks like $\mathcal{D}_1$.

		            $\mathcal{D}_1$ = \infer[\TIns]
		            {\G \V \Lambda\alpha.M\ B : \tau[\alpha \mapsto B] @ A}
		            {\ID{\G \V \Lambda\alpha.M\ : \forall\alpha.\tau @ A}}

		            Use "Inversion Lemma for $\Lambda$ type" to $\G \V \Lambda\alpha.M\ : \forall\alpha.\tau @ A$,
		            get $\G \V M : \tau @ A$ and $\alpha \notin \FTV(\G) \cup \FV(A)$.

		            Use "Stage Substituition Lemma" to $\G \V M : \tau @ A$,
		            get $\G[\alpha \mapsto B] \V M[\alpha \mapsto B] : \tau[\alpha \mapsto B] @ A[\alpha \mapsto B]$.

		            Because $\alpha \notin \FTV(\G) \cup \FV(A)$, $\G[\alpha \mapsto B] = \G$ and $A[\alpha \mapsto B] = A$.

		            So, we can rewrite $\G[\alpha \mapsto B] \V M[\alpha \mapsto B] : \tau[\alpha \mapsto B] @ A[\alpha \mapsto B]$ to
		            $\G \V M[\alpha \mapsto B] : \tau[\alpha \mapsto B] @ A$.
		      \item $M\ B \R M'\ B$

		            We can use induction hypothesis directly.
	      \end{itemize}

	\item Otherwise

	      we can use induction hypothesis directly.
\end{itemize}

\begin{dfn}[$\natural$ translation]
	$\natural$ translation is a translation from $\lambda^\text{MD}$ to $\lambda^\to$.
	\begin{itemize}
		\item Term
		      \begin{flalign*}
			      \natural(x) &= x & \\
			      \natural(\lambda x:\tau.M) &= \lambda x:\natural(\tau).\natural(M) & \\
			      \natural(M\ N) &= \natural(M)\ \natural(N)& \\
			      \natural(\TB_\alpha M) &= \natural(M) & \\
			      \natural(\TBL_\alpha M) &= \natural(M)& \\
			      \natural(\Lambda\alpha.M) &= \natural(M)& \\
			      \natural(M\ B) &= \natural(M) &
		      \end{flalign*}
		\item Type
		      \begin{flalign*}
			      \natural(X) &= X & \\
			      \natural(\Pi x:\tau.\sigma) &= \natural(\tau) \to \natural(\sigma) & \\
			      \natural(\tau\ x) &= \natural(\tau) & \\
			      \natural(\TW_\alpha \tau) &= \natural(\tau) & \\
			      \natural(\forall \alpha.\tau) &= \natural(\tau) &
		      \end{flalign*}
		\item Kind
		      \begin{flalign*}
			      \natural(K) &= * &
		      \end{flalign*}
		\item Context
		      \begin{flalign*}
			      \natural(\phi) &= \phi & \\
			      \natural(\G, x:T@A) &= \natural(\G), \natural(x):\natural(\tau) & \\
			      \natural(\G, X:K@A) &= \natural(\G) &
		      \end{flalign*}
	\end{itemize}
\end{dfn}

\begin{lemma}[Preservation of equality in $\natural$]
	If $\G \V \tau \E \sigma @ A$ then $\natural(\tau) = \natural(\sigma)$.
\end{lemma}

Prove by induction on the derivation tree.

\begin{lemma}[Preservation of typing in $\natural$]
	If $\G \V M:\tau@A$ in $\lambda^{\text{MD}}$ then $\natural(\G) \V \natural(M): \natural(\tau)$ in $\lambda^\to$.
\end{lemma}

Prove by induction on the type derivation tree.

\begin{itemize}
	\item \TApp

	      We have a derivation tree $\MD{1}$.

	      $\MD{1}$ = \infer[\TApp]
	      {\G \V M N : \tau[x \mapsto N] @A}
	      {\ID{\G \V M : (\Pi(x:\sigma).\tau) @ A} \andalso \ID{\G \V N :\sigma @A}}

	      From the induction hypothesis, we have $\natural(\G) \V \natural(M) : \natural(\sigma) \to \natural(\tau)$ and $\natural(\G) \V \natural(N) : \natural(\sigma)$.
	      Use the Application rule in $\lambda^\to$, we get $\natural(\G) \V \natural(M)\ \natural(N) : \natural(\tau)$.
	      Because $\natural(M)\ \natural(N) = \natural(M\ N)$ from the definition of $\natural$, $\natural(\G) \V \natural(M\ N) : \natural(\tau)$ in $\lambda^\to$.
	\item \TConv

	      We have a derivation tree $\MD{1}$.

	      $\MD{1}$ = \infer[\TConv]
	      {\G\V M:\sigma@A}
	      {\ID{\G\V M:\tau@A} \andalso \ID{\G\V \tau\E\sigma@A}}

	      Use "Preservation of typing in $\natural$" to $\G\V \tau\E\sigma@A$, we get $\natural(\tau) = \natural(\sigma)$.
	      On the otherhand, $\natural(\G) \V \natural(M):\natural(\tau)$ from the induction hypothesis.
	      Then $\natural(\G) \V \natural(M):\natural(\sigma)$.

\end{itemize}

\begin{lemma}[Inversion Lemma for Application]
	\begin{item}
	      \item If $\G \V (\lambda x:\sigma.M)\ N: \tau@A$ then $\exists x, \rho$ such that
	      \begin{enumerate}
		      \item $\G, x:\sigma \V M : \rho @A$
		      \item $\G \V N:\sigma @ A$
	      \end{enumerate}
	\end{item}
\end{lemma}

Prove by induction on the derivation tree.

\begin{itemize}
	\item \TApp

	      Obvious.

	\item \TConv

	      The derivation is $\mathcal{D}_1$.

	      $\mathcal{D}_1$ = \infer[\TConv]
	      {\G \V (\lambda x:\sigma.M)\ N: \tau@A}
	      {\ID{\G \V (\lambda x:\sigma.M)\ N: \rho@A} \andalso \ID{\G \V \rho \E \tau : K @A}}

	      Use the induction hypothesis to $\G \V (\lambda x:\sigma.M)\ N: \rho@A$, get $\G, x:\sigma \V M : \pi @A$ and $\G \V N:\sigma @ A$.

	      Then we can fix $x, \pi$ as $x, \rho$ in the statement correspondingly.

\end{itemize}

\begin{lemma}[Preservation of substitution in $\natural$]
	If $\G, x:\sigma \V M:\tau@A$ and $\G \V N:\sigma@A$ in $\lambda^{\text{MD}}$
	then $\natural(M[x \mapsto N])$ = $\natural(M)[x\mapsto\natural(N)]$
\end{lemma}

Prove by induction on the type derivation tree of $\G, x:\sigma \V M:\tau@A$.

\begin{lemma}[Preservation of $\beta$ reduction in $\natural$]
	If $\G \V M:\tau@A$ and $M \longrightarrow_\beta N$ in $\lambda^{\text{MD}}$
	then $\natural(M) \longrightarrow_\beta^+ \natural(N)$.
\end{lemma}

Prove by induction on the derivation of $\beta$ reduction.

\begin{itemize}
	\newcommand{\R}{\longrightarrow_{\beta}}
	\item $(\lambda x:\tau.M)\ N \R M[x \mapsto N]$

	      From the definition of $\natural$, $\natural((\lambda x:\tau.M)\ N)$ = $\lambda x:\natural(\tau).\natural(M)\ \natural(N)$.

	      $\lambda x:\natural(\tau).\natural(M)\ \natural(N)$ is a typed term in $\lambda^\to$, we can do $\beta$ reduction from it.\\
	      As a result of the reduction, we get $\natural(M)[x\mapsto\natural(N)]$.

	      On the otherside, use "Inversion Lemma for Application" to $(\lambda x:\tau.M)\ N$, get $\G, x:\sigma \V M:\tau@A$ and $\G \V N:\sigma@A$.
	      From "Preservation of substitution in $\natural$", $\natural(M[x \mapsto N])$ = $\natural(M)[x\mapsto\natural(N)]$.

	\item Otherwise

	      Use the induction hypothesis.
\end{itemize}

\begin{theorem}[Strong Normalization]
	If $\G\V^A t:T$ then there is no infinite sequence of terms $(t_i)_{i\ge1}$ and $t_i \longrightarrow_{\beta, \TBL \TB,\Lambda} t_{i+1}$ for $i\ge 1$
\end{theorem}

Prove if there are infinite reductions in $\lambda^{\text{MD}}$ then there are infinite beta reductions in $\lambda^{\text{MD}}$.\\
This is because other reductions reduce the size of term.\\

Now, we can conclude there are no infinite reductions in a typed $\lambda^{\text{MD}}$ term. \\

This is because if there are no infinite reduction in a typed $\lambda^{\text{MD}}$ term $M$,
we can construct a typed term of simply typed lambda calculus $\natural(M)$ from "Preservation of typing in $\natural$".
And $\natural(M)$ has infinite reductions from "Preservation of $\beta$ reduction in $\natural$".\\

But, indeed, $\lambda^\to$ has a property of Strong Normalization, so there is no infinite reductions.

\begin{theorem}[Confluence(Church-Rosser Property)]
	Define $M \longrightarrow N$ as $M \longrightarrow_{\beta} N$ or $M\longrightarrow_\blacklozenge N$ or  $M \longrightarrow_{\Lambda} N$.\\
	For any term $M$, if $M \longrightarrow^* N$ and $M \longrightarrow^* L$,
	there exists $O$ that satisfies $N \longrightarrow^* O$ and $L \longrightarrow^* O$.
\end{theorem}

\textsc{Proof.}

Because we show the Strong Normalization of $\lambda^{\text{MD}}$, we can use Newman's lemma to prove Church-Rosser property of $\lambda^{\text{MD}}$.
Then, what we must show is Weak Church-Rosser Property now.

When we consider two dirfferent redux in a $\lambda^{\text{MD}}$ term, they can only be disjoint, or one is a part of the other.
In short, they are never overlapped each other.
So, we can reduce one of them after we reduce another.\\

\figheader{Values and Redexes}{}
$A \neq \varepsilon$\\
\begin{align*}
	\textrm{Values}                              &   & v^\varepsilon \in V^\varepsilon & ::= \lambda x:\tau.M \mid\ \TB_\alpha v^\alpha \mid \Lambda\alpha.v^\varepsilon              & \\
	                                             &   & v^A \in V^A                     & ::= x \mid \lambda x:\tau.v^A \mid v^A\ v^A \mid\ \TB_\alpha v^{A\alpha}
	\mid \Lambda\alpha.v^A \mid v^A\ \varepsilon &                                                                                                                                      \\
	                                             &   &                                 & \quad\   \mid\ \TBL_\alpha v^{A'} (\text{if } A'\alpha = A \text{ and } A' \neq \varepsilon) & \\
	                                             &   &                                 & \quad\   \mid \%_\alpha v^{A'} (\text{if } A'\alpha = A)                                     & \\
	\textrm{Redexes}                             &   & R^\varepsilon                   & ::= (\lambda x:\tau.M)\ v^\varepsilon \mid (\Lambda\alpha.v^\varepsilon)\ \varepsilon        & \\
	                                             &   & R^\alpha                        & ::=\ \TBL_\alpha \TB_\alpha v^\alpha                                                         & \\
\end{align*}

\begin{dfn}[Reduction]
	$ M \longrightarrow M'$ iff \\
	$ M \longrightarrow_\Lambda M' $, $ M \longrightarrow_\blacklozenge M' $ or $ M \longrightarrow_\beta M' $.
\end{dfn}

\begin{theorem}[Progress]
	If $x:\tau@\varepsilon \notin \G$ and $\G \V M : \tau @ A$ then $ M \in V^A $ or $\exists M'$ such that $M \longrightarrow M'$.
\end{theorem}

Prove by induction on the type derivation tree of $\G \V M:\tau@A$.

\begin{itemize}
	\item \TVar
	      \begin{itemize}
		      \item $ A = \varepsilon$

		            This case is impossible because $x:\tau@\varepsilon \notin \G$.
		      \item Otherwise

		            This case is obvious because $x \in V^A$.
	      \end{itemize}

	\item \TTBL

	      The derivation is $\MD{1}$.

	      $\MD{1}$ = \infer[\TTBL]
	      {\G \V \TBL_\alpha M :\tau @ A\alpha}
	      {\ID{\G \V M : \TW_\alpha \tau @ A}}

	      There are two cases for the induction hypothesis.

	      \begin{itemize}

		      \item $ M \in V^A $

		            \begin{itemize}
			            \item $ A = \varepsilon $

			                  Use "Inversion Lemma" for all cases of $v^\varepsilon$, the case of $ M = \TB_\alpha v^\alpha $ is only reasonable.\\
			                  From the definition of $ \longrightarrow $, $\TBL_\alpha \TB_\alpha v^\alpha \longrightarrow v^\alpha$

			            \item Otherwise

			                  $ \TBL_\alpha M \in V^{A\alpha}$.
		            \end{itemize}

		      \item $\exists M'$ such that $M \longrightarrow M'$

		            From the definition of $ \longrightarrow $, $\TBL_\alpha M \longrightarrow \TBL_\alpha M'$

	      \end{itemize}

	\item \TApp

	      The derivation looks like $\MD{1}$.

	      $\MD{1}$ = \infer[\TApp]
	      {\G \V M\ N :\tau[x\mapsto N] @ A}
	      {\ID{\G \V M : (\Pi x:\sigma.\tau) @ A} \andalso \ID{\G \V N : \sigma @ A}}

	      \begin{itemize}
		      \item $M \in V^A$ and $N \in V^A$

		            \begin{itemize}
			            \item $A=\varepsilon$

			                  $M = \lambda x:\sigma.L$ from the definition of $V^\varepsilon$ and Inversion Lemma.\\
			                  Then, $\lambda x:\sigma.L\ N \longrightarrow_\beta L[x\mapsto N]$.
			            \item Otherwise

			                  $M\ N \in V^A$.
		            \end{itemize}
		      \item Otherwise

		            $M\ N \longrightarrow M'\ N$ or $M\ N \longrightarrow M\ N'$.
	      \end{itemize}

	\item \TIns

	      The derivation looks like $\MD{1}$.

	      $\MD{1}$ = \infer[\TIns]
	      {\G \V M\ B : \tau[\alpha \mapsto B] @ A}
	      {\ID{\G \V M : \forall\alpha.\tau @ A}}

	      \begin{itemize}
		      \item $ M \in V^A $

		            \begin{itemize}
			            \item $A=\varepsilon$

			                  $M = \Lambda\alpha.v^\varepsilon$ from the definition of $V^\varepsilon$ and Inversion Lemma.\\
			                  Then, $\Lambda\alpha.v^\varepsilon\ B \longrightarrow_\Lambda v^\varepsilon[x\mapsto B]$.

			            \item Otherwise

			                  $v^A B \in V^A$.
		            \end{itemize}

		      \item $\exists M'$ such that $M \longrightarrow M'$

		            $M\ \varepsilon \longrightarrow M'\ B$
	      \end{itemize}

	\item Others

	      It is obvious from the induction hypothesis.

\end{itemize}

\section{ Staged Semantics }

\AI{You don't have to repeat definitions.}

\figheader{Staged Reduction}{}
$A \neq \varepsilon$\\
\begin{align*}
	E^A_\varepsilon [(\lambda x:\tau.M)\ v^\varepsilon]          & \longrightarrow_s E^A_\varepsilon[M[x\mapsto v^\varepsilon]]                \\
	E^A_\varepsilon [(\Lambda\alpha.v^\varepsilon)\ B] & \longrightarrow_s E^A_\varepsilon[v^\varepsilon[\alpha\mapsto B]] \\
	E^A_\alpha [\TBL_\alpha \TB_\alpha v^\alpha]                 & \longrightarrow_s E^A_\alpha[v^\alpha]                                      \\
\end{align*}

\figheader{Evaluation Context}{}
$A \neq \varepsilon$\\
\begin{align*}
	E^\varepsilon_B \in ECtx^\varepsilon_B & ::= \square\ (\text{if\ } B = \varepsilon) \mid E^\varepsilon_B\ M \mid v^e\ E^\varepsilon_B
	\mid \TB_\alpha E^\alpha_B \mid \Lambda\alpha.E^\varepsilon_B
	\mid E^\varepsilon_B\ B                                                                                                     \\
	E^A_B \in ECtx^A_B                     & ::= \square\ (\text{if } A = B) \mid \lambda x:\tau.E^A_B \mid E^A_B\ M \mid v^A\ E^A_B
	\mid E^\varepsilon_B \mid \TB_\alpha E^{A\alpha}_B
	\mid \TBL_\alpha E^{A'}_B \ (\text{where } A'\alpha = A)                                                                              \\
	                                       & \quad \mid \Lambda\alpha.E^\varepsilon_B
	\mid E^A_B\ B \mid \%_\alpha\ E^{A'}_B \ (\text{where } A'\alpha = A)                                                       \\
\end{align*}

\begin{lemma}[Unique Decomposition]
	If $x:\tau@\varepsilon \notin \G$ and $\G \V M : \tau @ A$ then 1 or 2 is true.
	\begin{enumerate}
		\item $ M \in V^A$
		\item $\exists ! B, E^A_B, R^B$ such that ($B = \varepsilon$ or $B = \beta$) and $M = E^A_B[R^B]$.
	\end{enumerate}
\end{lemma}

Prove by induction on the type derivation tree of $\G \V M:\tau@A$.

\begin{itemize}
	\item \TVar
	      \begin{itemize}
		      \item $ A = \varepsilon$

		            This case is impossible because $x:\tau@\varepsilon \notin \G$.
		      \item Otherwise

		            This case is obvious because $x \in V^A$.
	      \end{itemize}

	\item \TTBL
	      \begin{itemize}
		      \item $ A = \varepsilon$

		            This case is impossible because the stage of the conclusion of \TTBL\ cannot be $\varepsilon$.

		      \item Otherwise

		            The derivation is $\MD{1}$.

		            $\MD{1}$ = \infer[\TTBL]
		            {\G \V \TBL_\alpha M :\tau @ \alpha}
		            {\ID{\G \V M : \TW_\alpha \tau @ \varepsilon}}

		            From the induction hypothesis, 1 or 2 is true.
		            \begin{enumerate}
			            \item $ M \in V^\varepsilon$
			            \item $\exists ! B, E^\varepsilon_B, R^B$ such that ($B = \varepsilon$ or $B = \beta$) and $M = E^\varepsilon_B[R^B]$.
		            \end{enumerate}

		            \begin{itemize}
			            \item $ M \in V^\varepsilon$ is true

			                  Use "Inversion Lemma" for all cases of $v^\varepsilon$, the case of $ M = \TB_\alpha v^\alpha $ is only reasonable.

			                  Then, $\TBL_\alpha \TB_\alpha v^\alpha = E^\alpha_\alpha [R^\alpha]$.

			            \item $\exists ! B, E^\varepsilon_B, R^B$ such that ($B = \varepsilon$ or $B = \beta$) and $M = E^\varepsilon_B[R^B]$ is true.

			                  \begin{itemize}
				                  \item $ M = \TB_\alpha E^\alpha_B[R^B] $

				                        $ \TBL_\alpha \TB_\alpha E^\alpha_B[R^B] \longrightarrow_s E^\alpha_B[R^B]$ doesn't hold because $ E^\alpha_B[R^B] \notin v^\alpha$.
				                        So, given $B, E^\varepsilon_B, R^B$ are the unique tuples satisfies the condition.
				                  \item Otherwise

				                        It is obvious from the induction hypothesis and the definition of $E^A_B$.
			                  \end{itemize}
		            \end{itemize}

	      \end{itemize}

	\item \TIns

	      \begin{itemize}
		      \item $ A = \varepsilon$

		            The derivation is $\MD{1}$.

		            $\MD{1}$ = \infer[\TIns]
		            {\G \V M\ C :\tau[\alpha \mapsto C] @ \varepsilon}
		            {\ID{\G \V M : \forall\alpha.\tau @ \varepsilon}}

		            From the induction hypothesis, 1 or 2 is true.

		            \begin{enumerate}
			            \item $ M \in V^\varepsilon$
			            \item $\exists ! B, E^\varepsilon_B, R^B$ such that ($B = \varepsilon$ or $B = \beta$) and $M = E^\varepsilon_B[R^B]$.
		            \end{enumerate}

		            \begin{itemize}
			            \item $ M \in V^\varepsilon$

			                  Use Inversion Lemma for all shape in $v^\varepsilon$, the case of $ M = \Lambda\alpha.v^\varepsilon$ is only reasonable.

			                  Then, $ \Lambda\alpha.v^\varepsilon\ C = E^\varepsilon_\varepsilon [R^\varepsilon]$
			            \item $\exists ! B, E^\varepsilon_B, R^B$ such that ($B = \varepsilon$ or $B = \beta$) and $M = E^\varepsilon_B[R^B]$

			                  Because $ E^\varepsilon_B[R^B] \neq \Lambda\alpha.v^\varepsilon$, we can decompose $E^\varepsilon_B[R^B]\ B$ uniquely.
		            \end{itemize}

		      \item $ A \neq \varepsilon $

		            The derivation is $\MD{1}$.

		            $\MD{1}$ = \infer[\TIns]
		            {\G \V M\ C :\tau[\alpha \mapsto C] @ A}
		            {\ID{\G \V M : \forall\alpha.\tau @ A}}

		            From the induction hypothesis, 1 or 2 is true.

		            \begin{enumerate}
			            \item $ M \in V^A$
			            \item $\exists ! B, E^A_B, R^B$ such that ($B = \varepsilon$ or $B = \beta$) and $M = E^A_B[R^B]$.
		            \end{enumerate}

		            \begin{itemize}
			            \item $ M \in V^A$

			                  It is clear that $v^A\ C \in V^A$.

			            \item $\exists ! B, E^A_B, R^B$ such that ($B = \varepsilon$ or $B = \beta$) and $M = E^A_B[R^B]$

			                  Because we cannot $\Lambda$ reduction at stage $A$, we can decompose $E^A_B[R^B]\ C$ uniquely.
		            \end{itemize}


	      \end{itemize}

	\item \TApp

	      \begin{itemize}
		      \item $ A = \varepsilon$

		            The derivation is $\MD{1}$.

		            $\MD{1}$ = \infer[\TApp]
		            {\G \V M\ N :\tau[\tau \mapsto N] @ \varepsilon}
		            {\ID{\G \V M : \Pi x:\sigma.\tau @ \varepsilon} \andalso \ID{\G \V N : \sigma @ \varepsilon}}

		            From the induction hypothesis, 1 or 2 is true.

		            \begin{enumerate}
			            \item $ M \in V^\varepsilon$ and $ N \in V^\varepsilon$
			            \item $ M \in V^\varepsilon$ and $\exists ! B, E^\varepsilon_B, R^B$ such that ($B = \varepsilon$ or $B = \beta$) and $N = E^\varepsilon_B[R^B]$
			            \item $\exists ! B, E^\varepsilon_B, R^B$ such that ($B = \varepsilon$ or $B = \beta$) and $M = E^\varepsilon_B[R^B]$ and $ N \in V^\varepsilon$
			            \item $\exists ! B, E^\varepsilon_B, R^B$ such that ($B = \varepsilon$ or $B = \beta$) and $M = E^\varepsilon_B[R^B]$ and $\exists ! B', E^\varepsilon_{B'}, R^{B'}$ such that ($B' = \varepsilon$ or $B' = \beta$) and $N = E^\varepsilon_{B'}[R^{B'}]$
		            \end{enumerate}

		            \begin{itemize}
			            \item Case of 1

			                  Use Inversion Lemma for all shape in $v^\varepsilon$, the case of $ M = \lambda x:\sigma.v^\varepsilon$ is only reasonable.

			                  Then $M N = R^\varepsilon$.

			            \item Otherwise

			                  It is clear.
		            \end{itemize}

		      \item $ A \neq \varepsilon $

		            $M N \notin R^\varepsilon$ because $ A \neq \varepsilon$.
		            So, we can decompose uniquely.

	      \end{itemize}


	\item \TConv

	      We can use the induction hypothesis directly.

\end{itemize}

\AI{Removed the work-in-progress part.}
\endinput
\blue{\huge{It is work in progress from here.}}

\begin{dfn}[\% Powerset of a Term]
	For a term $M$, $M^\%$ is a set of terms.\\
	$M' \in M^\%$ iff you can get $M'$ from $M$ by removing arbitary number $\%$.
\end{dfn}

\begin{lemma}[Equality and Reduction]
	$\G \V M \E N : \tau @A$ if and only if\\
	$\exists M' \in M^\%, M' \in M^\%, \exists N' \in N^\%, \exists L$
	such that $\G \V M' : \tau$, $\G \V N' : \tau$, $M' \longrightarrow^* L$ and $N' \longrightarrow^* L$.
\end{lemma}

\red{TODO}


% \section{ Deterministic Typechecking }
%
% \figheader{Well-formed kinds}{\rulefbox{\Gamma \vdash K\iskind}}
% \begin{center}
%     \infrule{}{\G\VT *\iskind @A}{\WStar} \andalso
%     \infrule{\G\VT \tau::*@A \andalso \G,x:\tau@A\VT K\iskind @A}{\G\VT(\Pi x:\tau.K)\iskind @A}{\WAbs}\andalso
%     \infrule{\G\VT K\iskind @A}{\G\VT K\iskind @A\alpha}{\WCsp}\\[2mm]
% \end{center}
% 
% \figheader{Kinding}{\rulefbox{\G \VT T::K}}
% \begin{center}
%     \infrule{X::K@A \in \G \andalso \G\VT K\iskind @A}{\G \VT X::K@A}{\KVar} \andalso
%     \infrule{\G\VT \tau :: *@A \andalso \G,x:\tau@A\VT \sigma::J@A}{\G\VT(\Pi x:\tau.\sigma) :: (\Pi x:\tau.J)@A}{\KAbs} \\[2mm]
%     \infrule{\G\VT \sigma:: (\Pi x:\tau.K)@A \andalso \G\VT M:\tau'@A \andalso \G\VT \tau\E\tau' @A}{\G\VT \sigma\ M::K[x\mapsto M]@A}{\KApp} \andalso
%     \infrule{\G\VT \tau::*@A\alpha}{\G\VT\TW_\alpha \tau::*@A}{\KTW}\andalso
%     \infrule{\G\VT \tau::K@A \andalso \alpha\notin\rm{FTV}(\G)\cup\rm{FTV}(A)}{\G\VT\forall\alpha.\tau::K@A}{\KGen} \andalso
%     \infrule{\G\VT \tau::K@A}{\G\VT \tau::K@A\alpha}{\KCsp} \end{center}
% 
% \figheader{Typing}{\rulefbox{\G\VT t:\tau}}
% \begin{center}
%     \infrule{x:\tau@A \in \G \andalso \G\VT \tau::*@A}{\G \VT x:\tau@A}{\TVar} \andalso
%     \infrule{\G\VT \sigma::*@A\andalso\G,x:\sigma@A\VT M:\tau@A}{\G\VT(\lambda (x:\sigma).M):(\Pi (x:\sigma).\tau)@A}{\textsc{T-Abs}} \\[2mm]
%     \infrule{\G\VT M:(\Pi (x:\sigma).\tau)@A \andalso \G\VT N:\sigma'@A \andalso \G\VT \sigma\E\sigma' @A}{\G\VT M\ N : \tau[x\mapsto N]@A}{\textsc{T-App}} \andalso
%     \infrule{\G\VT M:\tau@{A\alpha}}{\G\VT\TB_{\alpha}M:\TW_{\alpha}\tau@A}{\textsc{T-$\TB$}} \andalso
%     \infrule{\G\VT M:\TW_{\alpha}\tau@A}{\G\VT\TBL_{\alpha}M:\tau@{A\alpha}}{\TTBL} \\[2mm]
%     \infrule{\G\VT M:\tau@A \andalso \alpha\notin\rm{FTV}(\G)\cup\rm{FTV}(A)}{\G\VT\Lambda\alpha.M:\forall\alpha.\tau@A}{\textsc{T-Gen}} \andalso
%     \infrule{\G\VT M:\forall\alpha.\tau@A}{\G\VT M\ \varepsilon:\tau[\alpha \mapsto \varepsilon]@A}{\textsc{T-Ins}} \andalso
%     \infrule{\G\VT M:\tau@A}{\G\VT \%_\alpha M:\tau@{A\alpha}}{\textsc{T-Csp}} \andalso
% \end{center}
% 
% \figheader{Kind Equivalence}{\rulefbox{\G\VT K\E J@A}}
% \begin{center}
%     \infrule{\G\VT \tau \E \sigma :: *@A \andalso \G,x:\tau@A \VT K \E J@A}{\G\VT\Pi x:\tau.K \E \Pi x:\sigma.J@A}{\QKAbs} \andalso
%     \infrule{\G\VT K \E J@A}{\G\VT K \E J@{A\alpha}}{\textsc{QK-Csp}} \\[4mm]
% \end{center}
% 
% \figheader{Type Equivalence}{\rulefbox{\G\VT S\E T @A}}
% \begin{center}
%     \infrule{\G\VT \tau \E \sigma @A \andalso \G,x:\tau@A \VT \rho \E \pi @A}{\G\VT\Pi x:\tau.\rho \E \Pi x:\sigma.\pi @A}{\QTAbs} \andalso
%     \infrule{\G\VT \tau \E \sigma @A \andalso \G\VT M \E N @A}{\G\VT \tau\ M \E \sigma\ N @A}{\QTApp} \\[2mm]
%     \infrule{\G\VT \tau \E \sigma @{A\alpha}}{\G\VT \TW_{\alpha} \tau \E \TW_{\alpha} \sigma @A}{\textsc{QT-$\TW$}}\andalso
%     \infrule{\G\VT \tau \E \sigma @A \andalso \alpha\notin\rm{FTV}(\G)\cup\rm{FTV}(A)}{\G\VT \forall\alpha.\tau \E  \forall\alpha.\sigma @A}{\textsc{QT-Gen}} \andalso
%     \infrule{\G\VT \tau \E \sigma @A}{\G\VT \tau \E \sigma @{A\alpha}}{\textsc{QT-Csp}} \\[4mm]
% \end{center}
% 
% \figheader{Term Equivalence}{\rulefbox{\G\VT M\E N : \tau @A}}
% \begin{center}
%     Demands from syntax rules\\[2mm]
%     \infrule{\G\VT \tau \E \sigma :: *@A \andalso \G,x:\tau@A \VT M \E N : \rho @A}{\G\VT\lambda x:\tau.M \E \lambda x:\sigma.N : (\Pi x:\tau.\rho)@A}{\QAbs} \andalso
%     \infrule{\G\VT M \E L : (\Pi x:\sigma.\tau)@A \andalso \G\VT N \E O : \sigma@A}{\G\VT M\ N \E L\ O : \tau[x \mapsto N]@A}{\QApp} \\[2mm]
%     \infrule{\G\VT M \E N : \tau@{A\alpha}}{\G\VT \TB_\alpha M \E \TB_\alpha N : \TW_\alpha \tau@A}{\QTB} \andalso
%     \infrule{\G\VT M \E N : \TW_\alpha \tau@A}{\G\VT \TBL_\alpha M \E \TBL_\alpha N : \tau@{A\alpha}}{\QTBL} \\[2mm]
%     \infrule{\G\VT M\E N : \tau@A \andalso \alpha \notin \FTV(\G)\cup\FTV(A)}{\G\VT \Lambda\alpha.M \E \Lambda\alpha.N : \forall\alpha.\tau@A}{\QGen} \andalso
%     \infrule{\G\VT M \E N:\forall\alpha.\tau@A}{\G\VT M\ \varepsilon \E N\ \varepsilon : \tau[\alpha \mapsto \varepsilon]@A}{\QIns}\andalso
%     \infrule{\G\VT M \E N : \tau @A}{\G\VT\%_\alpha M \E \%_\alpha N : \tau@{A\alpha}}{\QCsp} \\[4mm]
%     Demands from equivalence relationship\\[2mm]
%     \infrule{\G\VT M:\tau@A}{\G\VT M\E M : \tau@A}{\QRefl} \andalso
%     \infrule{\G\VT M\E N : \tau@A}{\G\VT N\E M : \tau@A}{\QSym} \andalso
%     \infrule{\G\VT M\E N : \tau@A \andalso \G\VT N\E L : \tau@A}{\G\VT M\E L : \tau@A}{\QTrans} \\[4mm]
%     Demands from reduction rule\\[2mm]
%     \infrule{\G,x:\sigma@A\VT M:\tau@A \andalso \G\VT N:\sigma@A}{\G\VT(\lambda x:\sigma.M)\ N\E M[x\mapsto N] : \tau[x \mapsto N]@A}{\QBeta} \andalso
%     \infrule{\G\VT M:(\Pi x:\sigma.\tau)@A \andalso x\notin \text{FV}(M)}{\G\VT(\lambda x:\sigma.M\ x)\E M: (\Pi x:\sigma.\tau)@A}{\QEta} \\[2mm]
%     \infrule{\G\VT M \E N : \tau@A}{\G\VT \TBL_\alpha(\TB_\alpha M) \E N : \tau @A}{\QTBLTB} \andalso
%     \infrule{\G\VT (\Lambda\alpha.M) : \forall\alpha.\tau@A}{\G\VT (\Lambda\alpha.M)\ \varepsilon \E M[\alpha \mapsto \varepsilon] : \tau[\alpha \mapsto \varepsilon]@A}{\QLambda} \\[2mm]
%     \infrule{\G\VT M:\tau@{A\alpha} \andalso \G\VT M:\tau@A}{\G\VT\%_\alpha M \E M : \tau@{A\alpha}}{\QPercent} \andalso
% \end{center}

% \figheader{Well-formed kinds}{\rulefbox{\Gamma \vdash K\iskind}}
% \begin{center}
%     \infrule{}{\G\VT*\iskind @A}{\WStar} \andalso
%     \infrule{\G\VT \tau::*@A \andalso \G,x:\tau@A\VT K\iskind @A}{\G\VT(\Pi x:\tau.K)\iskind @A}{\WAbs}\andalso
%     \infrule{\G\VT K\iskind @A}{\G\VT K\iskind @A\alpha}{\WCsp}\\[2mm]
% \end{center}
% 
% \figheader{Kinding}{\rulefbox{\G \VT T::K}}
% \begin{center}
%     \infrule{X::K@A \in \G \andalso \G\VT K\iskind @A}{\G \VT X::K@A}{\KVar} \andalso
%     \infrule{\G\VT \tau :: *@A \andalso \G,x:\tau@A\VT \sigma::J@A}{\G\VT(\Pi x:\tau.\sigma) :: (\Pi x:\tau.J)@A}{\KAbs} \\[2mm]
%     \infrule{\G\VT \sigma:: (\Pi x:\tau.K)@A \andalso \G\VT M:\tau@A}{\G\VT \sigma\ M::K[x\mapsto M]@A}{\KApp} \andalso
%     \infrule{\G\VT \tau::K@A \andalso \G\VT K\equiv J@A}{\G\VT \tau::J@A}{\KConv} \\[2mm]
%     \infrule{\G\VT \tau::*@A\alpha}{\G\VT\TW_\alpha \tau::*@A}{\KTW}\andalso
%     \infrule{\G\VT \tau::K@A \andalso \alpha\notin\rm{FTV}(\G)\cup\rm{FTV}(A)}{\G\VT\forall\alpha.\tau::K@A}{\KGen} \andalso
%     \infrule{\G\VT \tau::K@A}{\G\VT \tau::K@A\alpha}{\KCsp}
% \end{center}
% 
% \figheader{Typing}{\rulefbox{\G\VT t:\tau}}
% \begin{center}
%     \infrule{x:\tau@A \in \G \andalso \G\VT \tau::*@A}{\G \VT x:\tau@A}{\TVar} \andalso
%     \infrule{\G\VT \sigma::*@A\andalso\G,x:\sigma@A\VT M:\tau@A}{\G\VT(\lambda (x:\sigma).M):(\Pi (x:\sigma).\tau)@A}{\textsc{T-Abs}} \\[2mm]
%     \infrule{\G\VT M:(\Pi (x:\sigma).\tau)@A \andalso \G\VT N:\sigma@A}{\G\VT M\ N : \tau[x\mapsto N]@A}{\textsc{T-App}} \andalso
%     \infrule{\G\VT M:\tau@A \andalso \G\VT \tau\equiv \sigma :: K@A}{\G\VT M:\sigma@A}{\textsc{T-Conv}} \\[2mm]
%     \infrule{\G\VT M:\tau@{A\alpha}}{\G\VT\TB_{\alpha}M:\TW_{\alpha}\tau@A}{\textsc{T-$\TB$}} \andalso
%     \infrule{\G\VT M:\TW_{\alpha}\tau@A}{\G\VT\TBL_{\alpha}M:\tau@{A\alpha}}{\TTBL} \\[2mm]
%     \infrule{\G\VT M:\tau@A \andalso \alpha\notin\rm{FTV}(\G)\cup\rm{FTV}(A)}{\G\VT\Lambda\alpha.M:\forall\alpha.\tau@A}{\textsc{T-Gen}} \andalso
%     \infrule{\G\VT M:\forall\alpha.\tau@A}{\G\VT M\ \varepsilon:\tau[\alpha \mapsto \varepsilon]@A}{\textsc{T-Ins}} \andalso
%     \infrule{\G\VT M:\tau@A}{\G\VT \%_\alpha M:\tau@{A\alpha}}{\textsc{T-Csp}} \andalso
% \end{center}
% 
% \figheader{Kind Equivalence}{\rulefbox{\G\VT K\E J@A}}
% \begin{center}
%     \infrule{\G\VT \tau \E \sigma :: *@A \andalso \G,x:\tau@A \VT K \E J@A}{\G\VT\Pi x:\tau.K \E \Pi x:\sigma.J@A}{\QKAbs} \andalso
%     \infrule{\G\VT K \E J@A}{\G\VT K \E J@{A\alpha}}{\textsc{QK-Csp}} \\[2mm]
%     \infrule{\G\VT K \iskind @A}{\G\VT K\E K@A}{\textsc{QK-Refl}} \andalso
%     % 簡約規則の要請\\[2mm]
% \end{center}
% 
% \figheader{Type Equivalence}{\rulefbox{\G\VT S\E T :: K @A}}
% \begin{center}
%     \infrule{\G\VT \tau \E \sigma :: *@A \andalso \G,x:\tau@A \VT \rho \E \pi :: *@A}{\G\VT\Pi x:\tau.\rho \E \Pi x:\sigma.\pi :: *@A}{\QTAbs} \andalso
%     \infrule{\G\VT \tau \E \sigma :: (\Pi x:\rho.K)@A \andalso \G\VT M \E N : \rho @A}{\G\VT \tau\ M \E \sigma\ N :: K[x \mapsto M]@A}{\QTApp} \\[2mm]
%     \infrule{\G\VT \tau \E \sigma :: *@{A\alpha}}{\G\VT \TW_{\alpha} \tau \E \TW_{\alpha} \sigma :: *@A}{\textsc{QT-$\TW$}}\andalso
%     \infrule{\G\VT \tau \E \sigma :: *@A \andalso \alpha\notin\rm{FTV}(\G)\cup\rm{FTV}(A)}{\G\VT \forall\alpha.\tau \E  \forall\alpha.\sigma :: *@A}{\textsc{QT-Gen}} \andalso
%     \infrule{\G\VT \tau \E \sigma :: K@A}{\G\VT \tau \E \sigma :: K@{A\alpha}}{\textsc{QT-Csp}} \\[2mm]
%     \infrule{\G\VT \tau::K@A}{\G\VT \tau\E\tau :: K@A}{\textsc{QT-Refl}} 
% \end{center}
% 
% \figheader{Term Equivalence}{\rulefbox{\G\VT M\E N : \tau @A}}
% \begin{center}
%     \infrule{\G\VT \tau \E \sigma :: *@A \andalso \G,x:\tau@A \VT M \E N : \rho @A}{\G\VT\lambda x:\tau.M \E \lambda x:\sigma.N : (\Pi x:\tau.\rho)@A}{\QAbs} \andalso
%     \infrule{\G\VT M \E L : (\Pi x:\sigma.\tau)@A \andalso \G\VT N \E O : \sigma@A}{\G\VT M\ N \E L\ O : \tau[x \mapsto N]@A}{\QApp} \\[2mm]
%     \infrule{\G\VT M \E N : \tau@{A\alpha}}{\G\VT \TB_\alpha M \E \TB_\alpha N : \TW_\alpha \tau@A}{\QTB} \andalso
%     \infrule{\G\VT M \E N : \TW_\alpha \tau@A}{\G\VT \TBL_\alpha M \E \TBL_\alpha N : \tau@{A\alpha}}{\QTBL} \\[2mm]
%     \infrule{\G\VT M\E N : \tau@A \andalso \alpha \notin \FTV(\G)\cup\FTV(A)}{\G\VT \Lambda\alpha.M \E \Lambda\alpha.N : \forall\alpha.\tau@A}{\QGen} \andalso
%     \infrule{\G\VT M \E N:\forall\alpha.\tau@A}{\G\VT M\ \varepsilon \E N\ \varepsilon : \tau[\alpha \mapsto \varepsilon]@A}{\QIns}\andalso
%     \infrule{\G\VT M \E N : \tau @A}{\G\VT\%_\alpha M \E \%_\alpha N : \tau@{A\alpha}}{\QCsp} \\[2mm]
%     \infrule{\G\VT M:\tau@A}{\G\VT M\E M : \tau@A}{\QRefl} \\[2mm]
%     \infrule{\G,x:\sigma@A\VT M:\tau@A \andalso \G\VT N:\sigma@A}{\G\VT(\lambda x:\sigma.M)\ N\E M[x\mapsto N] : \tau[x \mapsto N]@A}{\QBeta} \andalso
%     \infrule{\G,x:\sigma@A\VT M:\tau@A \andalso \G\VT N:\sigma@A}{\G\VT M[x\mapsto N] \E (\lambda x:\sigma.M)\ N : \tau[x \mapsto N]@A}{\QBeta T} \\[2mm]
%     \infrule{\G\VT M:(\Pi x:\sigma.\tau)@A \andalso x\notin \text{FV}(M)}{\G\VT(\lambda x:\sigma.M\ x)\E M: (\Pi x:\sigma.\tau)@A}{\QEta} \andalso
%     \infrule{\G\VT M:(\Pi x:\sigma.\tau)@A \andalso x\notin \text{FV}(M)}{\G\VT M \E (\lambda x:\sigma.M\ x) @A}{\QEta T} \\[2mm]
%     \infrule{\G\VT M \E N : \tau@A}{\G\VT \TBL_\alpha(\TB_\alpha M) \E N : \tau @A}{\QTBLTB} \andalso
%     \infrule{\G\VT M \E N : \tau@A}{\G\VT N \E \TBL_\alpha(\TB_\alpha M) : \tau @A}{\QTBLTB T} \\[2mm]
%     \infrule{\G\VT (\Lambda\alpha.M) : \forall\alpha.\tau@A}{\G\VT (\Lambda\alpha.M)\ \varepsilon \E M[\alpha \mapsto \varepsilon] : \tau[\alpha \mapsto \varepsilon]@A}{\QLambda} \andalso
%     \infrule{\G\VT (\Lambda\alpha.M) : \forall\alpha.\tau@A}{\G\VT M[\alpha \mapsto \varepsilon] \E (\Lambda\alpha.M)\ \varepsilon : \tau[\alpha \mapsto \varepsilon]@A}{\QLambda} \\[2mm]
%     \infrule{\G\VT M:\tau@{A\alpha} \andalso \G\VT M:\tau@A}{\G\VT\%_\alpha M \E M : \tau@{A\alpha}}{\QPercent} \\[2mm]
%     \infrule{\G\VT M:\tau@{A\alpha} \andalso \G\VT M:\tau@A}{\G\VT M \E \%_\alpha M : \tau@{A\alpha}}{\QPercent}
% \end{center}


\fi
\section{Appendix}

\blue{We solicit submissions in the form of regular research papers describing original scientific research results, including system development and case studies. Regular research papers should not exceed 18 pages in the Springer LNCS format, including bibliography and figures. This category encompasses both theoretical and implementation (also known as system descriptions) papers. In either case, submissions should clearly identify what has been accomplished and why it is significant. Submissions will be judged on the basis of significance, relevance, correctness, originality, and clarity. System descriptions papers should contain a link to a working system and will be judged on originality, usefulness, and design. In case of lack of space, proofs, experimental results, or any information supporting the technical results of the paper could be provided as an appendix or a link to a web page, but reviewers are not obliged to read them.}
\blue{Review Process
	APLAS 2019 will use a lightweight double-blind reviewing process. Following this process means that reviewers will not see the authors’ names or affiliations as they initially review a paper. The authors’ names will then be revealed to the reviewers only once their reviews have been submitted.
	To facilitate this process, submitted papers must adhere to the following:
	Author names and institutions must be omitted and
	References to the authors’ own related work should be in the third person (e.g., not “We build on our previous work …” but rather “We build on the work of …”).
	The purpose of this process is to help the reviewers come to an initial judgement about the paper without bias, not to make it impossible for them to discover the authors if they were to try. Nothing should be done in the name of anonymity that weakens the submission, makes the job of reviewing the paper more difficult, or interferes with the process of disseminating new ideas. For example, important background references should not be omitted or anonymized, even if they are written by the same authors and share common ideas, techniques, or infrastructure. Authors should feel free to disseminate their ideas or draft versions of their paper as they normally would. For instance, authors may post drafts of their papers on the web or give talks on their research ideas.}
\end{document}
